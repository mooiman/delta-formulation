\chapter{2D Wave numerical experiments}\label{sec:2d_numerical_experiments}
%------------------------------------------------------------------------------
\section{Gaussian hump}
In this section some numerical results are given for the
2D Gaussian hump.
The numerical tests are performed with some different time steps ($\Dt$), different space resolution (but with $\Dx=\Dy$) and different geometry, but all with the same constant bed level at $z_b = \bqty{-10}{\metre}$.
%--------------------------------------------------------------------------------
\subsection{Gaussian hump}\label{sec:num_exp_gaussian_hump}
The considered 2-D wave equation reads:
\begin{align}
    \pdiff{h}{t}  + \nabla \dotp \vec{q} & = 0 \qquad \textit{continuity eq.} \\
    \pdiff{q}{t}  + g h \nabla \zeta & = 0 \qquad \textit{momentum eq.}
\end{align}
%
With initial conditions
\begin{align}
    h(x,y,0) & = \zeta(x,y,0) - z_b(x,y), \quad \bunit{\metre} \\
    q(x,y,0) & = 0, \quad \bunit{\square\metre\per\second}
\end{align}
for the the water level a Gaussian hump is prescribed
\begin{align}
    \zeta(x,y) =  a_0 \exp\left[ -\left( \frac{(x - \mu_x)^2}{2\sigma^2_x} + \frac{(y - \mu_y)^2}{2\sigma^2_y}\right)\right], \quad \bunit{\metre}.
\end{align}
At the boundaries no ingoing signals are prescribed, so outgoing signals are leaving the domain unhampered, which means no reflections will be present other then numerical reflections.
%--------------------------------------------------------------------------------
\paragraph*{Numerical experiment}
The numerical experiment is performed with the following parameters:
\begin{itemize}
    \item Length of the domain, $L_x = L_y = \bqty{6000}{\metre}$, ranging from $\bqty{-3000}{\metre}$ to $\bqty{3000}{\metre}$.
    \item Bed level, $z_b = \bqty{-10}{\metre}$.
    \item Grid size, $\Dx = \Dy = \bqty{10}{\metre}$.
    \item Start time, $t_{start} = \bqty{0}{\second}$.
    \item End time, $t_{stop} = \bqty{1800}{\second}$.
    \item Timestep, $\Dt = \bqty{10}{\second}$.
    \item Amplitude of the Gaussian hump, $a_0 = \bqty{0.01}{\metre}$.
    \item Centre of the Gaussian hump, $\mu_x = \mu_y = \bqty{0}{\metre}$.
    \item Spreading of the Gaussian hump, $\sigma_x = \sigma_y = \bqty{350}{\metre}$.
\end{itemize}

%--------------------------------------------------------------------------------
\paragraph*{Results of the numerical experiments 1}
\notyet
%--------------------------------------------------------------------------------
\subsection{Gaussian hump, reflection}\label{sec:gaussian_hump_refelction}
The Gaussian hump test case is used to determine the reflection of a wave at the boundary. The schematization for this test is shown in \autoref{fig:two_mesh_reflection}.
\begin{figure}[H]
    \centering
    \includegraphics[width=0.9\textwidth]{figures/two_meshes_for_reflection.png}
    \caption{Schematization area: \emph{[-6000,6000]x[-6000, 6000]} \bunit{m} and \emph{[-6000, 6000]x\newline [-3000, 3000]} \bunit{\metre}. The observation points at $y=\bqty{3000}{\metre}$ are present on both meshes and are used to determine the reflection of a wave at these locations.}\label{fig:two_mesh_reflection}
\end{figure}

The reflection coefficient is determined by:
%\begin{align}
%    R = \frac{  \abs{\max(h_{\mathit small}) - \max(h_{\mathit large})}}{\abs{\max(h_{\mathit large})} } \times 100\ \%, \quad 0 < t < \bqty{1800}{\second}
%\end{align}
\begin{align}
    R = \frac{  \abs{\max(h_{\mathit small}(t,\vec{x}_{\mathit obs}}) - \max(h_{\mathit large}(t,\vec{x}_{\mathit obs})}{h_0} \times 100\ \%, \quad 0 < t < \bqty{1800}{\second}
\end{align}
with
\begin{symbollist}
    \item[$R$] Reflection coefficient, \bunit{-}.
    \item[$t$] Simulation time, \bunit{\second}.
    \item[$h_0$] Still water depth, in both models the same, here \bqty{10}{\metre}.
    \item[$h_{\mathit small}$] Total water depth, in the small domain.
    \item[$h_{\mathit large}$] Total water depth, in the large domain.
    \item[$\vec{x}_{\mathit obs}$] Location of the coincide observation points in the small and large domain.
\end{symbollist}
    \notyet
%------------------------------------------------------------------------------
\section{Schematic river section, constant bed level}
The schematic test case is based on river section of the river Waal.
In this section a test case is presented for an orthogonal mesh with unequal mesh size in $x$- and $y$-direction (i.e.\ $\Dx \neq \Dy$) but constant in the coordinate direction (\autoref{sec:ortho_river_section}), and a curvilinear mesh with the $y$-coordinate regularized (\autoref{sec:non_ortho_river_section}).

\paragraph*{Numerical experiment}

The numerical experiments are performed with the following initial conditions:
\begin{align}
    \zeta(x,y,0) & = 0.01, \quad \bunit{\metre} \\
    z_b(x,y,0) & = -4.0, \quad \bunit{\metre} \\
    q(x,y,0) & = 0, \quad \bunit{\square\metre\per\second}
\end{align}
and boundary conditions:
\begin{align}
    q(t) & = 4, \text{ left boundary},\quad \bunit{\square\metre\per\square\second} \\
    \zeta(t) & = 0,  \text{ right boundary},\quad \bunit{m}
\end{align}
and the following parameters
\begin{itemize}
    \item Length, \bqty{17.5}{\kilo\metre}
    \item Width, \bqty{2.0}{\kilo\metre}
\begin{itemize}
    \item Orthogonal mesh size, $\Dx = \bqty{80}{\metre}$ and $\Dy = \qty{40}{\metre}$.
    \item Non-orthogonal mesh size, 20 gridcell over the width of the river and the other 20 over the floodplains (\autoref{fig:non_ortho_grid_a}). The mesh shown in \autoref{fig:non_ortho_grid_b} is used for the computations.
\end{itemize}
    \item Bed friction, Ch\'ezy coefficient \bqty{50}{{\metre}^{1/2}\, s^{-1}}
\end{itemize}

%------------------------------------------------------------------------------
\subsection{Orthogonal mesh}\label{sec:ortho_river_section}
For this test case the following mesh is used, the river outline is also given in this figure because that is used in \autoref{sec:non_ortho_river_section}:
\begin{figure}[H]
    \centering
    \includegraphics[width=\textwidth]{figures/meander_cartesian.png}
    \caption{Constant quadrangular mesh with $\Dx = \bqty{80}{\metre}$ and $\Dy = \bqty{40}{\metre}$.}
\end{figure}

%------------------------------------------------------------------------------
\subsection{Non-orthogonal meandering mesh}\label{sec:non_ortho_river_section}

The non-orthogonal meandering test case is used to compute a uniform discharge over a non-orthogonal meandering mesh with constant $\Dx$.
To obtain the computational grid we start from the grid as shown in \autoref{fig:non_ortho_grid_a}, this mesh is regularized for the $y$-coordinate (\autoref{fig:non_ortho_grid_b}) and then used for the computations.
For this test case the following mesh is used:
\begin{figure}[H]
    \includegraphics[width=\textwidth]{figures/meander_constant_dx_5_river20.png}
    \caption{Initial constructed mesh with 20 grid cells over the width of the river}
    \label{fig:non_ortho_grid_a}
\end{figure}
\begin{figure}[H]
    \includegraphics[width=\textwidth]{figures/meander_constant_dx_5_river20_smoothed.png}
    \caption{Non orthogonal mesh, smooth $y$-coordinate over the river width}
    \label{fig:non_ortho_grid_b}
\end{figure}
%------------------------------------------------------------------------------
\subsection*{Results of the numerical experiments}
\notyet
%In this section some results are presented of the previous two sections (\nameref{sec:ortho_river_section} and \nameref{sec:non_ortho_river_section}).
%Time histories of the centre point of the model
%\begin{figure}[H]
%    \includegraphics[width=\textwidth]{figures/maze.pdf}
%    \caption{Water level time history at the centre point of the model}
%    \label{fig:non_ortho_grid_b}
%\end{figure}
%
%Vector plot of the velocity:
%\begin{figure}[H]
%    \includegraphics[width=\textwidth]{figures/maze.pdf}
%    \caption{Water level time history at the centre point of the model}
%    \label{fig:non_ortho_grid_b}
%\end{figure}
%%
%and an iso-line plot of the velocity magnitude:
%\begin{figure}[H]
%    \includegraphics[width=\textwidth]{figures/maze.pdf}
%    \caption{Water level time history at the centre point of the model}
%    \label{fig:non_ortho_grid_b}
%\end{figure}
