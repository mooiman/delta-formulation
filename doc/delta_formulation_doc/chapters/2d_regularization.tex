%--------------------------------------------------------------------------------
\chapter{Regularization of a two dimensional given function $u_{\mathit{giv}}$}
The regularization function read, based on \citet[eq.\ 6]{Borsboom1998}:
\begin{align}
    \widetilde{u} - \nabla \dotp \bPsi \nabla \widetilde{u} = u_{\mathit{giv}},
    \label{eq:regularization}
\end{align}
with $\bPsi$ the artificial viscosity and defined as:
\begin{align}
    \bPsi =
    \begin{pmatrix}
        \Psi_{11} & \Psi_{12} \\
        \Psi_{21} & \Psi_{22}
    \end{pmatrix}
    = c_\psi \begin{pmatrix}
        \Dx^2 & 0 \\
        0     & \Dy^2
    \end{pmatrix}
    E_\psi
\end{align}
where
\begin{symbollist}
    \item[$u_{giv}$] a given function, ex.\ bathymetry, viscosity, \ldots, [\si{\cdot}].
    \item[$\widetilde{u}$] Regularized/Smoothed function of $u_{giv}$, [\si{\cdot}].
    \item[$\Psi_{ij}$] (artificial) smoothing values ($i,j \in \{1,2\}$), [\si{\square\metre}]
    \item[$\Dx, \Dy$] Space discretization, $\Dx_{i} = x_{i+\half} - x_{i-\half}$ and $\Dy_{j} = y_{j+\half} - y_{j-\half}$, [\si{\metre}]
    \item[$c_{\Psi}$] Smooting factor (set by user), [\si{1/\cdot}]
    \item[$E_\Psi$] Error function, [\si{\cdot}]
\end{symbollist}
First the discretization of \autoref{eq:regularization} is discussed and second the determination of the artificial smoothing coefficient $\bPsi$.

Notation agreements:
\begin{symbollist}
    \item[$u$] Non-regularized/non-smoothed function, to be determined numerically.
    \item[$\widetilde u$] Regularized/smoothed function, denoted by the wavy line.
    \item[$\overline u$] Piecewise linear function, denoted by the bar.
    \item[$u_i$] Value of the numerical value at point $x_i$.
\end{symbollist}
%--------------------------------------------------------------------------------
\section{Discretization} % of \autoref{eq:regularization}}
The finite volume approach of this equation read:
\begin{align}
    \int_\Omega \widetilde{u}\, d\omega  - \int_\Omega \nabla \dotp \bPsi \nabla \widetilde{u}\, d\omega  = \int_\Omega  u_{\mathit{giv}}\, d\omega \label{eq:2d_regularization}
\end{align}

%--------------------------------------------------------------------------------
\subsection{Integration of the first term}
The discretization of first term of \autoref{eq:2d_regularization} read:
\begin{align}
    J\, \int_{\Omega}  \utilde\, d\omega
\end{align}
which will be approximated by the sum of the integral over the sub-control volumes.

On a structured grid one control volume ($cv$) around a node consist of four sub-control volumes ($scv_i$, $i\in\{0,1,2,3\}$).
\begin{align}
    J_{cv}\int_{cv} \utilde \, d\omega &=
    J_{scv_0}\int_{scv_0} \utilde\, d\omega +
    J_{scv_1}\int_{scv_1} \utilde\, d\omega +
    \nonumber \\*
    &\quad
    +J_{scv_2}\int_{scv_2} \utilde\, d\omega +
    J_{scv_3}\int_{scv_3} \utilde\, d\omega
\end{align}
Fore the discretization on a curvilinear grid we get:
\begin{align}
    J_{cv}\int_{cv}\utilde\, d\omega \approx &
    J_{scv_0} \utilde_{i-\quart, j-\quart} +
    J_{scv_1} \utilde_{i+\quart, j-\quart} +
    \nonumber \\*
    & +
    J_{scv_2} \utilde_{i+\quart, j+\quart} +
    J_{scv_3} \utilde_{i-\quart, j+\quart}
\end{align}
with $J_{scv_i}$ the area of the sub control volume $i$.
For cartesian grids we have $J_{scv_i} = \quart \Dx\Dy$.

For the space discretizations of an arbitrary function $u$ on the quadrature point of a sub-control volume the following space interpolation is used, $\utilde$ (1-point Gauss):
\begin{align}
    \left. u\right|_{i+\quart, j+\quart} & \approx \frac{1}{16}\left( 9u_{i, j} + 3 u_{i+1,j} + u_{i+1, j+1} + 3  u_{i, j+1}\right)
\end{align}
(see for the location of the coefficients in the grid \autoref{fig:2d_structured_grid}).
%--------------------------------------------------------------------------------
\subsection{Integration of the second term}
Integration of the second term of \autoref{eq:2d_regularization} read:
\begin{align}
    \int_\Omega & \nabla \dotp \left( \mat{\Psi} \nabla u \right) \, d\omega  =
    \oint_{\Omega}  \left(  \mat{\Psi} \nabla u \right) \dotp \vec{n}\, dl =
    \\
    = & \oint_{\Omega} \left(
    \left( \Psi_{11} \pdiff{{u}}{x} + \Psi_{12} \pdiff{{u}}{y}  \right) \nx
    + \left( \Psi_{21} \pdiff{{u}}{x} + \Psi_{22} \pdiff{{u}}{y}  \right) \ny
    \right) \norm{dl}
\end{align}
with $\vecn = (\nx, \ny)^T$ the outward normal vector.

The term within the integral:
\begin{align}
    \pdiff{}{x} \left( \Psi_{11} \pdiff{{u}}{x} + \Psi_{12} \pdiff{{u}}{y}  \right) +
    \pdiff{}{y} \left( \Psi_{21} \pdiff{{u}}{x} + \Psi_{22} \pdiff{{u}}{y}  \right)
\end{align}
can be transformed to\newline
%\textit{step 1a}
%\begin{align}
%    &\frac{1}{J} \Bigg[
%    y_n \pdiff{}{\xi} \left(  \Psi_{11} \pdiff{{u}}{x} + \Psi_{12} \pdiff{{u}}{y} \right) - y_\xi \pdiff{}{\eta} \left(  \Psi_{11} \pdiff{{u}}{x} + \Psi_{12} \pdiff{{u}}{y} \right)
%    +
%    \\*
%    & \qquad
%    -x_\eta \pdiff{}{\xi} \left(  \Psi_{21} \pdiff{{u}}{x} + \Psi_{22} \pdiff{{u}}{y}  \right) + x_\xi \pdiff{}{\eta} \left( \Psi_{21} \pdiff{{u}}{x} + \Psi_{22} \pdiff{{u}}{y}   \right)
%    \Bigg]
%\end{align}
%\textit{step 1b}\newline
%Assume $x_{\xi\eta} = x_{\eta\xi}$ and $y_{\xi\eta} = y_{\eta\xi}$
%\begin{align}
%    &\frac{1}{J} \Bigg[
%    \pdiff{}{\xi} \left( y_\eta \left( \Psi_{11} \pdiff{{u}}{x} + \Psi_{12} \pdiff{{u}}{y} \right) \right)
%    - \pdiff{}{\eta} \left( y_\xi \left(  \Psi_{11} \pdiff{{u}}{x} + \Psi_{12} \pdiff{{u}}{y} \right) \right)
%    +
%    \\*
%    & \qquad
%    -\pdiff{}{\xi} \left( x_\eta  \left( \Psi_{21} \pdiff{{u}}{x} + \Psi_{22} \pdiff{{u}}{y} \right) \right)
%    + \pdiff{}{\eta} \left( x_\xi \left( \Psi_{21} \pdiff{{u}}{x} + \Psi_{22} \pdiff{{u}}{y} \right) \right)
%    \Bigg]
%\end{align}
%Rearrange this equation to a divergence type in $\xi$- and $\eta$-direction to prepare to use Green's theorem, yields:
%\begin{align}
%    &\frac{1}{J} \Bigg[
%    \pdiff{}{\xi} \left( y_\eta \left( \Psi_{11} \pdiff{{u}}{x} + \Psi_{12} \pdiff{{u}}{y} \right)
%    - x_\eta  \left( \Psi_{21} \pdiff{{u}}{x} + \Psi_{22} \pdiff{{u}}{y} \right) \right)
%    +
%    \\*
%    & \qquad
%    + \pdiff{}{\eta} \left( -y_\xi \left(  \Psi_{11} \pdiff{{u}}{x} + \Psi_{12} \pdiff{{u}}{y} \right)
%    +   x_\xi \left( \Psi_{21} \pdiff{{u}}{x} + \Psi_{22} \pdiff{{u}}{y} \right) \right)
%    \Bigg]
%\end{align}
%Applying Green's theorem, and multiplying with $J$ (because the whole heat equation is multiplied by $J$)\newline
%\textit{step 2a}:
%\begin{align}
%    &
%    \left( y_\eta \left( \Psi_{11} \pdiff{{u}}{x} + \Psi_{12} \pdiff{{u}}{y} \right)
%    - x_\eta  \left( \Psi_{21} \pdiff{{u}}{x} + \Psi_{22} \pdiff{{u}}{y} \right) \right) \nxi
%    +
%    \\*
%    & \qquad
%    + \left( -y_\xi \left(  \Psi_{11} \pdiff{{u}}{x} + \Psi_{12} \pdiff{{u}}{y} \right)
%    +   x_\xi \left( \Psi_{21} \pdiff{{u}}{x} + \Psi_{22} \pdiff{{u}}{y} \right) \right) \neta
%\end{align}
%with $\vecn = (\nxi, \neta)^T$ the outward normal vector.
%
%\textit{step 2b}:
\begin{align}
    &
    \frac{1}{J} \Bigg[ y_\eta \left(
    \Psi_{11} \left( y_\eta \pdiff{u}{\xi} - y_\xi  \pdiff{u}{\eta}\right)
    + \Psi_{12} \left( - x_\eta \pdiff{u}{\xi} + x_\xi \pdiff{u}{\eta} \right) \right) +
    \\*
    & \quad - x_\eta \left(
    \Psi_{21} \left( y_\eta \pdiff{u}{\xi} - y_\xi  \pdiff{u}{\eta}\right)
    + \Psi_{22} \left( - x_\eta \pdiff{u}{\xi} + x_\xi \pdiff{u}{\eta} \right)
    \right)
    \Bigg] \nxi
    +
    \\
    &\frac{1}{J} \Bigg[
    -y_\xi \left(
    \Psi_{11} \left( y_\eta \pdiff{u}{\xi} - y_\xi  \pdiff{u}{\eta}\right)
    + \Psi_{12} \left( - x_\eta \pdiff{u}{\xi} + x_\xi \pdiff{u}{\eta} \right)  \right) +
    \\*
    &\quad +   x_\xi \left(
    \Psi_{21} \left( y_\eta \pdiff{u}{\xi} - y_\xi  \pdiff{u}{\eta}\right)
    + \Psi_{22} \left( - x_\eta \pdiff{u}{\xi} + x_\xi \pdiff{u}{\eta} \right)  \right)
    \Bigg] \neta
    \label{eq:diffusion_ugiven}
\end{align}
with $\vecn = (\nxi, \neta)^T$ the outward normal vector.



%--------------------------------------------------------------------------------
\subsubsection*{Orthotropic}
When the artificial viscosity ($\mat{\Psi}$) is orthotropic  ($\Psi_{12} = \Psi_{21} = 0$) \autoref{eq:diffusion_ugiven} reduces to:
%($\Psi_{11} = \Psi_{22} = \Psi(\xi, \eta)$ and $\Psi_{12} = \Psi_{21} = 0$)
\begin{align}
    &
    \frac{1}{J} \Bigg[ y_\eta
    \Psi_{11} \left( y_\eta \pdiff{u}{\xi} - y_\xi  \pdiff{u}{\eta}\right)
    - x_\eta
    \Psi_{22} \left( - x_\eta \pdiff{u}{\xi} + x_\xi \pdiff{u}{\eta} \right)
    \Bigg] \nxi
    +
    \\
    & + \frac{1}{J} \Bigg[
    -y_\xi
    \Psi_{11} \left( y_\eta \pdiff{u}{\xi} - y_\xi  \pdiff{u}{\eta}\right)
    +   x_\xi
    \Psi_{22} \left( - x_\eta \pdiff{u}{\xi} + x_\xi \pdiff{u}{\eta} \right)
    \Bigg] \neta
\end{align}

%--------------------------------------------------------------------------------
\subsubsection*{Orthogonal}
In case the artificial viscosity ($\mat{\Psi}$) is also orthogonal (i.e.\ $y_\xi = x_\eta =0$) it reads:
\begin{align}
    &
    \frac{1}{J} \Bigg[ y_\eta
    \Psi_{11} \left( y_\eta \pdiff{u}{\xi} \right)
    \Bigg] \nxi
    + \frac{1}{J} \Bigg[
    x_\xi
    \Psi_{22} \left( x_\xi \pdiff{u}{\eta} \right)
    \Bigg] \neta
\end{align}

%--------------------------------------------------------------------------------
\subsubsection*{Isotropic}
In case the artificial viscosity ($\mat{\Psi}$) is isotropic then the tensor reads:
\begin{align}
    \bPsi =
    \begin{pmatrix}
        \Psi & 0 \\
        0 & \Psi
    \end{pmatrix}
\end{align}
 (isotropic, $\Psi_{11} = \Psi_{22} = \Psi$ and $\Psi_{12} = \Psi_{21} = 0$).

%--------------------------------------------------------------------------------
\subsection{Integration of right hand side}
For the integration of the right hand side of \autoref{eq:2d_regularization}, we could use a smaller integration step size, to incorporate the sub-grid scale effects.

%--------------------------------------------------------------------------------
\section{Determination of artificial smoothing coefficient $\Psi$} \label{sec:artificial_viscosity}
The artificial smoothing coefficient (\autoref{eq:regularization}) is dependent on the second derivative of given function $u_{giv}$ \citep[eq.\ 8]{Borsboom1998}.
\textbf{Assuming} that $c_E$ is constant over the whole two dimensional area, thus isotropic.

\todo{Check next equations}

\begin{align}
        &  \Bigl( \frac{1}{64} E_{i-1,j-1} + \frac{6}{64} E_{i,j-1} + \frac{1}{64} E_{i+1,j-1} +
    \nonumber \\
    & + \frac{6}{64} E_{i-1,j} + \frac{36}{64} E_{i,j} + \frac{6}{64} E_{i+1,j} +
    \nonumber \\
    &  + \frac{1}{64} E_{i-1,j+1} +\frac{6}{64} E_{i,j+1} + \frac{1}{64} E_{i+1,j+1} \Bigl) +
    \nonumber \\
    &
    %
    c_E \left(
      \left. \frac{1}{2\Dy}  \pdiff{\widetilde{u}}{x}  \right|_{i+\half,j-\quart}
    + \left. \frac{1}{2\Dy}  \pdiff{\widetilde{u}}{x}  \right|_{i+\half,j+\quart} +
    \right. \\
    &
    + \left. \frac{1}{2\Dx}  \pdiff{\widetilde{u}}{y}  \right|_{i-\quart,j+\half}
    + \left. \frac{1}{2\Dx}  \pdiff{\widetilde{u}}{y}  \right|_{i+\quart,j+\half} +
    \\
    &
    - \left. \frac{1}{2\Dy}  \pdiff{\widetilde{u}}{x}  \right|_{i-\half,j-\quart}
    - \left. \frac{1}{2\Dy}  \pdiff{\widetilde{u}}{x}  \right|_{i-\half,j+\quart}
    \\
    &
    \left.
    - \left. \frac{1}{2\Dx}  \pdiff{\widetilde{u}}{y}   \right|_{i-\quart,j-\half}
    - \left. \frac{1}{2\Dx}  \pdiff{\widetilde{u}}{y}   \right|_{i+\quart,j-\half}
    \right)
    %
    =
    \nonumber \\
    & = \frac{1}{\Dx\Dy} \int_{j-\half}^{j+\half} \int_{i-\half}^{i+\half} \abs{D} \, dx\, dy
    \label{eq:error_regularization}
\end{align}
with \citep[eq.\ 2]{Borsboom1998} (omitting the influence of curvilinear coordinates)
\begin{align}
    D_{11} = \Dx^2 \pdiff[2]{u_{giv}}{x}
    \\
    D_{22} = \Dy^2 \pdiff[2]{u_{giv}}{y}
\end{align}
and
\begin{align}
    c_E & = \max \left(  c_{\Psi}\Dx^2 \pdiff[2]{u_{giv}}{x},  c_{\Psi}\Dy^2 \pdiff[2]{u_{giv}}{y} \right)
    \nonumber \\
    & =  \max\left(
      c_{\Psi}\abs{u_{{giv}_{i-1,j}} -2 u_{{giv}_{i,j}} + u_{{giv}_{i+1,j}}},
      c_{\Psi}\abs{u_{{giv}_{i,j-1}} -2 u_{{giv}_{i,j}} + u_{{giv}_{i,j+1}}}
    \right), \forall i
\end{align}
The right hand side is approximated by:
\begin{align}
    & \frac{1}{\Dx\Dy} \int_{j-\half}^{j+\half} \int_{i-\half}^{i+\half} \abs{D} \, dx\, dy
    =
\nonumber \\
& \quad =  \frac{1}{64} \abs{D}_{i-1,j-1} +
\frac{6}{64} \abs{D}_{i,j-1} +
\frac{1}{64} \abs{D}_{i+1,j-1} +
\nonumber \\
    & \quad +
\frac{6}{64} \abs{D}_{i-1,j} +
\frac{36}{64} \abs{D}_{i,j} +
\frac{6}{64} \abs{D}_{i+1,j} +
\nonumber \\
& \quad + \frac{1}{64} \abs{D}_{i-1,j+1} +
\frac{6}{64} \abs{D}_{i,j+1} +
\frac{1}{64} \abs{D}_{i+1,j+1}
\end{align}

Now system \autoref{eq:error_regularization} can be solved and $\bPsi$ is set to:
\begin{align}
    \Psi = c_{\Psi}(\Dx^2 + \Dy^2) E
\end{align}