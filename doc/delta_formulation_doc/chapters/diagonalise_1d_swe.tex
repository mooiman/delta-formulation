%--------------------------------------------------------------------------------
\chapter{Diagonalise 1-D wave equation with convection}\label{sec:diagonalise_conservative_wave_with_convection}

The one dimensional shallow water equations  with convection
read
%
\begin{align}
    \pdiff{h}{t}  + \pdiff{q}{x} & = 0 \label{eq:continuity_for_diagonalisation}\\
    \pdiff{q}{t}  + \pdiff{}{x} \left( \frac{q^2}{h} \right) + g h \pdiff{\zeta}{x} & = 0 \label{eq:momentum_for_diagonalisation}
\end{align}
Using $\zeta = h +z_b$ in the momentum equation the system of equations is written as:
\begin{align}
    \pdiff{h}{t}  + \pdiff{q}{x} & = 0 \
    \pdiff{q}{t}  + \pdiff{}{x} \left( \frac{q^2}{h} \right) + g h \pdiff{h}{x} & = -gh\pdiff{z_b}{x}
\end{align}
The convection term can rewritten in the linear form for the derivatives as:
\begin{align}
    \pdiff{}{x} \left( \frac{q^2}{h} \right) = \frac{2 q}{h} \pdiff{q}{x} - \frac{q^2}{h^2}\pdiff{h}{x}
    \label{eq:convection_linear_form}
\end{align}
In matrix notation it reads ($\zeta = h +z_b$ is used):
\begin{align}
    \begin{pmatrix} h \\ q \end{pmatrix}_t +
    \begin{pmatrix} 0 & 1  \\ gh-\frac{q^2}{h^2} & \frac{2q}{h} \end{pmatrix}
    \begin{pmatrix} h \\ q \end{pmatrix}_x =
    -gh \begin{pmatrix} 0 \\ z_b \end{pmatrix}_x
    0
\end{align}
To make the system of equations diagonal, we have to find the eigen values and the eigen vectors.
The eigenvalues of the matrix are (using \maplesoft)
\begin{align}
    \lambda_1 = \frac{q}{h} + \sqrt{gh} \qquad \mbox{ and } \qquad
    \lambda_2 = \frac{q}{h} - \sqrt{gh}
\end{align}
The eigenvectors are
\begin{align}
    \begin{pmatrix} 1 \\ \frac{q}{h} + \sqrt{gh} \end{pmatrix}  \qquad \mbox{ and } \qquad
    \begin{pmatrix} 1 \\ \frac{q}{h} - \sqrt{gh} \end{pmatrix}
\end{align}
The diagonalised  SWE with convection read (after multiplying by $2\sqrt{gh}$, using  \maplesoft)
\begin{align}
    &\begin{pmatrix} \sqrt{gh} - \frac{q}{h} & 1  \\ \sqrt{gh} + \frac{q}{h} & -1 \end{pmatrix}
    \begin{pmatrix} h \\ q \end{pmatrix}_t +
    \begin{pmatrix} \frac{q}{h} + \sqrt{gh}   & 0  \\
        0 & \frac{q}{h} - \sqrt{gh}  \end{pmatrix}
    \begin{pmatrix} \sqrt{gh} - \frac{q}{h} & 1  \\ \sqrt{gh} + \frac{q}{h} & -1 \end{pmatrix}
    \begin{pmatrix} h \\ q \end{pmatrix}_x =
    \nonumber \\*
&= -gh \begin{pmatrix} \sqrt{gh} - \frac{q}{h} & 1  \\ \sqrt{gh} + \frac{q}{h} & -1 \end{pmatrix}
  \begin{pmatrix} 0  \\ z_b \end{pmatrix}_x
\end{align}
%
Written in two separate equations
\begin{align}
    &\left(\sqrt{gh} - \frac{q}{h}\right)\pdiff{h}{t} + \pdiff{q}{t} +  \left( \frac{q}{h} + \sqrt{gh}\right)\left( \left(\sqrt{gh} - \frac{q}{h} \right)\pdiff{h}{x} + \pdiff{q}{x} \right) =
    \nonumber \\*
    &\qquad = -gh\pdiff{z_b}{x} \quad \text{right going}
    \\
    &\left(\sqrt{gh} + \frac{q}{h}\right)\pdiff{h}{t} - \pdiff{q}{t} +  \left( \frac{q}{h} - \sqrt{gh}\right)\left( \left(\sqrt{gh} + \frac{q}{h} \right)\pdiff{h}{x} - \pdiff{q}{x} \right) =
    \nonumber \\*
    & \qquad = gh \pdiff{z_b}{x} \quad \text{left going}
\end{align}
First rearrange the equation for the right going wave (keep in mind that also \autoref{eq:convection_linear_form} is used):
\begin{align}
    &\left(\sqrt{gh} - \frac{q}{h}\right)\pdiff{h}{t} + \pdiff{q}{t} +
     \left( gh - \frac{q^2}{h^2} \right)\pdiff{h}{x} + \left( \frac{q}{h} + \sqrt{gh}\right) \pdiff{q}{x} = -gh\pdiff{z_b}{x}
     \\*
%    &\left(\sqrt{gh} - \frac{q}{h}\right)\pdiff{h}{t} + \pdiff{q}{t} +
% gh\pdiff{\zeta}{x} - \frac{q^2}{h^2}\pdiff{h}{x} + \left( \frac{q}{h} + \sqrt{gh}\right) \pdiff{q}{x} = 0
%     \\*
%&\left(\sqrt{gh} - \frac{q}{h}\right)\pdiff{h}{t} + \pdiff{q}{t} +
%gh\pdiff{\zeta}{x} - \frac{2 q}{h} \pdiff{q}{x} + \pdiff{}{x} \left( \frac{q^2}{h} \right) + \left( \frac{q}{h} + \sqrt{gh}\right) \pdiff{q}{x} = 0
%      \\*
 &\left(\sqrt{gh} - \frac{q}{h}\right)\left(\pdiff{h}{t} + \pdiff{q}{x}\right) + \left( \pdiff{q}{t} + \pdiff{}{x} \left( \frac{q^2}{h} \right) +
 gh\pdiff{\zeta}{x}\right)   = 0
\end{align}
and second, rearrange the equation for the left going wave:
\begin{align}
    &\left(\sqrt{gh} + \frac{q}{h}\right)\pdiff{h}{t} - \pdiff{q}{t} +  \left( \frac{q^2}{h^2} - gh\right)\pdiff{h}{x} - \left( \frac{q}{h} - \sqrt{gh}\right)\pdiff{q}{x} = gh \pdiff{z_b}{x}
     \\*
%    &\left(\sqrt{gh} + \frac{q}{h}\right)\pdiff{h}{t} - \pdiff{q}{t} +  \frac{q^2}{h^2}\pdiff{h}{x} - gh \pdiff{\zeta}{x} - \left( \frac{q}{h} - \sqrt{gh}\right)\pdiff{q}{x} = 0
%     \\*
%    &\left(\sqrt{gh} + \frac{q}{h}\right)\pdiff{h}{t} - \pdiff{q}{t} +  \frac{2 q}{h} \pdiff{q}{x} - \pdiff{}{x} \left( \frac{q^2}{h} \right) - gh \pdiff{\zeta}{x} - \left( \frac{q}{h} - \sqrt{gh}\right)\pdiff{q}{x} = 0
%     \\*
    &\left(\sqrt{gh} + \frac{q}{h}\right)\left(\pdiff{h}{t} + \pdiff{q}{x}\right)  - \left( \pdiff{q}{t} + \pdiff{}{x} \left( \frac{q^2}{h} \right) + gh \pdiff{\zeta}{x} \right) = 0
\end{align}
Which can be written as a combination of the continuity \eqref{eq:continuity_for_diagonalisation} and momentum equation \eqref{eq:momentum_for_diagonalisation} (see also \citet[eq.\ 4]{Borsboom2001}),
\begin{align}
    \begin{matrix}
    \quad \text{right going} \\
    \quad \text{left going}
\end{matrix}
\qquad
    \begin{pmatrix}
        \sqrt{gh} - \frac{q}{h}  &  1 \\
        \sqrt{gh} + \frac{q}{h}  &  -1
    \end{pmatrix}
    \begin{pmatrix} \textit{continuity eq.} \\ \textit{momentum eq.} \end{pmatrix}    = 0
\end{align}
or, after multiplying with $h$
\begin{align}
    \begin{matrix}
    \quad \text{right going} \\
    \quad \text{left going}
\end{matrix}
\qquad
    \begin{pmatrix}
        h\sqrt{gh} - q  &  h \\
        h\sqrt{gh} + q  &  -h
    \end{pmatrix}
    \begin{pmatrix} \textit{continuity eq.} \\ \textit{momentum eq.} \end{pmatrix}    = 0
\end{align}
