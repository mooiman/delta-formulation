%--------------------------------------------------------------------------------
\chapter{Diagonalise 1D wave equation with convection}\label{sec:diagonalise_conservative_wave_with_convection}

The one dimensional shallow water equations  with convection
for flat bottom ($\half g\, \lpdiff{h^2}{x} = gh\, \lpdiff{h}{x}$ and $\lpdiff{z_b}{x} = 0$), reads
%
\begin{align}
    \pdiff{h}{t}  + \pdiff{q}{x} & = 0 \\
    \pdiff{q}{t}  + \pdiff{}{x} \left( \frac{q^2}{h} \right) + g h \pdiff{h}{x} & = 0
\end{align}
The convection term can rewritten in the linear form for the derivatives as:
\begin{align}
    \pdiff{}{x} \left( \frac{q^2}{h} \right) = \frac{2 q}{h} \pdiff{q}{x} - \frac{q^2}{h^2}\pdiff{h}{x}
    \label{eq:convection_linear_form}
\end{align}
In matrix notation it reads:
\begin{align}
    \left( \begin{matrix} h \\ q \end{matrix}  \right)_t +
    \left( \begin{matrix} 0 & 1  \\ gh-\frac{q^2}{h^2} & \frac{2q}{h} \end{matrix}  \right)
    \left( \begin{matrix} h \\ q \end{matrix} \right)_x = 0
\end{align}
To make the system of equations diagonal, we have to find the eigen values and the eigen vectors.
The eigenvalues of the matrix are (using \maplesoft)
\begin{align}
    \lambda_1 = \frac{q}{h} + \sqrt{gh} \qquad \mbox{ and } \qquad
    \lambda_2 = \frac{q}{h} - \sqrt{gh}
\end{align}
The eigenvectors are
\begin{align}
    \begin{pmatrix} 1 \\ \frac{q}{h} + \sqrt{gh} \end{pmatrix}  \qquad \mbox{ and } \qquad
    \begin{pmatrix} 1 \\ \frac{q}{h} - \sqrt{gh} \end{pmatrix}
\end{align}
The diagonalised  SWE with convection read (after multiplying by $2\sqrt{gh}$, using  \maplesoft)
\begin{align}
    \begin{pmatrix} \sqrt{gh} - \frac{q}{h} & 1  \\ \sqrt{gh} + \frac{q}{h} & -1 \end{pmatrix}
    \begin{pmatrix} h \\ q \end{pmatrix}_t +
    \begin{pmatrix} \frac{q}{h} + \sqrt{gh}   & 0  \\
        0 & \frac{q}{h} - \sqrt{gh}  \end{pmatrix}
    \begin{pmatrix} \sqrt{gh} - \frac{q}{h} & 1  \\ \sqrt{gh} + \frac{q}{h} & -1 \end{pmatrix}
    \begin{pmatrix} h \\ q \end{pmatrix}_x = 0
\end{align}
%
Written in two separate equations
\begin{align}
    &\left(\sqrt{gh} - \frac{q}{h}\right)\pdiff{h}{t} + \pdiff{q}{t} +  \left( \frac{q}{h} + \sqrt{gh}\right)\left( \left(\sqrt{gh} - \frac{q}{h} \right)\pdiff{h}{x} + \pdiff{q}{x} \right) = 0 \quad \text{right going}
    \\
    &\left(\sqrt{gh} + \frac{q}{h}\right)\pdiff{h}{t} - \pdiff{q}{t} +  \left( \frac{q}{h} - \sqrt{gh}\right)\left( \left(\sqrt{gh} + \frac{q}{h} \right)\pdiff{h}{x} - \pdiff{q}{x} \right) = 0 \quad \text{left going}
\end{align}
Which can be written as a combination of the continuity and momentum equation \citep[eq.\ 4]{Borsboom2001}, (keep in mind that also \autoref{eq:convection_linear_form} is used)
\begin{align}
    \begin{matrix}
    \quad \text{right going} \\
    \quad \text{left going}
\end{matrix}
\qquad
    \begin{pmatrix}
        \sqrt{gh} - \frac{q}{h}  &  1 \\
        \sqrt{gh} + \frac{q}{h}  &  -1
    \end{pmatrix}
    \begin{pmatrix} \textit{continuity eq.} \\ \textit{momentum eq.} \end{pmatrix}    = 0
    \label{eq:left_right_going_equations}
\end{align}
or, after multiplying with $h$
\begin{align}
    \begin{matrix}
    \quad \text{right going} \\
    \quad \text{left going}
\end{matrix}
\qquad
    \begin{pmatrix}
        h\sqrt{gh} - q  &  h \\
        h\sqrt{gh} + q  &  -h
    \end{pmatrix}
    \begin{pmatrix} \textit{continuity eq.} \\ \textit{momentum eq.} \end{pmatrix}    = 0
\end{align}
