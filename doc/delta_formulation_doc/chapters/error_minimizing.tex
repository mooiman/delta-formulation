%------------------------------------------------------------------------------
\chapter{Two-step numerical modeling, error minimizing}\label{sec:error_minimizing}
%{\color{gray}
%The error-minimizing integration method is based on the assumption that a function can be made smooth (regularized) so that the numerical discretization of the regularized function and the regularized function itself are so close that the numerical error is negligible for that function.
%Meaning that the numerical solution is close enough to the regularized function and that the error in the regularization step is larger then the numerical error in the discretization step.
%}

For the realization of our objective, an error analysis is required to gain insight in the relative importance of discretization errors.
This has to be in the form of power series expansions to be genuinely generally applicable.
Smoothness is required to ensure fast converging series and dominant lowest-order terms that can be used as a basis for reliable local error approximations.
Artificial smoothing is added to satisfy this requirement, if necessary.
To enable the physical interpretation of numerical errors afterwards, smoothing can only involve the artificial enhancement of physical dissipation.
Taylor-series expansions can be used to determine the leading terms of the residual.
The residual, however, is not a suitable error measure since it indi-
cates the local discretization error in the equations, not in the solution. In order
to be useful, the residual needs to be reformulated in terms of local solution
errors.
We did not find any existing scheme that allows for such a transformation, and so we developed a discretization method that does.
The result turns out to be a method of finite volume type.
The discretization consists of integrating the model equations over control volumes, using uniquely defined discrete approximations of all variables.
The proposed numerical modeling technique solves the conceptual model problem in two steps (\autoref{fig:two_step_method}: in the first step the difficult problem to be solved is changed into an easy problem by adding artificial smoothing; in the second step the easy problem is discretized.

Showing the two-step method in general
%
\begin{figure}[H]
    \begin{center}
        \def\svgwidth{1.0\textwidth} % scaling text
        \resizebox{0.9\textwidth}{!}{
            %LaTeX with PSTricks extensions
%%Creator: Inkscape 1.4 (86a8ad7, 2024-10-11)
%%Please note this file requires PSTricks extensions
\psset{xunit=.5pt,yunit=.5pt,runit=.5pt}
\begin{pspicture}(911.04030819,411.58216582)
{
\newrgbcolor{curcolor}{0 0 0}
\pscustom[linewidth=1.51181105,linecolor=curcolor]
{
\newpath
\moveto(666.03715284,270.82645619)
\lineto(884.53159576,270.82645619)
\curveto(898.79865104,270.82645619)(910.28440313,259.3407041)(910.28440313,245.07364881)
\lineto(910.28440313,196.57925395)
\curveto(910.28440313,182.31219867)(898.79865104,170.82644658)(884.53159576,170.82644658)
\lineto(666.03715284,170.82644658)
\curveto(651.77009755,170.82644658)(640.28434546,182.31219867)(640.28434546,196.57925395)
\lineto(640.28434546,245.07364881)
\curveto(640.28434546,259.3407041)(651.77009755,270.82645619)(666.03715284,270.82645619)
\closepath
}
}
{
\newrgbcolor{curcolor}{0 0 0}
\pscustom[linestyle=none,fillstyle=solid,fillcolor=curcolor]
{
\newpath
\moveto(57.63600714,208.24369652)
\lineto(57.65868431,208.33440519)
\lineto(57.68136147,208.42511385)
\lineto(57.70403864,208.51015322)
\lineto(57.7267158,208.59519259)
\lineto(57.74939297,208.66889337)
\lineto(57.77773943,208.74259416)
\lineto(57.80608588,208.81629495)
\lineto(57.83443234,208.87865715)
\lineto(57.86844809,208.94668865)
\lineto(57.90246384,209.00338156)
\lineto(57.94781817,209.06007448)
\lineto(57.9931725,209.1110981)
\lineto(58.04419612,209.16212172)
\lineto(58.10088903,209.20747605)
\lineto(58.16325124,209.25283038)
\lineto(58.23128273,209.29251542)
\lineto(58.30498352,209.32653117)
\lineto(58.39002289,209.36054692)
\lineto(58.48073155,209.39456267)
\lineto(58.5827788,209.42290912)
\lineto(58.63380242,209.43424771)
\lineto(58.69049533,209.44558629)
\lineto(58.74718825,209.45692487)
\lineto(58.80955045,209.46826345)
\lineto(58.87191266,209.47960204)
\lineto(58.93427486,209.49094062)
\lineto(59.00230636,209.5022792)
\lineto(59.07600714,209.50794849)
\lineto(59.14970793,209.51361778)
\lineto(59.22907801,209.52495637)
\lineto(59.30844809,209.53062566)
\lineto(59.38781817,209.53629495)
\lineto(59.47285754,209.54196424)
\lineto(59.5635662,209.54763353)
\lineto(59.65427486,209.55330282)
\lineto(59.75065281,209.55897211)
\lineto(59.84703077,209.55897211)
\lineto(59.94907801,209.56464141)
\lineto(60.05112525,209.56464141)
\lineto(60.15884179,209.5703107)
\lineto(60.27222762,209.5703107)
\lineto(60.38561344,209.5703107)
\lineto(60.50466856,209.5703107)
\lineto(60.62939297,209.5703107)
\curveto(61.47411738,209.5703107)(61.70088903,209.5703107)(61.70088903,210.11456267)
\curveto(61.70088903,210.44905086)(61.3947473,210.44905086)(61.25301502,210.44905086)
\curveto(60.31191266,210.44905086)(57.99884179,210.35834219)(57.06907801,210.35834219)
\curveto(56.21868431,210.35834219)(54.15506226,210.44905086)(53.31600714,210.44905086)
\curveto(53.11191266,210.44905086)(52.77175517,210.44905086)(52.77175517,209.88212172)
\curveto(52.77175517,209.5703107)(53.03254258,209.5703107)(53.55978667,209.5703107)
\curveto(53.62214888,209.5703107)(54.15506226,209.5703107)(54.64262132,209.51928708)
\curveto(55.14718825,209.45692487)(55.40230636,209.42857841)(55.40230636,209.06574377)
\curveto(55.40230636,208.95235794)(55.36829061,208.86164928)(55.28892053,208.52716109)
\lineto(51.50750321,193.36180676)
\curveto(51.22403864,192.25629495)(51.16167643,192.03519259)(48.92797565,192.03519259)
\curveto(48.45175517,192.03519259)(48.16829061,192.03519259)(48.16829061,191.46826345)
\curveto(48.16829061,191.15645243)(48.42340872,191.15645243)(48.92797565,191.15645243)
\lineto(62.01270006,191.15645243)
\curveto(62.68734573,191.15645243)(62.71002289,191.15645243)(62.89144021,191.6326729)
\lineto(65.11380242,197.73849967)
\curveto(65.22718825,198.0503107)(65.22718825,198.10133432)(65.22718825,198.13535007)
\curveto(65.22718825,198.2430666)(65.14781817,198.44716109)(64.89270006,198.44716109)
\curveto(64.63191266,198.44716109)(64.60923549,198.30542881)(64.41081029,197.8518855)
\curveto(63.44703077,195.24968078)(62.20545596,192.03519259)(57.31852683,192.03519259)
\lineto(54.67096777,192.03519259)
\curveto(54.26844809,192.03519259)(54.21742447,192.03519259)(54.04734573,192.05786975)
\curveto(53.76388116,192.08621621)(53.67884179,192.11456267)(53.67884179,192.34133432)
\curveto(53.67884179,192.4207044)(53.67884179,192.4830666)(53.82057407,192.98763353)
\closepath
}
}
{
\newrgbcolor{curcolor}{0 0 0}
\pscustom[linewidth=0,linecolor=curcolor]
{
\newpath
\moveto(57.63600714,208.24369652)
\lineto(57.65868431,208.33440519)
\lineto(57.68136147,208.42511385)
\lineto(57.70403864,208.51015322)
\lineto(57.7267158,208.59519259)
\lineto(57.74939297,208.66889337)
\lineto(57.77773943,208.74259416)
\lineto(57.80608588,208.81629495)
\lineto(57.83443234,208.87865715)
\lineto(57.86844809,208.94668865)
\lineto(57.90246384,209.00338156)
\lineto(57.94781817,209.06007448)
\lineto(57.9931725,209.1110981)
\lineto(58.04419612,209.16212172)
\lineto(58.10088903,209.20747605)
\lineto(58.16325124,209.25283038)
\lineto(58.23128273,209.29251542)
\lineto(58.30498352,209.32653117)
\lineto(58.39002289,209.36054692)
\lineto(58.48073155,209.39456267)
\lineto(58.5827788,209.42290912)
\lineto(58.63380242,209.43424771)
\lineto(58.69049533,209.44558629)
\lineto(58.74718825,209.45692487)
\lineto(58.80955045,209.46826345)
\lineto(58.87191266,209.47960204)
\lineto(58.93427486,209.49094062)
\lineto(59.00230636,209.5022792)
\lineto(59.07600714,209.50794849)
\lineto(59.14970793,209.51361778)
\lineto(59.22907801,209.52495637)
\lineto(59.30844809,209.53062566)
\lineto(59.38781817,209.53629495)
\lineto(59.47285754,209.54196424)
\lineto(59.5635662,209.54763353)
\lineto(59.65427486,209.55330282)
\lineto(59.75065281,209.55897211)
\lineto(59.84703077,209.55897211)
\lineto(59.94907801,209.56464141)
\lineto(60.05112525,209.56464141)
\lineto(60.15884179,209.5703107)
\lineto(60.27222762,209.5703107)
\lineto(60.38561344,209.5703107)
\lineto(60.50466856,209.5703107)
\lineto(60.62939297,209.5703107)
\curveto(61.47411738,209.5703107)(61.70088903,209.5703107)(61.70088903,210.11456267)
\curveto(61.70088903,210.44905086)(61.3947473,210.44905086)(61.25301502,210.44905086)
\curveto(60.31191266,210.44905086)(57.99884179,210.35834219)(57.06907801,210.35834219)
\curveto(56.21868431,210.35834219)(54.15506226,210.44905086)(53.31600714,210.44905086)
\curveto(53.11191266,210.44905086)(52.77175517,210.44905086)(52.77175517,209.88212172)
\curveto(52.77175517,209.5703107)(53.03254258,209.5703107)(53.55978667,209.5703107)
\curveto(53.62214888,209.5703107)(54.15506226,209.5703107)(54.64262132,209.51928708)
\curveto(55.14718825,209.45692487)(55.40230636,209.42857841)(55.40230636,209.06574377)
\curveto(55.40230636,208.95235794)(55.36829061,208.86164928)(55.28892053,208.52716109)
\lineto(51.50750321,193.36180676)
\curveto(51.22403864,192.25629495)(51.16167643,192.03519259)(48.92797565,192.03519259)
\curveto(48.45175517,192.03519259)(48.16829061,192.03519259)(48.16829061,191.46826345)
\curveto(48.16829061,191.15645243)(48.42340872,191.15645243)(48.92797565,191.15645243)
\lineto(62.01270006,191.15645243)
\curveto(62.68734573,191.15645243)(62.71002289,191.15645243)(62.89144021,191.6326729)
\lineto(65.11380242,197.73849967)
\curveto(65.22718825,198.0503107)(65.22718825,198.10133432)(65.22718825,198.13535007)
\curveto(65.22718825,198.2430666)(65.14781817,198.44716109)(64.89270006,198.44716109)
\curveto(64.63191266,198.44716109)(64.60923549,198.30542881)(64.41081029,197.8518855)
\curveto(63.44703077,195.24968078)(62.20545596,192.03519259)(57.31852683,192.03519259)
\lineto(54.67096777,192.03519259)
\curveto(54.26844809,192.03519259)(54.21742447,192.03519259)(54.04734573,192.05786975)
\curveto(53.76388116,192.08621621)(53.67884179,192.11456267)(53.67884179,192.34133432)
\curveto(53.67884179,192.4207044)(53.67884179,192.4830666)(53.82057407,192.98763353)
\closepath
}
}
{
\newrgbcolor{curcolor}{0 0 0}
\pscustom[linestyle=none,fillstyle=solid,fillcolor=curcolor]
{
\newpath
\moveto(75.6643536,184.38164928)
\lineto(75.6643536,184.39298786)
\lineto(75.6643536,184.40999574)
\lineto(75.6643536,184.41566503)
\lineto(75.6643536,184.42700361)
\lineto(75.65868431,184.4326729)
\lineto(75.65868431,184.44401148)
\lineto(75.65301502,184.44968078)
\lineto(75.65301502,184.46101936)
\lineto(75.64734573,184.47235794)
\lineto(75.64167643,184.48369652)
\lineto(75.63033785,184.49503511)
\lineto(75.62466856,184.51204298)
\lineto(75.61332998,184.52338156)
\lineto(75.60766069,184.54038944)
\lineto(75.59065281,184.55739731)
\lineto(75.57931423,184.57440519)
\lineto(75.56230636,184.59141306)
\lineto(75.54529848,184.61409023)
\lineto(75.53962919,184.62542881)
\lineto(75.52829061,184.63676739)
\lineto(75.52262132,184.64810597)
\lineto(75.51128273,184.65944456)
\lineto(75.49994415,184.67078314)
\lineto(75.48860557,184.68779101)
\lineto(75.47726699,184.6991296)
\lineto(75.46025911,184.71046818)
\lineto(75.44892053,184.72747605)
\lineto(75.43758195,184.73881463)
\lineto(75.42057407,184.75582251)
\lineto(75.40923549,184.77283038)
\lineto(75.39222762,184.78983826)
\lineto(75.37521974,184.80684613)
\lineto(75.36388116,184.823854)
\lineto(75.34687329,184.84086188)
\lineto(75.32419612,184.85786975)
\lineto(75.30718825,184.87487763)
\lineto(75.29018037,184.89755479)
\lineto(75.2731725,184.91456267)
\lineto(75.25049533,184.93723983)
\lineto(75.22781817,184.95424771)
\lineto(75.21081029,184.97692487)
\lineto(75.18813313,184.99960204)
\curveto(71.66183391,188.559917)(70.7547473,193.89472015)(70.7547473,198.21472015)
\curveto(70.7547473,203.12999574)(71.83191266,208.05094062)(75.30151895,211.57157054)
\curveto(75.6643536,211.91739731)(75.6643536,211.97409023)(75.6643536,212.0591296)
\curveto(75.6643536,212.25755479)(75.55663706,212.33692487)(75.39222762,212.33692487)
\curveto(75.10876305,212.33692487)(72.56325124,210.4207044)(70.89647958,206.83771227)
\curveto(69.45647958,203.72527133)(69.1163221,200.59015322)(69.1163221,198.21472015)
\curveto(69.1163221,196.00936582)(69.42813313,192.60212172)(70.98151895,189.40464141)
\curveto(72.67096777,185.92936582)(75.10876305,184.09818471)(75.39222762,184.09818471)
\curveto(75.55663706,184.09818471)(75.6643536,184.17755479)(75.6643536,184.38164928)
\closepath
}
}
{
\newrgbcolor{curcolor}{0 0 0}
\pscustom[linewidth=0,linecolor=curcolor]
{
\newpath
\moveto(75.6643536,184.38164928)
\lineto(75.6643536,184.39298786)
\lineto(75.6643536,184.40999574)
\lineto(75.6643536,184.41566503)
\lineto(75.6643536,184.42700361)
\lineto(75.65868431,184.4326729)
\lineto(75.65868431,184.44401148)
\lineto(75.65301502,184.44968078)
\lineto(75.65301502,184.46101936)
\lineto(75.64734573,184.47235794)
\lineto(75.64167643,184.48369652)
\lineto(75.63033785,184.49503511)
\lineto(75.62466856,184.51204298)
\lineto(75.61332998,184.52338156)
\lineto(75.60766069,184.54038944)
\lineto(75.59065281,184.55739731)
\lineto(75.57931423,184.57440519)
\lineto(75.56230636,184.59141306)
\lineto(75.54529848,184.61409023)
\lineto(75.53962919,184.62542881)
\lineto(75.52829061,184.63676739)
\lineto(75.52262132,184.64810597)
\lineto(75.51128273,184.65944456)
\lineto(75.49994415,184.67078314)
\lineto(75.48860557,184.68779101)
\lineto(75.47726699,184.6991296)
\lineto(75.46025911,184.71046818)
\lineto(75.44892053,184.72747605)
\lineto(75.43758195,184.73881463)
\lineto(75.42057407,184.75582251)
\lineto(75.40923549,184.77283038)
\lineto(75.39222762,184.78983826)
\lineto(75.37521974,184.80684613)
\lineto(75.36388116,184.823854)
\lineto(75.34687329,184.84086188)
\lineto(75.32419612,184.85786975)
\lineto(75.30718825,184.87487763)
\lineto(75.29018037,184.89755479)
\lineto(75.2731725,184.91456267)
\lineto(75.25049533,184.93723983)
\lineto(75.22781817,184.95424771)
\lineto(75.21081029,184.97692487)
\lineto(75.18813313,184.99960204)
\curveto(71.66183391,188.559917)(70.7547473,193.89472015)(70.7547473,198.21472015)
\curveto(70.7547473,203.12999574)(71.83191266,208.05094062)(75.30151895,211.57157054)
\curveto(75.6643536,211.91739731)(75.6643536,211.97409023)(75.6643536,212.0591296)
\curveto(75.6643536,212.25755479)(75.55663706,212.33692487)(75.39222762,212.33692487)
\curveto(75.10876305,212.33692487)(72.56325124,210.4207044)(70.89647958,206.83771227)
\curveto(69.45647958,203.72527133)(69.1163221,200.59015322)(69.1163221,198.21472015)
\curveto(69.1163221,196.00936582)(69.42813313,192.60212172)(70.98151895,189.40464141)
\curveto(72.67096777,185.92936582)(75.10876305,184.09818471)(75.39222762,184.09818471)
\curveto(75.55663706,184.09818471)(75.6643536,184.17755479)(75.6643536,184.38164928)
\closepath
}
}
{
\newrgbcolor{curcolor}{0 0 0}
\pscustom[linestyle=none,fillstyle=solid,fillcolor=curcolor]
{
\newpath
\moveto(87.15600714,192.743854)
\lineto(87.1843536,192.63613747)
\lineto(87.22403864,192.52842093)
\lineto(87.25805439,192.42637369)
\lineto(87.30340872,192.32432645)
\lineto(87.34876305,192.22794849)
\lineto(87.39978667,192.13723983)
\lineto(87.45081029,192.04653117)
\lineto(87.50750321,191.95582251)
\lineto(87.56986541,191.87078314)
\lineto(87.63222762,191.78574377)
\lineto(87.70025911,191.70637369)
\lineto(87.76829061,191.6326729)
\lineto(87.8419914,191.55897211)
\lineto(87.91569218,191.49094062)
\lineto(87.99506226,191.42290912)
\lineto(88.07443234,191.36054692)
\lineto(88.15947171,191.29818471)
\lineto(88.24451108,191.2414918)
\lineto(88.33521974,191.19046818)
\lineto(88.4259284,191.13944456)
\lineto(88.51663706,191.09409023)
\lineto(88.61301502,191.05440519)
\lineto(88.70939297,191.01472015)
\lineto(88.81144021,190.9807044)
\lineto(88.91348746,190.94668865)
\lineto(89.0155347,190.92401148)
\lineto(89.12325124,190.89566503)
\lineto(89.23096777,190.87865715)
\lineto(89.33868431,190.86731857)
\lineto(89.44640084,190.85597999)
\lineto(89.55978667,190.8503107)
\lineto(89.6731725,190.84464141)
\curveto(90.6539599,190.84464141)(91.31159769,191.49094062)(91.75947171,192.39802723)
\curveto(92.24136147,193.41283038)(92.60419612,195.14196424)(92.60419612,195.19298786)
\curveto(92.60419612,195.47645243)(92.3547473,195.47645243)(92.26970793,195.47645243)
\curveto(91.98624336,195.47645243)(91.95789691,195.3630666)(91.86718825,194.96621621)
\curveto(91.47600714,193.39015322)(90.93742447,191.46826345)(89.75254258,191.46826345)
\curveto(89.15726699,191.46826345)(88.885141,191.83676739)(88.885141,192.76653117)
\curveto(88.885141,193.39015322)(89.21962919,194.71676739)(89.44073155,195.69755479)
\lineto(90.24010163,198.75897211)
\curveto(90.31947171,199.18416897)(90.60293628,200.25566503)(90.7163221,200.68086188)
\curveto(90.85805439,201.32716109)(91.14151895,202.39865715)(91.14151895,202.5630666)
\curveto(91.14151895,203.07897211)(90.74466856,203.32842093)(90.31947171,203.32842093)
\curveto(90.17773943,203.32842093)(89.44073155,203.30007448)(89.21962919,202.34196424)
\curveto(88.68104651,200.27834219)(87.43947171,195.33472015)(87.10498352,193.83802723)
\curveto(87.06529848,193.72464141)(85.9427788,191.46826345)(83.87915675,191.46826345)
\curveto(82.405141,191.46826345)(82.12167643,192.743854)(82.12167643,193.78700361)
\curveto(82.12167643,195.3630666)(82.92104651,197.59676739)(83.65805439,199.54700361)
\curveto(83.99254258,200.39739731)(84.13427486,200.78290912)(84.13427486,201.32716109)
\curveto(84.13427486,202.59141306)(83.23285754,203.63456267)(81.8155347,203.63456267)
\curveto(79.12829061,203.63456267)(78.085141,199.54700361)(78.085141,199.28621621)
\curveto(78.085141,199.00275164)(78.36860557,199.00275164)(78.43096777,199.00275164)
\curveto(78.71443234,199.00275164)(78.7427788,199.06511385)(78.88451108,199.51865715)
\curveto(79.58183391,201.9734603)(80.66466856,203.01660991)(81.73616462,203.01660991)
\curveto(81.97994415,203.01660991)(82.44482604,202.98826345)(82.44482604,202.08684613)
\curveto(82.44482604,201.40653117)(82.12167643,200.59015322)(81.95726699,200.16495637)
\curveto(80.91411738,197.36432645)(80.31884179,195.61818471)(80.31884179,194.24054692)
\curveto(80.31884179,191.55330282)(82.26340872,190.84464141)(83.79978667,190.84464141)
\curveto(85.65931423,190.84464141)(86.67978667,192.11456267)(87.15600714,192.743854)
\closepath
}
}
{
\newrgbcolor{curcolor}{0 0 0}
\pscustom[linewidth=0,linecolor=curcolor]
{
\newpath
\moveto(87.15600714,192.743854)
\lineto(87.1843536,192.63613747)
\lineto(87.22403864,192.52842093)
\lineto(87.25805439,192.42637369)
\lineto(87.30340872,192.32432645)
\lineto(87.34876305,192.22794849)
\lineto(87.39978667,192.13723983)
\lineto(87.45081029,192.04653117)
\lineto(87.50750321,191.95582251)
\lineto(87.56986541,191.87078314)
\lineto(87.63222762,191.78574377)
\lineto(87.70025911,191.70637369)
\lineto(87.76829061,191.6326729)
\lineto(87.8419914,191.55897211)
\lineto(87.91569218,191.49094062)
\lineto(87.99506226,191.42290912)
\lineto(88.07443234,191.36054692)
\lineto(88.15947171,191.29818471)
\lineto(88.24451108,191.2414918)
\lineto(88.33521974,191.19046818)
\lineto(88.4259284,191.13944456)
\lineto(88.51663706,191.09409023)
\lineto(88.61301502,191.05440519)
\lineto(88.70939297,191.01472015)
\lineto(88.81144021,190.9807044)
\lineto(88.91348746,190.94668865)
\lineto(89.0155347,190.92401148)
\lineto(89.12325124,190.89566503)
\lineto(89.23096777,190.87865715)
\lineto(89.33868431,190.86731857)
\lineto(89.44640084,190.85597999)
\lineto(89.55978667,190.8503107)
\lineto(89.6731725,190.84464141)
\curveto(90.6539599,190.84464141)(91.31159769,191.49094062)(91.75947171,192.39802723)
\curveto(92.24136147,193.41283038)(92.60419612,195.14196424)(92.60419612,195.19298786)
\curveto(92.60419612,195.47645243)(92.3547473,195.47645243)(92.26970793,195.47645243)
\curveto(91.98624336,195.47645243)(91.95789691,195.3630666)(91.86718825,194.96621621)
\curveto(91.47600714,193.39015322)(90.93742447,191.46826345)(89.75254258,191.46826345)
\curveto(89.15726699,191.46826345)(88.885141,191.83676739)(88.885141,192.76653117)
\curveto(88.885141,193.39015322)(89.21962919,194.71676739)(89.44073155,195.69755479)
\lineto(90.24010163,198.75897211)
\curveto(90.31947171,199.18416897)(90.60293628,200.25566503)(90.7163221,200.68086188)
\curveto(90.85805439,201.32716109)(91.14151895,202.39865715)(91.14151895,202.5630666)
\curveto(91.14151895,203.07897211)(90.74466856,203.32842093)(90.31947171,203.32842093)
\curveto(90.17773943,203.32842093)(89.44073155,203.30007448)(89.21962919,202.34196424)
\curveto(88.68104651,200.27834219)(87.43947171,195.33472015)(87.10498352,193.83802723)
\curveto(87.06529848,193.72464141)(85.9427788,191.46826345)(83.87915675,191.46826345)
\curveto(82.405141,191.46826345)(82.12167643,192.743854)(82.12167643,193.78700361)
\curveto(82.12167643,195.3630666)(82.92104651,197.59676739)(83.65805439,199.54700361)
\curveto(83.99254258,200.39739731)(84.13427486,200.78290912)(84.13427486,201.32716109)
\curveto(84.13427486,202.59141306)(83.23285754,203.63456267)(81.8155347,203.63456267)
\curveto(79.12829061,203.63456267)(78.085141,199.54700361)(78.085141,199.28621621)
\curveto(78.085141,199.00275164)(78.36860557,199.00275164)(78.43096777,199.00275164)
\curveto(78.71443234,199.00275164)(78.7427788,199.06511385)(78.88451108,199.51865715)
\curveto(79.58183391,201.9734603)(80.66466856,203.01660991)(81.73616462,203.01660991)
\curveto(81.97994415,203.01660991)(82.44482604,202.98826345)(82.44482604,202.08684613)
\curveto(82.44482604,201.40653117)(82.12167643,200.59015322)(81.95726699,200.16495637)
\curveto(80.91411738,197.36432645)(80.31884179,195.61818471)(80.31884179,194.24054692)
\curveto(80.31884179,191.55330282)(82.26340872,190.84464141)(83.79978667,190.84464141)
\curveto(85.65931423,190.84464141)(86.67978667,192.11456267)(87.15600714,192.743854)
\closepath
}
}
{
\newrgbcolor{curcolor}{0 0 0}
\pscustom[linestyle=none,fillstyle=solid,fillcolor=curcolor]
{
\newpath
\moveto(101.61836935,198.21472015)
\lineto(101.61836935,198.42448393)
\lineto(101.61836935,198.64558629)
\lineto(101.61270006,198.86668865)
\lineto(101.60136147,199.0991296)
\lineto(101.57868431,199.57535007)
\lineto(101.54466856,200.07424771)
\lineto(101.49931423,200.59582251)
\lineto(101.43695203,201.13440519)
\lineto(101.36325124,201.68432645)
\lineto(101.27254258,202.25692487)
\lineto(101.15915675,202.83519259)
\lineto(101.02876305,203.4191296)
\lineto(100.87569218,204.01440519)
\lineto(100.70561344,204.61535007)
\lineto(100.50718825,205.22196424)
\lineto(100.28608588,205.82857841)
\lineto(100.16136147,206.12905086)
\lineto(100.03663706,206.4295233)
\lineto(99.90057407,206.72999574)
\lineto(99.75884179,207.03046818)
\curveto(98.06939297,210.50007448)(95.64293628,212.33692487)(95.35947171,212.33692487)
\curveto(95.18939297,212.33692487)(95.07600714,212.22920834)(95.07600714,212.0591296)
\curveto(95.07600714,211.97409023)(95.07600714,211.91739731)(95.60325124,211.40716109)
\curveto(98.3755347,208.61786975)(99.99128273,204.12212172)(99.99128273,198.21472015)
\curveto(99.99128273,193.39015322)(98.94246384,188.41818471)(95.43884179,184.85786975)
\curveto(95.07600714,184.52338156)(95.07600714,184.46101936)(95.07600714,184.38164928)
\curveto(95.07600714,184.21157054)(95.18939297,184.09818471)(95.35947171,184.09818471)
\curveto(95.64293628,184.09818471)(98.17144021,186.02007448)(99.84955045,189.6030666)
\curveto(101.28388116,192.70416897)(101.61836935,195.83928708)(101.61836935,198.21472015)
\closepath
}
}
{
\newrgbcolor{curcolor}{0 0 0}
\pscustom[linewidth=0,linecolor=curcolor]
{
\newpath
\moveto(101.61836935,198.21472015)
\lineto(101.61836935,198.42448393)
\lineto(101.61836935,198.64558629)
\lineto(101.61270006,198.86668865)
\lineto(101.60136147,199.0991296)
\lineto(101.57868431,199.57535007)
\lineto(101.54466856,200.07424771)
\lineto(101.49931423,200.59582251)
\lineto(101.43695203,201.13440519)
\lineto(101.36325124,201.68432645)
\lineto(101.27254258,202.25692487)
\lineto(101.15915675,202.83519259)
\lineto(101.02876305,203.4191296)
\lineto(100.87569218,204.01440519)
\lineto(100.70561344,204.61535007)
\lineto(100.50718825,205.22196424)
\lineto(100.28608588,205.82857841)
\lineto(100.16136147,206.12905086)
\lineto(100.03663706,206.4295233)
\lineto(99.90057407,206.72999574)
\lineto(99.75884179,207.03046818)
\curveto(98.06939297,210.50007448)(95.64293628,212.33692487)(95.35947171,212.33692487)
\curveto(95.18939297,212.33692487)(95.07600714,212.22920834)(95.07600714,212.0591296)
\curveto(95.07600714,211.97409023)(95.07600714,211.91739731)(95.60325124,211.40716109)
\curveto(98.3755347,208.61786975)(99.99128273,204.12212172)(99.99128273,198.21472015)
\curveto(99.99128273,193.39015322)(98.94246384,188.41818471)(95.43884179,184.85786975)
\curveto(95.07600714,184.52338156)(95.07600714,184.46101936)(95.07600714,184.38164928)
\curveto(95.07600714,184.21157054)(95.18939297,184.09818471)(95.35947171,184.09818471)
\curveto(95.64293628,184.09818471)(98.17144021,186.02007448)(99.84955045,189.6030666)
\curveto(101.28388116,192.70416897)(101.61836935,195.83928708)(101.61836935,198.21472015)
\closepath
}
}
{
\newrgbcolor{curcolor}{0 0 0}
\pscustom[linestyle=none,fillstyle=solid,fillcolor=curcolor]
{
\newpath
\moveto(131.71096777,200.39739731)
\lineto(131.79033785,200.39739731)
\lineto(131.87537722,200.39739731)
\lineto(131.9547473,200.39739731)
\lineto(132.00010163,200.4030666)
\lineto(132.03978667,200.4030666)
\lineto(132.07947171,200.40873589)
\lineto(132.11915675,200.41440519)
\lineto(132.15884179,200.42007448)
\lineto(132.19852683,200.42574377)
\lineto(132.23821187,200.43141306)
\lineto(132.27789691,200.44275164)
\lineto(132.31191266,200.45409023)
\lineto(132.3459284,200.46542881)
\lineto(132.37994415,200.47676739)
\lineto(132.4139599,200.49377526)
\lineto(132.44797565,200.51078314)
\lineto(132.4763221,200.5334603)
\lineto(132.50466856,200.55046818)
\lineto(132.53301502,200.57881463)
\lineto(132.55569218,200.6014918)
\lineto(132.57836935,200.62983826)
\lineto(132.58970793,200.64684613)
\lineto(132.60104651,200.663854)
\lineto(132.6067158,200.67519259)
\lineto(132.61805439,200.69220046)
\lineto(132.62372368,200.70920834)
\lineto(132.63506226,200.7318855)
\lineto(132.64073155,200.74889337)
\lineto(132.64640084,200.77157054)
\lineto(132.65207014,200.78857841)
\lineto(132.65773943,200.81125558)
\lineto(132.66340872,200.83393274)
\lineto(132.66340872,200.85660991)
\lineto(132.66907801,200.87928708)
\lineto(132.66907801,200.90196424)
\lineto(132.66907801,200.92464141)
\lineto(132.66907801,200.95298786)
\curveto(132.66907801,201.519917)(132.13049533,201.519917)(131.73931423,201.519917)
\lineto(114.82214888,201.519917)
\curveto(114.41962919,201.519917)(113.8923851,201.519917)(113.8923851,200.95298786)
\curveto(113.8923851,200.39739731)(114.41962919,200.39739731)(114.84482604,200.39739731)
\closepath
}
}
{
\newrgbcolor{curcolor}{0 0 0}
\pscustom[linewidth=0,linecolor=curcolor]
{
\newpath
\moveto(131.71096777,200.39739731)
\lineto(131.79033785,200.39739731)
\lineto(131.87537722,200.39739731)
\lineto(131.9547473,200.39739731)
\lineto(132.00010163,200.4030666)
\lineto(132.03978667,200.4030666)
\lineto(132.07947171,200.40873589)
\lineto(132.11915675,200.41440519)
\lineto(132.15884179,200.42007448)
\lineto(132.19852683,200.42574377)
\lineto(132.23821187,200.43141306)
\lineto(132.27789691,200.44275164)
\lineto(132.31191266,200.45409023)
\lineto(132.3459284,200.46542881)
\lineto(132.37994415,200.47676739)
\lineto(132.4139599,200.49377526)
\lineto(132.44797565,200.51078314)
\lineto(132.4763221,200.5334603)
\lineto(132.50466856,200.55046818)
\lineto(132.53301502,200.57881463)
\lineto(132.55569218,200.6014918)
\lineto(132.57836935,200.62983826)
\lineto(132.58970793,200.64684613)
\lineto(132.60104651,200.663854)
\lineto(132.6067158,200.67519259)
\lineto(132.61805439,200.69220046)
\lineto(132.62372368,200.70920834)
\lineto(132.63506226,200.7318855)
\lineto(132.64073155,200.74889337)
\lineto(132.64640084,200.77157054)
\lineto(132.65207014,200.78857841)
\lineto(132.65773943,200.81125558)
\lineto(132.66340872,200.83393274)
\lineto(132.66340872,200.85660991)
\lineto(132.66907801,200.87928708)
\lineto(132.66907801,200.90196424)
\lineto(132.66907801,200.92464141)
\lineto(132.66907801,200.95298786)
\curveto(132.66907801,201.519917)(132.13049533,201.519917)(131.73931423,201.519917)
\lineto(114.82214888,201.519917)
\curveto(114.41962919,201.519917)(113.8923851,201.519917)(113.8923851,200.95298786)
\curveto(113.8923851,200.39739731)(114.41962919,200.39739731)(114.84482604,200.39739731)
\closepath
}
}
{
\newrgbcolor{curcolor}{0 0 0}
\pscustom[linestyle=none,fillstyle=solid,fillcolor=curcolor]
{
\newpath
\moveto(131.73931423,194.9095233)
\lineto(131.77899927,194.9095233)
\lineto(131.81301502,194.9095233)
\lineto(131.8923851,194.91519259)
\lineto(131.97175517,194.91519259)
\lineto(132.01144021,194.91519259)
\lineto(132.05112525,194.92086188)
\lineto(132.09081029,194.92653117)
\lineto(132.13049533,194.92653117)
\lineto(132.16451108,194.93220046)
\lineto(132.20419612,194.94353904)
\lineto(132.24388116,194.94920834)
\lineto(132.27789691,194.96054692)
\lineto(132.31758195,194.9718855)
\lineto(132.35159769,194.98322408)
\lineto(132.38561344,194.99456267)
\lineto(132.41962919,195.01157054)
\lineto(132.44797565,195.02857841)
\lineto(132.4763221,195.05125558)
\lineto(132.50466856,195.07393274)
\lineto(132.53301502,195.09660991)
\lineto(132.55569218,195.12495637)
\lineto(132.56703077,195.13629495)
\lineto(132.57836935,195.15330282)
\lineto(132.58970793,195.16464141)
\lineto(132.60104651,195.18164928)
\lineto(132.6067158,195.19865715)
\lineto(132.61805439,195.21566503)
\lineto(132.62372368,195.2326729)
\lineto(132.63506226,195.24968078)
\lineto(132.64073155,195.27235794)
\lineto(132.64640084,195.28936582)
\lineto(132.65207014,195.31204298)
\lineto(132.65773943,195.33472015)
\lineto(132.66340872,195.35739731)
\lineto(132.66340872,195.38007448)
\lineto(132.66907801,195.40275164)
\lineto(132.66907801,195.42542881)
\lineto(132.66907801,195.45377526)
\lineto(132.66907801,195.47645243)
\curveto(132.66907801,196.04338156)(132.13049533,196.04338156)(131.71096777,196.04338156)
\lineto(114.84482604,196.04338156)
\curveto(114.41962919,196.04338156)(113.8923851,196.04338156)(113.8923851,195.47645243)
\curveto(113.8923851,194.9095233)(114.41962919,194.9095233)(114.82214888,194.9095233)
\closepath
}
}
{
\newrgbcolor{curcolor}{0 0 0}
\pscustom[linewidth=0,linecolor=curcolor]
{
\newpath
\moveto(131.73931423,194.9095233)
\lineto(131.77899927,194.9095233)
\lineto(131.81301502,194.9095233)
\lineto(131.8923851,194.91519259)
\lineto(131.97175517,194.91519259)
\lineto(132.01144021,194.91519259)
\lineto(132.05112525,194.92086188)
\lineto(132.09081029,194.92653117)
\lineto(132.13049533,194.92653117)
\lineto(132.16451108,194.93220046)
\lineto(132.20419612,194.94353904)
\lineto(132.24388116,194.94920834)
\lineto(132.27789691,194.96054692)
\lineto(132.31758195,194.9718855)
\lineto(132.35159769,194.98322408)
\lineto(132.38561344,194.99456267)
\lineto(132.41962919,195.01157054)
\lineto(132.44797565,195.02857841)
\lineto(132.4763221,195.05125558)
\lineto(132.50466856,195.07393274)
\lineto(132.53301502,195.09660991)
\lineto(132.55569218,195.12495637)
\lineto(132.56703077,195.13629495)
\lineto(132.57836935,195.15330282)
\lineto(132.58970793,195.16464141)
\lineto(132.60104651,195.18164928)
\lineto(132.6067158,195.19865715)
\lineto(132.61805439,195.21566503)
\lineto(132.62372368,195.2326729)
\lineto(132.63506226,195.24968078)
\lineto(132.64073155,195.27235794)
\lineto(132.64640084,195.28936582)
\lineto(132.65207014,195.31204298)
\lineto(132.65773943,195.33472015)
\lineto(132.66340872,195.35739731)
\lineto(132.66340872,195.38007448)
\lineto(132.66907801,195.40275164)
\lineto(132.66907801,195.42542881)
\lineto(132.66907801,195.45377526)
\lineto(132.66907801,195.47645243)
\curveto(132.66907801,196.04338156)(132.13049533,196.04338156)(131.71096777,196.04338156)
\lineto(114.84482604,196.04338156)
\curveto(114.41962919,196.04338156)(113.8923851,196.04338156)(113.8923851,195.47645243)
\curveto(113.8923851,194.9095233)(114.41962919,194.9095233)(114.82214888,194.9095233)
\closepath
}
}
{
\newrgbcolor{curcolor}{0 0 0}
\pscustom[linestyle=none,fillstyle=solid,fillcolor=curcolor]
{
\newpath
\moveto(155.09679454,200.19330282)
\lineto(155.09112525,200.61283038)
\lineto(155.08545596,201.03802723)
\lineto(155.07411738,201.46322408)
\lineto(155.05710951,201.88275164)
\lineto(155.02876305,202.30794849)
\lineto(154.9947473,202.72747605)
\lineto(154.95506226,203.14700361)
\lineto(154.89836935,203.56086188)
\lineto(154.83600714,203.97472015)
\lineto(154.75663706,204.38857841)
\lineto(154.6659284,204.79676739)
\lineto(154.55821187,205.20495637)
\lineto(154.43348746,205.61314534)
\lineto(154.29175517,206.00999574)
\lineto(154.13868431,206.40684613)
\lineto(154.05364494,206.60527133)
\lineto(153.96293628,206.79802723)
\curveto(152.66466856,209.51928708)(150.3459284,209.97283038)(149.16104651,209.97283038)
\curveto(147.4659284,209.97283038)(145.40230636,209.23582251)(144.24577092,206.60527133)
\curveto(143.33868431,204.65503511)(143.19695203,202.44968078)(143.19695203,200.19330282)
\curveto(143.19695203,198.07298786)(143.31033785,195.53314534)(144.47254258,193.39015322)
\curveto(145.68577092,191.10542881)(147.74939297,190.53849967)(149.13270006,190.53849967)
\lineto(149.13270006,191.15645243)
\curveto(148.03285754,191.15645243)(146.36041659,191.86511385)(145.85584966,194.57503511)
\curveto(145.54403864,196.26448393)(145.54403864,198.86101936)(145.54403864,200.5391296)
\curveto(145.54403864,202.34196424)(145.54403864,204.2014918)(145.77647958,205.72653117)
\curveto(146.30939297,209.09409023)(148.42403864,209.34353904)(149.13270006,209.34353904)
\curveto(150.06246384,209.34353904)(151.93332998,208.83897211)(152.46624336,206.03834219)
\curveto(152.74970793,204.4622792)(152.74970793,202.30794849)(152.74970793,200.5391296)
\curveto(152.74970793,198.41314534)(152.74970793,196.49692487)(152.43789691,194.68842093)
\curveto(152.01270006,192.00684613)(150.39695203,191.15645243)(149.13270006,191.15645243)
\lineto(149.13270006,191.15645243)
\lineto(149.13270006,190.53849967)
\curveto(150.65773943,190.53849967)(152.80073155,191.12810597)(154.04797565,193.81535007)
\curveto(154.95506226,195.759917)(155.09679454,197.95960204)(155.09679454,200.19330282)
\closepath
}
}
{
\newrgbcolor{curcolor}{0 0 0}
\pscustom[linewidth=0,linecolor=curcolor]
{
\newpath
\moveto(155.09679454,200.19330282)
\lineto(155.09112525,200.61283038)
\lineto(155.08545596,201.03802723)
\lineto(155.07411738,201.46322408)
\lineto(155.05710951,201.88275164)
\lineto(155.02876305,202.30794849)
\lineto(154.9947473,202.72747605)
\lineto(154.95506226,203.14700361)
\lineto(154.89836935,203.56086188)
\lineto(154.83600714,203.97472015)
\lineto(154.75663706,204.38857841)
\lineto(154.6659284,204.79676739)
\lineto(154.55821187,205.20495637)
\lineto(154.43348746,205.61314534)
\lineto(154.29175517,206.00999574)
\lineto(154.13868431,206.40684613)
\lineto(154.05364494,206.60527133)
\lineto(153.96293628,206.79802723)
\curveto(152.66466856,209.51928708)(150.3459284,209.97283038)(149.16104651,209.97283038)
\curveto(147.4659284,209.97283038)(145.40230636,209.23582251)(144.24577092,206.60527133)
\curveto(143.33868431,204.65503511)(143.19695203,202.44968078)(143.19695203,200.19330282)
\curveto(143.19695203,198.07298786)(143.31033785,195.53314534)(144.47254258,193.39015322)
\curveto(145.68577092,191.10542881)(147.74939297,190.53849967)(149.13270006,190.53849967)
\lineto(149.13270006,191.15645243)
\curveto(148.03285754,191.15645243)(146.36041659,191.86511385)(145.85584966,194.57503511)
\curveto(145.54403864,196.26448393)(145.54403864,198.86101936)(145.54403864,200.5391296)
\curveto(145.54403864,202.34196424)(145.54403864,204.2014918)(145.77647958,205.72653117)
\curveto(146.30939297,209.09409023)(148.42403864,209.34353904)(149.13270006,209.34353904)
\curveto(150.06246384,209.34353904)(151.93332998,208.83897211)(152.46624336,206.03834219)
\curveto(152.74970793,204.4622792)(152.74970793,202.30794849)(152.74970793,200.5391296)
\curveto(152.74970793,198.41314534)(152.74970793,196.49692487)(152.43789691,194.68842093)
\curveto(152.01270006,192.00684613)(150.39695203,191.15645243)(149.13270006,191.15645243)
\lineto(149.13270006,191.15645243)
\lineto(149.13270006,190.53849967)
\curveto(150.65773943,190.53849967)(152.80073155,191.12810597)(154.04797565,193.81535007)
\curveto(154.95506226,195.759917)(155.09679454,197.95960204)(155.09679454,200.19330282)
\closepath
}
}
{
\newrgbcolor{curcolor}{0 0 0}
\pscustom[linestyle=none,fillstyle=solid,fillcolor=curcolor]
{
\newpath
\moveto(161.97931423,191.18479889)
\lineto(161.97931423,191.35487763)
\lineto(161.97364494,191.52495637)
\lineto(161.96230636,191.68936582)
\lineto(161.94529848,191.84810597)
\lineto(161.92829061,192.00117684)
\lineto(161.90561344,192.14857841)
\lineto(161.88293628,192.29597999)
\lineto(161.85458982,192.43204298)
\lineto(161.82057407,192.56810597)
\lineto(161.78655832,192.69849967)
\lineto(161.74120399,192.82322408)
\lineto(161.70151895,192.9422792)
\lineto(161.65049533,193.05566503)
\lineto(161.605141,193.16338156)
\lineto(161.54844809,193.2710981)
\lineto(161.49175517,193.36747605)
\lineto(161.42939297,193.45818471)
\lineto(161.36703077,193.54889337)
\lineto(161.29899927,193.62826345)
\lineto(161.23096777,193.70196424)
\lineto(161.15726699,193.77566503)
\lineto(161.0835662,193.83802723)
\lineto(161.00419612,193.89472015)
\lineto(160.92482604,193.95141306)
\lineto(160.83978667,193.99676739)
\lineto(160.74907801,194.03645243)
\lineto(160.65836935,194.07046818)
\lineto(160.56766069,194.09881463)
\lineto(160.47128273,194.1214918)
\lineto(160.37490478,194.13849967)
\lineto(160.27285754,194.14416897)
\lineto(160.17081029,194.14983826)
\curveto(159.24104651,194.14983826)(158.67411738,193.44117684)(158.67411738,192.65314534)
\curveto(158.67411738,191.8934603)(159.24104651,191.15645243)(160.17081029,191.15645243)
\curveto(160.51096777,191.15645243)(160.87947171,191.26983826)(161.16293628,191.51928708)
\curveto(161.24230636,191.58164928)(161.2819914,191.60999574)(161.30466856,191.60999574)
\curveto(161.33301502,191.60999574)(161.36136147,191.58164928)(161.36136147,191.18479889)
\curveto(161.36136147,189.09283038)(160.37490478,187.39771227)(159.43947171,186.46794849)
\curveto(159.12766069,186.16180676)(159.12766069,186.09944456)(159.12766069,186.02007448)
\curveto(159.12766069,185.81597999)(159.26939297,185.70826345)(159.41112525,185.70826345)
\curveto(159.72293628,185.70826345)(161.97931423,187.87960204)(161.97931423,191.18479889)
\closepath
}
}
{
\newrgbcolor{curcolor}{0 0 0}
\pscustom[linewidth=0,linecolor=curcolor]
{
\newpath
\moveto(161.97931423,191.18479889)
\lineto(161.97931423,191.35487763)
\lineto(161.97364494,191.52495637)
\lineto(161.96230636,191.68936582)
\lineto(161.94529848,191.84810597)
\lineto(161.92829061,192.00117684)
\lineto(161.90561344,192.14857841)
\lineto(161.88293628,192.29597999)
\lineto(161.85458982,192.43204298)
\lineto(161.82057407,192.56810597)
\lineto(161.78655832,192.69849967)
\lineto(161.74120399,192.82322408)
\lineto(161.70151895,192.9422792)
\lineto(161.65049533,193.05566503)
\lineto(161.605141,193.16338156)
\lineto(161.54844809,193.2710981)
\lineto(161.49175517,193.36747605)
\lineto(161.42939297,193.45818471)
\lineto(161.36703077,193.54889337)
\lineto(161.29899927,193.62826345)
\lineto(161.23096777,193.70196424)
\lineto(161.15726699,193.77566503)
\lineto(161.0835662,193.83802723)
\lineto(161.00419612,193.89472015)
\lineto(160.92482604,193.95141306)
\lineto(160.83978667,193.99676739)
\lineto(160.74907801,194.03645243)
\lineto(160.65836935,194.07046818)
\lineto(160.56766069,194.09881463)
\lineto(160.47128273,194.1214918)
\lineto(160.37490478,194.13849967)
\lineto(160.27285754,194.14416897)
\lineto(160.17081029,194.14983826)
\curveto(159.24104651,194.14983826)(158.67411738,193.44117684)(158.67411738,192.65314534)
\curveto(158.67411738,191.8934603)(159.24104651,191.15645243)(160.17081029,191.15645243)
\curveto(160.51096777,191.15645243)(160.87947171,191.26983826)(161.16293628,191.51928708)
\curveto(161.24230636,191.58164928)(161.2819914,191.60999574)(161.30466856,191.60999574)
\curveto(161.33301502,191.60999574)(161.36136147,191.58164928)(161.36136147,191.18479889)
\curveto(161.36136147,189.09283038)(160.37490478,187.39771227)(159.43947171,186.46794849)
\curveto(159.12766069,186.16180676)(159.12766069,186.09944456)(159.12766069,186.02007448)
\curveto(159.12766069,185.81597999)(159.26939297,185.70826345)(159.41112525,185.70826345)
\curveto(159.72293628,185.70826345)(161.97931423,187.87960204)(161.97931423,191.18479889)
\closepath
}
}
{
\newrgbcolor{curcolor}{0 0 0}
\pscustom[linestyle=none,fillstyle=solid,fillcolor=curcolor]
{
\newpath
\moveto(183.3355347,210.44905086)
\lineto(183.3355347,210.44905086)
\lineto(183.3355347,210.45472015)
\lineto(183.3355347,210.45472015)
\lineto(183.3355347,210.46038944)
\lineto(183.3355347,210.47172802)
\lineto(183.3355347,210.47739731)
\lineto(183.3355347,210.48873589)
\lineto(183.3355347,210.50007448)
\lineto(183.32986541,210.51141306)
\lineto(183.32986541,210.52275164)
\lineto(183.32419612,210.53409023)
\lineto(183.32419612,210.54542881)
\lineto(183.31852683,210.55676739)
\lineto(183.31285754,210.57377526)
\lineto(183.30718825,210.58511385)
\lineto(183.30151895,210.60212172)
\lineto(183.29018037,210.6134603)
\lineto(183.28451108,210.63046818)
\lineto(183.2731725,210.64180676)
\lineto(183.26183391,210.65314534)
\lineto(183.25049533,210.67015322)
\lineto(183.23915675,210.67582251)
\lineto(183.23348746,210.6814918)
\lineto(183.22781817,210.68716109)
\lineto(183.21647958,210.69283038)
\lineto(183.21081029,210.69849967)
\lineto(183.19947171,210.70416897)
\lineto(183.19380242,210.70983826)
\lineto(183.18246384,210.71550755)
\lineto(183.17679454,210.72117684)
\lineto(183.16545596,210.72684613)
\lineto(183.15411738,210.72684613)
\lineto(183.1427788,210.73251542)
\lineto(183.13144021,210.73818471)
\lineto(183.12010163,210.73818471)
\lineto(183.10876305,210.743854)
\lineto(183.09175517,210.7495233)
\lineto(183.08041659,210.7495233)
\lineto(183.06907801,210.7495233)
\lineto(183.05207014,210.75519259)
\lineto(183.03506226,210.75519259)
\lineto(183.02372368,210.75519259)
\lineto(183.0067158,210.76086188)
\lineto(182.98970793,210.76086188)
\lineto(182.97270006,210.76086188)
\curveto(182.54750321,210.76086188)(179.8659284,210.50007448)(179.38403864,210.44905086)
\curveto(179.15726699,210.4207044)(178.98718825,210.27897211)(178.98718825,209.91046818)
\curveto(178.98718825,209.5703107)(179.24230636,209.5703107)(179.66750321,209.5703107)
\curveto(181.02246384,209.5703107)(181.07915675,209.37755479)(181.07915675,209.09409023)
\lineto(180.99978667,208.52716109)
\lineto(179.29899927,201.83172802)
\curveto(178.79443234,202.87487763)(177.9667158,203.63456267)(176.70246384,203.63456267)
\lineto(176.73081029,203.01660991)
\curveto(178.5619914,203.01660991)(178.95884179,200.70353904)(178.95884179,200.5391296)
\curveto(178.95884179,200.35771227)(178.90781817,200.19330282)(178.87947171,200.05157054)
\lineto(177.46214888,194.5126729)
\curveto(177.32041659,194.00810597)(177.32041659,193.95708235)(176.89521974,193.4695233)
\curveto(175.65931423,191.91613747)(174.49710951,191.46826345)(173.70907801,191.46826345)
\curveto(172.29742447,191.46826345)(171.90057407,193.01597999)(171.90057407,194.1214918)
\curveto(171.90057407,195.53314534)(172.8019914,199.00275164)(173.4539599,200.30668865)
\curveto(174.32703077,201.9734603)(175.60262132,203.01660991)(176.73081029,203.01660991)
\lineto(176.70246384,203.63456267)
\curveto(173.39726699,203.63456267)(169.89931423,199.4903107)(169.89931423,195.3630666)
\curveto(169.89931423,192.70416897)(171.44703077,190.84464141)(173.6523851,190.84464141)
\curveto(174.21364494,190.84464141)(175.63096777,190.96369652)(177.32041659,192.96495637)
\curveto(177.55285754,191.77440519)(178.53364494,190.84464141)(179.88860557,190.84464141)
\curveto(180.88073155,190.84464141)(181.53270006,191.49094062)(181.98057407,192.39802723)
\curveto(182.46246384,193.41283038)(182.83096777,195.14196424)(182.83096777,195.19298786)
\curveto(182.83096777,195.47645243)(182.57584966,195.47645243)(182.49647958,195.47645243)
\curveto(182.21301502,195.47645243)(182.18466856,195.3630666)(182.09962919,194.96621621)
\curveto(181.61773943,193.12936582)(181.10750321,191.46826345)(179.95096777,191.46826345)
\curveto(179.19128273,191.46826345)(179.10057407,192.19960204)(179.10057407,192.76653117)
\curveto(179.10057407,193.44117684)(179.15726699,193.64527133)(179.27065281,194.1214918)
\closepath
}
}
{
\newrgbcolor{curcolor}{0 0 0}
\pscustom[linewidth=0,linecolor=curcolor]
{
\newpath
\moveto(183.3355347,210.44905086)
\lineto(183.3355347,210.44905086)
\lineto(183.3355347,210.45472015)
\lineto(183.3355347,210.45472015)
\lineto(183.3355347,210.46038944)
\lineto(183.3355347,210.47172802)
\lineto(183.3355347,210.47739731)
\lineto(183.3355347,210.48873589)
\lineto(183.3355347,210.50007448)
\lineto(183.32986541,210.51141306)
\lineto(183.32986541,210.52275164)
\lineto(183.32419612,210.53409023)
\lineto(183.32419612,210.54542881)
\lineto(183.31852683,210.55676739)
\lineto(183.31285754,210.57377526)
\lineto(183.30718825,210.58511385)
\lineto(183.30151895,210.60212172)
\lineto(183.29018037,210.6134603)
\lineto(183.28451108,210.63046818)
\lineto(183.2731725,210.64180676)
\lineto(183.26183391,210.65314534)
\lineto(183.25049533,210.67015322)
\lineto(183.23915675,210.67582251)
\lineto(183.23348746,210.6814918)
\lineto(183.22781817,210.68716109)
\lineto(183.21647958,210.69283038)
\lineto(183.21081029,210.69849967)
\lineto(183.19947171,210.70416897)
\lineto(183.19380242,210.70983826)
\lineto(183.18246384,210.71550755)
\lineto(183.17679454,210.72117684)
\lineto(183.16545596,210.72684613)
\lineto(183.15411738,210.72684613)
\lineto(183.1427788,210.73251542)
\lineto(183.13144021,210.73818471)
\lineto(183.12010163,210.73818471)
\lineto(183.10876305,210.743854)
\lineto(183.09175517,210.7495233)
\lineto(183.08041659,210.7495233)
\lineto(183.06907801,210.7495233)
\lineto(183.05207014,210.75519259)
\lineto(183.03506226,210.75519259)
\lineto(183.02372368,210.75519259)
\lineto(183.0067158,210.76086188)
\lineto(182.98970793,210.76086188)
\lineto(182.97270006,210.76086188)
\curveto(182.54750321,210.76086188)(179.8659284,210.50007448)(179.38403864,210.44905086)
\curveto(179.15726699,210.4207044)(178.98718825,210.27897211)(178.98718825,209.91046818)
\curveto(178.98718825,209.5703107)(179.24230636,209.5703107)(179.66750321,209.5703107)
\curveto(181.02246384,209.5703107)(181.07915675,209.37755479)(181.07915675,209.09409023)
\lineto(180.99978667,208.52716109)
\lineto(179.29899927,201.83172802)
\curveto(178.79443234,202.87487763)(177.9667158,203.63456267)(176.70246384,203.63456267)
\lineto(176.73081029,203.01660991)
\curveto(178.5619914,203.01660991)(178.95884179,200.70353904)(178.95884179,200.5391296)
\curveto(178.95884179,200.35771227)(178.90781817,200.19330282)(178.87947171,200.05157054)
\lineto(177.46214888,194.5126729)
\curveto(177.32041659,194.00810597)(177.32041659,193.95708235)(176.89521974,193.4695233)
\curveto(175.65931423,191.91613747)(174.49710951,191.46826345)(173.70907801,191.46826345)
\curveto(172.29742447,191.46826345)(171.90057407,193.01597999)(171.90057407,194.1214918)
\curveto(171.90057407,195.53314534)(172.8019914,199.00275164)(173.4539599,200.30668865)
\curveto(174.32703077,201.9734603)(175.60262132,203.01660991)(176.73081029,203.01660991)
\lineto(176.70246384,203.63456267)
\curveto(173.39726699,203.63456267)(169.89931423,199.4903107)(169.89931423,195.3630666)
\curveto(169.89931423,192.70416897)(171.44703077,190.84464141)(173.6523851,190.84464141)
\curveto(174.21364494,190.84464141)(175.63096777,190.96369652)(177.32041659,192.96495637)
\curveto(177.55285754,191.77440519)(178.53364494,190.84464141)(179.88860557,190.84464141)
\curveto(180.88073155,190.84464141)(181.53270006,191.49094062)(181.98057407,192.39802723)
\curveto(182.46246384,193.41283038)(182.83096777,195.14196424)(182.83096777,195.19298786)
\curveto(182.83096777,195.47645243)(182.57584966,195.47645243)(182.49647958,195.47645243)
\curveto(182.21301502,195.47645243)(182.18466856,195.3630666)(182.09962919,194.96621621)
\curveto(181.61773943,193.12936582)(181.10750321,191.46826345)(179.95096777,191.46826345)
\curveto(179.19128273,191.46826345)(179.10057407,192.19960204)(179.10057407,192.76653117)
\curveto(179.10057407,193.44117684)(179.15726699,193.64527133)(179.27065281,194.1214918)
\closepath
}
}
{
\newrgbcolor{curcolor}{0 0 0}
\pscustom[linestyle=none,fillstyle=solid,fillcolor=curcolor]
{
\newpath
\moveto(193.99380242,201.83172802)
\lineto(193.9427788,201.92810597)
\lineto(193.89175517,202.02448393)
\lineto(193.84073155,202.11519259)
\lineto(193.78970793,202.20590125)
\lineto(193.73301502,202.29660991)
\lineto(193.67065281,202.38731857)
\lineto(193.6139599,202.47235794)
\lineto(193.55159769,202.55172802)
\lineto(193.48923549,202.63676739)
\lineto(193.42120399,202.71046818)
\lineto(193.3531725,202.78983826)
\lineto(193.27947171,202.86353904)
\lineto(193.21144021,202.93157054)
\lineto(193.13773943,202.99960204)
\lineto(193.05836935,203.06196424)
\lineto(192.97899927,203.12432645)
\lineto(192.89962919,203.18668865)
\lineto(192.81458982,203.23771227)
\lineto(192.72955045,203.29440519)
\lineto(192.63884179,203.33975952)
\lineto(192.54813313,203.38511385)
\lineto(192.45742447,203.43046818)
\lineto(192.36104651,203.46448393)
\lineto(192.26466856,203.50416897)
\lineto(192.16262132,203.53251542)
\lineto(192.06057407,203.56086188)
\lineto(191.95852683,203.58353904)
\lineto(191.85081029,203.60054692)
\lineto(191.74309376,203.61755479)
\lineto(191.62970793,203.62889337)
\lineto(191.5163221,203.63456267)
\lineto(191.39726699,203.63456267)
\lineto(191.42561344,203.01660991)
\curveto(193.25679454,203.01660991)(193.65364494,200.70353904)(193.65364494,200.5391296)
\curveto(193.65364494,200.35771227)(193.60262132,200.19330282)(193.57427486,200.05157054)
\lineto(192.15695203,194.5126729)
\curveto(192.01521974,194.00810597)(192.01521974,193.95708235)(191.59002289,193.4695233)
\curveto(190.34844809,191.91613747)(189.19191266,191.46826345)(188.40388116,191.46826345)
\curveto(186.99222762,191.46826345)(186.59537722,193.01597999)(186.59537722,194.1214918)
\curveto(186.59537722,195.53314534)(187.49679454,199.00275164)(188.14309376,200.30668865)
\curveto(189.02183391,201.9734603)(190.29742447,203.01660991)(191.42561344,203.01660991)
\lineto(191.39726699,203.63456267)
\curveto(188.09207014,203.63456267)(184.59411738,199.4903107)(184.59411738,195.3630666)
\curveto(184.59411738,192.70416897)(186.14183391,190.84464141)(188.34718825,190.84464141)
\curveto(188.90844809,190.84464141)(190.32577092,190.96369652)(192.01521974,192.96495637)
\curveto(192.24766069,191.77440519)(193.22844809,190.84464141)(194.58340872,190.84464141)
\curveto(195.5755347,190.84464141)(196.22183391,191.49094062)(196.67537722,192.39802723)
\curveto(197.15159769,193.41283038)(197.52577092,195.14196424)(197.52577092,195.19298786)
\curveto(197.52577092,195.47645243)(197.27065281,195.47645243)(197.19128273,195.47645243)
\curveto(196.90781817,195.47645243)(196.87947171,195.3630666)(196.78876305,194.96621621)
\curveto(196.31254258,193.12936582)(195.79663706,191.46826345)(194.64577092,191.46826345)
\curveto(193.88608588,191.46826345)(193.79537722,192.19960204)(193.79537722,192.76653117)
\curveto(193.79537722,193.39015322)(193.85207014,193.61125558)(194.15821187,194.85849967)
\curveto(194.47569218,196.04338156)(194.5323851,196.32684613)(194.78750321,197.39834219)
\lineto(195.79663706,201.34983826)
\curveto(196.00073155,202.13786975)(196.00073155,202.20023196)(196.00073155,202.30794849)
\curveto(196.00073155,202.79550755)(195.65490478,203.07897211)(195.17868431,203.07897211)
\curveto(194.50403864,203.07897211)(194.07884179,202.44968078)(193.99380242,201.83172802)
\closepath
}
}
{
\newrgbcolor{curcolor}{0 0 0}
\pscustom[linewidth=0,linecolor=curcolor]
{
\newpath
\moveto(193.99380242,201.83172802)
\lineto(193.9427788,201.92810597)
\lineto(193.89175517,202.02448393)
\lineto(193.84073155,202.11519259)
\lineto(193.78970793,202.20590125)
\lineto(193.73301502,202.29660991)
\lineto(193.67065281,202.38731857)
\lineto(193.6139599,202.47235794)
\lineto(193.55159769,202.55172802)
\lineto(193.48923549,202.63676739)
\lineto(193.42120399,202.71046818)
\lineto(193.3531725,202.78983826)
\lineto(193.27947171,202.86353904)
\lineto(193.21144021,202.93157054)
\lineto(193.13773943,202.99960204)
\lineto(193.05836935,203.06196424)
\lineto(192.97899927,203.12432645)
\lineto(192.89962919,203.18668865)
\lineto(192.81458982,203.23771227)
\lineto(192.72955045,203.29440519)
\lineto(192.63884179,203.33975952)
\lineto(192.54813313,203.38511385)
\lineto(192.45742447,203.43046818)
\lineto(192.36104651,203.46448393)
\lineto(192.26466856,203.50416897)
\lineto(192.16262132,203.53251542)
\lineto(192.06057407,203.56086188)
\lineto(191.95852683,203.58353904)
\lineto(191.85081029,203.60054692)
\lineto(191.74309376,203.61755479)
\lineto(191.62970793,203.62889337)
\lineto(191.5163221,203.63456267)
\lineto(191.39726699,203.63456267)
\lineto(191.42561344,203.01660991)
\curveto(193.25679454,203.01660991)(193.65364494,200.70353904)(193.65364494,200.5391296)
\curveto(193.65364494,200.35771227)(193.60262132,200.19330282)(193.57427486,200.05157054)
\lineto(192.15695203,194.5126729)
\curveto(192.01521974,194.00810597)(192.01521974,193.95708235)(191.59002289,193.4695233)
\curveto(190.34844809,191.91613747)(189.19191266,191.46826345)(188.40388116,191.46826345)
\curveto(186.99222762,191.46826345)(186.59537722,193.01597999)(186.59537722,194.1214918)
\curveto(186.59537722,195.53314534)(187.49679454,199.00275164)(188.14309376,200.30668865)
\curveto(189.02183391,201.9734603)(190.29742447,203.01660991)(191.42561344,203.01660991)
\lineto(191.39726699,203.63456267)
\curveto(188.09207014,203.63456267)(184.59411738,199.4903107)(184.59411738,195.3630666)
\curveto(184.59411738,192.70416897)(186.14183391,190.84464141)(188.34718825,190.84464141)
\curveto(188.90844809,190.84464141)(190.32577092,190.96369652)(192.01521974,192.96495637)
\curveto(192.24766069,191.77440519)(193.22844809,190.84464141)(194.58340872,190.84464141)
\curveto(195.5755347,190.84464141)(196.22183391,191.49094062)(196.67537722,192.39802723)
\curveto(197.15159769,193.41283038)(197.52577092,195.14196424)(197.52577092,195.19298786)
\curveto(197.52577092,195.47645243)(197.27065281,195.47645243)(197.19128273,195.47645243)
\curveto(196.90781817,195.47645243)(196.87947171,195.3630666)(196.78876305,194.96621621)
\curveto(196.31254258,193.12936582)(195.79663706,191.46826345)(194.64577092,191.46826345)
\curveto(193.88608588,191.46826345)(193.79537722,192.19960204)(193.79537722,192.76653117)
\curveto(193.79537722,193.39015322)(193.85207014,193.61125558)(194.15821187,194.85849967)
\curveto(194.47569218,196.04338156)(194.5323851,196.32684613)(194.78750321,197.39834219)
\lineto(195.79663706,201.34983826)
\curveto(196.00073155,202.13786975)(196.00073155,202.20023196)(196.00073155,202.30794849)
\curveto(196.00073155,202.79550755)(195.65490478,203.07897211)(195.17868431,203.07897211)
\curveto(194.50403864,203.07897211)(194.07884179,202.44968078)(193.99380242,201.83172802)
\closepath
}
}
{
\newrgbcolor{curcolor}{0 0 0}
\pscustom[linestyle=none,fillstyle=solid,fillcolor=curcolor]
{
\newpath
\moveto(204.17584966,202.44968078)
\lineto(206.8347473,202.44968078)
\curveto(207.39033785,202.44968078)(207.67380242,202.44968078)(207.67380242,203.01660991)
\curveto(207.67380242,203.32842093)(207.39033785,203.32842093)(206.88577092,203.32842093)
\lineto(204.39695203,203.32842093)
\curveto(205.41742447,207.3422792)(205.55915675,207.90920834)(205.55915675,208.07361778)
\curveto(205.55915675,208.55550755)(205.21899927,208.83897211)(204.7427788,208.83897211)
\curveto(204.65207014,208.83897211)(203.86403864,208.81062566)(203.60892053,207.81849967)
\lineto(202.50907801,203.32842093)
\lineto(199.85584966,203.32842093)
\curveto(199.28892053,203.32842093)(199.00545596,203.32842093)(199.00545596,202.79550755)
\curveto(199.00545596,202.44968078)(199.23222762,202.44968078)(199.79915675,202.44968078)
\lineto(202.28230636,202.44968078)
\curveto(200.25270006,194.43330282)(200.13931423,193.95708235)(200.13931423,193.44117684)
\curveto(200.13931423,191.91613747)(201.21081029,190.84464141)(202.73018037,190.84464141)
\curveto(205.61018037,190.84464141)(207.2259284,194.96621621)(207.2259284,195.19298786)
\curveto(207.2259284,195.47645243)(207.00482604,195.47645243)(206.88577092,195.47645243)
\curveto(206.63065281,195.47645243)(206.60230636,195.39141306)(206.46057407,195.07960204)
\curveto(205.24734573,192.14857841)(203.75065281,191.46826345)(202.79254258,191.46826345)
\curveto(202.20293628,191.46826345)(201.91947171,191.83676739)(201.91947171,192.76653117)
\curveto(201.91947171,193.44117684)(201.97049533,193.64527133)(202.08388116,194.1214918)
\closepath
}
}
{
\newrgbcolor{curcolor}{0 0 0}
\pscustom[linewidth=0,linecolor=curcolor]
{
\newpath
\moveto(204.17584966,202.44968078)
\lineto(206.8347473,202.44968078)
\curveto(207.39033785,202.44968078)(207.67380242,202.44968078)(207.67380242,203.01660991)
\curveto(207.67380242,203.32842093)(207.39033785,203.32842093)(206.88577092,203.32842093)
\lineto(204.39695203,203.32842093)
\curveto(205.41742447,207.3422792)(205.55915675,207.90920834)(205.55915675,208.07361778)
\curveto(205.55915675,208.55550755)(205.21899927,208.83897211)(204.7427788,208.83897211)
\curveto(204.65207014,208.83897211)(203.86403864,208.81062566)(203.60892053,207.81849967)
\lineto(202.50907801,203.32842093)
\lineto(199.85584966,203.32842093)
\curveto(199.28892053,203.32842093)(199.00545596,203.32842093)(199.00545596,202.79550755)
\curveto(199.00545596,202.44968078)(199.23222762,202.44968078)(199.79915675,202.44968078)
\lineto(202.28230636,202.44968078)
\curveto(200.25270006,194.43330282)(200.13931423,193.95708235)(200.13931423,193.44117684)
\curveto(200.13931423,191.91613747)(201.21081029,190.84464141)(202.73018037,190.84464141)
\curveto(205.61018037,190.84464141)(207.2259284,194.96621621)(207.2259284,195.19298786)
\curveto(207.2259284,195.47645243)(207.00482604,195.47645243)(206.88577092,195.47645243)
\curveto(206.63065281,195.47645243)(206.60230636,195.39141306)(206.46057407,195.07960204)
\curveto(205.24734573,192.14857841)(203.75065281,191.46826345)(202.79254258,191.46826345)
\curveto(202.20293628,191.46826345)(201.91947171,191.83676739)(201.91947171,192.76653117)
\curveto(201.91947171,193.44117684)(201.97049533,193.64527133)(202.08388116,194.1214918)
\closepath
}
}
{
\newrgbcolor{curcolor}{0 0 0}
\pscustom[linestyle=none,fillstyle=solid,fillcolor=curcolor]
{
\newpath
\moveto(219.09175517,201.83172802)
\lineto(219.04640084,201.92810597)
\lineto(218.99537722,202.02448393)
\lineto(218.9443536,202.11519259)
\lineto(218.88766069,202.20590125)
\lineto(218.83096777,202.29660991)
\lineto(218.77427486,202.38731857)
\lineto(218.71758195,202.47235794)
\lineto(218.65521974,202.55172802)
\lineto(218.58718825,202.63676739)
\lineto(218.52482604,202.71046818)
\lineto(218.45679454,202.78983826)
\lineto(218.38309376,202.86353904)
\lineto(218.30939297,202.93157054)
\lineto(218.23569218,202.99960204)
\lineto(218.1619914,203.06196424)
\lineto(218.08262132,203.12432645)
\lineto(217.99758195,203.18668865)
\lineto(217.91821187,203.23771227)
\lineto(217.8331725,203.29440519)
\lineto(217.74246384,203.33975952)
\lineto(217.65175517,203.38511385)
\lineto(217.56104651,203.43046818)
\lineto(217.46466856,203.46448393)
\lineto(217.36829061,203.50416897)
\lineto(217.26624336,203.53251542)
\lineto(217.16419612,203.56086188)
\lineto(217.06214888,203.58353904)
\lineto(216.95443234,203.60054692)
\lineto(216.84104651,203.61755479)
\lineto(216.72766069,203.62889337)
\lineto(216.61427486,203.63456267)
\lineto(216.50088903,203.63456267)
\lineto(216.5235662,203.01660991)
\curveto(218.36041659,203.01660991)(218.75726699,200.70353904)(218.75726699,200.5391296)
\curveto(218.75726699,200.35771227)(218.70624336,200.19330282)(218.67789691,200.05157054)
\lineto(217.26057407,194.5126729)
\curveto(217.11884179,194.00810597)(217.11884179,193.95708235)(216.69364494,193.4695233)
\curveto(215.45207014,191.91613747)(214.28986541,191.46826345)(213.50183391,191.46826345)
\curveto(212.09584966,191.46826345)(211.69899927,193.01597999)(211.69899927,194.1214918)
\curveto(211.69899927,195.53314534)(212.60041659,199.00275164)(213.2467158,200.30668865)
\curveto(214.12545596,201.9734603)(215.40104651,203.01660991)(216.5235662,203.01660991)
\lineto(216.50088903,203.63456267)
\curveto(213.19569218,203.63456267)(209.69773943,199.4903107)(209.69773943,195.3630666)
\curveto(209.69773943,192.70416897)(211.24545596,190.84464141)(213.45081029,190.84464141)
\curveto(214.00640084,190.84464141)(215.42372368,190.96369652)(217.11884179,192.96495637)
\curveto(217.35128273,191.77440519)(218.33207014,190.84464141)(219.68703077,190.84464141)
\curveto(220.67915675,190.84464141)(221.32545596,191.49094062)(221.77899927,192.39802723)
\curveto(222.25521974,193.41283038)(222.62939297,195.14196424)(222.62939297,195.19298786)
\curveto(222.62939297,195.47645243)(222.36860557,195.47645243)(222.28923549,195.47645243)
\curveto(222.00577092,195.47645243)(221.98309376,195.3630666)(221.8923851,194.96621621)
\curveto(221.41616462,193.12936582)(220.90025911,191.46826345)(219.74939297,191.46826345)
\curveto(218.98970793,191.46826345)(218.89899927,192.19960204)(218.89899927,192.76653117)
\curveto(218.89899927,193.39015322)(218.95002289,193.61125558)(219.26183391,194.85849967)
\curveto(219.57931423,196.04338156)(219.63600714,196.32684613)(219.89112525,197.39834219)
\lineto(220.90025911,201.34983826)
\curveto(221.1043536,202.13786975)(221.1043536,202.20023196)(221.1043536,202.30794849)
\curveto(221.1043536,202.79550755)(220.75852683,203.07897211)(220.28230636,203.07897211)
\curveto(219.60766069,203.07897211)(219.18246384,202.44968078)(219.09175517,201.83172802)
\closepath
}
}
{
\newrgbcolor{curcolor}{0 0 0}
\pscustom[linewidth=0,linecolor=curcolor]
{
\newpath
\moveto(219.09175517,201.83172802)
\lineto(219.04640084,201.92810597)
\lineto(218.99537722,202.02448393)
\lineto(218.9443536,202.11519259)
\lineto(218.88766069,202.20590125)
\lineto(218.83096777,202.29660991)
\lineto(218.77427486,202.38731857)
\lineto(218.71758195,202.47235794)
\lineto(218.65521974,202.55172802)
\lineto(218.58718825,202.63676739)
\lineto(218.52482604,202.71046818)
\lineto(218.45679454,202.78983826)
\lineto(218.38309376,202.86353904)
\lineto(218.30939297,202.93157054)
\lineto(218.23569218,202.99960204)
\lineto(218.1619914,203.06196424)
\lineto(218.08262132,203.12432645)
\lineto(217.99758195,203.18668865)
\lineto(217.91821187,203.23771227)
\lineto(217.8331725,203.29440519)
\lineto(217.74246384,203.33975952)
\lineto(217.65175517,203.38511385)
\lineto(217.56104651,203.43046818)
\lineto(217.46466856,203.46448393)
\lineto(217.36829061,203.50416897)
\lineto(217.26624336,203.53251542)
\lineto(217.16419612,203.56086188)
\lineto(217.06214888,203.58353904)
\lineto(216.95443234,203.60054692)
\lineto(216.84104651,203.61755479)
\lineto(216.72766069,203.62889337)
\lineto(216.61427486,203.63456267)
\lineto(216.50088903,203.63456267)
\lineto(216.5235662,203.01660991)
\curveto(218.36041659,203.01660991)(218.75726699,200.70353904)(218.75726699,200.5391296)
\curveto(218.75726699,200.35771227)(218.70624336,200.19330282)(218.67789691,200.05157054)
\lineto(217.26057407,194.5126729)
\curveto(217.11884179,194.00810597)(217.11884179,193.95708235)(216.69364494,193.4695233)
\curveto(215.45207014,191.91613747)(214.28986541,191.46826345)(213.50183391,191.46826345)
\curveto(212.09584966,191.46826345)(211.69899927,193.01597999)(211.69899927,194.1214918)
\curveto(211.69899927,195.53314534)(212.60041659,199.00275164)(213.2467158,200.30668865)
\curveto(214.12545596,201.9734603)(215.40104651,203.01660991)(216.5235662,203.01660991)
\lineto(216.50088903,203.63456267)
\curveto(213.19569218,203.63456267)(209.69773943,199.4903107)(209.69773943,195.3630666)
\curveto(209.69773943,192.70416897)(211.24545596,190.84464141)(213.45081029,190.84464141)
\curveto(214.00640084,190.84464141)(215.42372368,190.96369652)(217.11884179,192.96495637)
\curveto(217.35128273,191.77440519)(218.33207014,190.84464141)(219.68703077,190.84464141)
\curveto(220.67915675,190.84464141)(221.32545596,191.49094062)(221.77899927,192.39802723)
\curveto(222.25521974,193.41283038)(222.62939297,195.14196424)(222.62939297,195.19298786)
\curveto(222.62939297,195.47645243)(222.36860557,195.47645243)(222.28923549,195.47645243)
\curveto(222.00577092,195.47645243)(221.98309376,195.3630666)(221.8923851,194.96621621)
\curveto(221.41616462,193.12936582)(220.90025911,191.46826345)(219.74939297,191.46826345)
\curveto(218.98970793,191.46826345)(218.89899927,192.19960204)(218.89899927,192.76653117)
\curveto(218.89899927,193.39015322)(218.95002289,193.61125558)(219.26183391,194.85849967)
\curveto(219.57931423,196.04338156)(219.63600714,196.32684613)(219.89112525,197.39834219)
\lineto(220.90025911,201.34983826)
\curveto(221.1043536,202.13786975)(221.1043536,202.20023196)(221.1043536,202.30794849)
\curveto(221.1043536,202.79550755)(220.75852683,203.07897211)(220.28230636,203.07897211)
\curveto(219.60766069,203.07897211)(219.18246384,202.44968078)(219.09175517,201.83172802)
\closepath
}
}
{
\newrgbcolor{curcolor}{0 0 0}
\pscustom[linestyle=none,fillstyle=solid,fillcolor=curcolor]
{
\newpath
\moveto(384.95478425,218.54937211)
\lineto(384.69399685,218.33393904)
\lineto(384.43320945,218.12984456)
\lineto(384.16675276,217.93141936)
\lineto(383.90029606,217.74433274)
\lineto(383.63383937,217.56858471)
\lineto(383.36171339,217.40417526)
\lineto(383.0895874,217.2511044)
\lineto(382.81746142,217.10937211)
\lineto(382.68139843,217.04700991)
\lineto(382.54533543,216.98464771)
\lineto(382.40927244,216.92795479)
\lineto(382.27320945,216.87126188)
\lineto(382.13147717,216.82023826)
\lineto(381.99541417,216.77488393)
\lineto(381.85935118,216.73519889)
\lineto(381.72328819,216.69551385)
\lineto(381.5872252,216.6614981)
\lineto(381.4511622,216.63315164)
\lineto(381.31509921,216.60480519)
\lineto(381.17336693,216.58212802)
\lineto(381.03730394,216.56512015)
\lineto(380.90124094,216.55378156)
\lineto(380.77084724,216.54811227)
\lineto(380.63478425,216.54811227)
\curveto(379.50659528,216.54811227)(378.68454803,217.02433274)(377.69809134,217.59126188)
\curveto(376.87604409,218.06748235)(375.99730394,218.54937211)(374.87478425,218.54937211)
\curveto(374.16612283,218.54937211)(373.42911496,218.32826975)(372.8111622,218.04480519)
\curveto(372.24423307,217.76134062)(371.67730394,217.4495296)(371.20108346,217.0526792)
\lineto(369.67604409,215.8111044)
\lineto(370.07289449,215.33488393)
\curveto(371.45620157,216.48575007)(372.95289449,217.33614377)(374.3872252,217.33614377)
\curveto(375.52108346,217.33614377)(376.34313071,216.854254)(377.3295874,216.28732487)
\curveto(378.15163465,215.8111044)(379.01903622,215.33488393)(380.15289449,215.33488393)
\curveto(380.86155591,215.33488393)(381.59856378,215.55598629)(382.21651654,215.83945086)
\curveto(382.77777638,216.12291542)(383.34470551,216.42905715)(383.82092598,216.83157684)
\lineto(385.35163465,218.06748235)
\closepath
}
}
{
\newrgbcolor{curcolor}{0 0 0}
\pscustom[linewidth=0,linecolor=curcolor]
{
\newpath
\moveto(384.95478425,218.54937211)
\lineto(384.69399685,218.33393904)
\lineto(384.43320945,218.12984456)
\lineto(384.16675276,217.93141936)
\lineto(383.90029606,217.74433274)
\lineto(383.63383937,217.56858471)
\lineto(383.36171339,217.40417526)
\lineto(383.0895874,217.2511044)
\lineto(382.81746142,217.10937211)
\lineto(382.68139843,217.04700991)
\lineto(382.54533543,216.98464771)
\lineto(382.40927244,216.92795479)
\lineto(382.27320945,216.87126188)
\lineto(382.13147717,216.82023826)
\lineto(381.99541417,216.77488393)
\lineto(381.85935118,216.73519889)
\lineto(381.72328819,216.69551385)
\lineto(381.5872252,216.6614981)
\lineto(381.4511622,216.63315164)
\lineto(381.31509921,216.60480519)
\lineto(381.17336693,216.58212802)
\lineto(381.03730394,216.56512015)
\lineto(380.90124094,216.55378156)
\lineto(380.77084724,216.54811227)
\lineto(380.63478425,216.54811227)
\curveto(379.50659528,216.54811227)(378.68454803,217.02433274)(377.69809134,217.59126188)
\curveto(376.87604409,218.06748235)(375.99730394,218.54937211)(374.87478425,218.54937211)
\curveto(374.16612283,218.54937211)(373.42911496,218.32826975)(372.8111622,218.04480519)
\curveto(372.24423307,217.76134062)(371.67730394,217.4495296)(371.20108346,217.0526792)
\lineto(369.67604409,215.8111044)
\lineto(370.07289449,215.33488393)
\curveto(371.45620157,216.48575007)(372.95289449,217.33614377)(374.3872252,217.33614377)
\curveto(375.52108346,217.33614377)(376.34313071,216.854254)(377.3295874,216.28732487)
\curveto(378.15163465,215.8111044)(379.01903622,215.33488393)(380.15289449,215.33488393)
\curveto(380.86155591,215.33488393)(381.59856378,215.55598629)(382.21651654,215.83945086)
\curveto(382.77777638,216.12291542)(383.34470551,216.42905715)(383.82092598,216.83157684)
\lineto(385.35163465,218.06748235)
\closepath
}
}
{
\newrgbcolor{curcolor}{0 0 0}
\pscustom[linestyle=none,fillstyle=solid,fillcolor=curcolor]
{
\newpath
\moveto(377.6584063,208.10086818)
\lineto(377.68108346,208.19724613)
\lineto(377.70376063,208.2822855)
\lineto(377.7264378,208.36732487)
\lineto(377.74911496,208.45236424)
\lineto(377.77179213,208.52606503)
\lineto(377.80013858,208.59976582)
\lineto(377.82848504,208.6734666)
\lineto(377.8568315,208.7414981)
\lineto(377.89084724,208.8038603)
\lineto(377.92486299,208.86055322)
\lineto(377.97021732,208.91724613)
\lineto(378.01557165,208.96826975)
\lineto(378.06659528,209.01929337)
\lineto(378.12328819,209.06464771)
\lineto(378.18565039,209.11000204)
\lineto(378.25368189,209.14968708)
\lineto(378.32738268,209.18370282)
\lineto(378.41242205,209.21771857)
\lineto(378.50313071,209.25173432)
\lineto(378.60517795,209.28008078)
\lineto(378.65620157,209.29141936)
\lineto(378.71289449,209.30275794)
\lineto(378.7695874,209.31409652)
\lineto(378.83194961,209.32543511)
\lineto(378.89431181,209.33677369)
\lineto(378.95667402,209.34811227)
\lineto(379.02470551,209.35945086)
\lineto(379.0984063,209.36512015)
\lineto(379.17210709,209.37645873)
\lineto(379.25147717,209.38212802)
\lineto(379.33084724,209.38779731)
\lineto(379.41021732,209.3934666)
\lineto(379.49525669,209.39913589)
\lineto(379.58596535,209.40480519)
\lineto(379.67667402,209.41047448)
\lineto(379.77305197,209.41614377)
\lineto(379.86942992,209.41614377)
\lineto(379.97147717,209.42181306)
\lineto(380.07352441,209.42181306)
\lineto(380.18124094,209.42748235)
\lineto(380.29462677,209.42748235)
\lineto(380.4080126,209.42748235)
\lineto(380.52706772,209.42748235)
\lineto(380.65179213,209.42748235)
\curveto(381.49651654,209.42748235)(381.72328819,209.42748235)(381.72328819,209.97173432)
\curveto(381.72328819,210.30622251)(381.41714646,210.30622251)(381.27541417,210.30622251)
\curveto(380.33431181,210.30622251)(378.02124094,210.22118314)(377.09147717,210.22118314)
\curveto(376.24108346,210.22118314)(374.17746142,210.30622251)(373.3384063,210.30622251)
\curveto(373.13431181,210.30622251)(372.79415433,210.30622251)(372.79415433,209.73929337)
\curveto(372.79415433,209.42748235)(373.05494173,209.42748235)(373.58218583,209.42748235)
\curveto(373.64454803,209.42748235)(374.17746142,209.42748235)(374.66502047,209.37645873)
\curveto(375.1695874,209.31409652)(375.42470551,209.28575007)(375.42470551,208.92291542)
\curveto(375.42470551,208.8095296)(375.39068976,208.71882093)(375.31131969,208.38433274)
\lineto(371.52990236,193.21897841)
\curveto(371.2464378,192.1134666)(371.18407559,191.89236424)(368.9503748,191.89236424)
\curveto(368.47415433,191.89236424)(368.19068976,191.89236424)(368.19068976,191.32543511)
\curveto(368.19068976,191.01362408)(368.44580787,191.01362408)(368.9503748,191.01362408)
\lineto(382.03509921,191.01362408)
\curveto(382.70974488,191.01362408)(382.73242205,191.01362408)(382.91383937,191.48984456)
\lineto(385.13620157,197.59567133)
\curveto(385.2495874,197.90748235)(385.2495874,197.95850597)(385.2495874,197.99252172)
\curveto(385.2495874,198.10023826)(385.17021732,198.30433274)(384.91509921,198.30433274)
\curveto(384.65431181,198.30433274)(384.63163465,198.16260046)(384.43320945,197.70905715)
\curveto(383.46942992,195.10685243)(382.22785512,191.89236424)(377.34092598,191.89236424)
\lineto(374.69336693,191.89236424)
\curveto(374.29084724,191.89236424)(374.23982362,191.89236424)(374.06974488,191.91504141)
\curveto(373.78628031,191.94338786)(373.70124094,191.97173432)(373.70124094,192.19850597)
\curveto(373.70124094,192.27787605)(373.70124094,192.34023826)(373.84297323,192.84480519)
\closepath
}
}
{
\newrgbcolor{curcolor}{0 0 0}
\pscustom[linewidth=0,linecolor=curcolor]
{
\newpath
\moveto(377.6584063,208.10086818)
\lineto(377.68108346,208.19724613)
\lineto(377.70376063,208.2822855)
\lineto(377.7264378,208.36732487)
\lineto(377.74911496,208.45236424)
\lineto(377.77179213,208.52606503)
\lineto(377.80013858,208.59976582)
\lineto(377.82848504,208.6734666)
\lineto(377.8568315,208.7414981)
\lineto(377.89084724,208.8038603)
\lineto(377.92486299,208.86055322)
\lineto(377.97021732,208.91724613)
\lineto(378.01557165,208.96826975)
\lineto(378.06659528,209.01929337)
\lineto(378.12328819,209.06464771)
\lineto(378.18565039,209.11000204)
\lineto(378.25368189,209.14968708)
\lineto(378.32738268,209.18370282)
\lineto(378.41242205,209.21771857)
\lineto(378.50313071,209.25173432)
\lineto(378.60517795,209.28008078)
\lineto(378.65620157,209.29141936)
\lineto(378.71289449,209.30275794)
\lineto(378.7695874,209.31409652)
\lineto(378.83194961,209.32543511)
\lineto(378.89431181,209.33677369)
\lineto(378.95667402,209.34811227)
\lineto(379.02470551,209.35945086)
\lineto(379.0984063,209.36512015)
\lineto(379.17210709,209.37645873)
\lineto(379.25147717,209.38212802)
\lineto(379.33084724,209.38779731)
\lineto(379.41021732,209.3934666)
\lineto(379.49525669,209.39913589)
\lineto(379.58596535,209.40480519)
\lineto(379.67667402,209.41047448)
\lineto(379.77305197,209.41614377)
\lineto(379.86942992,209.41614377)
\lineto(379.97147717,209.42181306)
\lineto(380.07352441,209.42181306)
\lineto(380.18124094,209.42748235)
\lineto(380.29462677,209.42748235)
\lineto(380.4080126,209.42748235)
\lineto(380.52706772,209.42748235)
\lineto(380.65179213,209.42748235)
\curveto(381.49651654,209.42748235)(381.72328819,209.42748235)(381.72328819,209.97173432)
\curveto(381.72328819,210.30622251)(381.41714646,210.30622251)(381.27541417,210.30622251)
\curveto(380.33431181,210.30622251)(378.02124094,210.22118314)(377.09147717,210.22118314)
\curveto(376.24108346,210.22118314)(374.17746142,210.30622251)(373.3384063,210.30622251)
\curveto(373.13431181,210.30622251)(372.79415433,210.30622251)(372.79415433,209.73929337)
\curveto(372.79415433,209.42748235)(373.05494173,209.42748235)(373.58218583,209.42748235)
\curveto(373.64454803,209.42748235)(374.17746142,209.42748235)(374.66502047,209.37645873)
\curveto(375.1695874,209.31409652)(375.42470551,209.28575007)(375.42470551,208.92291542)
\curveto(375.42470551,208.8095296)(375.39068976,208.71882093)(375.31131969,208.38433274)
\lineto(371.52990236,193.21897841)
\curveto(371.2464378,192.1134666)(371.18407559,191.89236424)(368.9503748,191.89236424)
\curveto(368.47415433,191.89236424)(368.19068976,191.89236424)(368.19068976,191.32543511)
\curveto(368.19068976,191.01362408)(368.44580787,191.01362408)(368.9503748,191.01362408)
\lineto(382.03509921,191.01362408)
\curveto(382.70974488,191.01362408)(382.73242205,191.01362408)(382.91383937,191.48984456)
\lineto(385.13620157,197.59567133)
\curveto(385.2495874,197.90748235)(385.2495874,197.95850597)(385.2495874,197.99252172)
\curveto(385.2495874,198.10023826)(385.17021732,198.30433274)(384.91509921,198.30433274)
\curveto(384.65431181,198.30433274)(384.63163465,198.16260046)(384.43320945,197.70905715)
\curveto(383.46942992,195.10685243)(382.22785512,191.89236424)(377.34092598,191.89236424)
\lineto(374.69336693,191.89236424)
\curveto(374.29084724,191.89236424)(374.23982362,191.89236424)(374.06974488,191.91504141)
\curveto(373.78628031,191.94338786)(373.70124094,191.97173432)(373.70124094,192.19850597)
\curveto(373.70124094,192.27787605)(373.70124094,192.34023826)(373.84297323,192.84480519)
\closepath
}
}
{
\newrgbcolor{curcolor}{0 0 0}
\pscustom[linestyle=none,fillstyle=solid,fillcolor=curcolor]
{
\newpath
\moveto(395.68675276,184.23882093)
\lineto(395.68675276,184.25015952)
\lineto(395.68675276,184.26716739)
\lineto(395.68675276,184.27283668)
\lineto(395.68675276,184.28417526)
\lineto(395.68108346,184.28984456)
\lineto(395.68108346,184.30118314)
\lineto(395.67541417,184.31252172)
\lineto(395.67541417,184.31819101)
\lineto(395.66974488,184.3295296)
\lineto(395.66407559,184.34086818)
\lineto(395.65273701,184.35220676)
\lineto(395.64706772,184.36921463)
\lineto(395.63572913,184.38055322)
\lineto(395.63005984,184.39756109)
\lineto(395.61305197,184.41456897)
\lineto(395.60171339,184.43157684)
\lineto(395.58470551,184.44858471)
\lineto(395.56769764,184.47126188)
\lineto(395.56202835,184.48260046)
\lineto(395.55068976,184.49393904)
\lineto(395.54502047,184.50527763)
\lineto(395.53368189,184.51661621)
\lineto(395.52234331,184.52795479)
\lineto(395.51100472,184.54496267)
\lineto(395.49966614,184.55630125)
\lineto(395.48265827,184.56763983)
\lineto(395.47131969,184.58464771)
\lineto(395.4599811,184.59598629)
\lineto(395.44297323,184.61299416)
\lineto(395.43163465,184.63000204)
\lineto(395.41462677,184.64700991)
\lineto(395.3976189,184.66401778)
\lineto(395.38628031,184.68102566)
\lineto(395.36927244,184.69803353)
\lineto(395.34659528,184.71504141)
\lineto(395.3295874,184.73204928)
\lineto(395.31257953,184.75472645)
\lineto(395.29557165,184.77173432)
\lineto(395.27289449,184.79441148)
\lineto(395.25021732,184.81141936)
\lineto(395.23320945,184.83409652)
\lineto(395.21053228,184.85677369)
\curveto(391.68423307,188.41708865)(390.77714646,193.7518918)(390.77714646,198.0718918)
\curveto(390.77714646,202.99283668)(391.85431181,207.90811227)(395.32391811,211.43441148)
\curveto(395.68675276,211.78023826)(395.68675276,211.83126188)(395.68675276,211.92197054)
\curveto(395.68675276,212.11472645)(395.57903622,212.19409652)(395.41462677,212.19409652)
\curveto(395.1311622,212.19409652)(392.58565039,210.27787605)(390.91887874,206.69488393)
\curveto(389.47887874,203.58244298)(389.13872126,200.44732487)(389.13872126,198.0718918)
\curveto(389.13872126,195.86653747)(389.45053228,192.45929337)(391.00391811,189.26181306)
\curveto(392.69336693,185.78653747)(395.1311622,183.95535637)(395.41462677,183.95535637)
\curveto(395.57903622,183.95535637)(395.68675276,184.03472645)(395.68675276,184.23882093)
\closepath
}
}
{
\newrgbcolor{curcolor}{0 0 0}
\pscustom[linewidth=0,linecolor=curcolor]
{
\newpath
\moveto(395.68675276,184.23882093)
\lineto(395.68675276,184.25015952)
\lineto(395.68675276,184.26716739)
\lineto(395.68675276,184.27283668)
\lineto(395.68675276,184.28417526)
\lineto(395.68108346,184.28984456)
\lineto(395.68108346,184.30118314)
\lineto(395.67541417,184.31252172)
\lineto(395.67541417,184.31819101)
\lineto(395.66974488,184.3295296)
\lineto(395.66407559,184.34086818)
\lineto(395.65273701,184.35220676)
\lineto(395.64706772,184.36921463)
\lineto(395.63572913,184.38055322)
\lineto(395.63005984,184.39756109)
\lineto(395.61305197,184.41456897)
\lineto(395.60171339,184.43157684)
\lineto(395.58470551,184.44858471)
\lineto(395.56769764,184.47126188)
\lineto(395.56202835,184.48260046)
\lineto(395.55068976,184.49393904)
\lineto(395.54502047,184.50527763)
\lineto(395.53368189,184.51661621)
\lineto(395.52234331,184.52795479)
\lineto(395.51100472,184.54496267)
\lineto(395.49966614,184.55630125)
\lineto(395.48265827,184.56763983)
\lineto(395.47131969,184.58464771)
\lineto(395.4599811,184.59598629)
\lineto(395.44297323,184.61299416)
\lineto(395.43163465,184.63000204)
\lineto(395.41462677,184.64700991)
\lineto(395.3976189,184.66401778)
\lineto(395.38628031,184.68102566)
\lineto(395.36927244,184.69803353)
\lineto(395.34659528,184.71504141)
\lineto(395.3295874,184.73204928)
\lineto(395.31257953,184.75472645)
\lineto(395.29557165,184.77173432)
\lineto(395.27289449,184.79441148)
\lineto(395.25021732,184.81141936)
\lineto(395.23320945,184.83409652)
\lineto(395.21053228,184.85677369)
\curveto(391.68423307,188.41708865)(390.77714646,193.7518918)(390.77714646,198.0718918)
\curveto(390.77714646,202.99283668)(391.85431181,207.90811227)(395.32391811,211.43441148)
\curveto(395.68675276,211.78023826)(395.68675276,211.83126188)(395.68675276,211.92197054)
\curveto(395.68675276,212.11472645)(395.57903622,212.19409652)(395.41462677,212.19409652)
\curveto(395.1311622,212.19409652)(392.58565039,210.27787605)(390.91887874,206.69488393)
\curveto(389.47887874,203.58244298)(389.13872126,200.44732487)(389.13872126,198.0718918)
\curveto(389.13872126,195.86653747)(389.45053228,192.45929337)(391.00391811,189.26181306)
\curveto(392.69336693,185.78653747)(395.1311622,183.95535637)(395.41462677,183.95535637)
\curveto(395.57903622,183.95535637)(395.68675276,184.03472645)(395.68675276,184.23882093)
\closepath
}
}
{
\newrgbcolor{curcolor}{0 0 0}
\pscustom[linestyle=none,fillstyle=solid,fillcolor=curcolor]
{
\newpath
\moveto(413.59604409,211.40606503)
\lineto(413.33525669,211.19063196)
\lineto(413.0688,210.98653747)
\lineto(412.8080126,210.78811227)
\lineto(412.54155591,210.60102566)
\lineto(412.26942992,210.42527763)
\lineto(411.99730394,210.26086818)
\lineto(411.73084724,210.10779731)
\lineto(411.45872126,209.97173432)
\lineto(411.31698898,209.90370282)
\lineto(411.18092598,209.84134062)
\lineto(411.04486299,209.78464771)
\lineto(410.9088,209.73362408)
\lineto(410.77273701,209.68260046)
\lineto(410.63667402,209.63157684)
\lineto(410.50061102,209.5918918)
\lineto(410.36454803,209.55220676)
\lineto(410.22281575,209.51819101)
\lineto(410.08675276,209.48984456)
\lineto(409.95068976,209.4614981)
\lineto(409.81462677,209.44449023)
\lineto(409.67856378,209.42748235)
\lineto(409.54250079,209.41614377)
\lineto(409.4064378,209.40480519)
\lineto(409.2703748,209.40480519)
\curveto(408.14785512,209.40480519)(407.32013858,209.88102566)(406.33935118,210.44795479)
\curveto(405.51730394,210.92984456)(404.63856378,211.40606503)(403.51604409,211.40606503)
\curveto(402.80738268,211.40606503)(402.0703748,211.18496267)(401.45242205,210.9014981)
\curveto(400.88549291,210.61803353)(400.31856378,210.30622251)(399.83667402,209.90937211)
\lineto(398.31163465,208.66779731)
\lineto(398.71415433,208.19157684)
\curveto(400.09179213,209.34244298)(401.59415433,210.19283668)(403.02848504,210.19283668)
\curveto(404.16234331,210.19283668)(404.98439055,209.71094692)(405.96517795,209.14401778)
\curveto(406.79289449,208.66779731)(407.66029606,208.19157684)(408.79415433,208.19157684)
\curveto(409.50281575,208.19157684)(410.23982362,208.4126792)(410.85777638,208.69614377)
\curveto(411.41336693,208.97960834)(411.98029606,209.28575007)(412.46218583,209.68826975)
\lineto(413.99289449,210.92984456)
\closepath
}
}
{
\newrgbcolor{curcolor}{0 0 0}
\pscustom[linewidth=0,linecolor=curcolor]
{
\newpath
\moveto(413.59604409,211.40606503)
\lineto(413.33525669,211.19063196)
\lineto(413.0688,210.98653747)
\lineto(412.8080126,210.78811227)
\lineto(412.54155591,210.60102566)
\lineto(412.26942992,210.42527763)
\lineto(411.99730394,210.26086818)
\lineto(411.73084724,210.10779731)
\lineto(411.45872126,209.97173432)
\lineto(411.31698898,209.90370282)
\lineto(411.18092598,209.84134062)
\lineto(411.04486299,209.78464771)
\lineto(410.9088,209.73362408)
\lineto(410.77273701,209.68260046)
\lineto(410.63667402,209.63157684)
\lineto(410.50061102,209.5918918)
\lineto(410.36454803,209.55220676)
\lineto(410.22281575,209.51819101)
\lineto(410.08675276,209.48984456)
\lineto(409.95068976,209.4614981)
\lineto(409.81462677,209.44449023)
\lineto(409.67856378,209.42748235)
\lineto(409.54250079,209.41614377)
\lineto(409.4064378,209.40480519)
\lineto(409.2703748,209.40480519)
\curveto(408.14785512,209.40480519)(407.32013858,209.88102566)(406.33935118,210.44795479)
\curveto(405.51730394,210.92984456)(404.63856378,211.40606503)(403.51604409,211.40606503)
\curveto(402.80738268,211.40606503)(402.0703748,211.18496267)(401.45242205,210.9014981)
\curveto(400.88549291,210.61803353)(400.31856378,210.30622251)(399.83667402,209.90937211)
\lineto(398.31163465,208.66779731)
\lineto(398.71415433,208.19157684)
\curveto(400.09179213,209.34244298)(401.59415433,210.19283668)(403.02848504,210.19283668)
\curveto(404.16234331,210.19283668)(404.98439055,209.71094692)(405.96517795,209.14401778)
\curveto(406.79289449,208.66779731)(407.66029606,208.19157684)(408.79415433,208.19157684)
\curveto(409.50281575,208.19157684)(410.23982362,208.4126792)(410.85777638,208.69614377)
\curveto(411.41336693,208.97960834)(411.98029606,209.28575007)(412.46218583,209.68826975)
\lineto(413.99289449,210.92984456)
\closepath
}
}
{
\newrgbcolor{curcolor}{0 0 0}
\pscustom[linestyle=none,fillstyle=solid,fillcolor=curcolor]
{
\newpath
\moveto(407.1784063,192.60102566)
\lineto(407.20675276,192.49330912)
\lineto(407.2464378,192.38559259)
\lineto(407.28045354,192.28354534)
\lineto(407.32580787,192.18716739)
\lineto(407.3711622,192.08512015)
\lineto(407.42218583,191.99441148)
\lineto(407.47320945,191.90370282)
\lineto(407.52990236,191.81299416)
\lineto(407.59226457,191.72795479)
\lineto(407.65462677,191.64291542)
\lineto(407.72265827,191.56354534)
\lineto(407.79068976,191.48984456)
\lineto(407.86439055,191.41614377)
\lineto(407.93809134,191.34811227)
\lineto(408.01746142,191.28008078)
\lineto(408.0968315,191.21771857)
\lineto(408.18187087,191.15535637)
\lineto(408.26691024,191.09866345)
\lineto(408.3576189,191.04763983)
\lineto(408.44832756,190.99661621)
\lineto(408.53903622,190.95126188)
\lineto(408.63541417,190.91157684)
\lineto(408.73179213,190.8718918)
\lineto(408.83383937,190.83787605)
\lineto(408.93588661,190.8038603)
\lineto(409.03793386,190.78118314)
\lineto(409.14565039,190.75850597)
\lineto(409.25336693,190.73582881)
\lineto(409.36108346,190.72449023)
\lineto(409.4688,190.71315164)
\lineto(409.58218583,190.70748235)
\lineto(409.69557165,190.70181306)
\curveto(410.67635906,190.70181306)(411.33399685,191.34811227)(411.78187087,192.25519889)
\curveto(412.26376063,193.27000204)(412.62659528,194.99913589)(412.62659528,195.05582881)
\curveto(412.62659528,195.33929337)(412.37714646,195.33929337)(412.29210709,195.33929337)
\curveto(412.00864252,195.33929337)(411.98029606,195.22023826)(411.8895874,194.82338786)
\curveto(411.4984063,193.24732487)(410.95982362,191.32543511)(409.77494173,191.32543511)
\curveto(409.17966614,191.32543511)(408.90754016,191.69393904)(408.90754016,192.62370282)
\curveto(408.90754016,193.24732487)(409.24202835,194.57393904)(409.46313071,195.56039574)
\lineto(410.26250079,198.61614377)
\curveto(410.34187087,199.04134062)(410.62533543,200.11283668)(410.73872126,200.53803353)
\curveto(410.88045354,201.18433274)(411.16391811,202.25582881)(411.16391811,202.42590755)
\curveto(411.16391811,202.93614377)(410.76706772,203.18559259)(410.34187087,203.18559259)
\curveto(410.20013858,203.18559259)(409.46313071,203.15724613)(409.24202835,202.19913589)
\curveto(408.70344567,200.13551385)(407.46187087,195.19756109)(407.12738268,193.69519889)
\curveto(407.08769764,193.58181306)(405.96517795,191.32543511)(403.90155591,191.32543511)
\curveto(402.42754016,191.32543511)(402.14407559,192.60102566)(402.14407559,193.64417526)
\curveto(402.14407559,195.22023826)(402.94344567,197.45393904)(403.68045354,199.40417526)
\curveto(404.01494173,200.25456897)(404.15667402,200.64008078)(404.15667402,201.18433274)
\curveto(404.15667402,202.44858471)(403.25525669,203.49740361)(401.83793386,203.49740361)
\curveto(399.15068976,203.49740361)(398.10754016,199.40417526)(398.10754016,199.14338786)
\curveto(398.10754016,198.8599233)(398.39100472,198.8599233)(398.45336693,198.8599233)
\curveto(398.7368315,198.8599233)(398.76517795,198.9222855)(398.90691024,199.37582881)
\curveto(399.60423307,201.83063196)(400.68706772,202.87378156)(401.75856378,202.87378156)
\curveto(402.00234331,202.87378156)(402.4672252,202.8511044)(402.4672252,201.94401778)
\curveto(402.4672252,201.26370282)(402.14407559,200.44732487)(401.97966614,200.02212802)
\curveto(400.93651654,197.2214981)(400.34124094,195.48102566)(400.34124094,194.09771857)
\curveto(400.34124094,191.41047448)(402.28580787,190.70181306)(403.82218583,190.70181306)
\curveto(405.68171339,190.70181306)(406.70218583,191.97173432)(407.1784063,192.60102566)
\closepath
}
}
{
\newrgbcolor{curcolor}{0 0 0}
\pscustom[linewidth=0,linecolor=curcolor]
{
\newpath
\moveto(407.1784063,192.60102566)
\lineto(407.20675276,192.49330912)
\lineto(407.2464378,192.38559259)
\lineto(407.28045354,192.28354534)
\lineto(407.32580787,192.18716739)
\lineto(407.3711622,192.08512015)
\lineto(407.42218583,191.99441148)
\lineto(407.47320945,191.90370282)
\lineto(407.52990236,191.81299416)
\lineto(407.59226457,191.72795479)
\lineto(407.65462677,191.64291542)
\lineto(407.72265827,191.56354534)
\lineto(407.79068976,191.48984456)
\lineto(407.86439055,191.41614377)
\lineto(407.93809134,191.34811227)
\lineto(408.01746142,191.28008078)
\lineto(408.0968315,191.21771857)
\lineto(408.18187087,191.15535637)
\lineto(408.26691024,191.09866345)
\lineto(408.3576189,191.04763983)
\lineto(408.44832756,190.99661621)
\lineto(408.53903622,190.95126188)
\lineto(408.63541417,190.91157684)
\lineto(408.73179213,190.8718918)
\lineto(408.83383937,190.83787605)
\lineto(408.93588661,190.8038603)
\lineto(409.03793386,190.78118314)
\lineto(409.14565039,190.75850597)
\lineto(409.25336693,190.73582881)
\lineto(409.36108346,190.72449023)
\lineto(409.4688,190.71315164)
\lineto(409.58218583,190.70748235)
\lineto(409.69557165,190.70181306)
\curveto(410.67635906,190.70181306)(411.33399685,191.34811227)(411.78187087,192.25519889)
\curveto(412.26376063,193.27000204)(412.62659528,194.99913589)(412.62659528,195.05582881)
\curveto(412.62659528,195.33929337)(412.37714646,195.33929337)(412.29210709,195.33929337)
\curveto(412.00864252,195.33929337)(411.98029606,195.22023826)(411.8895874,194.82338786)
\curveto(411.4984063,193.24732487)(410.95982362,191.32543511)(409.77494173,191.32543511)
\curveto(409.17966614,191.32543511)(408.90754016,191.69393904)(408.90754016,192.62370282)
\curveto(408.90754016,193.24732487)(409.24202835,194.57393904)(409.46313071,195.56039574)
\lineto(410.26250079,198.61614377)
\curveto(410.34187087,199.04134062)(410.62533543,200.11283668)(410.73872126,200.53803353)
\curveto(410.88045354,201.18433274)(411.16391811,202.25582881)(411.16391811,202.42590755)
\curveto(411.16391811,202.93614377)(410.76706772,203.18559259)(410.34187087,203.18559259)
\curveto(410.20013858,203.18559259)(409.46313071,203.15724613)(409.24202835,202.19913589)
\curveto(408.70344567,200.13551385)(407.46187087,195.19756109)(407.12738268,193.69519889)
\curveto(407.08769764,193.58181306)(405.96517795,191.32543511)(403.90155591,191.32543511)
\curveto(402.42754016,191.32543511)(402.14407559,192.60102566)(402.14407559,193.64417526)
\curveto(402.14407559,195.22023826)(402.94344567,197.45393904)(403.68045354,199.40417526)
\curveto(404.01494173,200.25456897)(404.15667402,200.64008078)(404.15667402,201.18433274)
\curveto(404.15667402,202.44858471)(403.25525669,203.49740361)(401.83793386,203.49740361)
\curveto(399.15068976,203.49740361)(398.10754016,199.40417526)(398.10754016,199.14338786)
\curveto(398.10754016,198.8599233)(398.39100472,198.8599233)(398.45336693,198.8599233)
\curveto(398.7368315,198.8599233)(398.76517795,198.9222855)(398.90691024,199.37582881)
\curveto(399.60423307,201.83063196)(400.68706772,202.87378156)(401.75856378,202.87378156)
\curveto(402.00234331,202.87378156)(402.4672252,202.8511044)(402.4672252,201.94401778)
\curveto(402.4672252,201.26370282)(402.14407559,200.44732487)(401.97966614,200.02212802)
\curveto(400.93651654,197.2214981)(400.34124094,195.48102566)(400.34124094,194.09771857)
\curveto(400.34124094,191.41047448)(402.28580787,190.70181306)(403.82218583,190.70181306)
\curveto(405.68171339,190.70181306)(406.70218583,191.97173432)(407.1784063,192.60102566)
\closepath
}
}
{
\newrgbcolor{curcolor}{0 0 0}
\pscustom[linestyle=none,fillstyle=solid,fillcolor=curcolor]
{
\newpath
\moveto(421.6407685,198.0718918)
\lineto(421.6407685,198.28165558)
\lineto(421.6407685,198.50275794)
\lineto(421.63509921,198.7238603)
\lineto(421.62376063,198.95630125)
\lineto(421.60108346,199.43252172)
\lineto(421.56706772,199.93141936)
\lineto(421.52171339,200.45299416)
\lineto(421.45935118,200.99157684)
\lineto(421.38565039,201.54716739)
\lineto(421.29494173,202.11409652)
\lineto(421.18155591,202.69236424)
\lineto(421.0511622,203.27630125)
\lineto(420.89809134,203.87724613)
\lineto(420.7280126,204.47252172)
\lineto(420.5295874,205.07913589)
\lineto(420.30848504,205.68575007)
\lineto(420.18376063,205.98622251)
\lineto(420.05903622,206.28669495)
\lineto(419.92297323,206.58716739)
\lineto(419.78124094,206.88763983)
\curveto(418.09179213,210.36291542)(415.66533543,212.19409652)(415.38187087,212.19409652)
\curveto(415.21179213,212.19409652)(415.0984063,212.08637999)(415.0984063,211.92197054)
\curveto(415.0984063,211.83126188)(415.0984063,211.78023826)(415.62565039,211.26433274)
\curveto(418.39793386,208.47504141)(420.01368189,203.98496267)(420.01368189,198.0718918)
\curveto(420.01368189,193.24732487)(418.96486299,188.27535637)(415.46124094,184.71504141)
\curveto(415.0984063,184.38055322)(415.0984063,184.31819101)(415.0984063,184.23882093)
\curveto(415.0984063,184.06874219)(415.21179213,183.95535637)(415.38187087,183.95535637)
\curveto(415.66533543,183.95535637)(418.19383937,185.87724613)(419.87194961,189.46590755)
\curveto(421.30628031,192.56134062)(421.6407685,195.70212802)(421.6407685,198.0718918)
\closepath
}
}
{
\newrgbcolor{curcolor}{0 0 0}
\pscustom[linewidth=0,linecolor=curcolor]
{
\newpath
\moveto(421.6407685,198.0718918)
\lineto(421.6407685,198.28165558)
\lineto(421.6407685,198.50275794)
\lineto(421.63509921,198.7238603)
\lineto(421.62376063,198.95630125)
\lineto(421.60108346,199.43252172)
\lineto(421.56706772,199.93141936)
\lineto(421.52171339,200.45299416)
\lineto(421.45935118,200.99157684)
\lineto(421.38565039,201.54716739)
\lineto(421.29494173,202.11409652)
\lineto(421.18155591,202.69236424)
\lineto(421.0511622,203.27630125)
\lineto(420.89809134,203.87724613)
\lineto(420.7280126,204.47252172)
\lineto(420.5295874,205.07913589)
\lineto(420.30848504,205.68575007)
\lineto(420.18376063,205.98622251)
\lineto(420.05903622,206.28669495)
\lineto(419.92297323,206.58716739)
\lineto(419.78124094,206.88763983)
\curveto(418.09179213,210.36291542)(415.66533543,212.19409652)(415.38187087,212.19409652)
\curveto(415.21179213,212.19409652)(415.0984063,212.08637999)(415.0984063,211.92197054)
\curveto(415.0984063,211.83126188)(415.0984063,211.78023826)(415.62565039,211.26433274)
\curveto(418.39793386,208.47504141)(420.01368189,203.98496267)(420.01368189,198.0718918)
\curveto(420.01368189,193.24732487)(418.96486299,188.27535637)(415.46124094,184.71504141)
\curveto(415.0984063,184.38055322)(415.0984063,184.31819101)(415.0984063,184.23882093)
\curveto(415.0984063,184.06874219)(415.21179213,183.95535637)(415.38187087,183.95535637)
\curveto(415.66533543,183.95535637)(418.19383937,185.87724613)(419.87194961,189.46590755)
\curveto(421.30628031,192.56134062)(421.6407685,195.70212802)(421.6407685,198.0718918)
\closepath
}
}
{
\newrgbcolor{curcolor}{0 0 0}
\pscustom[linestyle=none,fillstyle=solid,fillcolor=curcolor]
{
\newpath
\moveto(451.73336693,200.25456897)
\lineto(451.81273701,200.25456897)
\lineto(451.89210709,200.25456897)
\lineto(451.97714646,200.25456897)
\lineto(452.0168315,200.26023826)
\lineto(452.05651654,200.26023826)
\lineto(452.10187087,200.26590755)
\lineto(452.14155591,200.27157684)
\lineto(452.18124094,200.27724613)
\lineto(452.22092598,200.28291542)
\lineto(452.26061102,200.28858471)
\lineto(452.29462677,200.2999233)
\lineto(452.33431181,200.31126188)
\lineto(452.36832756,200.32260046)
\lineto(452.40234331,200.33393904)
\lineto(452.43635906,200.35094692)
\lineto(452.46470551,200.36795479)
\lineto(452.49872126,200.39063196)
\lineto(452.52706772,200.41330912)
\lineto(452.54974488,200.43598629)
\lineto(452.57809134,200.45866345)
\lineto(452.6007685,200.48700991)
\lineto(452.61210709,200.50401778)
\lineto(452.61777638,200.52102566)
\lineto(452.62911496,200.53803353)
\lineto(452.64045354,200.55504141)
\lineto(452.64612283,200.57204928)
\lineto(452.65179213,200.58905715)
\lineto(452.65746142,200.60606503)
\lineto(452.6688,200.62874219)
\lineto(452.67446929,200.64575007)
\lineto(452.67446929,200.66842723)
\lineto(452.68013858,200.6911044)
\lineto(452.68580787,200.71378156)
\lineto(452.68580787,200.73645873)
\lineto(452.69147717,200.75913589)
\lineto(452.69147717,200.78748235)
\lineto(452.69147717,200.81015952)
\curveto(452.69147717,201.37708865)(452.1472252,201.37708865)(451.76171339,201.37708865)
\lineto(434.83887874,201.37708865)
\curveto(434.44202835,201.37708865)(433.90911496,201.37708865)(433.90911496,200.81015952)
\curveto(433.90911496,200.25456897)(434.44202835,200.25456897)(434.8672252,200.25456897)
\closepath
}
}
{
\newrgbcolor{curcolor}{0 0 0}
\pscustom[linewidth=0,linecolor=curcolor]
{
\newpath
\moveto(451.73336693,200.25456897)
\lineto(451.81273701,200.25456897)
\lineto(451.89210709,200.25456897)
\lineto(451.97714646,200.25456897)
\lineto(452.0168315,200.26023826)
\lineto(452.05651654,200.26023826)
\lineto(452.10187087,200.26590755)
\lineto(452.14155591,200.27157684)
\lineto(452.18124094,200.27724613)
\lineto(452.22092598,200.28291542)
\lineto(452.26061102,200.28858471)
\lineto(452.29462677,200.2999233)
\lineto(452.33431181,200.31126188)
\lineto(452.36832756,200.32260046)
\lineto(452.40234331,200.33393904)
\lineto(452.43635906,200.35094692)
\lineto(452.46470551,200.36795479)
\lineto(452.49872126,200.39063196)
\lineto(452.52706772,200.41330912)
\lineto(452.54974488,200.43598629)
\lineto(452.57809134,200.45866345)
\lineto(452.6007685,200.48700991)
\lineto(452.61210709,200.50401778)
\lineto(452.61777638,200.52102566)
\lineto(452.62911496,200.53803353)
\lineto(452.64045354,200.55504141)
\lineto(452.64612283,200.57204928)
\lineto(452.65179213,200.58905715)
\lineto(452.65746142,200.60606503)
\lineto(452.6688,200.62874219)
\lineto(452.67446929,200.64575007)
\lineto(452.67446929,200.66842723)
\lineto(452.68013858,200.6911044)
\lineto(452.68580787,200.71378156)
\lineto(452.68580787,200.73645873)
\lineto(452.69147717,200.75913589)
\lineto(452.69147717,200.78748235)
\lineto(452.69147717,200.81015952)
\curveto(452.69147717,201.37708865)(452.1472252,201.37708865)(451.76171339,201.37708865)
\lineto(434.83887874,201.37708865)
\curveto(434.44202835,201.37708865)(433.90911496,201.37708865)(433.90911496,200.81015952)
\curveto(433.90911496,200.25456897)(434.44202835,200.25456897)(434.8672252,200.25456897)
\closepath
}
}
{
\newrgbcolor{curcolor}{0 0 0}
\pscustom[linestyle=none,fillstyle=solid,fillcolor=curcolor]
{
\newpath
\moveto(451.76171339,194.77236424)
\lineto(451.79572913,194.77236424)
\lineto(451.83541417,194.77236424)
\lineto(451.91478425,194.77236424)
\lineto(451.98848504,194.77236424)
\lineto(452.02817008,194.77803353)
\lineto(452.06785512,194.77803353)
\lineto(452.10754016,194.78370282)
\lineto(452.1472252,194.78937211)
\lineto(452.18691024,194.79504141)
\lineto(452.22659528,194.8007107)
\lineto(452.26628031,194.80637999)
\lineto(452.30029606,194.81771857)
\lineto(452.33431181,194.82905715)
\lineto(452.37399685,194.84039574)
\lineto(452.4080126,194.85173432)
\lineto(452.43635906,194.86874219)
\lineto(452.4703748,194.88575007)
\lineto(452.49872126,194.90842723)
\lineto(452.52706772,194.9311044)
\lineto(452.55541417,194.95378156)
\lineto(452.57809134,194.98212802)
\lineto(452.58942992,194.9934666)
\lineto(452.6007685,195.01047448)
\lineto(452.61210709,195.02181306)
\lineto(452.61777638,195.03882093)
\lineto(452.62911496,195.05582881)
\lineto(452.64045354,195.07283668)
\lineto(452.64612283,195.08984456)
\lineto(452.65179213,195.11252172)
\lineto(452.65746142,195.1295296)
\lineto(452.6688,195.14653747)
\lineto(452.67446929,195.16921463)
\lineto(452.67446929,195.1918918)
\lineto(452.68013858,195.21456897)
\lineto(452.68580787,195.23724613)
\lineto(452.68580787,195.2599233)
\lineto(452.69147717,195.28260046)
\lineto(452.69147717,195.31094692)
\lineto(452.69147717,195.33929337)
\curveto(452.69147717,195.90622251)(452.1472252,195.90622251)(451.73336693,195.90622251)
\lineto(434.8672252,195.90622251)
\curveto(434.44202835,195.90622251)(433.90911496,195.90622251)(433.90911496,195.33929337)
\curveto(433.90911496,194.77236424)(434.44202835,194.77236424)(434.83887874,194.77236424)
\closepath
}
}
{
\newrgbcolor{curcolor}{0 0 0}
\pscustom[linewidth=0,linecolor=curcolor]
{
\newpath
\moveto(451.76171339,194.77236424)
\lineto(451.79572913,194.77236424)
\lineto(451.83541417,194.77236424)
\lineto(451.91478425,194.77236424)
\lineto(451.98848504,194.77236424)
\lineto(452.02817008,194.77803353)
\lineto(452.06785512,194.77803353)
\lineto(452.10754016,194.78370282)
\lineto(452.1472252,194.78937211)
\lineto(452.18691024,194.79504141)
\lineto(452.22659528,194.8007107)
\lineto(452.26628031,194.80637999)
\lineto(452.30029606,194.81771857)
\lineto(452.33431181,194.82905715)
\lineto(452.37399685,194.84039574)
\lineto(452.4080126,194.85173432)
\lineto(452.43635906,194.86874219)
\lineto(452.4703748,194.88575007)
\lineto(452.49872126,194.90842723)
\lineto(452.52706772,194.9311044)
\lineto(452.55541417,194.95378156)
\lineto(452.57809134,194.98212802)
\lineto(452.58942992,194.9934666)
\lineto(452.6007685,195.01047448)
\lineto(452.61210709,195.02181306)
\lineto(452.61777638,195.03882093)
\lineto(452.62911496,195.05582881)
\lineto(452.64045354,195.07283668)
\lineto(452.64612283,195.08984456)
\lineto(452.65179213,195.11252172)
\lineto(452.65746142,195.1295296)
\lineto(452.6688,195.14653747)
\lineto(452.67446929,195.16921463)
\lineto(452.67446929,195.1918918)
\lineto(452.68013858,195.21456897)
\lineto(452.68580787,195.23724613)
\lineto(452.68580787,195.2599233)
\lineto(452.69147717,195.28260046)
\lineto(452.69147717,195.31094692)
\lineto(452.69147717,195.33929337)
\curveto(452.69147717,195.90622251)(452.1472252,195.90622251)(451.73336693,195.90622251)
\lineto(434.8672252,195.90622251)
\curveto(434.44202835,195.90622251)(433.90911496,195.90622251)(433.90911496,195.33929337)
\curveto(433.90911496,194.77236424)(434.44202835,194.77236424)(434.83887874,194.77236424)
\closepath
}
}
{
\newrgbcolor{curcolor}{0 0 0}
\pscustom[linestyle=none,fillstyle=solid,fillcolor=curcolor]
{
\newpath
\moveto(475.1191937,200.05047448)
\lineto(475.11352441,200.47567133)
\lineto(475.10785512,200.89519889)
\lineto(475.09651654,201.32039574)
\lineto(475.07950866,201.7399233)
\lineto(475.0511622,202.16512015)
\lineto(475.01714646,202.58464771)
\lineto(474.97746142,203.00417526)
\lineto(474.9207685,203.41803353)
\lineto(474.8584063,203.8318918)
\lineto(474.77903622,204.24575007)
\lineto(474.68832756,204.65960834)
\lineto(474.58061102,205.06212802)
\lineto(474.45588661,205.470317)
\lineto(474.31415433,205.86716739)
\lineto(474.16108346,206.26401778)
\lineto(474.07604409,206.46244298)
\lineto(473.98533543,206.65519889)
\curveto(472.68706772,209.37645873)(470.36832756,209.83000204)(469.18344567,209.83000204)
\curveto(467.48832756,209.83000204)(465.42470551,209.09299416)(464.26817008,206.46244298)
\curveto(463.36108346,204.51220676)(463.21935118,202.30685243)(463.21935118,200.05047448)
\curveto(463.21935118,197.93015952)(463.33273701,195.390317)(464.49494173,193.24732487)
\curveto(465.70817008,190.96260046)(467.77179213,190.39567133)(469.15509921,190.39567133)
\lineto(469.15509921,191.01362408)
\curveto(468.05525669,191.01362408)(466.38281575,191.7222855)(465.87824882,194.43220676)
\curveto(465.5664378,196.12732487)(465.5664378,198.71819101)(465.5664378,200.39630125)
\curveto(465.5664378,202.19913589)(465.5664378,204.06433274)(465.79887874,205.58370282)
\curveto(466.33179213,208.95126188)(468.4464378,209.2007107)(469.15509921,209.2007107)
\curveto(470.08486299,209.2007107)(471.95572913,208.69614377)(472.48864252,205.89551385)
\curveto(472.77210709,204.31945086)(472.77210709,202.16512015)(472.77210709,200.39630125)
\curveto(472.77210709,198.270317)(472.77210709,196.35409652)(472.46029606,194.54559259)
\curveto(472.03509921,191.86401778)(470.41935118,191.01362408)(469.15509921,191.01362408)
\lineto(469.15509921,191.01362408)
\lineto(469.15509921,190.39567133)
\curveto(470.68013858,190.39567133)(472.82313071,190.98527763)(474.0703748,193.67252172)
\curveto(474.97746142,195.62275794)(475.1191937,197.81677369)(475.1191937,200.05047448)
\closepath
}
}
{
\newrgbcolor{curcolor}{0 0 0}
\pscustom[linewidth=0,linecolor=curcolor]
{
\newpath
\moveto(475.1191937,200.05047448)
\lineto(475.11352441,200.47567133)
\lineto(475.10785512,200.89519889)
\lineto(475.09651654,201.32039574)
\lineto(475.07950866,201.7399233)
\lineto(475.0511622,202.16512015)
\lineto(475.01714646,202.58464771)
\lineto(474.97746142,203.00417526)
\lineto(474.9207685,203.41803353)
\lineto(474.8584063,203.8318918)
\lineto(474.77903622,204.24575007)
\lineto(474.68832756,204.65960834)
\lineto(474.58061102,205.06212802)
\lineto(474.45588661,205.470317)
\lineto(474.31415433,205.86716739)
\lineto(474.16108346,206.26401778)
\lineto(474.07604409,206.46244298)
\lineto(473.98533543,206.65519889)
\curveto(472.68706772,209.37645873)(470.36832756,209.83000204)(469.18344567,209.83000204)
\curveto(467.48832756,209.83000204)(465.42470551,209.09299416)(464.26817008,206.46244298)
\curveto(463.36108346,204.51220676)(463.21935118,202.30685243)(463.21935118,200.05047448)
\curveto(463.21935118,197.93015952)(463.33273701,195.390317)(464.49494173,193.24732487)
\curveto(465.70817008,190.96260046)(467.77179213,190.39567133)(469.15509921,190.39567133)
\lineto(469.15509921,191.01362408)
\curveto(468.05525669,191.01362408)(466.38281575,191.7222855)(465.87824882,194.43220676)
\curveto(465.5664378,196.12732487)(465.5664378,198.71819101)(465.5664378,200.39630125)
\curveto(465.5664378,202.19913589)(465.5664378,204.06433274)(465.79887874,205.58370282)
\curveto(466.33179213,208.95126188)(468.4464378,209.2007107)(469.15509921,209.2007107)
\curveto(470.08486299,209.2007107)(471.95572913,208.69614377)(472.48864252,205.89551385)
\curveto(472.77210709,204.31945086)(472.77210709,202.16512015)(472.77210709,200.39630125)
\curveto(472.77210709,198.270317)(472.77210709,196.35409652)(472.46029606,194.54559259)
\curveto(472.03509921,191.86401778)(470.41935118,191.01362408)(469.15509921,191.01362408)
\lineto(469.15509921,191.01362408)
\lineto(469.15509921,190.39567133)
\curveto(470.68013858,190.39567133)(472.82313071,190.98527763)(474.0703748,193.67252172)
\curveto(474.97746142,195.62275794)(475.1191937,197.81677369)(475.1191937,200.05047448)
\closepath
}
}
{
\newrgbcolor{curcolor}{0 0 0}
\pscustom[linestyle=none,fillstyle=solid,fillcolor=curcolor]
{
\newpath
\moveto(482.00171339,191.04197054)
\lineto(482.00171339,191.21204928)
\lineto(481.99604409,191.38212802)
\lineto(481.98470551,191.54653747)
\lineto(481.96769764,191.70527763)
\lineto(481.95068976,191.85834849)
\lineto(481.9280126,192.00575007)
\lineto(481.90533543,192.15315164)
\lineto(481.87698898,192.29488393)
\lineto(481.84297323,192.42527763)
\lineto(481.80895748,192.55567133)
\lineto(481.76360315,192.68039574)
\lineto(481.72391811,192.79945086)
\lineto(481.67289449,192.91283668)
\lineto(481.62754016,193.02622251)
\lineto(481.57084724,193.12826975)
\lineto(481.51415433,193.22464771)
\lineto(481.45179213,193.31535637)
\lineto(481.38942992,193.40606503)
\lineto(481.32139843,193.48543511)
\lineto(481.25336693,193.55913589)
\lineto(481.17966614,193.63283668)
\lineto(481.10596535,193.69519889)
\lineto(481.02659528,193.7518918)
\lineto(480.9472252,193.80858471)
\lineto(480.86218583,193.85393904)
\lineto(480.77147717,193.89362408)
\lineto(480.6807685,193.92763983)
\lineto(480.59005984,193.95598629)
\lineto(480.49368189,193.97866345)
\lineto(480.39730394,193.99567133)
\lineto(480.29525669,194.00700991)
\lineto(480.19320945,194.00700991)
\curveto(479.26344567,194.00700991)(478.69651654,193.29834849)(478.69651654,192.510317)
\curveto(478.69651654,191.75063196)(479.26344567,191.01362408)(480.19320945,191.01362408)
\curveto(480.53336693,191.01362408)(480.90187087,191.12700991)(481.18533543,191.37645873)
\curveto(481.26470551,191.43882093)(481.30439055,191.46716739)(481.32706772,191.46716739)
\curveto(481.35541417,191.46716739)(481.38376063,191.43882093)(481.38376063,191.04197054)
\curveto(481.38376063,188.95000204)(480.39730394,187.26055322)(479.46187087,186.32512015)
\curveto(479.15005984,186.01897841)(479.15005984,185.95661621)(479.15005984,185.87724613)
\curveto(479.15005984,185.67315164)(479.29179213,185.56543511)(479.43352441,185.56543511)
\curveto(479.74533543,185.56543511)(482.00171339,187.73677369)(482.00171339,191.04197054)
\closepath
}
}
{
\newrgbcolor{curcolor}{0 0 0}
\pscustom[linewidth=0,linecolor=curcolor]
{
\newpath
\moveto(482.00171339,191.04197054)
\lineto(482.00171339,191.21204928)
\lineto(481.99604409,191.38212802)
\lineto(481.98470551,191.54653747)
\lineto(481.96769764,191.70527763)
\lineto(481.95068976,191.85834849)
\lineto(481.9280126,192.00575007)
\lineto(481.90533543,192.15315164)
\lineto(481.87698898,192.29488393)
\lineto(481.84297323,192.42527763)
\lineto(481.80895748,192.55567133)
\lineto(481.76360315,192.68039574)
\lineto(481.72391811,192.79945086)
\lineto(481.67289449,192.91283668)
\lineto(481.62754016,193.02622251)
\lineto(481.57084724,193.12826975)
\lineto(481.51415433,193.22464771)
\lineto(481.45179213,193.31535637)
\lineto(481.38942992,193.40606503)
\lineto(481.32139843,193.48543511)
\lineto(481.25336693,193.55913589)
\lineto(481.17966614,193.63283668)
\lineto(481.10596535,193.69519889)
\lineto(481.02659528,193.7518918)
\lineto(480.9472252,193.80858471)
\lineto(480.86218583,193.85393904)
\lineto(480.77147717,193.89362408)
\lineto(480.6807685,193.92763983)
\lineto(480.59005984,193.95598629)
\lineto(480.49368189,193.97866345)
\lineto(480.39730394,193.99567133)
\lineto(480.29525669,194.00700991)
\lineto(480.19320945,194.00700991)
\curveto(479.26344567,194.00700991)(478.69651654,193.29834849)(478.69651654,192.510317)
\curveto(478.69651654,191.75063196)(479.26344567,191.01362408)(480.19320945,191.01362408)
\curveto(480.53336693,191.01362408)(480.90187087,191.12700991)(481.18533543,191.37645873)
\curveto(481.26470551,191.43882093)(481.30439055,191.46716739)(481.32706772,191.46716739)
\curveto(481.35541417,191.46716739)(481.38376063,191.43882093)(481.38376063,191.04197054)
\curveto(481.38376063,188.95000204)(480.39730394,187.26055322)(479.46187087,186.32512015)
\curveto(479.15005984,186.01897841)(479.15005984,185.95661621)(479.15005984,185.87724613)
\curveto(479.15005984,185.67315164)(479.29179213,185.56543511)(479.43352441,185.56543511)
\curveto(479.74533543,185.56543511)(482.00171339,187.73677369)(482.00171339,191.04197054)
\closepath
}
}
{
\newrgbcolor{curcolor}{0 0 0}
\pscustom[linestyle=none,fillstyle=solid,fillcolor=curcolor]
{
\newpath
\moveto(536.30218583,219.67756109)
\lineto(535.64454803,219.46779731)
\lineto(534.95856378,219.24669495)
\lineto(534.24990236,219.02559259)
\lineto(533.51289449,218.80449023)
\lineto(532.75320945,218.58338786)
\lineto(531.97084724,218.3622855)
\lineto(531.17714646,218.15252172)
\lineto(530.3664378,217.95409652)
\lineto(529.53872126,217.76700991)
\lineto(528.69966614,217.59693117)
\lineto(527.85494173,217.4438603)
\lineto(526.99887874,217.30779731)
\lineto(526.14281575,217.20008078)
\lineto(525.28108346,217.1207107)
\lineto(524.41935118,217.06968708)
\lineto(523.56328819,217.0526792)
\curveto(520.59257953,217.0526792)(518.27383937,217.81803353)(518.08108346,217.86905715)
\lineto(515.22942992,218.81015952)
\curveto(514.32234331,219.11630125)(512.15667402,219.67756109)(509.57714646,219.67756109)
\curveto(505.93746142,219.67756109)(501.67415433,218.60606503)(498.6184063,217.61960834)
\lineto(495.7384063,216.68417526)
\lineto(495.9311622,216.08889967)
\curveto(499.35541417,217.19441148)(504.10061102,218.71945086)(508.67005984,218.71945086)
\curveto(511.6407685,218.71945086)(513.95950866,217.95976582)(514.15793386,217.9030729)
\lineto(517.0095874,216.96763983)
\curveto(517.91100472,216.65582881)(520.09368189,216.08889967)(522.66187087,216.08889967)
\curveto(526.30155591,216.08889967)(530.57053228,217.17173432)(533.61494173,218.15252172)
\lineto(536.49494173,219.0822855)
\closepath
}
}
{
\newrgbcolor{curcolor}{0 0 0}
\pscustom[linewidth=0,linecolor=curcolor]
{
\newpath
\moveto(536.30218583,219.67756109)
\lineto(535.64454803,219.46779731)
\lineto(534.95856378,219.24669495)
\lineto(534.24990236,219.02559259)
\lineto(533.51289449,218.80449023)
\lineto(532.75320945,218.58338786)
\lineto(531.97084724,218.3622855)
\lineto(531.17714646,218.15252172)
\lineto(530.3664378,217.95409652)
\lineto(529.53872126,217.76700991)
\lineto(528.69966614,217.59693117)
\lineto(527.85494173,217.4438603)
\lineto(526.99887874,217.30779731)
\lineto(526.14281575,217.20008078)
\lineto(525.28108346,217.1207107)
\lineto(524.41935118,217.06968708)
\lineto(523.56328819,217.0526792)
\curveto(520.59257953,217.0526792)(518.27383937,217.81803353)(518.08108346,217.86905715)
\lineto(515.22942992,218.81015952)
\curveto(514.32234331,219.11630125)(512.15667402,219.67756109)(509.57714646,219.67756109)
\curveto(505.93746142,219.67756109)(501.67415433,218.60606503)(498.6184063,217.61960834)
\lineto(495.7384063,216.68417526)
\lineto(495.9311622,216.08889967)
\curveto(499.35541417,217.19441148)(504.10061102,218.71945086)(508.67005984,218.71945086)
\curveto(511.6407685,218.71945086)(513.95950866,217.95976582)(514.15793386,217.9030729)
\lineto(517.0095874,216.96763983)
\curveto(517.91100472,216.65582881)(520.09368189,216.08889967)(522.66187087,216.08889967)
\curveto(526.30155591,216.08889967)(530.57053228,217.17173432)(533.61494173,218.15252172)
\lineto(536.49494173,219.0822855)
\closepath
}
}
{
\newrgbcolor{curcolor}{0 0 0}
\pscustom[linestyle=none,fillstyle=solid,fillcolor=curcolor]
{
\newpath
\moveto(503.36927244,210.30622251)
\lineto(503.36927244,210.30622251)
\lineto(503.36927244,210.3118918)
\lineto(503.36927244,210.31756109)
\lineto(503.36927244,210.32323038)
\lineto(503.36927244,210.32889967)
\lineto(503.36927244,210.33456897)
\lineto(503.36927244,210.34590755)
\lineto(503.36927244,210.35724613)
\lineto(503.36360315,210.36858471)
\lineto(503.36360315,210.3799233)
\lineto(503.35793386,210.39126188)
\lineto(503.35793386,210.40260046)
\lineto(503.35226457,210.41960834)
\lineto(503.34659528,210.43094692)
\lineto(503.34092598,210.4422855)
\lineto(503.33525669,210.45929337)
\lineto(503.32391811,210.47063196)
\lineto(503.31824882,210.48763983)
\lineto(503.30691024,210.49897841)
\lineto(503.29557165,210.51598629)
\lineto(503.28423307,210.52732487)
\lineto(503.27289449,210.53299416)
\lineto(503.2672252,210.53866345)
\lineto(503.26155591,210.54433274)
\lineto(503.25021732,210.55000204)
\lineto(503.24454803,210.55567133)
\lineto(503.23320945,210.56134062)
\lineto(503.22754016,210.56700991)
\lineto(503.21620157,210.5726792)
\lineto(503.20486299,210.57834849)
\lineto(503.1991937,210.58401778)
\lineto(503.18785512,210.58968708)
\lineto(503.17651654,210.58968708)
\lineto(503.16517795,210.59535637)
\lineto(503.15383937,210.60102566)
\lineto(503.14250079,210.60102566)
\lineto(503.12549291,210.60669495)
\lineto(503.11415433,210.60669495)
\lineto(503.10281575,210.61236424)
\lineto(503.08580787,210.61236424)
\lineto(503.0688,210.61236424)
\lineto(503.05746142,210.61803353)
\lineto(503.04045354,210.61803353)
\lineto(503.02344567,210.61803353)
\lineto(503.0064378,210.61803353)
\curveto(502.58124094,210.61803353)(499.89966614,210.36291542)(499.41777638,210.30622251)
\curveto(499.19100472,210.27787605)(499.02092598,210.13614377)(499.02092598,209.76763983)
\curveto(499.02092598,209.42748235)(499.27604409,209.42748235)(499.70124094,209.42748235)
\curveto(501.05620157,209.42748235)(501.11289449,209.23472645)(501.11289449,208.95126188)
\lineto(501.03352441,208.38433274)
\lineto(499.33273701,201.68889967)
\curveto(498.82817008,202.73204928)(498.00045354,203.49740361)(496.73620157,203.49740361)
\lineto(496.76454803,202.87378156)
\curveto(498.59572913,202.87378156)(498.99257953,200.5607107)(498.99257953,200.39630125)
\curveto(498.99257953,200.21488393)(498.94155591,200.05047448)(498.91320945,199.90874219)
\lineto(497.49588661,194.36984456)
\curveto(497.35415433,193.86527763)(497.35415433,193.814254)(496.92895748,193.32669495)
\curveto(495.68738268,191.77330912)(494.53084724,191.32543511)(493.74281575,191.32543511)
\curveto(492.3311622,191.32543511)(491.93431181,192.87315164)(491.93431181,193.97866345)
\curveto(491.93431181,195.390317)(492.83572913,198.8599233)(493.48769764,200.1638603)
\curveto(494.3607685,201.83063196)(495.63635906,202.87378156)(496.76454803,202.87378156)
\lineto(496.73620157,203.49740361)
\curveto(493.43100472,203.49740361)(489.93305197,199.34748235)(489.93305197,195.22023826)
\curveto(489.93305197,192.56134062)(491.4807685,190.70181306)(493.68612283,190.70181306)
\curveto(494.24738268,190.70181306)(495.66470551,190.82086818)(497.35415433,192.82212802)
\curveto(497.58659528,191.63157684)(498.56738268,190.70181306)(499.92234331,190.70181306)
\curveto(500.91446929,190.70181306)(501.5664378,191.34811227)(502.01431181,192.25519889)
\curveto(502.49053228,193.27000204)(502.86470551,194.99913589)(502.86470551,195.05582881)
\curveto(502.86470551,195.33929337)(502.6095874,195.33929337)(502.53021732,195.33929337)
\curveto(502.24675276,195.33929337)(502.2184063,195.22023826)(502.13336693,194.82338786)
\curveto(501.65147717,192.98653747)(501.14124094,191.32543511)(499.98470551,191.32543511)
\curveto(499.22502047,191.32543511)(499.13431181,192.05677369)(499.13431181,192.62370282)
\curveto(499.13431181,193.29834849)(499.19100472,193.50244298)(499.30439055,193.97866345)
\closepath
}
}
{
\newrgbcolor{curcolor}{0 0 0}
\pscustom[linewidth=0,linecolor=curcolor]
{
\newpath
\moveto(503.36927244,210.30622251)
\lineto(503.36927244,210.30622251)
\lineto(503.36927244,210.3118918)
\lineto(503.36927244,210.31756109)
\lineto(503.36927244,210.32323038)
\lineto(503.36927244,210.32889967)
\lineto(503.36927244,210.33456897)
\lineto(503.36927244,210.34590755)
\lineto(503.36927244,210.35724613)
\lineto(503.36360315,210.36858471)
\lineto(503.36360315,210.3799233)
\lineto(503.35793386,210.39126188)
\lineto(503.35793386,210.40260046)
\lineto(503.35226457,210.41960834)
\lineto(503.34659528,210.43094692)
\lineto(503.34092598,210.4422855)
\lineto(503.33525669,210.45929337)
\lineto(503.32391811,210.47063196)
\lineto(503.31824882,210.48763983)
\lineto(503.30691024,210.49897841)
\lineto(503.29557165,210.51598629)
\lineto(503.28423307,210.52732487)
\lineto(503.27289449,210.53299416)
\lineto(503.2672252,210.53866345)
\lineto(503.26155591,210.54433274)
\lineto(503.25021732,210.55000204)
\lineto(503.24454803,210.55567133)
\lineto(503.23320945,210.56134062)
\lineto(503.22754016,210.56700991)
\lineto(503.21620157,210.5726792)
\lineto(503.20486299,210.57834849)
\lineto(503.1991937,210.58401778)
\lineto(503.18785512,210.58968708)
\lineto(503.17651654,210.58968708)
\lineto(503.16517795,210.59535637)
\lineto(503.15383937,210.60102566)
\lineto(503.14250079,210.60102566)
\lineto(503.12549291,210.60669495)
\lineto(503.11415433,210.60669495)
\lineto(503.10281575,210.61236424)
\lineto(503.08580787,210.61236424)
\lineto(503.0688,210.61236424)
\lineto(503.05746142,210.61803353)
\lineto(503.04045354,210.61803353)
\lineto(503.02344567,210.61803353)
\lineto(503.0064378,210.61803353)
\curveto(502.58124094,210.61803353)(499.89966614,210.36291542)(499.41777638,210.30622251)
\curveto(499.19100472,210.27787605)(499.02092598,210.13614377)(499.02092598,209.76763983)
\curveto(499.02092598,209.42748235)(499.27604409,209.42748235)(499.70124094,209.42748235)
\curveto(501.05620157,209.42748235)(501.11289449,209.23472645)(501.11289449,208.95126188)
\lineto(501.03352441,208.38433274)
\lineto(499.33273701,201.68889967)
\curveto(498.82817008,202.73204928)(498.00045354,203.49740361)(496.73620157,203.49740361)
\lineto(496.76454803,202.87378156)
\curveto(498.59572913,202.87378156)(498.99257953,200.5607107)(498.99257953,200.39630125)
\curveto(498.99257953,200.21488393)(498.94155591,200.05047448)(498.91320945,199.90874219)
\lineto(497.49588661,194.36984456)
\curveto(497.35415433,193.86527763)(497.35415433,193.814254)(496.92895748,193.32669495)
\curveto(495.68738268,191.77330912)(494.53084724,191.32543511)(493.74281575,191.32543511)
\curveto(492.3311622,191.32543511)(491.93431181,192.87315164)(491.93431181,193.97866345)
\curveto(491.93431181,195.390317)(492.83572913,198.8599233)(493.48769764,200.1638603)
\curveto(494.3607685,201.83063196)(495.63635906,202.87378156)(496.76454803,202.87378156)
\lineto(496.73620157,203.49740361)
\curveto(493.43100472,203.49740361)(489.93305197,199.34748235)(489.93305197,195.22023826)
\curveto(489.93305197,192.56134062)(491.4807685,190.70181306)(493.68612283,190.70181306)
\curveto(494.24738268,190.70181306)(495.66470551,190.82086818)(497.35415433,192.82212802)
\curveto(497.58659528,191.63157684)(498.56738268,190.70181306)(499.92234331,190.70181306)
\curveto(500.91446929,190.70181306)(501.5664378,191.34811227)(502.01431181,192.25519889)
\curveto(502.49053228,193.27000204)(502.86470551,194.99913589)(502.86470551,195.05582881)
\curveto(502.86470551,195.33929337)(502.6095874,195.33929337)(502.53021732,195.33929337)
\curveto(502.24675276,195.33929337)(502.2184063,195.22023826)(502.13336693,194.82338786)
\curveto(501.65147717,192.98653747)(501.14124094,191.32543511)(499.98470551,191.32543511)
\curveto(499.22502047,191.32543511)(499.13431181,192.05677369)(499.13431181,192.62370282)
\curveto(499.13431181,193.29834849)(499.19100472,193.50244298)(499.30439055,193.97866345)
\closepath
}
}
{
\newrgbcolor{curcolor}{0 0 0}
\pscustom[linestyle=none,fillstyle=solid,fillcolor=curcolor]
{
\newpath
\moveto(514.02754016,201.68889967)
\lineto(513.97651654,201.78527763)
\lineto(513.92549291,201.88165558)
\lineto(513.87446929,201.97236424)
\lineto(513.81777638,202.06874219)
\lineto(513.76675276,202.15378156)
\lineto(513.70439055,202.24449023)
\lineto(513.64769764,202.3295296)
\lineto(513.58533543,202.40889967)
\lineto(513.52297323,202.49393904)
\lineto(513.45494173,202.56763983)
\lineto(513.38691024,202.64700991)
\lineto(513.31320945,202.7207107)
\lineto(513.24517795,202.78874219)
\lineto(513.17147717,202.85677369)
\lineto(513.09210709,202.91913589)
\lineto(513.01273701,202.9814981)
\lineto(512.93336693,203.0438603)
\lineto(512.84832756,203.10055322)
\lineto(512.76328819,203.15157684)
\lineto(512.67257953,203.19693117)
\lineto(512.58187087,203.2422855)
\lineto(512.4911622,203.28763983)
\lineto(512.39478425,203.32732487)
\lineto(512.2984063,203.36134062)
\lineto(512.19635906,203.38968708)
\lineto(512.09431181,203.41803353)
\lineto(511.99226457,203.4407107)
\lineto(511.88454803,203.45771857)
\lineto(511.7768315,203.47472645)
\lineto(511.66344567,203.48606503)
\lineto(511.54439055,203.49173432)
\lineto(511.43100472,203.49740361)
\lineto(511.45368189,202.87378156)
\curveto(513.29053228,202.87378156)(513.68738268,200.5607107)(513.68738268,200.39630125)
\curveto(513.68738268,200.21488393)(513.63635906,200.05047448)(513.6080126,199.90874219)
\lineto(512.19068976,194.36984456)
\curveto(512.04895748,193.86527763)(512.04895748,193.814254)(511.62376063,193.32669495)
\curveto(510.38218583,191.77330912)(509.22565039,191.32543511)(508.4376189,191.32543511)
\curveto(507.02596535,191.32543511)(506.62911496,192.87315164)(506.62911496,193.97866345)
\curveto(506.62911496,195.390317)(507.53053228,198.8599233)(508.1768315,200.1638603)
\curveto(509.05557165,201.83063196)(510.3311622,202.87378156)(511.45368189,202.87378156)
\lineto(511.43100472,203.49740361)
\curveto(508.12580787,203.49740361)(504.62785512,199.34748235)(504.62785512,195.22023826)
\curveto(504.62785512,192.56134062)(506.17557165,190.70181306)(508.38092598,190.70181306)
\curveto(508.94218583,190.70181306)(510.35950866,190.82086818)(512.04895748,192.82212802)
\curveto(512.28139843,191.63157684)(513.26218583,190.70181306)(514.61714646,190.70181306)
\curveto(515.60927244,190.70181306)(516.25557165,191.34811227)(516.70911496,192.25519889)
\curveto(517.18533543,193.27000204)(517.55950866,194.99913589)(517.55950866,195.05582881)
\curveto(517.55950866,195.33929337)(517.30439055,195.33929337)(517.22502047,195.33929337)
\curveto(516.94155591,195.33929337)(516.91320945,195.22023826)(516.82250079,194.82338786)
\curveto(516.34628031,192.98653747)(515.8303748,191.32543511)(514.67950866,191.32543511)
\curveto(513.91982362,191.32543511)(513.82911496,192.05677369)(513.82911496,192.62370282)
\curveto(513.82911496,193.24732487)(513.88580787,193.46842723)(514.19194961,194.71567133)
\curveto(514.50942992,195.90622251)(514.56612283,196.18968708)(514.82124094,197.26118314)
\lineto(515.8303748,201.20700991)
\curveto(516.03446929,202.0007107)(516.03446929,202.05740361)(516.03446929,202.16512015)
\curveto(516.03446929,202.6526792)(515.68864252,202.93614377)(515.21242205,202.93614377)
\curveto(514.53777638,202.93614377)(514.11257953,202.30685243)(514.02754016,201.68889967)
\closepath
}
}
{
\newrgbcolor{curcolor}{0 0 0}
\pscustom[linewidth=0,linecolor=curcolor]
{
\newpath
\moveto(514.02754016,201.68889967)
\lineto(513.97651654,201.78527763)
\lineto(513.92549291,201.88165558)
\lineto(513.87446929,201.97236424)
\lineto(513.81777638,202.06874219)
\lineto(513.76675276,202.15378156)
\lineto(513.70439055,202.24449023)
\lineto(513.64769764,202.3295296)
\lineto(513.58533543,202.40889967)
\lineto(513.52297323,202.49393904)
\lineto(513.45494173,202.56763983)
\lineto(513.38691024,202.64700991)
\lineto(513.31320945,202.7207107)
\lineto(513.24517795,202.78874219)
\lineto(513.17147717,202.85677369)
\lineto(513.09210709,202.91913589)
\lineto(513.01273701,202.9814981)
\lineto(512.93336693,203.0438603)
\lineto(512.84832756,203.10055322)
\lineto(512.76328819,203.15157684)
\lineto(512.67257953,203.19693117)
\lineto(512.58187087,203.2422855)
\lineto(512.4911622,203.28763983)
\lineto(512.39478425,203.32732487)
\lineto(512.2984063,203.36134062)
\lineto(512.19635906,203.38968708)
\lineto(512.09431181,203.41803353)
\lineto(511.99226457,203.4407107)
\lineto(511.88454803,203.45771857)
\lineto(511.7768315,203.47472645)
\lineto(511.66344567,203.48606503)
\lineto(511.54439055,203.49173432)
\lineto(511.43100472,203.49740361)
\lineto(511.45368189,202.87378156)
\curveto(513.29053228,202.87378156)(513.68738268,200.5607107)(513.68738268,200.39630125)
\curveto(513.68738268,200.21488393)(513.63635906,200.05047448)(513.6080126,199.90874219)
\lineto(512.19068976,194.36984456)
\curveto(512.04895748,193.86527763)(512.04895748,193.814254)(511.62376063,193.32669495)
\curveto(510.38218583,191.77330912)(509.22565039,191.32543511)(508.4376189,191.32543511)
\curveto(507.02596535,191.32543511)(506.62911496,192.87315164)(506.62911496,193.97866345)
\curveto(506.62911496,195.390317)(507.53053228,198.8599233)(508.1768315,200.1638603)
\curveto(509.05557165,201.83063196)(510.3311622,202.87378156)(511.45368189,202.87378156)
\lineto(511.43100472,203.49740361)
\curveto(508.12580787,203.49740361)(504.62785512,199.34748235)(504.62785512,195.22023826)
\curveto(504.62785512,192.56134062)(506.17557165,190.70181306)(508.38092598,190.70181306)
\curveto(508.94218583,190.70181306)(510.35950866,190.82086818)(512.04895748,192.82212802)
\curveto(512.28139843,191.63157684)(513.26218583,190.70181306)(514.61714646,190.70181306)
\curveto(515.60927244,190.70181306)(516.25557165,191.34811227)(516.70911496,192.25519889)
\curveto(517.18533543,193.27000204)(517.55950866,194.99913589)(517.55950866,195.05582881)
\curveto(517.55950866,195.33929337)(517.30439055,195.33929337)(517.22502047,195.33929337)
\curveto(516.94155591,195.33929337)(516.91320945,195.22023826)(516.82250079,194.82338786)
\curveto(516.34628031,192.98653747)(515.8303748,191.32543511)(514.67950866,191.32543511)
\curveto(513.91982362,191.32543511)(513.82911496,192.05677369)(513.82911496,192.62370282)
\curveto(513.82911496,193.24732487)(513.88580787,193.46842723)(514.19194961,194.71567133)
\curveto(514.50942992,195.90622251)(514.56612283,196.18968708)(514.82124094,197.26118314)
\lineto(515.8303748,201.20700991)
\curveto(516.03446929,202.0007107)(516.03446929,202.05740361)(516.03446929,202.16512015)
\curveto(516.03446929,202.6526792)(515.68864252,202.93614377)(515.21242205,202.93614377)
\curveto(514.53777638,202.93614377)(514.11257953,202.30685243)(514.02754016,201.68889967)
\closepath
}
}
{
\newrgbcolor{curcolor}{0 0 0}
\pscustom[linestyle=none,fillstyle=solid,fillcolor=curcolor]
{
\newpath
\moveto(524.2095874,202.30685243)
\lineto(526.86848504,202.30685243)
\curveto(527.42407559,202.30685243)(527.70754016,202.30685243)(527.70754016,202.87378156)
\curveto(527.70754016,203.18559259)(527.42407559,203.18559259)(526.91950866,203.18559259)
\lineto(524.43068976,203.18559259)
\curveto(525.4511622,207.19945086)(525.59289449,207.76637999)(525.59289449,207.93078944)
\curveto(525.59289449,208.4126792)(525.25273701,208.69614377)(524.77651654,208.69614377)
\curveto(524.68580787,208.69614377)(523.89777638,208.66779731)(523.64265827,207.67567133)
\lineto(522.54281575,203.18559259)
\lineto(519.88391811,203.18559259)
\curveto(519.31698898,203.18559259)(519.03352441,203.18559259)(519.03352441,202.6526792)
\curveto(519.03352441,202.30685243)(519.26596535,202.30685243)(519.83289449,202.30685243)
\lineto(522.31604409,202.30685243)
\curveto(520.2864378,194.29047448)(520.16738268,193.814254)(520.16738268,193.29834849)
\curveto(520.16738268,191.77330912)(521.24454803,190.70181306)(522.76391811,190.70181306)
\curveto(525.64391811,190.70181306)(527.25399685,194.82338786)(527.25399685,195.05582881)
\curveto(527.25399685,195.33929337)(527.03856378,195.33929337)(526.91950866,195.33929337)
\curveto(526.66439055,195.33929337)(526.63604409,195.24858471)(526.49431181,194.93677369)
\curveto(525.28108346,192.00575007)(523.78439055,191.32543511)(522.82628031,191.32543511)
\curveto(522.23667402,191.32543511)(521.95320945,191.69393904)(521.95320945,192.62370282)
\curveto(521.95320945,193.29834849)(522.00423307,193.50244298)(522.1176189,193.97866345)
\closepath
}
}
{
\newrgbcolor{curcolor}{0 0 0}
\pscustom[linewidth=0,linecolor=curcolor]
{
\newpath
\moveto(524.2095874,202.30685243)
\lineto(526.86848504,202.30685243)
\curveto(527.42407559,202.30685243)(527.70754016,202.30685243)(527.70754016,202.87378156)
\curveto(527.70754016,203.18559259)(527.42407559,203.18559259)(526.91950866,203.18559259)
\lineto(524.43068976,203.18559259)
\curveto(525.4511622,207.19945086)(525.59289449,207.76637999)(525.59289449,207.93078944)
\curveto(525.59289449,208.4126792)(525.25273701,208.69614377)(524.77651654,208.69614377)
\curveto(524.68580787,208.69614377)(523.89777638,208.66779731)(523.64265827,207.67567133)
\lineto(522.54281575,203.18559259)
\lineto(519.88391811,203.18559259)
\curveto(519.31698898,203.18559259)(519.03352441,203.18559259)(519.03352441,202.6526792)
\curveto(519.03352441,202.30685243)(519.26596535,202.30685243)(519.83289449,202.30685243)
\lineto(522.31604409,202.30685243)
\curveto(520.2864378,194.29047448)(520.16738268,193.814254)(520.16738268,193.29834849)
\curveto(520.16738268,191.77330912)(521.24454803,190.70181306)(522.76391811,190.70181306)
\curveto(525.64391811,190.70181306)(527.25399685,194.82338786)(527.25399685,195.05582881)
\curveto(527.25399685,195.33929337)(527.03856378,195.33929337)(526.91950866,195.33929337)
\curveto(526.66439055,195.33929337)(526.63604409,195.24858471)(526.49431181,194.93677369)
\curveto(525.28108346,192.00575007)(523.78439055,191.32543511)(522.82628031,191.32543511)
\curveto(522.23667402,191.32543511)(521.95320945,191.69393904)(521.95320945,192.62370282)
\curveto(521.95320945,193.29834849)(522.00423307,193.50244298)(522.1176189,193.97866345)
\closepath
}
}
{
\newrgbcolor{curcolor}{0 0 0}
\pscustom[linestyle=none,fillstyle=solid,fillcolor=curcolor]
{
\newpath
\moveto(539.12549291,201.68889967)
\lineto(539.08013858,201.78527763)
\lineto(539.02911496,201.88165558)
\lineto(538.97809134,201.97236424)
\lineto(538.92139843,202.06874219)
\lineto(538.86470551,202.15378156)
\lineto(538.8080126,202.24449023)
\lineto(538.75131969,202.3295296)
\lineto(538.68895748,202.40889967)
\lineto(538.62092598,202.49393904)
\lineto(538.55856378,202.56763983)
\lineto(538.49053228,202.64700991)
\lineto(538.4168315,202.7207107)
\lineto(538.34313071,202.78874219)
\lineto(538.26942992,202.85677369)
\lineto(538.19572913,202.91913589)
\lineto(538.11635906,202.9814981)
\lineto(538.03131969,203.0438603)
\lineto(537.95194961,203.10055322)
\lineto(537.86691024,203.15157684)
\lineto(537.77620157,203.19693117)
\lineto(537.68549291,203.2422855)
\lineto(537.59478425,203.28763983)
\lineto(537.4984063,203.32732487)
\lineto(537.40202835,203.36134062)
\lineto(537.2999811,203.38968708)
\lineto(537.19793386,203.41803353)
\lineto(537.09588661,203.4407107)
\lineto(536.98817008,203.45771857)
\lineto(536.87478425,203.47472645)
\lineto(536.76139843,203.48606503)
\lineto(536.6480126,203.49173432)
\lineto(536.53462677,203.49740361)
\lineto(536.55730394,202.87378156)
\curveto(538.39415433,202.87378156)(538.79100472,200.5607107)(538.79100472,200.39630125)
\curveto(538.79100472,200.21488393)(538.73431181,200.05047448)(538.71163465,199.90874219)
\lineto(537.29431181,194.36984456)
\curveto(537.15257953,193.86527763)(537.15257953,193.814254)(536.72738268,193.32669495)
\curveto(535.48580787,191.77330912)(534.32360315,191.32543511)(533.53557165,191.32543511)
\curveto(532.1295874,191.32543511)(531.73273701,192.87315164)(531.73273701,193.97866345)
\curveto(531.73273701,195.390317)(532.63415433,198.8599233)(533.28045354,200.1638603)
\curveto(534.1591937,201.83063196)(535.43478425,202.87378156)(536.55730394,202.87378156)
\lineto(536.53462677,203.49740361)
\curveto(533.22942992,203.49740361)(529.73147717,199.34748235)(529.73147717,195.22023826)
\curveto(529.73147717,192.56134062)(531.2791937,190.70181306)(533.48454803,190.70181306)
\curveto(534.04013858,190.70181306)(535.45746142,190.82086818)(537.15257953,192.82212802)
\curveto(537.38502047,191.63157684)(538.36580787,190.70181306)(539.7207685,190.70181306)
\curveto(540.71289449,190.70181306)(541.3591937,191.34811227)(541.81273701,192.25519889)
\curveto(542.28895748,193.27000204)(542.66313071,194.99913589)(542.66313071,195.05582881)
\curveto(542.66313071,195.33929337)(542.40234331,195.33929337)(542.32297323,195.33929337)
\curveto(542.03950866,195.33929337)(542.0168315,195.22023826)(541.92612283,194.82338786)
\curveto(541.44990236,192.98653747)(540.93399685,191.32543511)(539.78313071,191.32543511)
\curveto(539.01777638,191.32543511)(538.93273701,192.05677369)(538.93273701,192.62370282)
\curveto(538.93273701,193.24732487)(538.98376063,193.46842723)(539.29557165,194.71567133)
\curveto(539.61305197,195.90622251)(539.66974488,196.18968708)(539.92486299,197.26118314)
\lineto(540.93399685,201.20700991)
\curveto(541.13809134,202.0007107)(541.13809134,202.05740361)(541.13809134,202.16512015)
\curveto(541.13809134,202.6526792)(540.79226457,202.93614377)(540.31604409,202.93614377)
\curveto(539.64139843,202.93614377)(539.21620157,202.30685243)(539.12549291,201.68889967)
\closepath
}
}
{
\newrgbcolor{curcolor}{0 0 0}
\pscustom[linewidth=0,linecolor=curcolor]
{
\newpath
\moveto(539.12549291,201.68889967)
\lineto(539.08013858,201.78527763)
\lineto(539.02911496,201.88165558)
\lineto(538.97809134,201.97236424)
\lineto(538.92139843,202.06874219)
\lineto(538.86470551,202.15378156)
\lineto(538.8080126,202.24449023)
\lineto(538.75131969,202.3295296)
\lineto(538.68895748,202.40889967)
\lineto(538.62092598,202.49393904)
\lineto(538.55856378,202.56763983)
\lineto(538.49053228,202.64700991)
\lineto(538.4168315,202.7207107)
\lineto(538.34313071,202.78874219)
\lineto(538.26942992,202.85677369)
\lineto(538.19572913,202.91913589)
\lineto(538.11635906,202.9814981)
\lineto(538.03131969,203.0438603)
\lineto(537.95194961,203.10055322)
\lineto(537.86691024,203.15157684)
\lineto(537.77620157,203.19693117)
\lineto(537.68549291,203.2422855)
\lineto(537.59478425,203.28763983)
\lineto(537.4984063,203.32732487)
\lineto(537.40202835,203.36134062)
\lineto(537.2999811,203.38968708)
\lineto(537.19793386,203.41803353)
\lineto(537.09588661,203.4407107)
\lineto(536.98817008,203.45771857)
\lineto(536.87478425,203.47472645)
\lineto(536.76139843,203.48606503)
\lineto(536.6480126,203.49173432)
\lineto(536.53462677,203.49740361)
\lineto(536.55730394,202.87378156)
\curveto(538.39415433,202.87378156)(538.79100472,200.5607107)(538.79100472,200.39630125)
\curveto(538.79100472,200.21488393)(538.73431181,200.05047448)(538.71163465,199.90874219)
\lineto(537.29431181,194.36984456)
\curveto(537.15257953,193.86527763)(537.15257953,193.814254)(536.72738268,193.32669495)
\curveto(535.48580787,191.77330912)(534.32360315,191.32543511)(533.53557165,191.32543511)
\curveto(532.1295874,191.32543511)(531.73273701,192.87315164)(531.73273701,193.97866345)
\curveto(531.73273701,195.390317)(532.63415433,198.8599233)(533.28045354,200.1638603)
\curveto(534.1591937,201.83063196)(535.43478425,202.87378156)(536.55730394,202.87378156)
\lineto(536.53462677,203.49740361)
\curveto(533.22942992,203.49740361)(529.73147717,199.34748235)(529.73147717,195.22023826)
\curveto(529.73147717,192.56134062)(531.2791937,190.70181306)(533.48454803,190.70181306)
\curveto(534.04013858,190.70181306)(535.45746142,190.82086818)(537.15257953,192.82212802)
\curveto(537.38502047,191.63157684)(538.36580787,190.70181306)(539.7207685,190.70181306)
\curveto(540.71289449,190.70181306)(541.3591937,191.34811227)(541.81273701,192.25519889)
\curveto(542.28895748,193.27000204)(542.66313071,194.99913589)(542.66313071,195.05582881)
\curveto(542.66313071,195.33929337)(542.40234331,195.33929337)(542.32297323,195.33929337)
\curveto(542.03950866,195.33929337)(542.0168315,195.22023826)(541.92612283,194.82338786)
\curveto(541.44990236,192.98653747)(540.93399685,191.32543511)(539.78313071,191.32543511)
\curveto(539.01777638,191.32543511)(538.93273701,192.05677369)(538.93273701,192.62370282)
\curveto(538.93273701,193.24732487)(538.98376063,193.46842723)(539.29557165,194.71567133)
\curveto(539.61305197,195.90622251)(539.66974488,196.18968708)(539.92486299,197.26118314)
\lineto(540.93399685,201.20700991)
\curveto(541.13809134,202.0007107)(541.13809134,202.05740361)(541.13809134,202.16512015)
\curveto(541.13809134,202.6526792)(540.79226457,202.93614377)(540.31604409,202.93614377)
\curveto(539.64139843,202.93614377)(539.21620157,202.30685243)(539.12549291,201.68889967)
\closepath
}
}
{
\newrgbcolor{curcolor}{0 0 0}
\pscustom[linestyle=none,fillstyle=solid,fillcolor=curcolor]
{
\newpath
\moveto(705.37809638,214.49582881)
\lineto(697.3787263,219.014254)
\lineto(689.3906948,214.49582881)
\lineto(689.70250583,213.87787605)
\lineto(697.35604913,217.34748235)
\lineto(705.04360819,213.87787605)
\closepath
}
}
{
\newrgbcolor{curcolor}{0 0 0}
\pscustom[linewidth=0,linecolor=curcolor]
{
\newpath
\moveto(705.37809638,214.49582881)
\lineto(697.3787263,219.014254)
\lineto(689.3906948,214.49582881)
\lineto(689.70250583,213.87787605)
\lineto(697.35604913,217.34748235)
\lineto(705.04360819,213.87787605)
\closepath
}
}
{
\newrgbcolor{curcolor}{0 0 0}
\pscustom[linestyle=none,fillstyle=solid,fillcolor=curcolor]
{
\newpath
\moveto(697.51478929,207.94779731)
\lineto(697.53746646,208.04417526)
\lineto(697.56014362,208.12921463)
\lineto(697.58282079,208.214254)
\lineto(697.60549795,208.29929337)
\lineto(697.62817512,208.37299416)
\lineto(697.65652157,208.44669495)
\lineto(697.68486803,208.52039574)
\lineto(697.71321449,208.58842723)
\lineto(697.74723024,208.65078944)
\lineto(697.78124598,208.70748235)
\lineto(697.82660031,208.76417526)
\lineto(697.87195465,208.81519889)
\lineto(697.92297827,208.86622251)
\lineto(697.97967118,208.91157684)
\lineto(698.04203339,208.95693117)
\lineto(698.11006488,208.99661621)
\lineto(698.18376567,209.03063196)
\lineto(698.26880504,209.06464771)
\lineto(698.3595137,209.09866345)
\lineto(698.46156094,209.12700991)
\lineto(698.51258457,209.13834849)
\lineto(698.56927748,209.14968708)
\lineto(698.62597039,209.16102566)
\lineto(698.6883326,209.17236424)
\lineto(698.7506948,209.18370282)
\lineto(698.81305701,209.19504141)
\lineto(698.8810885,209.20637999)
\lineto(698.95478929,209.21204928)
\lineto(699.02849008,209.22338786)
\lineto(699.10786016,209.22905715)
\lineto(699.18723024,209.23472645)
\lineto(699.26660031,209.24039574)
\lineto(699.35163969,209.24606503)
\lineto(699.44234835,209.25173432)
\lineto(699.53305701,209.25740361)
\lineto(699.62943496,209.2630729)
\lineto(699.72581291,209.2630729)
\lineto(699.82786016,209.26874219)
\lineto(699.9299074,209.26874219)
\lineto(700.03762394,209.27441148)
\lineto(700.15100976,209.27441148)
\lineto(700.26439559,209.27441148)
\lineto(700.38345071,209.27441148)
\lineto(700.50817512,209.27441148)
\curveto(701.35289953,209.27441148)(701.57967118,209.27441148)(701.57967118,209.81866345)
\curveto(701.57967118,210.15315164)(701.27352945,210.15315164)(701.13179717,210.15315164)
\curveto(700.1906948,210.15315164)(697.87762394,210.06811227)(696.94786016,210.06811227)
\curveto(696.09746646,210.06811227)(694.03384441,210.15315164)(693.19478929,210.15315164)
\curveto(692.9906948,210.15315164)(692.65053732,210.15315164)(692.65053732,209.58622251)
\curveto(692.65053732,209.27441148)(692.91132472,209.27441148)(693.43856882,209.27441148)
\curveto(693.50093102,209.27441148)(694.03384441,209.27441148)(694.52140346,209.22338786)
\curveto(695.02597039,209.16102566)(695.2810885,209.1326792)(695.2810885,208.76984456)
\curveto(695.2810885,208.65645873)(695.24707276,208.56575007)(695.16770268,208.23126188)
\lineto(691.38628535,193.06590755)
\curveto(691.10282079,191.96039574)(691.04045858,191.73929337)(688.8067578,191.73929337)
\curveto(688.33053732,191.73929337)(688.04707276,191.73929337)(688.04707276,191.17236424)
\curveto(688.04707276,190.86055322)(688.30219087,190.86055322)(688.8067578,190.86055322)
\lineto(701.8914822,190.86055322)
\curveto(702.56612787,190.86055322)(702.58880504,190.86055322)(702.77022236,191.33677369)
\lineto(704.99258457,197.44260046)
\curveto(705.10597039,197.75441148)(705.10597039,197.80543511)(705.10597039,197.83945086)
\curveto(705.10597039,197.94716739)(705.02660031,198.15126188)(704.7714822,198.15126188)
\curveto(704.5106948,198.15126188)(704.48801764,198.0095296)(704.28959244,197.55598629)
\curveto(703.32581291,194.95378156)(702.08423811,191.73929337)(697.19730898,191.73929337)
\lineto(694.54974992,191.73929337)
\curveto(694.14723024,191.73929337)(694.09620661,191.73929337)(693.92612787,191.76197054)
\curveto(693.64266331,191.790317)(693.55762394,191.81866345)(693.55762394,192.04543511)
\curveto(693.55762394,192.12480519)(693.55762394,192.18716739)(693.69935622,192.69173432)
\closepath
}
}
{
\newrgbcolor{curcolor}{0 0 0}
\pscustom[linewidth=0,linecolor=curcolor]
{
\newpath
\moveto(697.51478929,207.94779731)
\lineto(697.53746646,208.04417526)
\lineto(697.56014362,208.12921463)
\lineto(697.58282079,208.214254)
\lineto(697.60549795,208.29929337)
\lineto(697.62817512,208.37299416)
\lineto(697.65652157,208.44669495)
\lineto(697.68486803,208.52039574)
\lineto(697.71321449,208.58842723)
\lineto(697.74723024,208.65078944)
\lineto(697.78124598,208.70748235)
\lineto(697.82660031,208.76417526)
\lineto(697.87195465,208.81519889)
\lineto(697.92297827,208.86622251)
\lineto(697.97967118,208.91157684)
\lineto(698.04203339,208.95693117)
\lineto(698.11006488,208.99661621)
\lineto(698.18376567,209.03063196)
\lineto(698.26880504,209.06464771)
\lineto(698.3595137,209.09866345)
\lineto(698.46156094,209.12700991)
\lineto(698.51258457,209.13834849)
\lineto(698.56927748,209.14968708)
\lineto(698.62597039,209.16102566)
\lineto(698.6883326,209.17236424)
\lineto(698.7506948,209.18370282)
\lineto(698.81305701,209.19504141)
\lineto(698.8810885,209.20637999)
\lineto(698.95478929,209.21204928)
\lineto(699.02849008,209.22338786)
\lineto(699.10786016,209.22905715)
\lineto(699.18723024,209.23472645)
\lineto(699.26660031,209.24039574)
\lineto(699.35163969,209.24606503)
\lineto(699.44234835,209.25173432)
\lineto(699.53305701,209.25740361)
\lineto(699.62943496,209.2630729)
\lineto(699.72581291,209.2630729)
\lineto(699.82786016,209.26874219)
\lineto(699.9299074,209.26874219)
\lineto(700.03762394,209.27441148)
\lineto(700.15100976,209.27441148)
\lineto(700.26439559,209.27441148)
\lineto(700.38345071,209.27441148)
\lineto(700.50817512,209.27441148)
\curveto(701.35289953,209.27441148)(701.57967118,209.27441148)(701.57967118,209.81866345)
\curveto(701.57967118,210.15315164)(701.27352945,210.15315164)(701.13179717,210.15315164)
\curveto(700.1906948,210.15315164)(697.87762394,210.06811227)(696.94786016,210.06811227)
\curveto(696.09746646,210.06811227)(694.03384441,210.15315164)(693.19478929,210.15315164)
\curveto(692.9906948,210.15315164)(692.65053732,210.15315164)(692.65053732,209.58622251)
\curveto(692.65053732,209.27441148)(692.91132472,209.27441148)(693.43856882,209.27441148)
\curveto(693.50093102,209.27441148)(694.03384441,209.27441148)(694.52140346,209.22338786)
\curveto(695.02597039,209.16102566)(695.2810885,209.1326792)(695.2810885,208.76984456)
\curveto(695.2810885,208.65645873)(695.24707276,208.56575007)(695.16770268,208.23126188)
\lineto(691.38628535,193.06590755)
\curveto(691.10282079,191.96039574)(691.04045858,191.73929337)(688.8067578,191.73929337)
\curveto(688.33053732,191.73929337)(688.04707276,191.73929337)(688.04707276,191.17236424)
\curveto(688.04707276,190.86055322)(688.30219087,190.86055322)(688.8067578,190.86055322)
\lineto(701.8914822,190.86055322)
\curveto(702.56612787,190.86055322)(702.58880504,190.86055322)(702.77022236,191.33677369)
\lineto(704.99258457,197.44260046)
\curveto(705.10597039,197.75441148)(705.10597039,197.80543511)(705.10597039,197.83945086)
\curveto(705.10597039,197.94716739)(705.02660031,198.15126188)(704.7714822,198.15126188)
\curveto(704.5106948,198.15126188)(704.48801764,198.0095296)(704.28959244,197.55598629)
\curveto(703.32581291,194.95378156)(702.08423811,191.73929337)(697.19730898,191.73929337)
\lineto(694.54974992,191.73929337)
\curveto(694.14723024,191.73929337)(694.09620661,191.73929337)(693.92612787,191.76197054)
\curveto(693.64266331,191.790317)(693.55762394,191.81866345)(693.55762394,192.04543511)
\curveto(693.55762394,192.12480519)(693.55762394,192.18716739)(693.69935622,192.69173432)
\closepath
}
}
{
\newrgbcolor{curcolor}{0 0 0}
\pscustom[linestyle=none,fillstyle=solid,fillcolor=curcolor]
{
\newpath
\moveto(715.54313575,184.08575007)
\lineto(715.54313575,184.09708865)
\lineto(715.54313575,184.11409652)
\lineto(715.54313575,184.11976582)
\lineto(715.54313575,184.1311044)
\lineto(715.53746646,184.13677369)
\lineto(715.53746646,184.14811227)
\lineto(715.53179717,184.15945086)
\lineto(715.53179717,184.16512015)
\lineto(715.52612787,184.17645873)
\lineto(715.52045858,184.18779731)
\lineto(715.50912,184.19913589)
\lineto(715.50345071,184.21614377)
\lineto(715.49211213,184.22748235)
\lineto(715.48644283,184.24449023)
\lineto(715.46943496,184.2614981)
\lineto(715.45809638,184.27850597)
\lineto(715.4410885,184.29551385)
\lineto(715.42408063,184.31819101)
\lineto(715.41841134,184.3295296)
\lineto(715.40707276,184.34086818)
\lineto(715.40140346,184.35220676)
\lineto(715.39006488,184.36354534)
\lineto(715.3787263,184.37488393)
\lineto(715.36738772,184.3918918)
\lineto(715.35604913,184.40323038)
\lineto(715.33904126,184.41456897)
\lineto(715.32770268,184.43157684)
\lineto(715.31636409,184.44291542)
\lineto(715.29935622,184.4599233)
\lineto(715.28801764,184.47693117)
\lineto(715.27100976,184.49393904)
\lineto(715.25400189,184.51094692)
\lineto(715.24266331,184.52795479)
\lineto(715.22565543,184.54496267)
\lineto(715.20297827,184.56197054)
\lineto(715.18597039,184.57897841)
\lineto(715.16896252,184.60165558)
\lineto(715.15195465,184.61866345)
\lineto(715.12927748,184.64134062)
\lineto(715.10660031,184.65834849)
\lineto(715.08959244,184.68102566)
\lineto(715.06691528,184.70370282)
\curveto(711.54061606,188.26401778)(710.63352945,193.59882093)(710.63352945,197.91882093)
\curveto(710.63352945,202.83976582)(711.7106948,207.75504141)(715.1803011,211.28134062)
\curveto(715.54313575,211.62716739)(715.54313575,211.67819101)(715.54313575,211.76889967)
\curveto(715.54313575,211.96165558)(715.43541921,212.04102566)(715.27100976,212.04102566)
\curveto(714.9875452,212.04102566)(712.44203339,210.12480519)(710.77526173,206.54181306)
\curveto(709.33526173,203.42937211)(708.99510425,200.294254)(708.99510425,197.91882093)
\curveto(708.99510425,195.7134666)(709.30691528,192.30622251)(710.8603011,189.10874219)
\curveto(712.54974992,185.6334666)(714.9875452,183.8022855)(715.27100976,183.8022855)
\curveto(715.43541921,183.8022855)(715.54313575,183.88165558)(715.54313575,184.08575007)
\closepath
}
}
{
\newrgbcolor{curcolor}{0 0 0}
\pscustom[linewidth=0,linecolor=curcolor]
{
\newpath
\moveto(715.54313575,184.08575007)
\lineto(715.54313575,184.09708865)
\lineto(715.54313575,184.11409652)
\lineto(715.54313575,184.11976582)
\lineto(715.54313575,184.1311044)
\lineto(715.53746646,184.13677369)
\lineto(715.53746646,184.14811227)
\lineto(715.53179717,184.15945086)
\lineto(715.53179717,184.16512015)
\lineto(715.52612787,184.17645873)
\lineto(715.52045858,184.18779731)
\lineto(715.50912,184.19913589)
\lineto(715.50345071,184.21614377)
\lineto(715.49211213,184.22748235)
\lineto(715.48644283,184.24449023)
\lineto(715.46943496,184.2614981)
\lineto(715.45809638,184.27850597)
\lineto(715.4410885,184.29551385)
\lineto(715.42408063,184.31819101)
\lineto(715.41841134,184.3295296)
\lineto(715.40707276,184.34086818)
\lineto(715.40140346,184.35220676)
\lineto(715.39006488,184.36354534)
\lineto(715.3787263,184.37488393)
\lineto(715.36738772,184.3918918)
\lineto(715.35604913,184.40323038)
\lineto(715.33904126,184.41456897)
\lineto(715.32770268,184.43157684)
\lineto(715.31636409,184.44291542)
\lineto(715.29935622,184.4599233)
\lineto(715.28801764,184.47693117)
\lineto(715.27100976,184.49393904)
\lineto(715.25400189,184.51094692)
\lineto(715.24266331,184.52795479)
\lineto(715.22565543,184.54496267)
\lineto(715.20297827,184.56197054)
\lineto(715.18597039,184.57897841)
\lineto(715.16896252,184.60165558)
\lineto(715.15195465,184.61866345)
\lineto(715.12927748,184.64134062)
\lineto(715.10660031,184.65834849)
\lineto(715.08959244,184.68102566)
\lineto(715.06691528,184.70370282)
\curveto(711.54061606,188.26401778)(710.63352945,193.59882093)(710.63352945,197.91882093)
\curveto(710.63352945,202.83976582)(711.7106948,207.75504141)(715.1803011,211.28134062)
\curveto(715.54313575,211.62716739)(715.54313575,211.67819101)(715.54313575,211.76889967)
\curveto(715.54313575,211.96165558)(715.43541921,212.04102566)(715.27100976,212.04102566)
\curveto(714.9875452,212.04102566)(712.44203339,210.12480519)(710.77526173,206.54181306)
\curveto(709.33526173,203.42937211)(708.99510425,200.294254)(708.99510425,197.91882093)
\curveto(708.99510425,195.7134666)(709.30691528,192.30622251)(710.8603011,189.10874219)
\curveto(712.54974992,185.6334666)(714.9875452,183.8022855)(715.27100976,183.8022855)
\curveto(715.43541921,183.8022855)(715.54313575,183.88165558)(715.54313575,184.08575007)
\closepath
}
}
{
\newrgbcolor{curcolor}{0 0 0}
\pscustom[linestyle=none,fillstyle=solid,fillcolor=curcolor]
{
\newpath
\moveto(734.01935622,207.35252172)
\lineto(726.01998614,211.87094692)
\lineto(718.02628535,207.35252172)
\lineto(718.33809638,206.73456897)
\lineto(725.99163969,210.20984456)
\lineto(733.67919874,206.73456897)
\closepath
}
}
{
\newrgbcolor{curcolor}{0 0 0}
\pscustom[linewidth=0,linecolor=curcolor]
{
\newpath
\moveto(734.01935622,207.35252172)
\lineto(726.01998614,211.87094692)
\lineto(718.02628535,207.35252172)
\lineto(718.33809638,206.73456897)
\lineto(725.99163969,210.20984456)
\lineto(733.67919874,206.73456897)
\closepath
}
}
{
\newrgbcolor{curcolor}{0 0 0}
\pscustom[linestyle=none,fillstyle=solid,fillcolor=curcolor]
{
\newpath
\moveto(727.03478929,192.44795479)
\lineto(727.06313575,192.34023826)
\lineto(727.10282079,192.23252172)
\lineto(727.13683654,192.13047448)
\lineto(727.18219087,192.03409652)
\lineto(727.2275452,191.93204928)
\lineto(727.27856882,191.84134062)
\lineto(727.32959244,191.75063196)
\lineto(727.38628535,191.6599233)
\lineto(727.44864756,191.57488393)
\lineto(727.51100976,191.48984456)
\lineto(727.57904126,191.41047448)
\lineto(727.64707276,191.33677369)
\lineto(727.72077354,191.2630729)
\lineto(727.79447433,191.19504141)
\lineto(727.87384441,191.12700991)
\lineto(727.95321449,191.06464771)
\lineto(728.03825386,191.0022855)
\lineto(728.12329323,190.94559259)
\lineto(728.21400189,190.89456897)
\lineto(728.30471055,190.84354534)
\lineto(728.39541921,190.79819101)
\lineto(728.49179717,190.75850597)
\lineto(728.58817512,190.71882093)
\lineto(728.69022236,190.68480519)
\lineto(728.79226961,190.65078944)
\lineto(728.89431685,190.62811227)
\lineto(729.00203339,190.60543511)
\lineto(729.10974992,190.58275794)
\lineto(729.21746646,190.57141936)
\lineto(729.32518299,190.56008078)
\lineto(729.43856882,190.55441148)
\lineto(729.55195465,190.54874219)
\curveto(730.53274205,190.54874219)(731.19037984,191.19504141)(731.63825386,192.10212802)
\curveto(732.12014362,193.11693117)(732.48297827,194.84606503)(732.48297827,194.90275794)
\curveto(732.48297827,195.18622251)(732.23352945,195.18622251)(732.14849008,195.18622251)
\curveto(731.86502551,195.18622251)(731.83667906,195.06716739)(731.74597039,194.670317)
\curveto(731.35478929,193.094254)(730.81620661,191.17236424)(729.63132472,191.17236424)
\curveto(729.03604913,191.17236424)(728.76392315,191.54086818)(728.76392315,192.47063196)
\curveto(728.76392315,193.094254)(729.09841134,194.42086818)(729.3195137,195.40732487)
\lineto(730.11888378,198.4630729)
\curveto(730.19825386,198.88826975)(730.48171843,199.95976582)(730.59510425,200.38496267)
\curveto(730.73683654,201.03126188)(731.0203011,202.10275794)(731.0203011,202.27283668)
\curveto(731.0203011,202.7830729)(730.62345071,203.03252172)(730.19825386,203.03252172)
\curveto(730.05652157,203.03252172)(729.3195137,203.00417526)(729.09841134,202.04606503)
\curveto(728.55982866,199.98244298)(727.31825386,195.04449023)(726.98376567,193.54212802)
\curveto(726.94408063,193.42874219)(725.82156094,191.17236424)(723.7579389,191.17236424)
\curveto(722.28392315,191.17236424)(722.00045858,192.44795479)(722.00045858,193.4911044)
\curveto(722.00045858,195.06716739)(722.79982866,197.30086818)(723.53683654,199.2511044)
\curveto(723.87132472,200.1014981)(724.01305701,200.48700991)(724.01305701,201.03126188)
\curveto(724.01305701,202.29551385)(723.11163969,203.34433274)(721.69431685,203.34433274)
\curveto(719.00707276,203.34433274)(717.96392315,199.2511044)(717.96392315,198.990317)
\curveto(717.96392315,198.70685243)(718.24738772,198.70685243)(718.30974992,198.70685243)
\curveto(718.59321449,198.70685243)(718.62156094,198.76921463)(718.76329323,199.22275794)
\curveto(719.46061606,201.67756109)(720.54345071,202.7207107)(721.61494677,202.7207107)
\curveto(721.8587263,202.7207107)(722.32360819,202.69803353)(722.32360819,201.79094692)
\curveto(722.32360819,201.11063196)(722.00045858,200.294254)(721.83604913,199.86905715)
\curveto(720.79289953,197.06842723)(720.19762394,195.32795479)(720.19762394,193.94464771)
\curveto(720.19762394,191.25740361)(722.14219087,190.54874219)(723.67856882,190.54874219)
\curveto(725.53809638,190.54874219)(726.55856882,191.81866345)(727.03478929,192.44795479)
\closepath
}
}
{
\newrgbcolor{curcolor}{0 0 0}
\pscustom[linewidth=0,linecolor=curcolor]
{
\newpath
\moveto(727.03478929,192.44795479)
\lineto(727.06313575,192.34023826)
\lineto(727.10282079,192.23252172)
\lineto(727.13683654,192.13047448)
\lineto(727.18219087,192.03409652)
\lineto(727.2275452,191.93204928)
\lineto(727.27856882,191.84134062)
\lineto(727.32959244,191.75063196)
\lineto(727.38628535,191.6599233)
\lineto(727.44864756,191.57488393)
\lineto(727.51100976,191.48984456)
\lineto(727.57904126,191.41047448)
\lineto(727.64707276,191.33677369)
\lineto(727.72077354,191.2630729)
\lineto(727.79447433,191.19504141)
\lineto(727.87384441,191.12700991)
\lineto(727.95321449,191.06464771)
\lineto(728.03825386,191.0022855)
\lineto(728.12329323,190.94559259)
\lineto(728.21400189,190.89456897)
\lineto(728.30471055,190.84354534)
\lineto(728.39541921,190.79819101)
\lineto(728.49179717,190.75850597)
\lineto(728.58817512,190.71882093)
\lineto(728.69022236,190.68480519)
\lineto(728.79226961,190.65078944)
\lineto(728.89431685,190.62811227)
\lineto(729.00203339,190.60543511)
\lineto(729.10974992,190.58275794)
\lineto(729.21746646,190.57141936)
\lineto(729.32518299,190.56008078)
\lineto(729.43856882,190.55441148)
\lineto(729.55195465,190.54874219)
\curveto(730.53274205,190.54874219)(731.19037984,191.19504141)(731.63825386,192.10212802)
\curveto(732.12014362,193.11693117)(732.48297827,194.84606503)(732.48297827,194.90275794)
\curveto(732.48297827,195.18622251)(732.23352945,195.18622251)(732.14849008,195.18622251)
\curveto(731.86502551,195.18622251)(731.83667906,195.06716739)(731.74597039,194.670317)
\curveto(731.35478929,193.094254)(730.81620661,191.17236424)(729.63132472,191.17236424)
\curveto(729.03604913,191.17236424)(728.76392315,191.54086818)(728.76392315,192.47063196)
\curveto(728.76392315,193.094254)(729.09841134,194.42086818)(729.3195137,195.40732487)
\lineto(730.11888378,198.4630729)
\curveto(730.19825386,198.88826975)(730.48171843,199.95976582)(730.59510425,200.38496267)
\curveto(730.73683654,201.03126188)(731.0203011,202.10275794)(731.0203011,202.27283668)
\curveto(731.0203011,202.7830729)(730.62345071,203.03252172)(730.19825386,203.03252172)
\curveto(730.05652157,203.03252172)(729.3195137,203.00417526)(729.09841134,202.04606503)
\curveto(728.55982866,199.98244298)(727.31825386,195.04449023)(726.98376567,193.54212802)
\curveto(726.94408063,193.42874219)(725.82156094,191.17236424)(723.7579389,191.17236424)
\curveto(722.28392315,191.17236424)(722.00045858,192.44795479)(722.00045858,193.4911044)
\curveto(722.00045858,195.06716739)(722.79982866,197.30086818)(723.53683654,199.2511044)
\curveto(723.87132472,200.1014981)(724.01305701,200.48700991)(724.01305701,201.03126188)
\curveto(724.01305701,202.29551385)(723.11163969,203.34433274)(721.69431685,203.34433274)
\curveto(719.00707276,203.34433274)(717.96392315,199.2511044)(717.96392315,198.990317)
\curveto(717.96392315,198.70685243)(718.24738772,198.70685243)(718.30974992,198.70685243)
\curveto(718.59321449,198.70685243)(718.62156094,198.76921463)(718.76329323,199.22275794)
\curveto(719.46061606,201.67756109)(720.54345071,202.7207107)(721.61494677,202.7207107)
\curveto(721.8587263,202.7207107)(722.32360819,202.69803353)(722.32360819,201.79094692)
\curveto(722.32360819,201.11063196)(722.00045858,200.294254)(721.83604913,199.86905715)
\curveto(720.79289953,197.06842723)(720.19762394,195.32795479)(720.19762394,193.94464771)
\curveto(720.19762394,191.25740361)(722.14219087,190.54874219)(723.67856882,190.54874219)
\curveto(725.53809638,190.54874219)(726.55856882,191.81866345)(727.03478929,192.44795479)
\closepath
}
}
{
\newrgbcolor{curcolor}{0 0 0}
\pscustom[linestyle=none,fillstyle=solid,fillcolor=curcolor]
{
\newpath
\moveto(741.4971515,197.91882093)
\lineto(741.4971515,198.12858471)
\lineto(741.4971515,198.34968708)
\lineto(741.4914822,198.57078944)
\lineto(741.48014362,198.80323038)
\lineto(741.45746646,199.27945086)
\lineto(741.42345071,199.77834849)
\lineto(741.37809638,200.2999233)
\lineto(741.31573417,200.83850597)
\lineto(741.24203339,201.39409652)
\lineto(741.15132472,201.96102566)
\lineto(741.0379389,202.53929337)
\lineto(740.9075452,203.12323038)
\lineto(740.75447433,203.72417526)
\lineto(740.58439559,204.31945086)
\lineto(740.38597039,204.92606503)
\lineto(740.16486803,205.5326792)
\lineto(740.04014362,205.83315164)
\lineto(739.91541921,206.13362408)
\lineto(739.77935622,206.43409652)
\lineto(739.63762394,206.73456897)
\curveto(737.94817512,210.20984456)(735.52171843,212.04102566)(735.23825386,212.04102566)
\curveto(735.06817512,212.04102566)(734.95478929,211.93330912)(734.95478929,211.76889967)
\curveto(734.95478929,211.67819101)(734.95478929,211.62716739)(735.48203339,211.11126188)
\curveto(738.25431685,208.32197054)(739.87006488,203.8318918)(739.87006488,197.91882093)
\curveto(739.87006488,193.094254)(738.82124598,188.1222855)(735.31762394,184.56197054)
\curveto(734.95478929,184.22748235)(734.95478929,184.16512015)(734.95478929,184.08575007)
\curveto(734.95478929,183.91567133)(735.06817512,183.8022855)(735.23825386,183.8022855)
\curveto(735.52171843,183.8022855)(738.05022236,185.72417526)(739.7283326,189.31283668)
\curveto(741.16266331,192.40826975)(741.4971515,195.54905715)(741.4971515,197.91882093)
\closepath
}
}
{
\newrgbcolor{curcolor}{0 0 0}
\pscustom[linewidth=0,linecolor=curcolor]
{
\newpath
\moveto(741.4971515,197.91882093)
\lineto(741.4971515,198.12858471)
\lineto(741.4971515,198.34968708)
\lineto(741.4914822,198.57078944)
\lineto(741.48014362,198.80323038)
\lineto(741.45746646,199.27945086)
\lineto(741.42345071,199.77834849)
\lineto(741.37809638,200.2999233)
\lineto(741.31573417,200.83850597)
\lineto(741.24203339,201.39409652)
\lineto(741.15132472,201.96102566)
\lineto(741.0379389,202.53929337)
\lineto(740.9075452,203.12323038)
\lineto(740.75447433,203.72417526)
\lineto(740.58439559,204.31945086)
\lineto(740.38597039,204.92606503)
\lineto(740.16486803,205.5326792)
\lineto(740.04014362,205.83315164)
\lineto(739.91541921,206.13362408)
\lineto(739.77935622,206.43409652)
\lineto(739.63762394,206.73456897)
\curveto(737.94817512,210.20984456)(735.52171843,212.04102566)(735.23825386,212.04102566)
\curveto(735.06817512,212.04102566)(734.95478929,211.93330912)(734.95478929,211.76889967)
\curveto(734.95478929,211.67819101)(734.95478929,211.62716739)(735.48203339,211.11126188)
\curveto(738.25431685,208.32197054)(739.87006488,203.8318918)(739.87006488,197.91882093)
\curveto(739.87006488,193.094254)(738.82124598,188.1222855)(735.31762394,184.56197054)
\curveto(734.95478929,184.22748235)(734.95478929,184.16512015)(734.95478929,184.08575007)
\curveto(734.95478929,183.91567133)(735.06817512,183.8022855)(735.23825386,183.8022855)
\curveto(735.52171843,183.8022855)(738.05022236,185.72417526)(739.7283326,189.31283668)
\curveto(741.16266331,192.40826975)(741.4971515,195.54905715)(741.4971515,197.91882093)
\closepath
}
}
{
\newrgbcolor{curcolor}{0 0 0}
\pscustom[linestyle=none,fillstyle=solid,fillcolor=curcolor]
{
\newpath
\moveto(771.58974992,200.1014981)
\lineto(771.66912,200.1014981)
\lineto(771.74849008,200.1014981)
\lineto(771.83352945,200.1014981)
\lineto(771.87321449,200.10716739)
\lineto(771.91289953,200.10716739)
\lineto(771.95825386,200.11283668)
\lineto(771.9979389,200.11850597)
\lineto(772.03762394,200.12417526)
\lineto(772.07730898,200.12984456)
\lineto(772.11699402,200.13551385)
\lineto(772.15100976,200.14685243)
\lineto(772.1906948,200.15819101)
\lineto(772.22471055,200.1695296)
\lineto(772.2587263,200.18086818)
\lineto(772.29274205,200.19787605)
\lineto(772.3210885,200.21488393)
\lineto(772.35510425,200.23756109)
\lineto(772.38345071,200.26023826)
\lineto(772.40612787,200.28291542)
\lineto(772.43447433,200.30559259)
\lineto(772.4571515,200.33393904)
\lineto(772.46849008,200.35094692)
\lineto(772.47415937,200.36795479)
\lineto(772.48549795,200.38496267)
\lineto(772.49683654,200.40197054)
\lineto(772.50250583,200.41897841)
\lineto(772.50817512,200.43598629)
\lineto(772.51384441,200.45299416)
\lineto(772.52518299,200.47567133)
\lineto(772.53085228,200.4926792)
\lineto(772.53085228,200.51535637)
\lineto(772.53652157,200.53803353)
\lineto(772.54219087,200.5607107)
\lineto(772.54219087,200.58338786)
\lineto(772.54786016,200.60606503)
\lineto(772.54786016,200.63441148)
\lineto(772.54786016,200.65708865)
\curveto(772.54786016,201.22401778)(772.00360819,201.22401778)(771.61809638,201.22401778)
\lineto(754.69526173,201.22401778)
\curveto(754.29841134,201.22401778)(753.76549795,201.22401778)(753.76549795,200.65708865)
\curveto(753.76549795,200.1014981)(754.29841134,200.1014981)(754.72360819,200.1014981)
\closepath
}
}
{
\newrgbcolor{curcolor}{0 0 0}
\pscustom[linewidth=0,linecolor=curcolor]
{
\newpath
\moveto(771.58974992,200.1014981)
\lineto(771.66912,200.1014981)
\lineto(771.74849008,200.1014981)
\lineto(771.83352945,200.1014981)
\lineto(771.87321449,200.10716739)
\lineto(771.91289953,200.10716739)
\lineto(771.95825386,200.11283668)
\lineto(771.9979389,200.11850597)
\lineto(772.03762394,200.12417526)
\lineto(772.07730898,200.12984456)
\lineto(772.11699402,200.13551385)
\lineto(772.15100976,200.14685243)
\lineto(772.1906948,200.15819101)
\lineto(772.22471055,200.1695296)
\lineto(772.2587263,200.18086818)
\lineto(772.29274205,200.19787605)
\lineto(772.3210885,200.21488393)
\lineto(772.35510425,200.23756109)
\lineto(772.38345071,200.26023826)
\lineto(772.40612787,200.28291542)
\lineto(772.43447433,200.30559259)
\lineto(772.4571515,200.33393904)
\lineto(772.46849008,200.35094692)
\lineto(772.47415937,200.36795479)
\lineto(772.48549795,200.38496267)
\lineto(772.49683654,200.40197054)
\lineto(772.50250583,200.41897841)
\lineto(772.50817512,200.43598629)
\lineto(772.51384441,200.45299416)
\lineto(772.52518299,200.47567133)
\lineto(772.53085228,200.4926792)
\lineto(772.53085228,200.51535637)
\lineto(772.53652157,200.53803353)
\lineto(772.54219087,200.5607107)
\lineto(772.54219087,200.58338786)
\lineto(772.54786016,200.60606503)
\lineto(772.54786016,200.63441148)
\lineto(772.54786016,200.65708865)
\curveto(772.54786016,201.22401778)(772.00360819,201.22401778)(771.61809638,201.22401778)
\lineto(754.69526173,201.22401778)
\curveto(754.29841134,201.22401778)(753.76549795,201.22401778)(753.76549795,200.65708865)
\curveto(753.76549795,200.1014981)(754.29841134,200.1014981)(754.72360819,200.1014981)
\closepath
}
}
{
\newrgbcolor{curcolor}{0 0 0}
\pscustom[linestyle=none,fillstyle=solid,fillcolor=curcolor]
{
\newpath
\moveto(771.61809638,194.61929337)
\lineto(771.65211213,194.61929337)
\lineto(771.69179717,194.61929337)
\lineto(771.77116724,194.61929337)
\lineto(771.84486803,194.61929337)
\lineto(771.88455307,194.62496267)
\lineto(771.92423811,194.62496267)
\lineto(771.96392315,194.63063196)
\lineto(772.00360819,194.63630125)
\lineto(772.04329323,194.64197054)
\lineto(772.08297827,194.64763983)
\lineto(772.12266331,194.65330912)
\lineto(772.15667906,194.66464771)
\lineto(772.1906948,194.67598629)
\lineto(772.23037984,194.68732487)
\lineto(772.26439559,194.69866345)
\lineto(772.29274205,194.71567133)
\lineto(772.3267578,194.7326792)
\lineto(772.35510425,194.75535637)
\lineto(772.38345071,194.77803353)
\lineto(772.41179717,194.8007107)
\lineto(772.43447433,194.82905715)
\lineto(772.44581291,194.84039574)
\lineto(772.4571515,194.85740361)
\lineto(772.46849008,194.86874219)
\lineto(772.47415937,194.88575007)
\lineto(772.48549795,194.90275794)
\lineto(772.49683654,194.91976582)
\lineto(772.50250583,194.93677369)
\lineto(772.50817512,194.95945086)
\lineto(772.51384441,194.97645873)
\lineto(772.52518299,194.9934666)
\lineto(772.53085228,195.01614377)
\lineto(772.53085228,195.03882093)
\lineto(772.53652157,195.0614981)
\lineto(772.54219087,195.08417526)
\lineto(772.54219087,195.10685243)
\lineto(772.54786016,195.1295296)
\lineto(772.54786016,195.15787605)
\lineto(772.54786016,195.18622251)
\curveto(772.54786016,195.75315164)(772.00360819,195.75315164)(771.58974992,195.75315164)
\lineto(754.72360819,195.75315164)
\curveto(754.29841134,195.75315164)(753.76549795,195.75315164)(753.76549795,195.18622251)
\curveto(753.76549795,194.61929337)(754.29841134,194.61929337)(754.69526173,194.61929337)
\closepath
}
}
{
\newrgbcolor{curcolor}{0 0 0}
\pscustom[linewidth=0,linecolor=curcolor]
{
\newpath
\moveto(771.61809638,194.61929337)
\lineto(771.65211213,194.61929337)
\lineto(771.69179717,194.61929337)
\lineto(771.77116724,194.61929337)
\lineto(771.84486803,194.61929337)
\lineto(771.88455307,194.62496267)
\lineto(771.92423811,194.62496267)
\lineto(771.96392315,194.63063196)
\lineto(772.00360819,194.63630125)
\lineto(772.04329323,194.64197054)
\lineto(772.08297827,194.64763983)
\lineto(772.12266331,194.65330912)
\lineto(772.15667906,194.66464771)
\lineto(772.1906948,194.67598629)
\lineto(772.23037984,194.68732487)
\lineto(772.26439559,194.69866345)
\lineto(772.29274205,194.71567133)
\lineto(772.3267578,194.7326792)
\lineto(772.35510425,194.75535637)
\lineto(772.38345071,194.77803353)
\lineto(772.41179717,194.8007107)
\lineto(772.43447433,194.82905715)
\lineto(772.44581291,194.84039574)
\lineto(772.4571515,194.85740361)
\lineto(772.46849008,194.86874219)
\lineto(772.47415937,194.88575007)
\lineto(772.48549795,194.90275794)
\lineto(772.49683654,194.91976582)
\lineto(772.50250583,194.93677369)
\lineto(772.50817512,194.95945086)
\lineto(772.51384441,194.97645873)
\lineto(772.52518299,194.9934666)
\lineto(772.53085228,195.01614377)
\lineto(772.53085228,195.03882093)
\lineto(772.53652157,195.0614981)
\lineto(772.54219087,195.08417526)
\lineto(772.54219087,195.10685243)
\lineto(772.54786016,195.1295296)
\lineto(772.54786016,195.15787605)
\lineto(772.54786016,195.18622251)
\curveto(772.54786016,195.75315164)(772.00360819,195.75315164)(771.58974992,195.75315164)
\lineto(754.72360819,195.75315164)
\curveto(754.29841134,195.75315164)(753.76549795,195.75315164)(753.76549795,195.18622251)
\curveto(753.76549795,194.61929337)(754.29841134,194.61929337)(754.69526173,194.61929337)
\closepath
}
}
{
\newrgbcolor{curcolor}{0 0 0}
\pscustom[linestyle=none,fillstyle=solid,fillcolor=curcolor]
{
\newpath
\moveto(794.97557669,199.89740361)
\lineto(794.9699074,200.32260046)
\lineto(794.96423811,200.74212802)
\lineto(794.95289953,201.16732487)
\lineto(794.93589165,201.58685243)
\lineto(794.9075452,202.01204928)
\lineto(794.87352945,202.43157684)
\lineto(794.83384441,202.8511044)
\lineto(794.7771515,203.26496267)
\lineto(794.71478929,203.67882093)
\lineto(794.63541921,204.0926792)
\lineto(794.54471055,204.50653747)
\lineto(794.43699402,204.90905715)
\lineto(794.31226961,205.31724613)
\lineto(794.17053732,205.71409652)
\lineto(794.01746646,206.11094692)
\lineto(793.93242709,206.30937211)
\lineto(793.84171843,206.50212802)
\curveto(792.54345071,209.22338786)(790.22471055,209.67693117)(789.03982866,209.67693117)
\curveto(787.34471055,209.67693117)(785.2810885,208.9399233)(784.12455307,206.30937211)
\curveto(783.21746646,204.35913589)(783.07573417,202.15378156)(783.07573417,199.89740361)
\curveto(783.07573417,197.77708865)(783.18912,195.23724613)(784.35132472,193.094254)
\curveto(785.56455307,190.8095296)(787.62817512,190.24260046)(789.0114822,190.24260046)
\lineto(789.0114822,190.86055322)
\curveto(787.91163969,190.86055322)(786.23919874,191.56921463)(785.73463181,194.27913589)
\curveto(785.42282079,195.974254)(785.42282079,198.56512015)(785.42282079,200.24323038)
\curveto(785.42282079,202.04606503)(785.42282079,203.91126188)(785.65526173,205.43063196)
\curveto(786.18817512,208.79819101)(788.30282079,209.04763983)(789.0114822,209.04763983)
\curveto(789.94124598,209.04763983)(791.81211213,208.5430729)(792.34502551,205.74244298)
\curveto(792.62849008,204.16637999)(792.62849008,202.01204928)(792.62849008,200.24323038)
\curveto(792.62849008,198.11724613)(792.62849008,196.20102566)(792.31667906,194.39252172)
\curveto(791.8914822,191.71094692)(790.27573417,190.86055322)(789.0114822,190.86055322)
\lineto(789.0114822,190.86055322)
\lineto(789.0114822,190.24260046)
\curveto(790.53652157,190.24260046)(792.6795137,190.83220676)(793.9267578,193.51945086)
\curveto(794.83384441,195.46968708)(794.97557669,197.66370282)(794.97557669,199.89740361)
\closepath
}
}
{
\newrgbcolor{curcolor}{0 0 0}
\pscustom[linewidth=0,linecolor=curcolor]
{
\newpath
\moveto(794.97557669,199.89740361)
\lineto(794.9699074,200.32260046)
\lineto(794.96423811,200.74212802)
\lineto(794.95289953,201.16732487)
\lineto(794.93589165,201.58685243)
\lineto(794.9075452,202.01204928)
\lineto(794.87352945,202.43157684)
\lineto(794.83384441,202.8511044)
\lineto(794.7771515,203.26496267)
\lineto(794.71478929,203.67882093)
\lineto(794.63541921,204.0926792)
\lineto(794.54471055,204.50653747)
\lineto(794.43699402,204.90905715)
\lineto(794.31226961,205.31724613)
\lineto(794.17053732,205.71409652)
\lineto(794.01746646,206.11094692)
\lineto(793.93242709,206.30937211)
\lineto(793.84171843,206.50212802)
\curveto(792.54345071,209.22338786)(790.22471055,209.67693117)(789.03982866,209.67693117)
\curveto(787.34471055,209.67693117)(785.2810885,208.9399233)(784.12455307,206.30937211)
\curveto(783.21746646,204.35913589)(783.07573417,202.15378156)(783.07573417,199.89740361)
\curveto(783.07573417,197.77708865)(783.18912,195.23724613)(784.35132472,193.094254)
\curveto(785.56455307,190.8095296)(787.62817512,190.24260046)(789.0114822,190.24260046)
\lineto(789.0114822,190.86055322)
\curveto(787.91163969,190.86055322)(786.23919874,191.56921463)(785.73463181,194.27913589)
\curveto(785.42282079,195.974254)(785.42282079,198.56512015)(785.42282079,200.24323038)
\curveto(785.42282079,202.04606503)(785.42282079,203.91126188)(785.65526173,205.43063196)
\curveto(786.18817512,208.79819101)(788.30282079,209.04763983)(789.0114822,209.04763983)
\curveto(789.94124598,209.04763983)(791.81211213,208.5430729)(792.34502551,205.74244298)
\curveto(792.62849008,204.16637999)(792.62849008,202.01204928)(792.62849008,200.24323038)
\curveto(792.62849008,198.11724613)(792.62849008,196.20102566)(792.31667906,194.39252172)
\curveto(791.8914822,191.71094692)(790.27573417,190.86055322)(789.0114822,190.86055322)
\lineto(789.0114822,190.86055322)
\lineto(789.0114822,190.24260046)
\curveto(790.53652157,190.24260046)(792.6795137,190.83220676)(793.9267578,193.51945086)
\curveto(794.83384441,195.46968708)(794.97557669,197.66370282)(794.97557669,199.89740361)
\closepath
}
}
{
\newrgbcolor{curcolor}{0 0 0}
\pscustom[linestyle=none,fillstyle=solid,fillcolor=curcolor]
{
\newpath
\moveto(801.85809638,190.88889967)
\lineto(801.85809638,191.05897841)
\lineto(801.85242709,191.22905715)
\lineto(801.8410885,191.3934666)
\lineto(801.82408063,191.55220676)
\lineto(801.80707276,191.70527763)
\lineto(801.78439559,191.8526792)
\lineto(801.76171843,192.00008078)
\lineto(801.73337197,192.14181306)
\lineto(801.69935622,192.27220676)
\lineto(801.66534047,192.40260046)
\lineto(801.61998614,192.52732487)
\lineto(801.5803011,192.64637999)
\lineto(801.52927748,192.75976582)
\lineto(801.48392315,192.87315164)
\lineto(801.42723024,192.97519889)
\lineto(801.37053732,193.07157684)
\lineto(801.30817512,193.1622855)
\lineto(801.24581291,193.25299416)
\lineto(801.17778142,193.33236424)
\lineto(801.10974992,193.40606503)
\lineto(801.03604913,193.47976582)
\lineto(800.96234835,193.54212802)
\lineto(800.88297827,193.59882093)
\lineto(800.80360819,193.65551385)
\lineto(800.71856882,193.70086818)
\lineto(800.62786016,193.74055322)
\lineto(800.5371515,193.77456897)
\lineto(800.44644283,193.80291542)
\lineto(800.35006488,193.82559259)
\lineto(800.25368693,193.84260046)
\lineto(800.15163969,193.85393904)
\lineto(800.04959244,193.85393904)
\curveto(799.11982866,193.85393904)(798.55289953,193.14527763)(798.55289953,192.35724613)
\curveto(798.55289953,191.59756109)(799.11982866,190.86055322)(800.04959244,190.86055322)
\curveto(800.38974992,190.86055322)(800.75825386,190.97393904)(801.04171843,191.22338786)
\curveto(801.1210885,191.28575007)(801.16077354,191.31409652)(801.18345071,191.31409652)
\curveto(801.21179717,191.31409652)(801.24014362,191.28575007)(801.24014362,190.88889967)
\curveto(801.24014362,188.79693117)(800.25368693,187.10748235)(799.31825386,186.17204928)
\curveto(799.00644283,185.86590755)(799.00644283,185.80354534)(799.00644283,185.72417526)
\curveto(799.00644283,185.52008078)(799.14817512,185.41236424)(799.2899074,185.41236424)
\curveto(799.60171843,185.41236424)(801.85809638,187.58370282)(801.85809638,190.88889967)
\closepath
}
}
{
\newrgbcolor{curcolor}{0 0 0}
\pscustom[linewidth=0,linecolor=curcolor]
{
\newpath
\moveto(801.85809638,190.88889967)
\lineto(801.85809638,191.05897841)
\lineto(801.85242709,191.22905715)
\lineto(801.8410885,191.3934666)
\lineto(801.82408063,191.55220676)
\lineto(801.80707276,191.70527763)
\lineto(801.78439559,191.8526792)
\lineto(801.76171843,192.00008078)
\lineto(801.73337197,192.14181306)
\lineto(801.69935622,192.27220676)
\lineto(801.66534047,192.40260046)
\lineto(801.61998614,192.52732487)
\lineto(801.5803011,192.64637999)
\lineto(801.52927748,192.75976582)
\lineto(801.48392315,192.87315164)
\lineto(801.42723024,192.97519889)
\lineto(801.37053732,193.07157684)
\lineto(801.30817512,193.1622855)
\lineto(801.24581291,193.25299416)
\lineto(801.17778142,193.33236424)
\lineto(801.10974992,193.40606503)
\lineto(801.03604913,193.47976582)
\lineto(800.96234835,193.54212802)
\lineto(800.88297827,193.59882093)
\lineto(800.80360819,193.65551385)
\lineto(800.71856882,193.70086818)
\lineto(800.62786016,193.74055322)
\lineto(800.5371515,193.77456897)
\lineto(800.44644283,193.80291542)
\lineto(800.35006488,193.82559259)
\lineto(800.25368693,193.84260046)
\lineto(800.15163969,193.85393904)
\lineto(800.04959244,193.85393904)
\curveto(799.11982866,193.85393904)(798.55289953,193.14527763)(798.55289953,192.35724613)
\curveto(798.55289953,191.59756109)(799.11982866,190.86055322)(800.04959244,190.86055322)
\curveto(800.38974992,190.86055322)(800.75825386,190.97393904)(801.04171843,191.22338786)
\curveto(801.1210885,191.28575007)(801.16077354,191.31409652)(801.18345071,191.31409652)
\curveto(801.21179717,191.31409652)(801.24014362,191.28575007)(801.24014362,190.88889967)
\curveto(801.24014362,188.79693117)(800.25368693,187.10748235)(799.31825386,186.17204928)
\curveto(799.00644283,185.86590755)(799.00644283,185.80354534)(799.00644283,185.72417526)
\curveto(799.00644283,185.52008078)(799.14817512,185.41236424)(799.2899074,185.41236424)
\curveto(799.60171843,185.41236424)(801.85809638,187.58370282)(801.85809638,190.88889967)
\closepath
}
}
{
\newrgbcolor{curcolor}{0 0 0}
\pscustom[linestyle=none,fillstyle=solid,fillcolor=curcolor]
{
\newpath
\moveto(835.95888378,220.15378156)
\lineto(815.50408063,215.26118314)
\lineto(815.68549795,214.58086818)
\lineto(835.98723024,218.48134062)
\lineto(856.27195465,214.58086818)
\lineto(856.44203339,215.26118314)
\closepath
}
}
{
\newrgbcolor{curcolor}{0 0 0}
\pscustom[linewidth=0,linecolor=curcolor]
{
\newpath
\moveto(835.95888378,220.15378156)
\lineto(815.50408063,215.26118314)
\lineto(815.68549795,214.58086818)
\lineto(835.98723024,218.48134062)
\lineto(856.27195465,214.58086818)
\lineto(856.44203339,215.26118314)
\closepath
}
}
{
\newrgbcolor{curcolor}{0 0 0}
\pscustom[linestyle=none,fillstyle=solid,fillcolor=curcolor]
{
\newpath
\moveto(823.22565543,210.15315164)
\lineto(823.22565543,210.15315164)
\lineto(823.22565543,210.15882093)
\lineto(823.22565543,210.16449023)
\lineto(823.22565543,210.17015952)
\lineto(823.22565543,210.17582881)
\lineto(823.22565543,210.1814981)
\lineto(823.22565543,210.19283668)
\lineto(823.22565543,210.20417526)
\lineto(823.21998614,210.21551385)
\lineto(823.21998614,210.22685243)
\lineto(823.21431685,210.23819101)
\lineto(823.21431685,210.2495296)
\lineto(823.20864756,210.26653747)
\lineto(823.20297827,210.27787605)
\lineto(823.19730898,210.28921463)
\lineto(823.19163969,210.30622251)
\lineto(823.1803011,210.31756109)
\lineto(823.17463181,210.33456897)
\lineto(823.16329323,210.34590755)
\lineto(823.15195465,210.36291542)
\lineto(823.14061606,210.374254)
\lineto(823.12927748,210.3799233)
\lineto(823.12360819,210.38559259)
\lineto(823.1179389,210.39126188)
\lineto(823.10660031,210.39693117)
\lineto(823.10093102,210.40260046)
\lineto(823.08959244,210.40826975)
\lineto(823.08392315,210.41393904)
\lineto(823.07258457,210.41960834)
\lineto(823.06124598,210.42527763)
\lineto(823.05557669,210.43094692)
\lineto(823.04423811,210.43661621)
\lineto(823.03289953,210.43661621)
\lineto(823.02156094,210.4422855)
\lineto(823.01022236,210.44795479)
\lineto(822.99888378,210.44795479)
\lineto(822.98187591,210.45362408)
\lineto(822.97053732,210.45362408)
\lineto(822.95919874,210.45929337)
\lineto(822.94219087,210.45929337)
\lineto(822.92518299,210.45929337)
\lineto(822.91384441,210.46496267)
\lineto(822.89683654,210.46496267)
\lineto(822.87982866,210.46496267)
\lineto(822.86282079,210.46496267)
\curveto(822.43762394,210.46496267)(819.75604913,210.20984456)(819.27415937,210.15315164)
\curveto(819.04738772,210.12480519)(818.87730898,209.9830729)(818.87730898,209.61456897)
\curveto(818.87730898,209.27441148)(819.13242709,209.27441148)(819.55762394,209.27441148)
\curveto(820.91258457,209.27441148)(820.96927748,209.08165558)(820.96927748,208.79819101)
\lineto(820.8899074,208.23126188)
\lineto(819.18912,201.53582881)
\curveto(818.68455307,202.57897841)(817.85683654,203.34433274)(816.59258457,203.34433274)
\lineto(816.62093102,202.7207107)
\curveto(818.45211213,202.7207107)(818.84896252,200.40763983)(818.84896252,200.24323038)
\curveto(818.84896252,200.06181306)(818.7979389,199.89740361)(818.76959244,199.75567133)
\lineto(817.35226961,194.21677369)
\curveto(817.21053732,193.71220676)(817.21053732,193.66118314)(816.78534047,193.17362408)
\curveto(815.54376567,191.62023826)(814.38723024,191.17236424)(813.59919874,191.17236424)
\curveto(812.1875452,191.17236424)(811.7906948,192.72008078)(811.7906948,193.82559259)
\curveto(811.7906948,195.23724613)(812.69211213,198.70685243)(813.34408063,200.01078944)
\curveto(814.2171515,201.67756109)(815.49274205,202.7207107)(816.62093102,202.7207107)
\lineto(816.59258457,203.34433274)
\curveto(813.28738772,203.34433274)(809.78943496,199.19441148)(809.78943496,195.06716739)
\curveto(809.78943496,192.40826975)(811.3371515,190.54874219)(813.54250583,190.54874219)
\curveto(814.10376567,190.54874219)(815.5210885,190.66779731)(817.21053732,192.66905715)
\curveto(817.44297827,191.47850597)(818.42376567,190.54874219)(819.7787263,190.54874219)
\curveto(820.77085228,190.54874219)(821.42282079,191.19504141)(821.8706948,192.10212802)
\curveto(822.34691528,193.11693117)(822.7210885,194.84606503)(822.7210885,194.90275794)
\curveto(822.7210885,195.18622251)(822.46597039,195.18622251)(822.38660031,195.18622251)
\curveto(822.10313575,195.18622251)(822.07478929,195.06716739)(821.98974992,194.670317)
\curveto(821.50786016,192.8334666)(820.99762394,191.17236424)(819.8410885,191.17236424)
\curveto(819.08140346,191.17236424)(818.9906948,191.90370282)(818.9906948,192.47063196)
\curveto(818.9906948,193.14527763)(819.04738772,193.34937211)(819.16077354,193.82559259)
\closepath
}
}
{
\newrgbcolor{curcolor}{0 0 0}
\pscustom[linewidth=0,linecolor=curcolor]
{
\newpath
\moveto(823.22565543,210.15315164)
\lineto(823.22565543,210.15315164)
\lineto(823.22565543,210.15882093)
\lineto(823.22565543,210.16449023)
\lineto(823.22565543,210.17015952)
\lineto(823.22565543,210.17582881)
\lineto(823.22565543,210.1814981)
\lineto(823.22565543,210.19283668)
\lineto(823.22565543,210.20417526)
\lineto(823.21998614,210.21551385)
\lineto(823.21998614,210.22685243)
\lineto(823.21431685,210.23819101)
\lineto(823.21431685,210.2495296)
\lineto(823.20864756,210.26653747)
\lineto(823.20297827,210.27787605)
\lineto(823.19730898,210.28921463)
\lineto(823.19163969,210.30622251)
\lineto(823.1803011,210.31756109)
\lineto(823.17463181,210.33456897)
\lineto(823.16329323,210.34590755)
\lineto(823.15195465,210.36291542)
\lineto(823.14061606,210.374254)
\lineto(823.12927748,210.3799233)
\lineto(823.12360819,210.38559259)
\lineto(823.1179389,210.39126188)
\lineto(823.10660031,210.39693117)
\lineto(823.10093102,210.40260046)
\lineto(823.08959244,210.40826975)
\lineto(823.08392315,210.41393904)
\lineto(823.07258457,210.41960834)
\lineto(823.06124598,210.42527763)
\lineto(823.05557669,210.43094692)
\lineto(823.04423811,210.43661621)
\lineto(823.03289953,210.43661621)
\lineto(823.02156094,210.4422855)
\lineto(823.01022236,210.44795479)
\lineto(822.99888378,210.44795479)
\lineto(822.98187591,210.45362408)
\lineto(822.97053732,210.45362408)
\lineto(822.95919874,210.45929337)
\lineto(822.94219087,210.45929337)
\lineto(822.92518299,210.45929337)
\lineto(822.91384441,210.46496267)
\lineto(822.89683654,210.46496267)
\lineto(822.87982866,210.46496267)
\lineto(822.86282079,210.46496267)
\curveto(822.43762394,210.46496267)(819.75604913,210.20984456)(819.27415937,210.15315164)
\curveto(819.04738772,210.12480519)(818.87730898,209.9830729)(818.87730898,209.61456897)
\curveto(818.87730898,209.27441148)(819.13242709,209.27441148)(819.55762394,209.27441148)
\curveto(820.91258457,209.27441148)(820.96927748,209.08165558)(820.96927748,208.79819101)
\lineto(820.8899074,208.23126188)
\lineto(819.18912,201.53582881)
\curveto(818.68455307,202.57897841)(817.85683654,203.34433274)(816.59258457,203.34433274)
\lineto(816.62093102,202.7207107)
\curveto(818.45211213,202.7207107)(818.84896252,200.40763983)(818.84896252,200.24323038)
\curveto(818.84896252,200.06181306)(818.7979389,199.89740361)(818.76959244,199.75567133)
\lineto(817.35226961,194.21677369)
\curveto(817.21053732,193.71220676)(817.21053732,193.66118314)(816.78534047,193.17362408)
\curveto(815.54376567,191.62023826)(814.38723024,191.17236424)(813.59919874,191.17236424)
\curveto(812.1875452,191.17236424)(811.7906948,192.72008078)(811.7906948,193.82559259)
\curveto(811.7906948,195.23724613)(812.69211213,198.70685243)(813.34408063,200.01078944)
\curveto(814.2171515,201.67756109)(815.49274205,202.7207107)(816.62093102,202.7207107)
\lineto(816.59258457,203.34433274)
\curveto(813.28738772,203.34433274)(809.78943496,199.19441148)(809.78943496,195.06716739)
\curveto(809.78943496,192.40826975)(811.3371515,190.54874219)(813.54250583,190.54874219)
\curveto(814.10376567,190.54874219)(815.5210885,190.66779731)(817.21053732,192.66905715)
\curveto(817.44297827,191.47850597)(818.42376567,190.54874219)(819.7787263,190.54874219)
\curveto(820.77085228,190.54874219)(821.42282079,191.19504141)(821.8706948,192.10212802)
\curveto(822.34691528,193.11693117)(822.7210885,194.84606503)(822.7210885,194.90275794)
\curveto(822.7210885,195.18622251)(822.46597039,195.18622251)(822.38660031,195.18622251)
\curveto(822.10313575,195.18622251)(822.07478929,195.06716739)(821.98974992,194.670317)
\curveto(821.50786016,192.8334666)(820.99762394,191.17236424)(819.8410885,191.17236424)
\curveto(819.08140346,191.17236424)(818.9906948,191.90370282)(818.9906948,192.47063196)
\curveto(818.9906948,193.14527763)(819.04738772,193.34937211)(819.16077354,193.82559259)
\closepath
}
}
{
\newrgbcolor{curcolor}{0 0 0}
\pscustom[linestyle=none,fillstyle=solid,fillcolor=curcolor]
{
\newpath
\moveto(833.88392315,201.53582881)
\lineto(833.83289953,201.63220676)
\lineto(833.78187591,201.72858471)
\lineto(833.73085228,201.81929337)
\lineto(833.67415937,201.91567133)
\lineto(833.62313575,202.0007107)
\lineto(833.56077354,202.09141936)
\lineto(833.50408063,202.17645873)
\lineto(833.44171843,202.25582881)
\lineto(833.37935622,202.34086818)
\lineto(833.31132472,202.41456897)
\lineto(833.24329323,202.49393904)
\lineto(833.16959244,202.56763983)
\lineto(833.10156094,202.63567133)
\lineto(833.02786016,202.70370282)
\lineto(832.94849008,202.76606503)
\lineto(832.86912,202.82842723)
\lineto(832.78974992,202.89078944)
\lineto(832.70471055,202.94748235)
\lineto(832.61967118,202.99850597)
\lineto(832.52896252,203.0438603)
\lineto(832.43825386,203.08921463)
\lineto(832.3475452,203.13456897)
\lineto(832.25116724,203.174254)
\lineto(832.15478929,203.20826975)
\lineto(832.05274205,203.23661621)
\lineto(831.9506948,203.26496267)
\lineto(831.84864756,203.28763983)
\lineto(831.74093102,203.30464771)
\lineto(831.63321449,203.32165558)
\lineto(831.51982866,203.33299416)
\lineto(831.40077354,203.33866345)
\lineto(831.28738772,203.34433274)
\lineto(831.31006488,202.7207107)
\curveto(833.14691528,202.7207107)(833.54376567,200.40763983)(833.54376567,200.24323038)
\curveto(833.54376567,200.06181306)(833.49274205,199.89740361)(833.46439559,199.75567133)
\lineto(832.04707276,194.21677369)
\curveto(831.90534047,193.71220676)(831.90534047,193.66118314)(831.48014362,193.17362408)
\curveto(830.23856882,191.62023826)(829.08203339,191.17236424)(828.29400189,191.17236424)
\curveto(826.88234835,191.17236424)(826.48549795,192.72008078)(826.48549795,193.82559259)
\curveto(826.48549795,195.23724613)(827.38691528,198.70685243)(828.03321449,200.01078944)
\curveto(828.91195465,201.67756109)(830.1875452,202.7207107)(831.31006488,202.7207107)
\lineto(831.28738772,203.34433274)
\curveto(827.98219087,203.34433274)(824.48423811,199.19441148)(824.48423811,195.06716739)
\curveto(824.48423811,192.40826975)(826.03195465,190.54874219)(828.23730898,190.54874219)
\curveto(828.79856882,190.54874219)(830.21589165,190.66779731)(831.90534047,192.66905715)
\curveto(832.13778142,191.47850597)(833.11856882,190.54874219)(834.47352945,190.54874219)
\curveto(835.46565543,190.54874219)(836.11195465,191.19504141)(836.56549795,192.10212802)
\curveto(837.04171843,193.11693117)(837.41589165,194.84606503)(837.41589165,194.90275794)
\curveto(837.41589165,195.18622251)(837.16077354,195.18622251)(837.08140346,195.18622251)
\curveto(836.7979389,195.18622251)(836.76959244,195.06716739)(836.67888378,194.670317)
\curveto(836.20266331,192.8334666)(835.6867578,191.17236424)(834.53589165,191.17236424)
\curveto(833.77620661,191.17236424)(833.68549795,191.90370282)(833.68549795,192.47063196)
\curveto(833.68549795,193.094254)(833.74219087,193.31535637)(834.0483326,194.56260046)
\curveto(834.36581291,195.75315164)(834.42250583,196.03661621)(834.67762394,197.10811227)
\lineto(835.6867578,201.05393904)
\curveto(835.89085228,201.84763983)(835.89085228,201.90433274)(835.89085228,202.01204928)
\curveto(835.89085228,202.49960834)(835.54502551,202.7830729)(835.06880504,202.7830729)
\curveto(834.39415937,202.7830729)(833.96896252,202.15378156)(833.88392315,201.53582881)
\closepath
}
}
{
\newrgbcolor{curcolor}{0 0 0}
\pscustom[linewidth=0,linecolor=curcolor]
{
\newpath
\moveto(833.88392315,201.53582881)
\lineto(833.83289953,201.63220676)
\lineto(833.78187591,201.72858471)
\lineto(833.73085228,201.81929337)
\lineto(833.67415937,201.91567133)
\lineto(833.62313575,202.0007107)
\lineto(833.56077354,202.09141936)
\lineto(833.50408063,202.17645873)
\lineto(833.44171843,202.25582881)
\lineto(833.37935622,202.34086818)
\lineto(833.31132472,202.41456897)
\lineto(833.24329323,202.49393904)
\lineto(833.16959244,202.56763983)
\lineto(833.10156094,202.63567133)
\lineto(833.02786016,202.70370282)
\lineto(832.94849008,202.76606503)
\lineto(832.86912,202.82842723)
\lineto(832.78974992,202.89078944)
\lineto(832.70471055,202.94748235)
\lineto(832.61967118,202.99850597)
\lineto(832.52896252,203.0438603)
\lineto(832.43825386,203.08921463)
\lineto(832.3475452,203.13456897)
\lineto(832.25116724,203.174254)
\lineto(832.15478929,203.20826975)
\lineto(832.05274205,203.23661621)
\lineto(831.9506948,203.26496267)
\lineto(831.84864756,203.28763983)
\lineto(831.74093102,203.30464771)
\lineto(831.63321449,203.32165558)
\lineto(831.51982866,203.33299416)
\lineto(831.40077354,203.33866345)
\lineto(831.28738772,203.34433274)
\lineto(831.31006488,202.7207107)
\curveto(833.14691528,202.7207107)(833.54376567,200.40763983)(833.54376567,200.24323038)
\curveto(833.54376567,200.06181306)(833.49274205,199.89740361)(833.46439559,199.75567133)
\lineto(832.04707276,194.21677369)
\curveto(831.90534047,193.71220676)(831.90534047,193.66118314)(831.48014362,193.17362408)
\curveto(830.23856882,191.62023826)(829.08203339,191.17236424)(828.29400189,191.17236424)
\curveto(826.88234835,191.17236424)(826.48549795,192.72008078)(826.48549795,193.82559259)
\curveto(826.48549795,195.23724613)(827.38691528,198.70685243)(828.03321449,200.01078944)
\curveto(828.91195465,201.67756109)(830.1875452,202.7207107)(831.31006488,202.7207107)
\lineto(831.28738772,203.34433274)
\curveto(827.98219087,203.34433274)(824.48423811,199.19441148)(824.48423811,195.06716739)
\curveto(824.48423811,192.40826975)(826.03195465,190.54874219)(828.23730898,190.54874219)
\curveto(828.79856882,190.54874219)(830.21589165,190.66779731)(831.90534047,192.66905715)
\curveto(832.13778142,191.47850597)(833.11856882,190.54874219)(834.47352945,190.54874219)
\curveto(835.46565543,190.54874219)(836.11195465,191.19504141)(836.56549795,192.10212802)
\curveto(837.04171843,193.11693117)(837.41589165,194.84606503)(837.41589165,194.90275794)
\curveto(837.41589165,195.18622251)(837.16077354,195.18622251)(837.08140346,195.18622251)
\curveto(836.7979389,195.18622251)(836.76959244,195.06716739)(836.67888378,194.670317)
\curveto(836.20266331,192.8334666)(835.6867578,191.17236424)(834.53589165,191.17236424)
\curveto(833.77620661,191.17236424)(833.68549795,191.90370282)(833.68549795,192.47063196)
\curveto(833.68549795,193.094254)(833.74219087,193.31535637)(834.0483326,194.56260046)
\curveto(834.36581291,195.75315164)(834.42250583,196.03661621)(834.67762394,197.10811227)
\lineto(835.6867578,201.05393904)
\curveto(835.89085228,201.84763983)(835.89085228,201.90433274)(835.89085228,202.01204928)
\curveto(835.89085228,202.49960834)(835.54502551,202.7830729)(835.06880504,202.7830729)
\curveto(834.39415937,202.7830729)(833.96896252,202.15378156)(833.88392315,201.53582881)
\closepath
}
}
{
\newrgbcolor{curcolor}{0 0 0}
\pscustom[linestyle=none,fillstyle=solid,fillcolor=curcolor]
{
\newpath
\moveto(844.06597039,202.15378156)
\lineto(846.72486803,202.15378156)
\curveto(847.28045858,202.15378156)(847.56392315,202.15378156)(847.56392315,202.7207107)
\curveto(847.56392315,203.03252172)(847.28045858,203.03252172)(846.77589165,203.03252172)
\lineto(844.28707276,203.03252172)
\curveto(845.3075452,207.04637999)(845.44927748,207.61330912)(845.44927748,207.77771857)
\curveto(845.44927748,208.25960834)(845.10912,208.5430729)(844.63289953,208.5430729)
\curveto(844.54219087,208.5430729)(843.75415937,208.51472645)(843.49904126,207.52260046)
\lineto(842.39919874,203.03252172)
\lineto(839.7403011,203.03252172)
\curveto(839.17337197,203.03252172)(838.8899074,203.03252172)(838.8899074,202.49960834)
\curveto(838.8899074,202.15378156)(839.12234835,202.15378156)(839.68927748,202.15378156)
\lineto(842.17242709,202.15378156)
\curveto(840.14282079,194.13740361)(840.02376567,193.66118314)(840.02376567,193.14527763)
\curveto(840.02376567,191.62023826)(841.10093102,190.54874219)(842.6203011,190.54874219)
\curveto(845.5003011,190.54874219)(847.11037984,194.670317)(847.11037984,194.90275794)
\curveto(847.11037984,195.18622251)(846.89494677,195.18622251)(846.77589165,195.18622251)
\curveto(846.52077354,195.18622251)(846.49242709,195.09551385)(846.3506948,194.78370282)
\curveto(845.13746646,191.8526792)(843.64077354,191.17236424)(842.68266331,191.17236424)
\curveto(842.09305701,191.17236424)(841.80959244,191.54086818)(841.80959244,192.47063196)
\curveto(841.80959244,193.14527763)(841.86061606,193.34937211)(841.97400189,193.82559259)
\closepath
}
}
{
\newrgbcolor{curcolor}{0 0 0}
\pscustom[linewidth=0,linecolor=curcolor]
{
\newpath
\moveto(844.06597039,202.15378156)
\lineto(846.72486803,202.15378156)
\curveto(847.28045858,202.15378156)(847.56392315,202.15378156)(847.56392315,202.7207107)
\curveto(847.56392315,203.03252172)(847.28045858,203.03252172)(846.77589165,203.03252172)
\lineto(844.28707276,203.03252172)
\curveto(845.3075452,207.04637999)(845.44927748,207.61330912)(845.44927748,207.77771857)
\curveto(845.44927748,208.25960834)(845.10912,208.5430729)(844.63289953,208.5430729)
\curveto(844.54219087,208.5430729)(843.75415937,208.51472645)(843.49904126,207.52260046)
\lineto(842.39919874,203.03252172)
\lineto(839.7403011,203.03252172)
\curveto(839.17337197,203.03252172)(838.8899074,203.03252172)(838.8899074,202.49960834)
\curveto(838.8899074,202.15378156)(839.12234835,202.15378156)(839.68927748,202.15378156)
\lineto(842.17242709,202.15378156)
\curveto(840.14282079,194.13740361)(840.02376567,193.66118314)(840.02376567,193.14527763)
\curveto(840.02376567,191.62023826)(841.10093102,190.54874219)(842.6203011,190.54874219)
\curveto(845.5003011,190.54874219)(847.11037984,194.670317)(847.11037984,194.90275794)
\curveto(847.11037984,195.18622251)(846.89494677,195.18622251)(846.77589165,195.18622251)
\curveto(846.52077354,195.18622251)(846.49242709,195.09551385)(846.3506948,194.78370282)
\curveto(845.13746646,191.8526792)(843.64077354,191.17236424)(842.68266331,191.17236424)
\curveto(842.09305701,191.17236424)(841.80959244,191.54086818)(841.80959244,192.47063196)
\curveto(841.80959244,193.14527763)(841.86061606,193.34937211)(841.97400189,193.82559259)
\closepath
}
}
{
\newrgbcolor{curcolor}{0 0 0}
\pscustom[linestyle=none,fillstyle=solid,fillcolor=curcolor]
{
\newpath
\moveto(858.98187591,201.53582881)
\lineto(858.93652157,201.63220676)
\lineto(858.88549795,201.72858471)
\lineto(858.83447433,201.81929337)
\lineto(858.77778142,201.91567133)
\lineto(858.7210885,202.0007107)
\lineto(858.66439559,202.09141936)
\lineto(858.60770268,202.17645873)
\lineto(858.54534047,202.25582881)
\lineto(858.47730898,202.34086818)
\lineto(858.41494677,202.41456897)
\lineto(858.34691528,202.49393904)
\lineto(858.27321449,202.56763983)
\lineto(858.1995137,202.63567133)
\lineto(858.12581291,202.70370282)
\lineto(858.05211213,202.76606503)
\lineto(857.97274205,202.82842723)
\lineto(857.88770268,202.89078944)
\lineto(857.8083326,202.94748235)
\lineto(857.72329323,202.99850597)
\lineto(857.63258457,203.0438603)
\lineto(857.54187591,203.08921463)
\lineto(857.45116724,203.13456897)
\lineto(857.35478929,203.174254)
\lineto(857.25841134,203.20826975)
\lineto(857.15636409,203.23661621)
\lineto(857.05431685,203.26496267)
\lineto(856.95226961,203.28763983)
\lineto(856.84455307,203.30464771)
\lineto(856.73116724,203.32165558)
\lineto(856.61778142,203.33299416)
\lineto(856.50439559,203.33866345)
\lineto(856.39100976,203.34433274)
\lineto(856.41368693,202.7207107)
\curveto(858.25053732,202.7207107)(858.64738772,200.40763983)(858.64738772,200.24323038)
\curveto(858.64738772,200.06181306)(858.5906948,199.89740361)(858.56801764,199.75567133)
\lineto(857.1506948,194.21677369)
\curveto(857.00896252,193.71220676)(857.00896252,193.66118314)(856.58376567,193.17362408)
\curveto(855.34219087,191.62023826)(854.17998614,191.17236424)(853.39195465,191.17236424)
\curveto(851.98597039,191.17236424)(851.58912,192.72008078)(851.58912,193.82559259)
\curveto(851.58912,195.23724613)(852.49053732,198.70685243)(853.13683654,200.01078944)
\curveto(854.01557669,201.67756109)(855.29116724,202.7207107)(856.41368693,202.7207107)
\lineto(856.39100976,203.34433274)
\curveto(853.08581291,203.34433274)(849.58786016,199.19441148)(849.58786016,195.06716739)
\curveto(849.58786016,192.40826975)(851.13557669,190.54874219)(853.34093102,190.54874219)
\curveto(853.89652157,190.54874219)(855.31384441,190.66779731)(857.00896252,192.66905715)
\curveto(857.24140346,191.47850597)(858.22219087,190.54874219)(859.5771515,190.54874219)
\curveto(860.56927748,190.54874219)(861.21557669,191.19504141)(861.66912,192.10212802)
\curveto(862.14534047,193.11693117)(862.5195137,194.84606503)(862.5195137,194.90275794)
\curveto(862.5195137,195.18622251)(862.2587263,195.18622251)(862.17935622,195.18622251)
\curveto(861.89589165,195.18622251)(861.87321449,195.06716739)(861.78250583,194.670317)
\curveto(861.30628535,192.8334666)(860.79037984,191.17236424)(859.6395137,191.17236424)
\curveto(858.87415937,191.17236424)(858.78912,191.90370282)(858.78912,192.47063196)
\curveto(858.78912,193.094254)(858.84014362,193.31535637)(859.15195465,194.56260046)
\curveto(859.46943496,195.75315164)(859.52612787,196.03661621)(859.78124598,197.10811227)
\lineto(860.79037984,201.05393904)
\curveto(860.99447433,201.84763983)(860.99447433,201.90433274)(860.99447433,202.01204928)
\curveto(860.99447433,202.49960834)(860.64864756,202.7830729)(860.17242709,202.7830729)
\curveto(859.49778142,202.7830729)(859.07258457,202.15378156)(858.98187591,201.53582881)
\closepath
}
}
{
\newrgbcolor{curcolor}{0 0 0}
\pscustom[linewidth=0,linecolor=curcolor]
{
\newpath
\moveto(858.98187591,201.53582881)
\lineto(858.93652157,201.63220676)
\lineto(858.88549795,201.72858471)
\lineto(858.83447433,201.81929337)
\lineto(858.77778142,201.91567133)
\lineto(858.7210885,202.0007107)
\lineto(858.66439559,202.09141936)
\lineto(858.60770268,202.17645873)
\lineto(858.54534047,202.25582881)
\lineto(858.47730898,202.34086818)
\lineto(858.41494677,202.41456897)
\lineto(858.34691528,202.49393904)
\lineto(858.27321449,202.56763983)
\lineto(858.1995137,202.63567133)
\lineto(858.12581291,202.70370282)
\lineto(858.05211213,202.76606503)
\lineto(857.97274205,202.82842723)
\lineto(857.88770268,202.89078944)
\lineto(857.8083326,202.94748235)
\lineto(857.72329323,202.99850597)
\lineto(857.63258457,203.0438603)
\lineto(857.54187591,203.08921463)
\lineto(857.45116724,203.13456897)
\lineto(857.35478929,203.174254)
\lineto(857.25841134,203.20826975)
\lineto(857.15636409,203.23661621)
\lineto(857.05431685,203.26496267)
\lineto(856.95226961,203.28763983)
\lineto(856.84455307,203.30464771)
\lineto(856.73116724,203.32165558)
\lineto(856.61778142,203.33299416)
\lineto(856.50439559,203.33866345)
\lineto(856.39100976,203.34433274)
\lineto(856.41368693,202.7207107)
\curveto(858.25053732,202.7207107)(858.64738772,200.40763983)(858.64738772,200.24323038)
\curveto(858.64738772,200.06181306)(858.5906948,199.89740361)(858.56801764,199.75567133)
\lineto(857.1506948,194.21677369)
\curveto(857.00896252,193.71220676)(857.00896252,193.66118314)(856.58376567,193.17362408)
\curveto(855.34219087,191.62023826)(854.17998614,191.17236424)(853.39195465,191.17236424)
\curveto(851.98597039,191.17236424)(851.58912,192.72008078)(851.58912,193.82559259)
\curveto(851.58912,195.23724613)(852.49053732,198.70685243)(853.13683654,200.01078944)
\curveto(854.01557669,201.67756109)(855.29116724,202.7207107)(856.41368693,202.7207107)
\lineto(856.39100976,203.34433274)
\curveto(853.08581291,203.34433274)(849.58786016,199.19441148)(849.58786016,195.06716739)
\curveto(849.58786016,192.40826975)(851.13557669,190.54874219)(853.34093102,190.54874219)
\curveto(853.89652157,190.54874219)(855.31384441,190.66779731)(857.00896252,192.66905715)
\curveto(857.24140346,191.47850597)(858.22219087,190.54874219)(859.5771515,190.54874219)
\curveto(860.56927748,190.54874219)(861.21557669,191.19504141)(861.66912,192.10212802)
\curveto(862.14534047,193.11693117)(862.5195137,194.84606503)(862.5195137,194.90275794)
\curveto(862.5195137,195.18622251)(862.2587263,195.18622251)(862.17935622,195.18622251)
\curveto(861.89589165,195.18622251)(861.87321449,195.06716739)(861.78250583,194.670317)
\curveto(861.30628535,192.8334666)(860.79037984,191.17236424)(859.6395137,191.17236424)
\curveto(858.87415937,191.17236424)(858.78912,191.90370282)(858.78912,192.47063196)
\curveto(858.78912,193.094254)(858.84014362,193.31535637)(859.15195465,194.56260046)
\curveto(859.46943496,195.75315164)(859.52612787,196.03661621)(859.78124598,197.10811227)
\lineto(860.79037984,201.05393904)
\curveto(860.99447433,201.84763983)(860.99447433,201.90433274)(860.99447433,202.01204928)
\curveto(860.99447433,202.49960834)(860.64864756,202.7830729)(860.17242709,202.7830729)
\curveto(859.49778142,202.7830729)(859.07258457,202.15378156)(858.98187591,201.53582881)
\closepath
}
}
{
\newrgbcolor{curcolor}{0 0 0}
\pscustom[linestyle=none,fillstyle=solid,fillcolor=curcolor]
{
\newpath
\moveto(45.13575767,233.11209303)
\lineto(45.13575767,235.03136355)
\curveto(44.17134811,233.52268324)(42.75337959,232.76834309)(40.88185212,232.76834309)
\curveto(39.66917871,232.76834309)(38.5519914,233.10254442)(37.53029018,233.77094709)
\curveto(36.51813756,234.43934976)(35.73037728,235.37033918)(35.16700932,236.56391538)
\curveto(34.61318996,237.76704018)(34.33628029,239.14681426)(34.33628029,240.70323761)
\curveto(34.33628029,242.22146653)(34.58931844,243.5964663)(35.09539474,244.82823693)
\curveto(35.60147105,246.06955617)(36.36058551,247.01964282)(37.37273812,247.67849687)
\curveto(38.38489073,248.33735093)(39.51640096,248.66677796)(40.76726881,248.66677796)
\curveto(41.68393532,248.66677796)(42.50034144,248.47103147)(43.21648715,248.07953847)
\curveto(43.93263287,247.69759409)(44.51509805,247.19629209)(44.9638827,246.57563247)
\lineto(44.9638827,254.10948539)
\lineto(47.52768436,254.10948539)
\lineto(47.52768436,233.11209303)
\closepath
\moveto(36.98601943,240.70323761)
\curveto(36.98601943,238.75532126)(37.39660964,237.29915831)(38.21779006,236.33474875)
\curveto(39.03897048,235.37033918)(40.00815435,234.8881344)(41.12534166,234.8881344)
\curveto(42.25207759,234.8881344)(43.20693854,235.34646766)(43.98992452,236.26313418)
\curveto(44.78245912,237.1893493)(45.17872641,238.59776921)(45.17872641,240.4883939)
\curveto(45.17872641,242.56999077)(44.77768481,244.0977683)(43.97560161,245.07172647)
\curveto(43.17351841,246.04568464)(42.18523732,246.53266373)(41.01075835,246.53266373)
\curveto(39.86492521,246.53266373)(38.90528995,246.06478186)(38.13185258,245.12901813)
\curveto(37.36796381,244.19325439)(36.98601943,242.71799422)(36.98601943,240.70323761)
\closepath
}
}
{
\newrgbcolor{curcolor}{0 0 0}
\pscustom[linestyle=none,fillstyle=solid,fillcolor=curcolor]
{
\newpath
\moveto(51.59539239,251.14464213)
\lineto(51.59539239,254.10948539)
\lineto(54.17351696,254.10948539)
\lineto(54.17351696,251.14464213)
\closepath
\moveto(51.59539239,233.11209303)
\lineto(51.59539239,248.32302802)
\lineto(54.17351696,248.32302802)
\lineto(54.17351696,233.11209303)
\closepath
}
}
{
\newrgbcolor{curcolor}{0 0 0}
\pscustom[linestyle=none,fillstyle=solid,fillcolor=curcolor]
{
\newpath
\moveto(58.71388027,233.11209303)
\lineto(58.71388027,246.31782002)
\lineto(56.4365369,246.31782002)
\lineto(56.4365369,248.32302802)
\lineto(58.71388027,248.32302802)
\lineto(58.71388027,249.94151733)
\curveto(58.71388027,250.96321855)(58.80459206,251.72233301)(58.98601564,252.21886071)
\curveto(59.23427949,252.88726337)(59.66874123,253.42675981)(60.28940085,253.83735002)
\curveto(60.91960907,254.25748884)(61.79808115,254.46755825)(62.92481708,254.46755825)
\curveto(63.6505114,254.46755825)(64.4525946,254.38162077)(65.33106668,254.20974579)
\lineto(64.94434799,251.96104825)
\curveto(64.40962586,252.05653435)(63.90354955,252.10427739)(63.42611908,252.10427739)
\curveto(62.6431331,252.10427739)(62.08931374,251.93717673)(61.76466102,251.60297539)
\curveto(61.44000829,251.26877406)(61.27768193,250.64334013)(61.27768193,249.72667362)
\lineto(61.27768193,248.32302802)
\lineto(64.24252519,248.32302802)
\lineto(64.24252519,246.31782002)
\lineto(61.27768193,246.31782002)
\lineto(61.27768193,233.11209303)
\closepath
}
}
{
\newrgbcolor{curcolor}{0 0 0}
\pscustom[linestyle=none,fillstyle=solid,fillcolor=curcolor]
{
\newpath
\moveto(66.33367018,233.11209303)
\lineto(66.33367018,246.31782002)
\lineto(64.0563268,246.31782002)
\lineto(64.0563268,248.32302802)
\lineto(66.33367018,248.32302802)
\lineto(66.33367018,249.94151733)
\curveto(66.33367018,250.96321855)(66.42438197,251.72233301)(66.60580555,252.21886071)
\curveto(66.8540694,252.88726337)(67.28853113,253.42675981)(67.90919075,253.83735002)
\curveto(68.53939898,254.25748884)(69.41787106,254.46755825)(70.54460698,254.46755825)
\curveto(71.27030131,254.46755825)(72.07238451,254.38162077)(72.95085658,254.20974579)
\lineto(72.5641379,251.96104825)
\curveto(72.02941576,252.05653435)(71.52333946,252.10427739)(71.04590898,252.10427739)
\curveto(70.262923,252.10427739)(69.70910365,251.93717673)(69.38445092,251.60297539)
\curveto(69.0597982,251.26877406)(68.89747184,250.64334013)(68.89747184,249.72667362)
\lineto(68.89747184,248.32302802)
\lineto(71.8623151,248.32302802)
\lineto(71.8623151,246.31782002)
\lineto(68.89747184,246.31782002)
\lineto(68.89747184,233.11209303)
\closepath
}
}
{
\newrgbcolor{curcolor}{0 0 0}
\pscustom[linestyle=none,fillstyle=solid,fillcolor=curcolor]
{
\newpath
\moveto(73.88184744,251.14464213)
\lineto(73.88184744,254.10948539)
\lineto(76.45997201,254.10948539)
\lineto(76.45997201,251.14464213)
\closepath
\moveto(73.88184744,233.11209303)
\lineto(73.88184744,248.32302802)
\lineto(76.45997201,248.32302802)
\lineto(76.45997201,233.11209303)
\closepath
}
}
{
\newrgbcolor{curcolor}{0 0 0}
\pscustom[linestyle=none,fillstyle=solid,fillcolor=curcolor]
{
\newpath
\moveto(90.31022962,238.68370669)
\lineto(92.84538545,238.35427966)
\curveto(92.56847577,236.60688412)(91.85710436,235.23665865)(90.71127122,234.24360326)
\curveto(89.57498668,233.26009648)(88.17611539,232.76834309)(86.51465733,232.76834309)
\curveto(84.43306045,232.76834309)(82.75727948,233.44629436)(81.48731441,234.80219692)
\curveto(80.22689795,236.16764808)(79.59668972,238.12033873)(79.59668972,240.66026887)
\curveto(79.59668972,242.30262971)(79.86882509,243.73969544)(80.41309584,244.97146607)
\curveto(80.95736658,246.2032367)(81.7833213,247.12467752)(82.89096001,247.73578853)
\curveto(84.00814733,248.35644815)(85.22082074,248.66677796)(86.52898024,248.66677796)
\curveto(88.18088969,248.66677796)(89.53201794,248.24663914)(90.58236499,247.4063615)
\curveto(91.63271204,246.57563247)(92.30588901,245.39160489)(92.60189591,243.85427876)
\lineto(90.0953859,243.46756007)
\curveto(89.85667066,244.48926129)(89.43175754,245.25792436)(88.82064653,245.77354927)
\curveto(88.21908413,246.28917419)(87.4886155,246.54698664)(86.62924064,246.54698664)
\curveto(85.33062975,246.54698664)(84.27550839,246.07910478)(83.46387658,245.14334104)
\curveto(82.65224477,244.21712592)(82.24642887,242.74664005)(82.24642887,240.73188344)
\curveto(82.24642887,238.688481)(82.63792186,237.20367222)(83.42090784,236.27745709)
\curveto(84.20389382,235.35124197)(85.22559504,234.8881344)(86.4860115,234.8881344)
\curveto(87.49816411,234.8881344)(88.34321605,235.19846421)(89.02116733,235.81912383)
\curveto(89.69911861,236.43978345)(90.12880604,237.39464441)(90.31022962,238.68370669)
\closepath
}
}
{
\newrgbcolor{curcolor}{0 0 0}
\pscustom[linestyle=none,fillstyle=solid,fillcolor=curcolor]
{
\newpath
\moveto(105.0198635,233.11209303)
\lineto(105.0198635,235.34646766)
\curveto(103.83583592,233.62771795)(102.22689522,232.76834309)(100.19304139,232.76834309)
\curveto(99.29547209,232.76834309)(98.45519445,232.94021806)(97.67220847,233.283968)
\curveto(96.8987711,233.62771795)(96.32108022,234.05740537)(95.93913584,234.57303029)
\curveto(95.56674007,235.09820381)(95.3041533,235.73796065)(95.15137555,236.49230081)
\curveto(95.04634085,236.99837711)(94.99382349,237.80046031)(94.99382349,238.89855041)
\lineto(94.99382349,248.32302802)
\lineto(97.57194807,248.32302802)
\lineto(97.57194807,239.88683149)
\curveto(97.57194807,238.54047755)(97.62446542,237.63335964)(97.72950013,237.16547778)
\curveto(97.89182649,236.4875265)(98.23557643,235.95280437)(98.76074996,235.56131138)
\curveto(99.28592348,235.17936699)(99.93522893,234.9883948)(100.7086663,234.9883948)
\curveto(101.48210367,234.9883948)(102.207798,235.1841413)(102.88574927,235.57563429)
\curveto(103.56370055,235.97667589)(104.04113103,236.51617233)(104.3180407,237.19412361)
\curveto(104.60449899,237.88162349)(104.74772813,238.87467888)(104.74772813,240.17328978)
\lineto(104.74772813,248.32302802)
\lineto(107.32585271,248.32302802)
\lineto(107.32585271,233.11209303)
\closepath
}
}
{
\newrgbcolor{curcolor}{0 0 0}
\pscustom[linestyle=none,fillstyle=solid,fillcolor=curcolor]
{
\newpath
\moveto(111.30762326,233.11209303)
\lineto(111.30762326,254.10948539)
\lineto(113.88574783,254.10948539)
\lineto(113.88574783,233.11209303)
\closepath
}
}
{
\newrgbcolor{curcolor}{0 0 0}
\pscustom[linestyle=none,fillstyle=solid,fillcolor=curcolor]
{
\newpath
\moveto(123.51074571,235.41808223)
\lineto(123.88314149,233.14073886)
\curveto(123.15744716,232.98796111)(122.50814171,232.91157223)(121.93522514,232.91157223)
\curveto(120.99946141,232.91157223)(120.27376708,233.05957568)(119.75814217,233.35558257)
\curveto(119.24251725,233.65158947)(118.87967009,234.03830816)(118.66960068,234.51573863)
\curveto(118.45953127,235.00271772)(118.35449657,236.01964463)(118.35449657,237.56651938)
\lineto(118.35449657,246.31782002)
\lineto(116.46387188,246.31782002)
\lineto(116.46387188,248.32302802)
\lineto(118.35449657,248.32302802)
\lineto(118.35449657,252.08995448)
\lineto(120.91829823,253.63682922)
\lineto(120.91829823,248.32302802)
\lineto(123.51074571,248.32302802)
\lineto(123.51074571,246.31782002)
\lineto(120.91829823,246.31782002)
\lineto(120.91829823,237.42329023)
\curveto(120.91829823,236.6880473)(120.96126697,236.21539113)(121.04720445,236.00532172)
\curveto(121.14269055,235.79525231)(121.290694,235.62815164)(121.4912148,235.50401972)
\curveto(121.70128421,235.37988779)(121.9972911,235.31782183)(122.37923548,235.31782183)
\curveto(122.66569377,235.31782183)(123.04286385,235.35124197)(123.51074571,235.41808223)
\closepath
}
}
{
\newrgbcolor{curcolor}{0 0 0}
\pscustom[linestyle=none,fillstyle=solid,fillcolor=curcolor]
{
}
}
{
\newrgbcolor{curcolor}{0 0 0}
\pscustom[linestyle=none,fillstyle=solid,fillcolor=curcolor]
{
\newpath
\moveto(134.18131251,227.28266691)
\lineto(134.18131251,248.32302802)
\lineto(136.53027046,248.32302802)
\lineto(136.53027046,246.34646584)
\curveto(137.08408981,247.11990322)(137.70952373,247.69759409)(138.40657223,248.07953847)
\curveto(139.10362073,248.47103147)(139.94867267,248.66677796)(140.94172806,248.66677796)
\curveto(142.24033896,248.66677796)(143.3861721,248.33257663)(144.37922749,247.66417396)
\curveto(145.37228289,246.99577129)(146.12184873,246.05045895)(146.62792504,244.82823693)
\curveto(147.13400134,243.61556352)(147.3870395,242.28353249)(147.3870395,240.83214384)
\curveto(147.3870395,239.27572048)(147.10535552,237.87207488)(146.54198755,236.62120703)
\curveto(145.9881682,235.37988779)(145.17653639,234.42502684)(144.10709212,233.75662417)
\curveto(143.04719646,233.09777012)(141.93000915,232.76834309)(140.75553018,232.76834309)
\curveto(139.89615532,232.76834309)(139.12271795,232.94976667)(138.43521806,233.31261383)
\curveto(137.75726678,233.67546099)(137.19867312,234.13379425)(136.75943709,234.6876136)
\lineto(136.75943709,227.28266691)
\closepath
\moveto(136.51594754,240.63162304)
\curveto(136.51594754,238.67415808)(136.91221484,237.22754374)(137.70474943,236.29178)
\curveto(138.49728402,235.35601627)(139.45691928,234.8881344)(140.5836552,234.8881344)
\curveto(141.72948835,234.8881344)(142.70822083,235.37033918)(143.51985264,236.33474875)
\curveto(144.34103306,237.30870692)(144.75162327,238.81261292)(144.75162327,240.84646675)
\curveto(144.75162327,242.78483449)(144.35058167,244.23622314)(143.54849846,245.2006327)
\curveto(142.75596387,246.16504226)(141.80587722,246.64724704)(140.69823852,246.64724704)
\curveto(139.60014842,246.64724704)(138.62619025,246.13162213)(137.776364,245.1003723)
\curveto(136.93608636,244.07867108)(136.51594754,242.58908799)(136.51594754,240.63162304)
\closepath
}
}
{
\newrgbcolor{curcolor}{0 0 0}
\pscustom[linestyle=none,fillstyle=solid,fillcolor=curcolor]
{
\newpath
\moveto(150.46647005,233.11209303)
\lineto(150.46647005,248.32302802)
\lineto(152.78678216,248.32302802)
\lineto(152.78678216,246.01703881)
\curveto(153.37879596,247.09603169)(153.9230667,247.8074031)(154.41959439,248.15115305)
\curveto(154.9256707,248.49490299)(155.47949005,248.66677796)(156.08105245,248.66677796)
\curveto(156.94997592,248.66677796)(157.8332223,248.38986828)(158.7307916,247.83604893)
\lineto(157.84277091,245.44412224)
\curveto(157.21256268,245.81651801)(156.58235445,246.0027159)(155.95214623,246.0027159)
\curveto(155.38877826,246.0027159)(154.88270196,245.83084093)(154.43391731,245.48709099)
\curveto(153.98513266,245.15288965)(153.66525424,244.68500779)(153.47428205,244.08344538)
\curveto(153.18782377,243.16677887)(153.04459462,242.16417487)(153.04459462,241.07563338)
\lineto(153.04459462,233.11209303)
\closepath
}
}
{
\newrgbcolor{curcolor}{0 0 0}
\pscustom[linestyle=none,fillstyle=solid,fillcolor=curcolor]
{
\newpath
\moveto(159.30370322,240.71756052)
\curveto(159.30370322,243.53440034)(160.0866892,245.62077152)(161.65266116,246.97667407)
\curveto(162.96082067,248.10341)(164.55543846,248.66677796)(166.43651454,248.66677796)
\curveto(168.52766003,248.66677796)(170.23686113,247.97927807)(171.56411786,246.6042783)
\curveto(172.89137458,245.23882714)(173.55500295,243.34820245)(173.55500295,240.93240424)
\curveto(173.55500295,238.97493928)(173.25899605,237.43283884)(172.66698226,236.30610292)
\curveto(172.08451708,235.1889156)(171.22991652,234.31999214)(170.1031806,233.69933252)
\curveto(168.98599328,233.0786729)(167.76377126,232.76834309)(166.43651454,232.76834309)
\curveto(164.30717461,232.76834309)(162.58365059,233.45106867)(161.26594248,234.81651983)
\curveto(159.95778297,236.181971)(159.30370322,238.14898456)(159.30370322,240.71756052)
\closepath
\moveto(161.95344236,240.71756052)
\curveto(161.95344236,238.76964418)(162.37835549,237.30870692)(163.22818174,236.33474875)
\curveto(164.07800798,235.37033918)(165.14745225,234.8881344)(166.43651454,234.8881344)
\curveto(167.71602822,234.8881344)(168.78069818,235.37511349)(169.63052443,236.34907166)
\curveto(170.48035068,237.32302983)(170.9052638,238.80783862)(170.9052638,240.80349801)
\curveto(170.9052638,242.68457409)(170.47557637,244.10731691)(169.61620151,245.07172647)
\curveto(168.76637526,246.04568464)(167.70647961,246.53266373)(166.43651454,246.53266373)
\curveto(165.14745225,246.53266373)(164.07800798,246.05045895)(163.22818174,245.08604939)
\curveto(162.37835549,244.12163982)(161.95344236,242.66547687)(161.95344236,240.71756052)
\closepath
}
}
{
\newrgbcolor{curcolor}{0 0 0}
\pscustom[linestyle=none,fillstyle=solid,fillcolor=curcolor]
{
\newpath
\moveto(178.95474561,233.11209303)
\lineto(176.56281893,233.11209303)
\lineto(176.56281893,254.10948539)
\lineto(179.1409435,254.10948539)
\lineto(179.1409435,246.61860122)
\curveto(180.22948499,247.98405238)(181.61880767,248.66677796)(183.30891156,248.66677796)
\curveto(184.2446753,248.66677796)(185.12792168,248.47580577)(185.95865071,248.09386139)
\curveto(186.79892835,247.72146562)(187.48642823,247.19151779)(188.02115037,246.5040179)
\curveto(188.56542111,245.82606662)(188.99033423,245.0048862)(189.29588974,244.04047664)
\curveto(189.60144524,243.07606708)(189.754223,242.04481725)(189.754223,240.94672715)
\curveto(189.754223,238.33995675)(189.10969185,236.32520014)(187.82062957,234.90245732)
\curveto(186.53156728,233.4797145)(184.98469253,232.76834309)(183.18000533,232.76834309)
\curveto(181.38486674,232.76834309)(179.97644683,233.51790894)(178.95474561,235.01704063)
\closepath
\moveto(178.92609979,240.83214384)
\curveto(178.92609979,239.00835942)(179.17436363,237.6906513)(179.67089133,236.87901949)
\curveto(180.48252314,235.55176277)(181.58061324,234.8881344)(182.96516162,234.8881344)
\curveto(184.09189754,234.8881344)(185.06585572,235.37511349)(185.88703614,236.34907166)
\curveto(186.70821656,237.33257844)(187.11880677,238.7935157)(187.11880677,240.73188344)
\curveto(187.11880677,242.71799422)(186.72253947,244.18370578)(185.93000488,245.12901813)
\curveto(185.1470189,246.07433047)(184.19693225,246.54698664)(183.07974493,246.54698664)
\curveto(181.95300901,246.54698664)(180.97905084,246.05523325)(180.15787042,245.07172647)
\curveto(179.33669,244.0977683)(178.92609979,242.68457409)(178.92609979,240.83214384)
\closepath
}
}
{
\newrgbcolor{curcolor}{0 0 0}
\pscustom[linestyle=none,fillstyle=solid,fillcolor=curcolor]
{
\newpath
\moveto(192.83365355,233.11209303)
\lineto(192.83365355,254.10948539)
\lineto(195.41177812,254.10948539)
\lineto(195.41177812,233.11209303)
\closepath
}
}
{
\newrgbcolor{curcolor}{0 0 0}
\pscustom[linestyle=none,fillstyle=solid,fillcolor=curcolor]
{
\newpath
\moveto(209.82062938,238.01052972)
\lineto(212.48469144,237.68110269)
\curveto(212.06455262,236.12467934)(211.28634095,234.91678023)(210.15005641,234.05740537)
\curveto(209.01377188,233.19803052)(207.56238323,232.76834309)(205.79589046,232.76834309)
\curveto(203.57106444,232.76834309)(201.80457168,233.45106867)(200.49641217,234.81651983)
\curveto(199.19780128,236.1915196)(198.54849583,238.11556443)(198.54849583,240.5886543)
\curveto(198.54849583,243.14768165)(199.20734989,245.13379243)(200.525058,246.54698664)
\curveto(201.84276612,247.96018086)(203.55196722,248.66677796)(205.65266132,248.66677796)
\curveto(207.68651515,248.66677796)(209.34797321,247.97450377)(210.6370355,246.58995539)
\curveto(211.92609778,245.205407)(212.57062893,243.25749066)(212.57062893,240.74620635)
\curveto(212.57062893,240.5934286)(212.56585462,240.36426197)(212.55630601,240.05870647)
\lineto(201.21255789,240.05870647)
\curveto(201.30804398,238.3876998)(201.78070016,237.10818612)(202.6305264,236.22016543)
\curveto(203.48035265,235.33214475)(204.54024831,234.8881344)(205.81021338,234.8881344)
\curveto(206.75552572,234.8881344)(207.56238323,235.13639825)(208.2307859,235.63292595)
\curveto(208.89918856,236.12945364)(209.42913639,236.92198823)(209.82062938,238.01052972)
\closepath
\moveto(201.35578703,242.17849778)
\lineto(209.84927521,242.17849778)
\curveto(209.7346919,243.45801146)(209.41003917,244.41764672)(208.87531704,245.05740356)
\curveto(208.05413662,246.05045895)(206.98946666,246.54698664)(205.68130715,246.54698664)
\curveto(204.49727957,246.54698664)(203.49944987,246.15071935)(202.68781806,245.35818476)
\curveto(201.88573486,244.56565017)(201.44172452,243.50575451)(201.35578703,242.17849778)
\closepath
}
}
{
\newrgbcolor{curcolor}{0 0 0}
\pscustom[linestyle=none,fillstyle=solid,fillcolor=curcolor]
{
\newpath
\moveto(215.72167405,233.11209303)
\lineto(215.72167405,248.32302802)
\lineto(218.02766325,248.32302802)
\lineto(218.02766325,246.18891379)
\curveto(218.50509373,246.93370533)(219.14007626,247.53049343)(219.93261086,247.97927807)
\curveto(220.72514545,248.43761133)(221.62748905,248.66677796)(222.63964166,248.66677796)
\curveto(223.76637758,248.66677796)(224.6878184,248.43283703)(225.40396412,247.96495516)
\curveto(226.12965844,247.49707329)(226.64050905,246.84299354)(226.93651595,246.0027159)
\curveto(228.13964075,247.77875727)(229.70561271,248.66677796)(231.63443184,248.66677796)
\curveto(233.14311215,248.66677796)(234.3032682,248.24663914)(235.11490001,247.4063615)
\curveto(235.92653182,246.57563247)(236.33234773,245.29134449)(236.33234773,243.55349756)
\lineto(236.33234773,233.11209303)
\lineto(233.76854607,233.11209303)
\lineto(233.76854607,242.6941227)
\curveto(233.76854607,243.72537253)(233.68260858,244.46538977)(233.51073361,244.91417441)
\curveto(233.34840725,245.37250767)(233.04762605,245.74012914)(232.60839001,246.01703881)
\curveto(232.16915397,246.29394849)(231.65352906,246.43240333)(231.06151527,246.43240333)
\curveto(229.992071,246.43240333)(229.10405031,246.07433047)(228.39745321,245.35818476)
\curveto(227.6908561,244.65158765)(227.33755755,243.51530312)(227.33755755,241.94933115)
\lineto(227.33755755,233.11209303)
\lineto(224.75943297,233.11209303)
\lineto(224.75943297,242.9949039)
\curveto(224.75943297,244.14073704)(224.54936357,245.0001119)(224.12922475,245.57302847)
\curveto(223.70908593,246.14594504)(223.02158604,246.43240333)(222.06672509,246.43240333)
\curveto(221.34103076,246.43240333)(220.66785379,246.24143114)(220.04719417,245.85948676)
\curveto(219.43608316,245.47754238)(218.99207282,244.91894872)(218.71516314,244.18370578)
\curveto(218.43825346,243.44846285)(218.29979863,242.38856719)(218.29979863,241.00401881)
\lineto(218.29979863,233.11209303)
\closepath
}
}
{
\newrgbcolor{curcolor}{0 0 0}
\pscustom[linestyle=none,fillstyle=solid,fillcolor=curcolor]
{
\newpath
\moveto(380.08137926,238.22477754)
\lineto(382.74544132,237.89535051)
\curveto(382.3253025,236.33892715)(381.54709082,235.13102805)(380.41080629,234.27165319)
\curveto(379.27452175,233.41227833)(377.82313311,232.9825909)(376.05664034,232.9825909)
\curveto(373.83181432,232.9825909)(372.06532156,233.66531648)(370.75716205,235.03076765)
\curveto(369.45855115,236.40576742)(368.80924571,238.32981224)(368.80924571,240.80290211)
\curveto(368.80924571,243.36192947)(369.46809976,245.34804025)(370.78580788,246.76123446)
\curveto(372.10351599,248.17442867)(373.8127171,248.88102578)(375.9134112,248.88102578)
\curveto(377.94726503,248.88102578)(379.60872309,248.18875159)(380.89778537,246.8042032)
\curveto(382.18684766,245.41965482)(382.83137881,243.47173848)(382.83137881,240.96045417)
\curveto(382.83137881,240.80767642)(382.8266045,240.57850979)(382.81705589,240.27295428)
\lineto(371.47330777,240.27295428)
\curveto(371.56879386,238.60194761)(372.04145003,237.32243394)(372.89127628,236.43441325)
\curveto(373.74110253,235.54639256)(374.80099819,235.10238222)(376.07096326,235.10238222)
\curveto(377.0162756,235.10238222)(377.82313311,235.35064607)(378.49153577,235.84717376)
\curveto(379.15993844,236.34370146)(379.68988627,237.13623605)(380.08137926,238.22477754)
\closepath
\moveto(371.61653691,242.3927456)
\lineto(380.11002509,242.3927456)
\curveto(379.99544177,243.67225928)(379.67078905,244.63189453)(379.13606692,245.27165137)
\curveto(378.3148865,246.26470676)(377.25021653,246.76123446)(375.94205703,246.76123446)
\curveto(374.75802944,246.76123446)(373.76019975,246.36496716)(372.94856794,245.57243257)
\curveto(372.14648474,244.77989798)(371.70247439,243.72000232)(371.61653691,242.3927456)
\closepath
}
}
{
\newrgbcolor{curcolor}{0 0 0}
\pscustom[linestyle=none,fillstyle=solid,fillcolor=curcolor]
{
\newpath
\moveto(395.90819993,235.20264262)
\curveto(394.95333898,234.39101081)(394.03189816,233.81809424)(393.14387747,233.4838929)
\curveto(392.2654054,233.14969157)(391.32009305,232.9825909)(390.30794044,232.9825909)
\curveto(388.63693377,232.9825909)(387.35264579,233.38840681)(386.4550765,234.20003862)
\curveto(385.5575072,235.02121904)(385.10872255,236.06679178)(385.10872255,237.33675685)
\curveto(385.10872255,238.08154839)(385.27582322,238.75949967)(385.61002455,239.37061068)
\curveto(385.9537745,239.9912703)(386.39778484,240.487798)(386.94205558,240.86019377)
\curveto(387.49587493,241.23258954)(388.11653455,241.51427352)(388.80403444,241.70524571)
\curveto(389.31011075,241.83892625)(390.07399951,241.96783247)(391.09570073,242.0919644)
\curveto(393.17729761,242.34022825)(394.70984944,242.63623514)(395.69335622,242.97998508)
\curveto(395.70290483,243.33328364)(395.70767913,243.55767596)(395.70767913,243.65316206)
\curveto(395.70767913,244.70350911)(395.46418959,245.44352634)(394.9772105,245.87321377)
\curveto(394.31835645,246.45567896)(393.33962397,246.74691155)(392.04101307,246.74691155)
\curveto(390.82833966,246.74691155)(389.93077037,246.53206783)(389.34830518,246.1023804)
\curveto(388.77538861,245.68224158)(388.35047549,244.93267573)(388.07356581,243.85368286)
\lineto(385.55273289,244.1974328)
\curveto(385.78189952,245.27642568)(386.1590696,246.14534915)(386.68424312,246.8042032)
\curveto(387.20941665,247.47260587)(387.96853111,247.98345648)(388.9615865,248.33675503)
\curveto(389.95464189,248.6996022)(391.10524934,248.88102578)(392.41340884,248.88102578)
\curveto(393.71201974,248.88102578)(394.76714109,248.72824802)(395.57877291,248.42269252)
\curveto(396.39040472,248.11713701)(396.98719281,247.73041833)(397.36913719,247.26253646)
\curveto(397.75108157,246.8042032)(398.01844264,246.22173802)(398.17122039,245.51514092)
\curveto(398.25715788,245.07590488)(398.30012662,244.28337029)(398.30012662,243.13753714)
\lineto(398.30012662,239.70003771)
\curveto(398.30012662,237.30333672)(398.35264397,235.7851078)(398.45767868,235.14535096)
\curveto(398.57226199,234.51514273)(398.79188001,233.90880603)(399.11653274,233.32634085)
\lineto(396.42382485,233.32634085)
\curveto(396.15646378,233.86106298)(395.98458881,234.4864969)(395.90819993,235.20264262)
\closepath
\moveto(395.69335622,240.96045417)
\curveto(394.75759249,240.57850979)(393.35394688,240.25385706)(391.48241942,239.986496)
\curveto(390.42252376,239.83371824)(389.67295791,239.66184327)(389.23372187,239.47087108)
\curveto(388.79448583,239.27989889)(388.45551019,238.99821491)(388.21679495,238.62581914)
\curveto(387.97807972,238.26297198)(387.8587221,237.85715607)(387.8587221,237.40837142)
\curveto(387.8587221,236.72087154)(388.11653455,236.14795496)(388.63215947,235.68962171)
\curveto(389.15733299,235.23128845)(389.92122176,235.00212182)(390.92382576,235.00212182)
\curveto(391.91688115,235.00212182)(392.80012753,235.21696553)(393.5735649,235.64665296)
\curveto(394.34700228,236.085889)(394.91514454,236.6826771)(395.2779917,237.43701725)
\curveto(395.55490138,238.01948243)(395.69335622,238.87885729)(395.69335622,240.01514182)
\closepath
}
}
{
\newrgbcolor{curcolor}{0 0 0}
\pscustom[linestyle=none,fillstyle=solid,fillcolor=curcolor]
{
\newpath
\moveto(401.26497025,237.86670468)
\lineto(403.814449,238.26774628)
\curveto(403.95767814,237.24604506)(404.35394544,236.46305908)(405.00325089,235.91878833)
\curveto(405.66210494,235.37451759)(406.57877146,235.10238222)(407.75325043,235.10238222)
\curveto(408.93727801,235.10238222)(409.81575009,235.34109746)(410.38866666,235.81852793)
\curveto(410.96158324,236.30550702)(411.24804152,236.87364929)(411.24804152,237.52295474)
\curveto(411.24804152,238.10541992)(410.99500337,238.56375318)(410.48892706,238.89795451)
\curveto(410.13562851,239.12712114)(409.25715643,239.41835373)(407.85351083,239.77165228)
\curveto(405.96288615,240.24908276)(404.64995233,240.65967297)(403.9147094,241.00342291)
\curveto(403.18901508,241.35672146)(402.63519572,241.83892625)(402.25325134,242.45003726)
\curveto(401.88085557,243.07069688)(401.69465768,243.75342246)(401.69465768,244.498214)
\curveto(401.69465768,245.17616528)(401.84743544,245.8015992)(402.15299094,246.37451577)
\curveto(402.46809506,246.95698096)(402.89300818,247.43918574)(403.42773031,247.82113012)
\curveto(403.82877191,248.11713701)(404.37304266,248.36540086)(405.06054254,248.56592166)
\curveto(405.75759104,248.77599107)(406.50238258,248.88102578)(407.29491718,248.88102578)
\curveto(408.48849337,248.88102578)(409.53406611,248.70915081)(410.43163541,248.36540086)
\curveto(411.33875331,248.02165092)(412.00715598,247.55376905)(412.43684341,246.96175526)
\curveto(412.86653084,246.37929008)(413.16253773,245.5963041)(413.3248641,244.61279732)
\lineto(410.80403118,244.26904737)
\curveto(410.68944786,245.05203335)(410.35524653,245.66314436)(409.80142718,246.1023804)
\curveto(409.25715643,246.54161644)(408.48371906,246.76123446)(407.48111506,246.76123446)
\curveto(406.29708748,246.76123446)(405.45203554,246.56548796)(404.94595923,246.17399497)
\curveto(404.43988292,245.78250198)(404.18684477,245.32416873)(404.18684477,244.7989952)
\curveto(404.18684477,244.46479387)(404.29187948,244.16401267)(404.50194889,243.8966516)
\curveto(404.7120183,243.61974192)(405.04144533,243.39057529)(405.49022997,243.20915171)
\curveto(405.74804243,243.11366562)(406.50715689,242.8940476)(407.76757335,242.55029766)
\curveto(409.59135777,242.06331857)(410.86132284,241.66227697)(411.57746855,241.34717285)
\curveto(412.30316288,241.04161735)(412.87130514,240.5928327)(413.28189535,240.00081891)
\curveto(413.69248556,239.40880512)(413.89778067,238.67356219)(413.89778067,237.79509011)
\curveto(413.89778067,236.93571525)(413.64474252,236.12408344)(413.13866621,235.36019468)
\curveto(412.64213851,234.60585452)(411.92121849,234.01861504)(410.97590615,233.59847622)
\curveto(410.03059381,233.18788601)(408.96114954,232.9825909)(407.76757335,232.9825909)
\curveto(405.79101117,232.9825909)(404.28233087,233.39318111)(403.24153243,234.21436153)
\curveto(402.2102826,235.03554195)(401.55142854,236.25298967)(401.26497025,237.86670468)
\closepath
}
}
{
\newrgbcolor{curcolor}{0 0 0}
\pscustom[linestyle=none,fillstyle=solid,fillcolor=curcolor]
{
\newpath
\moveto(416.84830191,227.4682689)
\lineto(416.56184363,229.88884141)
\curveto(417.12521159,229.73606366)(417.61696498,229.65967479)(418.0371038,229.65967479)
\curveto(418.61002037,229.65967479)(419.06835363,229.75516088)(419.41210357,229.94613307)
\curveto(419.75585352,230.13710526)(420.0375375,230.40446633)(420.25715552,230.74821627)
\curveto(420.41948188,231.00602873)(420.68206864,231.64578557)(421.0449158,232.66748679)
\curveto(421.09265885,232.81071593)(421.16904773,233.02078534)(421.27408243,233.29769502)
\lineto(415.50194797,248.53727583)
\lineto(418.28059334,248.53727583)
\lineto(421.44595741,239.72868354)
\curveto(421.85654762,238.61149622)(422.22416908,237.43701725)(422.54882181,236.20524662)
\curveto(422.8448287,237.3892742)(423.19812725,238.54465596)(423.60871746,239.67139188)
\lineto(426.86001901,248.53727583)
\lineto(429.43814358,248.53727583)
\lineto(423.65168621,233.06852839)
\curveto(423.03102659,231.39752172)(422.54882181,230.24691427)(422.20507186,229.61670604)
\curveto(421.74673861,228.76687979)(421.22156508,228.14622017)(420.62955129,227.75472718)
\curveto(420.0375375,227.35368558)(419.33094039,227.15316478)(418.50975997,227.15316478)
\curveto(418.01323228,227.15316478)(417.45941292,227.25819949)(416.84830191,227.4682689)
\closepath
}
}
{
\newrgbcolor{curcolor}{0 0 0}
\pscustom[linestyle=none,fillstyle=solid,fillcolor=curcolor]
{
}
}
{
\newrgbcolor{curcolor}{0 0 0}
\pscustom[linestyle=none,fillstyle=solid,fillcolor=curcolor]
{
\newpath
\moveto(439.77928643,227.49691473)
\lineto(439.77928643,248.53727583)
\lineto(442.12824438,248.53727583)
\lineto(442.12824438,246.56071366)
\curveto(442.68206373,247.33415103)(443.30749766,247.91184191)(444.00454615,248.29378629)
\curveto(444.70159465,248.68527928)(445.54664659,248.88102578)(446.53970198,248.88102578)
\curveto(447.83831288,248.88102578)(448.98414602,248.54682444)(449.97720141,247.87842178)
\curveto(450.97025681,247.21001911)(451.71982265,246.26470676)(452.22589896,245.04248474)
\curveto(452.73197527,243.82981133)(452.98501342,242.4977803)(452.98501342,241.04639165)
\curveto(452.98501342,239.4899683)(452.70332944,238.0863227)(452.13996147,236.83545485)
\curveto(451.58614212,235.59413561)(450.77451031,234.63927466)(449.70506604,233.97087199)
\curveto(448.64517038,233.31201793)(447.52798307,232.9825909)(446.3535041,232.9825909)
\curveto(445.49412924,232.9825909)(444.72069187,233.16401448)(444.03319198,233.52686165)
\curveto(443.3552407,233.88970881)(442.79664705,234.34804207)(442.35741101,234.90186142)
\lineto(442.35741101,227.49691473)
\closepath
\moveto(442.11392146,240.84587085)
\curveto(442.11392146,238.8884059)(442.51018876,237.44179156)(443.30272335,236.50602782)
\curveto(444.09525794,235.57026409)(445.0548932,235.10238222)(446.18162913,235.10238222)
\curveto(447.32746227,235.10238222)(448.30619475,235.584587)(449.11782656,236.54899656)
\curveto(449.93900698,237.52295474)(450.34959719,239.02686074)(450.34959719,241.06071457)
\curveto(450.34959719,242.9990823)(449.94855559,244.45047095)(449.14647239,245.41488052)
\curveto(448.35393779,246.37929008)(447.40385115,246.86149486)(446.29621244,246.86149486)
\curveto(445.19812234,246.86149486)(444.22416417,246.34586995)(443.37433792,245.31462012)
\curveto(442.53406028,244.2929189)(442.11392146,242.80333581)(442.11392146,240.84587085)
\closepath
}
}
{
\newrgbcolor{curcolor}{0 0 0}
\pscustom[linestyle=none,fillstyle=solid,fillcolor=curcolor]
{
\newpath
\moveto(456.06444397,233.32634085)
\lineto(456.06444397,248.53727583)
\lineto(458.38475609,248.53727583)
\lineto(458.38475609,246.23128663)
\curveto(458.97676988,247.31027951)(459.52104062,248.02165092)(460.01756832,248.36540086)
\curveto(460.52364462,248.70915081)(461.07746397,248.88102578)(461.67902637,248.88102578)
\curveto(462.54794984,248.88102578)(463.43119622,248.6041161)(464.32876552,248.05029675)
\lineto(463.44074483,245.65837006)
\curveto(462.8105366,246.03076583)(462.18032838,246.21696372)(461.55012015,246.21696372)
\curveto(460.98675218,246.21696372)(460.48067588,246.04508875)(460.03189123,245.7013388)
\curveto(459.58310658,245.36713747)(459.26322816,244.8992556)(459.07225597,244.2976932)
\curveto(458.78579769,243.38102669)(458.64256854,242.37842268)(458.64256854,241.2898812)
\lineto(458.64256854,233.32634085)
\closepath
}
}
{
\newrgbcolor{curcolor}{0 0 0}
\pscustom[linestyle=none,fillstyle=solid,fillcolor=curcolor]
{
\newpath
\moveto(464.90167714,240.93180834)
\curveto(464.90167714,243.74864815)(465.68466312,245.83501934)(467.25063508,247.19092189)
\curveto(468.55879459,248.31765781)(470.15341238,248.88102578)(472.03448846,248.88102578)
\curveto(474.12563395,248.88102578)(475.83483505,248.19352589)(477.16209178,246.81852612)
\curveto(478.4893485,245.45307495)(479.15297687,243.56245027)(479.15297687,241.14665205)
\curveto(479.15297687,239.1891871)(478.85696997,237.64708666)(478.26495618,236.52035074)
\curveto(477.682491,235.40316342)(476.82789045,234.53423995)(475.70115452,233.91358033)
\curveto(474.58396721,233.29292071)(473.36174519,232.9825909)(472.03448846,232.9825909)
\curveto(469.90514853,232.9825909)(468.18162451,233.66531648)(466.8639164,235.03076765)
\curveto(465.55575689,236.39621881)(464.90167714,238.36323238)(464.90167714,240.93180834)
\closepath
\moveto(467.55141628,240.93180834)
\curveto(467.55141628,238.983892)(467.97632941,237.52295474)(468.82615566,236.54899656)
\curveto(469.67598191,235.584587)(470.74542617,235.10238222)(472.03448846,235.10238222)
\curveto(473.31400214,235.10238222)(474.3786721,235.58936131)(475.22849835,236.56331948)
\curveto(476.0783246,237.53727765)(476.50323772,239.02208643)(476.50323772,241.01774583)
\curveto(476.50323772,242.8988219)(476.07355029,244.32156472)(475.21417543,245.28597429)
\curveto(474.36434919,246.25993246)(473.30445353,246.74691155)(472.03448846,246.74691155)
\curveto(470.74542617,246.74691155)(469.67598191,246.26470676)(468.82615566,245.3002972)
\curveto(467.97632941,244.33588764)(467.55141628,242.87972468)(467.55141628,240.93180834)
\closepath
}
}
{
\newrgbcolor{curcolor}{0 0 0}
\pscustom[linestyle=none,fillstyle=solid,fillcolor=curcolor]
{
\newpath
\moveto(484.55271954,233.32634085)
\lineto(482.16079285,233.32634085)
\lineto(482.16079285,254.32373321)
\lineto(484.73891742,254.32373321)
\lineto(484.73891742,246.83284903)
\curveto(485.82745891,248.1983002)(487.2167816,248.88102578)(488.90688548,248.88102578)
\curveto(489.84264922,248.88102578)(490.7258956,248.69005359)(491.55662463,248.3081092)
\curveto(492.39690227,247.93571343)(493.08440215,247.4057656)(493.61912429,246.71826572)
\curveto(494.16339503,246.04031444)(494.58830815,245.21913402)(494.89386366,244.25472446)
\curveto(495.19941917,243.29031489)(495.35219692,242.25906507)(495.35219692,241.16097497)
\curveto(495.35219692,238.55420457)(494.70766577,236.53944795)(493.41860349,235.11670513)
\curveto(492.1295412,233.69396231)(490.58266646,232.9825909)(488.77797925,232.9825909)
\curveto(486.98284066,232.9825909)(485.57442076,233.73215675)(484.55271954,235.23128845)
\closepath
\moveto(484.52407371,241.04639165)
\curveto(484.52407371,239.22260723)(484.77233755,237.90489912)(485.26886525,237.09326731)
\curveto(486.08049706,235.76601058)(487.17858716,235.10238222)(488.56313554,235.10238222)
\curveto(489.68987146,235.10238222)(490.66382964,235.58936131)(491.48501006,236.56331948)
\curveto(492.30619048,237.54682626)(492.71678069,239.00776352)(492.71678069,240.94613125)
\curveto(492.71678069,242.93224204)(492.32051339,244.3979536)(491.5279788,245.34326594)
\curveto(490.74499282,246.28857829)(489.79490617,246.76123446)(488.67771885,246.76123446)
\curveto(487.55098293,246.76123446)(486.57702476,246.26948107)(485.75584434,245.28597429)
\curveto(484.93466392,244.31201611)(484.52407371,242.8988219)(484.52407371,241.04639165)
\closepath
}
}
{
\newrgbcolor{curcolor}{0 0 0}
\pscustom[linestyle=none,fillstyle=solid,fillcolor=curcolor]
{
\newpath
\moveto(498.43162026,233.32634085)
\lineto(498.43162026,254.32373321)
\lineto(501.00974483,254.32373321)
\lineto(501.00974483,233.32634085)
\closepath
}
}
{
\newrgbcolor{curcolor}{0 0 0}
\pscustom[linestyle=none,fillstyle=solid,fillcolor=curcolor]
{
\newpath
\moveto(515.4185961,238.22477754)
\lineto(518.08265815,237.89535051)
\curveto(517.66251934,236.33892715)(516.88430766,235.13102805)(515.74802312,234.27165319)
\curveto(514.61173859,233.41227833)(513.16034994,232.9825909)(511.39385718,232.9825909)
\curveto(509.16903116,232.9825909)(507.40253839,233.66531648)(506.09437889,235.03076765)
\curveto(504.79576799,236.40576742)(504.14646254,238.32981224)(504.14646254,240.80290211)
\curveto(504.14646254,243.36192947)(504.8053166,245.34804025)(506.12302471,246.76123446)
\curveto(507.44073283,248.17442867)(509.14993394,248.88102578)(511.25062803,248.88102578)
\curveto(513.28448186,248.88102578)(514.94593992,248.18875159)(516.23500221,246.8042032)
\curveto(517.5240645,245.41965482)(518.16859564,243.47173848)(518.16859564,240.96045417)
\curveto(518.16859564,240.80767642)(518.16382134,240.57850979)(518.15427273,240.27295428)
\lineto(506.8105246,240.27295428)
\curveto(506.9060107,238.60194761)(507.37866687,237.32243394)(508.22849312,236.43441325)
\curveto(509.07831936,235.54639256)(510.13821502,235.10238222)(511.40818009,235.10238222)
\curveto(512.35349243,235.10238222)(513.16034994,235.35064607)(513.82875261,235.84717376)
\curveto(514.49715528,236.34370146)(515.0271031,237.13623605)(515.4185961,238.22477754)
\closepath
\moveto(506.95375374,242.3927456)
\lineto(515.44724192,242.3927456)
\curveto(515.33265861,243.67225928)(515.00800589,244.63189453)(514.47328375,245.27165137)
\curveto(513.65210333,246.26470676)(512.58743337,246.76123446)(511.27927386,246.76123446)
\curveto(510.09524628,246.76123446)(509.09741658,246.36496716)(508.28578477,245.57243257)
\curveto(507.48370157,244.77989798)(507.03969123,243.72000232)(506.95375374,242.3927456)
\closepath
}
}
{
\newrgbcolor{curcolor}{0 0 0}
\pscustom[linestyle=none,fillstyle=solid,fillcolor=curcolor]
{
\newpath
\moveto(521.31964076,233.32634085)
\lineto(521.31964076,248.53727583)
\lineto(523.62562997,248.53727583)
\lineto(523.62562997,246.4031616)
\curveto(524.10306044,247.14795315)(524.73804298,247.74474124)(525.53057757,248.19352589)
\curveto(526.32311216,248.65185915)(527.22545576,248.88102578)(528.23760837,248.88102578)
\curveto(529.3643443,248.88102578)(530.28578512,248.64708484)(531.00193083,248.17920298)
\curveto(531.72762516,247.71132111)(532.23847577,247.05724136)(532.53448266,246.21696372)
\curveto(533.73760746,247.99300509)(535.30357943,248.88102578)(537.23239855,248.88102578)
\curveto(538.74107886,248.88102578)(539.90123492,248.46088696)(540.71286673,247.62060932)
\curveto(541.52449854,246.78988029)(541.93031444,245.50559231)(541.93031444,243.76774537)
\lineto(541.93031444,233.32634085)
\lineto(539.36651278,233.32634085)
\lineto(539.36651278,242.90837051)
\curveto(539.36651278,243.93962034)(539.2805753,244.67963758)(539.10870032,245.12842223)
\curveto(538.94637396,245.58675549)(538.64559276,245.95437695)(538.20635672,246.23128663)
\curveto(537.76712069,246.50819631)(537.25149577,246.64665115)(536.65948198,246.64665115)
\curveto(535.59003771,246.64665115)(534.70201702,246.28857829)(533.99541992,245.57243257)
\curveto(533.28882281,244.86583547)(532.93552426,243.72955093)(532.93552426,242.16357897)
\lineto(532.93552426,233.32634085)
\lineto(530.35739969,233.32634085)
\lineto(530.35739969,243.20915171)
\curveto(530.35739969,244.35498486)(530.14733028,245.21435972)(529.72719146,245.78727629)
\curveto(529.30705264,246.36019286)(528.61955275,246.64665115)(527.6646918,246.64665115)
\curveto(526.93899747,246.64665115)(526.2658205,246.45567896)(525.64516088,246.07373457)
\curveto(525.03404987,245.69179019)(524.59003953,245.13319653)(524.31312985,244.3979536)
\curveto(524.03622018,243.66271067)(523.89776534,242.60281501)(523.89776534,241.21826663)
\lineto(523.89776534,233.32634085)
\closepath
}
}
{
\newrgbcolor{curcolor}{0 0 0}
\pscustom[linestyle=none,fillstyle=solid,fillcolor=curcolor]
{
\newpath
\moveto(664.01680118,238.08192644)
\lineto(666.68086324,237.75249941)
\curveto(666.26072442,236.19607605)(665.48251274,234.98817695)(664.34622821,234.12880209)
\curveto(663.20994368,233.26942723)(661.75855503,232.8397398)(659.99206226,232.8397398)
\curveto(657.76723624,232.8397398)(656.00074348,233.52246538)(654.69258397,234.88791655)
\curveto(653.39397307,236.26291632)(652.74466763,238.18696114)(652.74466763,240.66005101)
\curveto(652.74466763,243.21907837)(653.40352168,245.20518915)(654.7212298,246.61838336)
\curveto(656.03893792,248.03157757)(657.74813902,248.73817468)(659.84883312,248.73817468)
\curveto(661.88268695,248.73817468)(663.54414501,248.04590048)(664.8332073,246.6613521)
\curveto(666.12226958,245.27680372)(666.76680073,243.32888737)(666.76680073,240.81760307)
\curveto(666.76680073,240.66482531)(666.76202642,240.43565869)(666.75247781,240.13010318)
\lineto(655.40872969,240.13010318)
\curveto(655.50421578,238.45909651)(655.97687195,237.17958284)(656.8266982,236.29156215)
\curveto(657.67652445,235.40354146)(658.73642011,234.95953112)(660.00638518,234.95953112)
\curveto(660.95169752,234.95953112)(661.75855503,235.20779497)(662.42695769,235.70432266)
\curveto(663.09536036,236.20085036)(663.62530819,236.99338495)(664.01680118,238.08192644)
\closepath
\moveto(655.55195883,242.2498945)
\lineto(664.04544701,242.2498945)
\curveto(663.9308637,243.52940818)(663.60621097,244.48904343)(663.07148884,245.12880027)
\curveto(662.25030842,246.12185566)(661.18563845,246.61838336)(659.87747895,246.61838336)
\curveto(658.69345137,246.61838336)(657.69562167,246.22211606)(656.88398986,245.42958147)
\curveto(656.08190666,244.63704688)(655.63789632,243.57715122)(655.55195883,242.2498945)
\closepath
}
}
{
\newrgbcolor{curcolor}{0 0 0}
\pscustom[linestyle=none,fillstyle=solid,fillcolor=curcolor]
{
\newpath
\moveto(679.61445523,227.35406363)
\lineto(679.61445523,234.80197906)
\curveto(679.21341363,234.2386111)(678.65004566,233.77072923)(677.92435134,233.39833346)
\curveto(677.20820562,233.02593769)(676.44431686,232.8397398)(675.63268505,232.8397398)
\curveto(673.82799785,232.8397398)(672.27157449,233.56065982)(670.96341499,235.00249986)
\curveto(669.66480409,236.4443399)(669.01549864,238.42090207)(669.01549864,240.93218638)
\curveto(669.01549864,242.45996391)(669.27808541,243.83018938)(669.80325893,245.04286279)
\curveto(670.33798106,246.2555362)(671.10664413,247.17220271)(672.10924813,247.79286233)
\curveto(673.12140074,248.42307056)(674.22903945,248.73817468)(675.43216425,248.73817468)
\curveto(677.31324033,248.73817468)(678.79327481,247.94564008)(679.87226768,246.3605709)
\lineto(679.87226768,248.39442473)
\lineto(682.1925798,248.39442473)
\lineto(682.1925798,227.35406363)
\closepath
\moveto(671.66523779,240.83192598)
\curveto(671.66523779,238.87446103)(672.075828,237.40397516)(672.89700842,236.42046838)
\curveto(673.71818884,235.4465102)(674.70169562,234.95953112)(675.84752877,234.95953112)
\curveto(676.94561886,234.95953112)(677.89093121,235.42263868)(678.6834658,236.34885381)
\curveto(679.47600039,237.28461754)(679.87226768,238.70258606)(679.87226768,240.60275935)
\curveto(679.87226768,242.62706457)(679.45212886,244.15006779)(678.61185123,245.17176901)
\curveto(677.7811222,246.19347023)(676.80238972,246.70432084)(675.67565379,246.70432084)
\curveto(674.55846648,246.70432084)(673.60837983,246.22689037)(672.82539385,245.27202941)
\curveto(672.05195648,244.32671707)(671.66523779,242.84668259)(671.66523779,240.83192598)
\closepath
}
}
{
\newrgbcolor{curcolor}{0 0 0}
\pscustom[linestyle=none,fillstyle=solid,fillcolor=curcolor]
{
\newpath
\moveto(696.20039036,233.18348975)
\lineto(696.20039036,235.41786438)
\curveto(695.01636278,233.69911466)(693.40742207,232.8397398)(691.37356824,232.8397398)
\curveto(690.47599894,232.8397398)(689.6357213,233.01161477)(688.85273532,233.35536472)
\curveto(688.07929795,233.69911466)(687.50160707,234.12880209)(687.11966269,234.644427)
\curveto(686.74726692,235.16960053)(686.48468016,235.80935737)(686.33190241,236.56369752)
\curveto(686.2268677,237.06977383)(686.17435035,237.87185703)(686.17435035,238.96994712)
\lineto(686.17435035,248.39442473)
\lineto(688.75247492,248.39442473)
\lineto(688.75247492,239.95822821)
\curveto(688.75247492,238.61187427)(688.80499227,237.70475636)(688.91002698,237.23687449)
\curveto(689.07235334,236.55892322)(689.41610328,236.02420108)(689.94127681,235.63270809)
\curveto(690.46645033,235.25076371)(691.11575578,235.05979152)(691.88919315,235.05979152)
\curveto(692.66263053,235.05979152)(693.38832485,235.25553801)(694.06627613,235.64703101)
\curveto(694.7442274,236.04807261)(695.22165788,236.58756904)(695.49856756,237.26552032)
\curveto(695.78502584,237.95302021)(695.92825499,238.9460756)(695.92825499,240.2446865)
\lineto(695.92825499,248.39442473)
\lineto(698.50637956,248.39442473)
\lineto(698.50637956,233.18348975)
\closepath
}
}
{
\newrgbcolor{curcolor}{0 0 0}
\pscustom[linestyle=none,fillstyle=solid,fillcolor=curcolor]
{
\newpath
\moveto(702.55976468,251.21603885)
\lineto(702.55976468,254.18088211)
\lineto(705.13788925,254.18088211)
\lineto(705.13788925,251.21603885)
\closepath
\moveto(702.55976468,233.18348975)
\lineto(702.55976468,248.39442473)
\lineto(705.13788925,248.39442473)
\lineto(705.13788925,233.18348975)
\closepath
}
}
{
\newrgbcolor{curcolor}{0 0 0}
\pscustom[linestyle=none,fillstyle=solid,fillcolor=curcolor]
{
\newpath
\moveto(713.28762697,233.18348975)
\lineto(707.50116959,248.39442473)
\lineto(710.22252331,248.39442473)
\lineto(713.48814777,239.28505124)
\curveto(713.84144632,238.30154446)(714.16609904,237.27984324)(714.46210594,236.21994758)
\curveto(714.69127257,237.02203078)(715.01115099,237.98644034)(715.4217412,239.11317627)
\lineto(718.80194897,248.39442473)
\lineto(721.45168812,248.39442473)
\lineto(715.69387657,233.18348975)
\closepath
}
}
{
\newrgbcolor{curcolor}{0 0 0}
\pscustom[linestyle=none,fillstyle=solid,fillcolor=curcolor]
{
\newpath
\moveto(733.654812,235.05979152)
\curveto(732.69995105,234.24815971)(731.77851023,233.67524314)(730.89048954,233.3410418)
\curveto(730.01201747,233.00684047)(729.06670512,232.8397398)(728.05455251,232.8397398)
\curveto(726.38354584,232.8397398)(725.09925786,233.24555571)(724.20168856,234.05718752)
\curveto(723.30411927,234.87836794)(722.85533462,235.92394068)(722.85533462,237.19390575)
\curveto(722.85533462,237.93869729)(723.02243529,238.61664857)(723.35663662,239.22775958)
\curveto(723.70038656,239.8484192)(724.14439691,240.3449469)(724.68866765,240.71734267)
\curveto(725.242487,241.08973844)(725.86314662,241.37142242)(726.55064651,241.56239461)
\curveto(727.05672281,241.69607514)(727.82061158,241.82498137)(728.8423128,241.9491133)
\curveto(730.92390968,242.19737715)(732.45646151,242.49338404)(733.43996829,242.83713398)
\curveto(733.4495169,243.19043254)(733.4542912,243.41482486)(733.4542912,243.51031096)
\curveto(733.4542912,244.560658)(733.21080166,245.30067524)(732.72382257,245.73036267)
\curveto(732.06496851,246.31282785)(731.08623604,246.60406044)(729.78762514,246.60406044)
\curveto(728.57495173,246.60406044)(727.67738243,246.38921673)(727.09491725,245.9595293)
\curveto(726.52200068,245.53939048)(726.09708756,244.78982463)(725.82017788,243.71083176)
\lineto(723.29934496,244.0545817)
\curveto(723.52851159,245.13357458)(723.90568167,246.00249804)(724.43085519,246.6613521)
\curveto(724.95602872,247.32975477)(725.71514318,247.84060538)(726.70819857,248.19390393)
\curveto(727.70125396,248.55675109)(728.85186141,248.73817468)(730.16002091,248.73817468)
\curveto(731.45863181,248.73817468)(732.51375316,248.58539692)(733.32538497,248.27984142)
\curveto(734.13701678,247.97428591)(734.73380488,247.58756723)(735.11574926,247.11968536)
\curveto(735.49769364,246.6613521)(735.76505471,246.07888692)(735.91783246,245.37228982)
\curveto(736.00376995,244.93305378)(736.04673869,244.14051919)(736.04673869,242.99468604)
\lineto(736.04673869,239.55718661)
\curveto(736.04673869,237.16048562)(736.09925604,235.6422567)(736.20429075,235.00249986)
\curveto(736.31887406,234.37229163)(736.53849208,233.76595493)(736.86314481,233.18348975)
\lineto(734.17043692,233.18348975)
\curveto(733.90307585,233.71821188)(733.73120088,234.3436458)(733.654812,235.05979152)
\closepath
\moveto(733.43996829,240.81760307)
\curveto(732.50420455,240.43565869)(731.10055895,240.11100596)(729.22903148,239.8436449)
\curveto(728.16913583,239.69086714)(727.41956998,239.51899217)(726.98033394,239.32801998)
\curveto(726.5410979,239.13704779)(726.20212226,238.85536381)(725.96340702,238.48296804)
\curveto(725.72469178,238.12012087)(725.60533417,237.71430497)(725.60533417,237.26552032)
\curveto(725.60533417,236.57802043)(725.86314662,236.00510386)(726.37877154,235.5467706)
\curveto(726.90394506,235.08843735)(727.66783382,234.85927072)(728.67043783,234.85927072)
\curveto(729.66349322,234.85927072)(730.5467396,235.07411443)(731.32017697,235.50380186)
\curveto(732.09361434,235.9430379)(732.66175661,236.539826)(733.02460377,237.29416615)
\curveto(733.30151345,237.87663133)(733.43996829,238.73600619)(733.43996829,239.87229072)
\closepath
}
}
{
\newrgbcolor{curcolor}{0 0 0}
\pscustom[linestyle=none,fillstyle=solid,fillcolor=curcolor]
{
\newpath
\moveto(739.98553689,233.18348975)
\lineto(739.98553689,254.18088211)
\lineto(742.56366146,254.18088211)
\lineto(742.56366146,233.18348975)
\closepath
}
}
{
\newrgbcolor{curcolor}{0 0 0}
\pscustom[linestyle=none,fillstyle=solid,fillcolor=curcolor]
{
\newpath
\moveto(756.97251273,238.08192644)
\lineto(759.63657479,237.75249941)
\curveto(759.21643597,236.19607605)(758.43822429,234.98817695)(757.30193975,234.12880209)
\curveto(756.16565522,233.26942723)(754.71426657,232.8397398)(752.94777381,232.8397398)
\curveto(750.72294779,232.8397398)(748.95645502,233.52246538)(747.64829552,234.88791655)
\curveto(746.34968462,236.26291632)(745.70037917,238.18696114)(745.70037917,240.66005101)
\curveto(745.70037917,243.21907837)(746.35923323,245.20518915)(747.67694134,246.61838336)
\curveto(748.99464946,248.03157757)(750.70385057,248.73817468)(752.80454466,248.73817468)
\curveto(754.8383985,248.73817468)(756.49985655,248.04590048)(757.78891884,246.6613521)
\curveto(759.07798113,245.27680372)(759.72251227,243.32888737)(759.72251227,240.81760307)
\curveto(759.72251227,240.66482531)(759.71773797,240.43565869)(759.70818936,240.13010318)
\lineto(748.36444123,240.13010318)
\curveto(748.45992733,238.45909651)(748.9325835,237.17958284)(749.78240975,236.29156215)
\curveto(750.632236,235.40354146)(751.69213165,234.95953112)(752.96209672,234.95953112)
\curveto(753.90740907,234.95953112)(754.71426657,235.20779497)(755.38266924,235.70432266)
\curveto(756.05107191,236.20085036)(756.58101973,236.99338495)(756.97251273,238.08192644)
\closepath
\moveto(748.50767037,242.2498945)
\lineto(757.00115855,242.2498945)
\curveto(756.88657524,243.52940818)(756.56192252,244.48904343)(756.02720038,245.12880027)
\curveto(755.20601996,246.12185566)(754.14135,246.61838336)(752.83319049,246.61838336)
\curveto(751.64916291,246.61838336)(750.65133321,246.22211606)(749.8397014,245.42958147)
\curveto(749.0376182,244.63704688)(748.59360786,243.57715122)(748.50767037,242.2498945)
\closepath
}
}
{
\newrgbcolor{curcolor}{0 0 0}
\pscustom[linestyle=none,fillstyle=solid,fillcolor=curcolor]
{
\newpath
\moveto(762.87355739,233.18348975)
\lineto(762.87355739,248.39442473)
\lineto(765.19386951,248.39442473)
\lineto(765.19386951,246.23166467)
\curveto(766.31105683,247.90267134)(767.92477184,248.73817468)(770.03501454,248.73817468)
\curveto(770.95168106,248.73817468)(771.7919587,248.57107401)(772.55584746,248.23687268)
\curveto(773.32928483,247.91221995)(773.90697571,247.48253252)(774.28892009,246.94781039)
\curveto(774.67086447,246.41308825)(774.93822554,245.77810572)(775.09100329,245.04286279)
\curveto(775.18648939,244.56543231)(775.23423244,243.72992898)(775.23423244,242.53635278)
\lineto(775.23423244,233.18348975)
\lineto(772.65610786,233.18348975)
\lineto(772.65610786,242.43609238)
\curveto(772.65610786,243.48643943)(772.55584746,244.26942541)(772.35532666,244.78505033)
\curveto(772.15480586,245.31022385)(771.796733,245.72558837)(771.28110809,246.03114387)
\curveto(770.77503178,246.34624799)(770.17824369,246.50380004)(769.4907438,246.50380004)
\curveto(768.3926537,246.50380004)(767.44256706,246.1552758)(766.64048386,245.4582273)
\curveto(765.84794926,244.7611788)(765.45168197,243.43869638)(765.45168197,241.49078004)
\lineto(765.45168197,233.18348975)
\closepath
}
}
{
\newrgbcolor{curcolor}{0 0 0}
\pscustom[linestyle=none,fillstyle=solid,fillcolor=curcolor]
{
\newpath
\moveto(784.81626608,235.48947895)
\lineto(785.18866185,233.21213557)
\curveto(784.46296753,233.05935782)(783.81366208,232.98296895)(783.24074551,232.98296895)
\curveto(782.30498177,232.98296895)(781.57928745,233.13097239)(781.06366253,233.42697929)
\curveto(780.54803762,233.72298618)(780.18519046,234.10970487)(779.97512105,234.58713535)
\curveto(779.76505164,235.07411443)(779.66001693,236.09104135)(779.66001693,237.63791609)
\lineto(779.66001693,246.38921673)
\lineto(777.76939224,246.38921673)
\lineto(777.76939224,248.39442473)
\lineto(779.66001693,248.39442473)
\lineto(779.66001693,252.16135119)
\lineto(782.22381859,253.70822594)
\lineto(782.22381859,248.39442473)
\lineto(784.81626608,248.39442473)
\lineto(784.81626608,246.38921673)
\lineto(782.22381859,246.38921673)
\lineto(782.22381859,237.49468695)
\curveto(782.22381859,236.75944402)(782.26678733,236.28678784)(782.35272482,236.07671843)
\curveto(782.44821091,235.86664902)(782.59621436,235.69954836)(782.79673516,235.57541643)
\curveto(783.00680457,235.45128451)(783.30281147,235.38921855)(783.68475585,235.38921855)
\curveto(783.97121414,235.38921855)(784.34838421,235.42263868)(784.81626608,235.48947895)
\closepath
}
}
{
\newrgbcolor{curcolor}{0 0 0}
\pscustom[linestyle=none,fillstyle=solid,fillcolor=curcolor]
{
}
}
{
\newrgbcolor{curcolor}{0 0 0}
\pscustom[linestyle=none,fillstyle=solid,fillcolor=curcolor]
{
\newpath
\moveto(795.48683288,227.35406363)
\lineto(795.48683288,248.39442473)
\lineto(797.83579082,248.39442473)
\lineto(797.83579082,246.41786256)
\curveto(798.38961018,247.19129993)(799.0150441,247.76899081)(799.7120926,248.15093519)
\curveto(800.40914109,248.54242818)(801.25419304,248.73817468)(802.24724843,248.73817468)
\curveto(803.54585932,248.73817468)(804.69169247,248.40397334)(805.68474786,247.73557067)
\curveto(806.67780325,247.06716801)(807.4273691,246.12185566)(807.9334454,244.89963364)
\curveto(808.43952171,243.68696023)(808.69255986,242.3549292)(808.69255986,240.90354055)
\curveto(808.69255986,239.3471172)(808.41087588,237.9434716)(807.84750792,236.69260375)
\curveto(807.29368857,235.45128451)(806.48205676,234.49642356)(805.41261249,233.82802089)
\curveto(804.35271683,233.16916683)(803.23552951,232.8397398)(802.06105054,232.8397398)
\curveto(801.20167568,232.8397398)(800.42823831,233.02116338)(799.74073842,233.38401055)
\curveto(799.06278715,233.74685771)(798.50419349,234.20519097)(798.06495745,234.75901032)
\lineto(798.06495745,227.35406363)
\closepath
\moveto(797.82146791,240.70301975)
\curveto(797.82146791,238.7455548)(798.2177352,237.29894045)(799.0102698,236.36317672)
\curveto(799.80280439,235.42741299)(800.76243964,234.95953112)(801.88917557,234.95953112)
\curveto(803.03500871,234.95953112)(804.01374119,235.4417359)(804.825373,236.40614546)
\curveto(805.64655342,237.38010364)(806.05714363,238.88400964)(806.05714363,240.91786347)
\curveto(806.05714363,242.8562312)(805.65610203,244.30761985)(804.85401883,245.27202941)
\curveto(804.06148424,246.23643898)(803.11139759,246.71864376)(802.00375888,246.71864376)
\curveto(800.90566879,246.71864376)(799.93171062,246.20301884)(799.08188437,245.17176901)
\curveto(798.24160673,244.15006779)(797.82146791,242.66048471)(797.82146791,240.70301975)
\closepath
}
}
{
\newrgbcolor{curcolor}{0 0 0}
\pscustom[linestyle=none,fillstyle=solid,fillcolor=curcolor]
{
\newpath
\moveto(811.77199041,233.18348975)
\lineto(811.77199041,248.39442473)
\lineto(814.09230253,248.39442473)
\lineto(814.09230253,246.08843553)
\curveto(814.68431632,247.16742841)(815.22858706,247.87879982)(815.72511476,248.22254976)
\curveto(816.23119107,248.5662997)(816.78501042,248.73817468)(817.38657282,248.73817468)
\curveto(818.25549629,248.73817468)(819.13874267,248.461265)(820.03631196,247.90744565)
\lineto(819.14829128,245.51551896)
\curveto(818.51808305,245.88791473)(817.88787482,246.07411262)(817.25766659,246.07411262)
\curveto(816.69429863,246.07411262)(816.18822232,245.90223764)(815.73943767,245.5584877)
\curveto(815.29065303,245.22428637)(814.97077461,244.7564045)(814.77980242,244.1548421)
\curveto(814.49334413,243.23817558)(814.35011499,242.23557158)(814.35011499,241.1470301)
\lineto(814.35011499,233.18348975)
\closepath
}
}
{
\newrgbcolor{curcolor}{0 0 0}
\pscustom[linestyle=none,fillstyle=solid,fillcolor=curcolor]
{
\newpath
\moveto(820.60923079,240.78895724)
\curveto(820.60923079,243.60579705)(821.39221677,245.69216823)(822.95818874,247.04807079)
\curveto(824.26634824,248.17480671)(825.86096603,248.73817468)(827.74204211,248.73817468)
\curveto(829.8331876,248.73817468)(831.54238871,248.05067479)(832.86964543,246.67567502)
\curveto(834.19690216,245.31022385)(834.86053052,243.41959917)(834.86053052,241.00380095)
\curveto(834.86053052,239.046336)(834.56452362,237.50423556)(833.97250983,236.37749963)
\curveto(833.39004465,235.26031232)(832.5354441,234.39138885)(831.40870817,233.77072923)
\curveto(830.29152086,233.15006961)(829.06929884,232.8397398)(827.74204211,232.8397398)
\curveto(825.61270219,232.8397398)(823.88917817,233.52246538)(822.57147005,234.88791655)
\curveto(821.26331054,236.25336771)(820.60923079,238.22038127)(820.60923079,240.78895724)
\closepath
\moveto(823.25896994,240.78895724)
\curveto(823.25896994,238.84104089)(823.68388306,237.38010364)(824.53370931,236.40614546)
\curveto(825.38353556,235.4417359)(826.45297983,234.95953112)(827.74204211,234.95953112)
\curveto(829.02155579,234.95953112)(830.08622575,235.4465102)(830.936052,236.42046838)
\curveto(831.78587825,237.39442655)(832.21079137,238.87923533)(832.21079137,240.87489472)
\curveto(832.21079137,242.7559708)(831.78110395,244.17871362)(830.92172909,245.14312319)
\curveto(830.07190284,246.11708136)(829.01200718,246.60406044)(827.74204211,246.60406044)
\curveto(826.45297983,246.60406044)(825.38353556,246.12185566)(824.53370931,245.1574461)
\curveto(823.68388306,244.19303654)(823.25896994,242.73687358)(823.25896994,240.78895724)
\closepath
}
}
{
\newrgbcolor{curcolor}{0 0 0}
\pscustom[linestyle=none,fillstyle=solid,fillcolor=curcolor]
{
\newpath
\moveto(840.26025877,233.18348975)
\lineto(837.86833208,233.18348975)
\lineto(837.86833208,254.18088211)
\lineto(840.44645666,254.18088211)
\lineto(840.44645666,246.68999793)
\curveto(841.53499814,248.05544909)(842.92432083,248.73817468)(844.61442472,248.73817468)
\curveto(845.55018845,248.73817468)(846.43343483,248.54720248)(847.26416386,248.1652581)
\curveto(848.1044415,247.79286233)(848.79194139,247.2629145)(849.32666352,246.57541462)
\curveto(849.87093427,245.89746334)(850.29584739,245.07628292)(850.6014029,244.11187336)
\curveto(850.9069584,243.14746379)(851.05973615,242.11621396)(851.05973615,241.01812387)
\curveto(851.05973615,238.41135346)(850.41520501,236.39659685)(849.12614272,234.97385403)
\curveto(847.83708044,233.55111121)(846.29020569,232.8397398)(844.48551849,232.8397398)
\curveto(842.6903799,232.8397398)(841.28195999,233.58930565)(840.26025877,235.08843735)
\closepath
\moveto(840.23161294,240.90354055)
\curveto(840.23161294,239.07975613)(840.47987679,237.76204802)(840.97640449,236.95041621)
\curveto(841.7880363,235.62315948)(842.88612639,234.95953112)(844.27067477,234.95953112)
\curveto(845.3974107,234.95953112)(846.37136887,235.4465102)(847.19254929,236.42046838)
\curveto(848.01372971,237.40397516)(848.42431992,238.86491242)(848.42431992,240.80328015)
\curveto(848.42431992,242.78939094)(848.02805263,244.2551025)(847.23551803,245.20041484)
\curveto(846.45253205,246.14572719)(845.5024454,246.61838336)(844.38525809,246.61838336)
\curveto(843.25852216,246.61838336)(842.28456399,246.12662997)(841.46338357,245.14312319)
\curveto(840.64220315,244.16916501)(840.23161294,242.7559708)(840.23161294,240.90354055)
\closepath
}
}
{
\newrgbcolor{curcolor}{0 0 0}
\pscustom[linestyle=none,fillstyle=solid,fillcolor=curcolor]
{
\newpath
\moveto(854.1391667,233.18348975)
\lineto(854.1391667,254.18088211)
\lineto(856.71729128,254.18088211)
\lineto(856.71729128,233.18348975)
\closepath
}
}
{
\newrgbcolor{curcolor}{0 0 0}
\pscustom[linestyle=none,fillstyle=solid,fillcolor=curcolor]
{
\newpath
\moveto(871.12614254,238.08192644)
\lineto(873.7902046,237.75249941)
\curveto(873.37006578,236.19607605)(872.5918541,234.98817695)(871.45556957,234.12880209)
\curveto(870.31928503,233.26942723)(868.86789638,232.8397398)(867.10140362,232.8397398)
\curveto(864.8765776,232.8397398)(863.11008484,233.52246538)(861.80192533,234.88791655)
\curveto(860.50331443,236.26291632)(859.85400898,238.18696114)(859.85400898,240.66005101)
\curveto(859.85400898,243.21907837)(860.51286304,245.20518915)(861.83057116,246.61838336)
\curveto(863.14827927,248.03157757)(864.85748038,248.73817468)(866.95817448,248.73817468)
\curveto(868.99202831,248.73817468)(870.65348637,248.04590048)(871.94254865,246.6613521)
\curveto(873.23161094,245.27680372)(873.87614209,243.32888737)(873.87614209,240.81760307)
\curveto(873.87614209,240.66482531)(873.87136778,240.43565869)(873.86181917,240.13010318)
\lineto(862.51807104,240.13010318)
\curveto(862.61355714,238.45909651)(863.08621331,237.17958284)(863.93603956,236.29156215)
\curveto(864.78586581,235.40354146)(865.84576147,234.95953112)(867.11572654,234.95953112)
\curveto(868.06103888,234.95953112)(868.86789638,235.20779497)(869.53629905,235.70432266)
\curveto(870.20470172,236.20085036)(870.73464955,236.99338495)(871.12614254,238.08192644)
\closepath
\moveto(862.66130019,242.2498945)
\lineto(871.15478837,242.2498945)
\curveto(871.04020505,243.52940818)(870.71555233,244.48904343)(870.1808302,245.12880027)
\curveto(869.35964978,246.12185566)(868.29497981,246.61838336)(866.98682031,246.61838336)
\curveto(865.80279272,246.61838336)(864.80496303,246.22211606)(863.99333122,245.42958147)
\curveto(863.19124802,244.63704688)(862.74723767,243.57715122)(862.66130019,242.2498945)
\closepath
}
}
{
\newrgbcolor{curcolor}{0 0 0}
\pscustom[linestyle=none,fillstyle=solid,fillcolor=curcolor]
{
\newpath
\moveto(877.02718721,233.18348975)
\lineto(877.02718721,248.39442473)
\lineto(879.33317641,248.39442473)
\lineto(879.33317641,246.2603105)
\curveto(879.81060689,247.00510205)(880.44558942,247.60189014)(881.23812401,248.05067479)
\curveto(882.0306586,248.50900805)(882.9330022,248.73817468)(883.94515482,248.73817468)
\curveto(885.07189074,248.73817468)(885.99333156,248.50423374)(886.70947728,248.03635187)
\curveto(887.4351716,247.56847001)(887.94602221,246.91439025)(888.24202911,246.07411262)
\curveto(889.44515391,247.85015399)(891.01112587,248.73817468)(892.939945,248.73817468)
\curveto(894.4486253,248.73817468)(895.60878136,248.31803586)(896.42041317,247.47775822)
\curveto(897.23204498,246.64702919)(897.63786089,245.36274121)(897.63786089,243.62489427)
\lineto(897.63786089,233.18348975)
\lineto(895.07405923,233.18348975)
\lineto(895.07405923,242.76551941)
\curveto(895.07405923,243.79676924)(894.98812174,244.53678648)(894.81624677,244.98557113)
\curveto(894.65392041,245.44390439)(894.35313921,245.81152585)(893.91390317,246.08843553)
\curveto(893.47466713,246.36534521)(892.95904221,246.50380004)(892.36702842,246.50380004)
\curveto(891.29758416,246.50380004)(890.40956347,246.14572719)(889.70296636,245.42958147)
\curveto(888.99636926,244.72298437)(888.64307071,243.58669983)(888.64307071,242.02072787)
\lineto(888.64307071,233.18348975)
\lineto(886.06494613,233.18348975)
\lineto(886.06494613,243.06630061)
\curveto(886.06494613,244.21213376)(885.85487672,245.07150861)(885.4347379,245.64442519)
\curveto(885.01459908,246.21734176)(884.3270992,246.50380004)(883.37223824,246.50380004)
\curveto(882.64654392,246.50380004)(881.97336695,246.31282785)(881.35270733,245.93088347)
\curveto(880.74159632,245.54893909)(880.29758597,244.99034543)(880.0206763,244.2551025)
\curveto(879.74376662,243.51985957)(879.60531178,242.45996391)(879.60531178,241.07541552)
\lineto(879.60531178,233.18348975)
\closepath
}
}
{
\newrgbcolor{curcolor}{0 0 0}
\pscustom[linewidth=1.51181105,linecolor=curcolor]
{
\newpath
\moveto(346.17978287,270.96936496)
\lineto(564.67422578,270.96936496)
\curveto(578.94128107,270.96936496)(590.42703316,259.48361287)(590.42703316,245.21655758)
\lineto(590.42703316,196.72216273)
\curveto(590.42703316,182.45510744)(578.94128107,170.96935535)(564.67422578,170.96935535)
\lineto(346.17978287,170.96935535)
\curveto(331.91272758,170.96935535)(320.42697549,182.45510744)(320.42697549,196.72216273)
\lineto(320.42697549,245.21655758)
\curveto(320.42697549,259.48361287)(331.91272758,270.96936496)(346.17978287,270.96936496)
\closepath
}
}
{
\newrgbcolor{curcolor}{0 0 0}
\pscustom[linewidth=1.51181105,linecolor=curcolor,linestyle=dashed,dash=1.20000005 1.20000005]
{
\newpath
\moveto(590.03458772,221.11147605)
\lineto(637.2843515,221.11147605)
}
}
{
\newrgbcolor{curcolor}{0 0 0}
\pscustom[linestyle=none,fillstyle=solid,fillcolor=curcolor]
{
\newpath
\moveto(629.37583709,224.77042331)
\lineto(639.28655428,221.12600147)
\lineto(629.37583655,217.48158042)
\curveto(630.9591548,219.63324514)(630.95003169,222.57709592)(629.37583709,224.77042331)
\closepath
}
}
{
\newrgbcolor{curcolor}{0 0 0}
\pscustom[linewidth=0.56692914,linecolor=curcolor]
{
\newpath
\moveto(629.37583709,224.77042331)
\lineto(639.28655428,221.12600147)
\lineto(629.37583655,217.48158042)
\curveto(630.9591548,219.63324514)(630.95003169,222.57709592)(629.37583709,224.77042331)
\closepath
}
}
{
\newrgbcolor{curcolor}{0 0 0}
\pscustom[linewidth=1.51181105,linecolor=curcolor]
{
\newpath
\moveto(26.5087119,271.54071169)
\lineto(245.00315481,271.54071169)
\curveto(259.2702101,271.54071169)(270.75596219,260.0549596)(270.75596219,245.78790432)
\lineto(270.75596219,197.29350946)
\curveto(270.75596219,183.02645417)(259.2702101,171.54070208)(245.00315481,171.54070208)
\lineto(26.5087119,171.54070208)
\curveto(12.24165661,171.54070208)(0.75590452,183.02645417)(0.75590452,197.29350946)
\lineto(0.75590452,245.78790432)
\curveto(0.75590452,260.0549596)(12.24165661,271.54071169)(26.5087119,271.54071169)
\closepath
}
}
{
\newrgbcolor{curcolor}{0 0 0}
\pscustom[linewidth=1.51181105,linecolor=curcolor]
{
\newpath
\moveto(506.03821014,410.82627356)
\lineto(724.53265306,410.82627356)
\curveto(738.79970834,410.82627356)(750.28546043,399.34052147)(750.28546043,385.07346619)
\lineto(750.28546043,336.57907133)
\curveto(750.28546043,322.31201604)(738.79970834,310.82626395)(724.53265306,310.82626395)
\lineto(506.03821014,310.82626395)
\curveto(491.77115485,310.82626395)(480.28540276,322.31201604)(480.28540276,336.57907133)
\lineto(480.28540276,385.07346619)
\curveto(480.28540276,399.34052147)(491.77115485,410.82627356)(506.03821014,410.82627356)
\closepath
}
}
{
\newrgbcolor{curcolor}{0 0 0}
\pscustom[linestyle=none,fillstyle=solid,fillcolor=curcolor]
{
\newpath
\moveto(500.2594668,373.18348013)
\lineto(500.2594668,375.10275065)
\curveto(499.29505724,373.59407034)(497.87708872,372.83973019)(496.00556125,372.83973019)
\curveto(494.79288784,372.83973019)(493.67570053,373.17393152)(492.65399931,373.84233419)
\curveto(491.6418467,374.51073686)(490.85408641,375.44172629)(490.29071845,376.63530248)
\curveto(489.73689909,377.83842728)(489.45998942,379.21820136)(489.45998942,380.77462471)
\curveto(489.45998942,382.29285363)(489.71302757,383.6678534)(490.21910387,384.89962403)
\curveto(490.72518018,386.14094327)(491.48429464,387.09102992)(492.49644725,387.74988398)
\curveto(493.50859986,388.40873803)(494.64011009,388.73816506)(495.89097794,388.73816506)
\curveto(496.80764445,388.73816506)(497.62405057,388.54241857)(498.34019628,388.15092558)
\curveto(499.056342,387.7689812)(499.63880718,387.2676792)(500.08759183,386.64701958)
\lineto(500.08759183,394.1808725)
\lineto(502.65139349,394.1808725)
\lineto(502.65139349,373.18348013)
\closepath
\moveto(492.10972856,380.77462471)
\curveto(492.10972856,378.82670837)(492.52031877,377.37054541)(493.34149919,376.40613585)
\curveto(494.16267961,375.44172629)(495.13186348,374.95952151)(496.2490508,374.95952151)
\curveto(497.37578672,374.95952151)(498.33064767,375.41785476)(499.11363366,376.33452128)
\curveto(499.90616825,377.2607364)(500.30243554,378.66915631)(500.30243554,380.559781)
\curveto(500.30243554,382.64137788)(499.90139394,384.1691554)(499.09931074,385.14311357)
\curveto(498.29722754,386.11707175)(497.30894645,386.60405083)(496.13446748,386.60405083)
\curveto(494.98863434,386.60405083)(494.02899908,386.13616897)(493.25556171,385.20040523)
\curveto(492.49167294,384.2646415)(492.10972856,382.78938132)(492.10972856,380.77462471)
\closepath
}
}
{
\newrgbcolor{curcolor}{0 0 0}
\pscustom[linestyle=none,fillstyle=solid,fillcolor=curcolor]
{
\newpath
\moveto(506.71910152,391.21602924)
\lineto(506.71910152,394.1808725)
\lineto(509.2972261,394.1808725)
\lineto(509.2972261,391.21602924)
\closepath
\moveto(506.71910152,373.18348013)
\lineto(506.71910152,388.39441512)
\lineto(509.2972261,388.39441512)
\lineto(509.2972261,373.18348013)
\closepath
}
}
{
\newrgbcolor{curcolor}{0 0 0}
\pscustom[linestyle=none,fillstyle=solid,fillcolor=curcolor]
{
\newpath
\moveto(512.19045426,377.72384397)
\lineto(514.739933,378.12488557)
\curveto(514.88316215,377.10318435)(515.27942944,376.32019837)(515.92873489,375.77592762)
\curveto(516.58758895,375.23165688)(517.50425546,374.95952151)(518.67873444,374.95952151)
\curveto(519.86276202,374.95952151)(520.7412341,375.19823675)(521.31415067,375.67566722)
\curveto(521.88706724,376.16264631)(522.17352553,376.73078858)(522.17352553,377.38009402)
\curveto(522.17352553,377.96255921)(521.92048737,378.42089246)(521.41441107,378.7550938)
\curveto(521.06111252,378.98426043)(520.18264044,379.27549302)(518.77899484,379.62879157)
\curveto(516.88837015,380.10622205)(515.57543634,380.51681226)(514.8401934,380.8605622)
\curveto(514.11449908,381.21386075)(513.56067973,381.69606553)(513.17873535,382.30717654)
\curveto(512.80633957,382.92783616)(512.62014169,383.61056174)(512.62014169,384.35535329)
\curveto(512.62014169,385.03330456)(512.77291944,385.65873849)(513.07847495,386.23165506)
\curveto(513.39357906,386.81412024)(513.81849218,387.29632502)(514.35321432,387.67826941)
\curveto(514.75425592,387.9742763)(515.29852666,388.22254015)(515.98602655,388.42306095)
\curveto(516.68307504,388.63313036)(517.42786659,388.73816506)(518.22040118,388.73816506)
\curveto(519.41397737,388.73816506)(520.45955011,388.56629009)(521.35711941,388.22254015)
\curveto(522.26423732,387.87879021)(522.93263998,387.41090834)(523.36232741,386.81889455)
\curveto(523.79201484,386.23642937)(524.08802174,385.45344338)(524.2503481,384.4699366)
\lineto(521.72951518,384.12618666)
\curveto(521.61493187,384.90917264)(521.28073053,385.52028365)(520.72691118,385.95951969)
\curveto(520.18264044,386.39875573)(519.40920307,386.61837375)(518.40659907,386.61837375)
\curveto(517.22257148,386.61837375)(516.37751954,386.42262725)(515.87144323,386.03113426)
\curveto(515.36536693,385.63964127)(515.11232878,385.18130801)(515.11232878,384.65613449)
\curveto(515.11232878,384.32193315)(515.21736348,384.02115195)(515.42743289,383.75379089)
\curveto(515.6375023,383.47688121)(515.96692933,383.24771458)(516.41571398,383.066291)
\curveto(516.67352643,382.97080491)(517.43264089,382.75118689)(518.69305735,382.40743694)
\curveto(520.51684177,381.92045786)(521.78680684,381.51941626)(522.50295255,381.20431214)
\curveto(523.22864688,380.89875664)(523.79678915,380.44997199)(524.20737936,379.8579582)
\curveto(524.61796957,379.26594441)(524.82326467,378.53070147)(524.82326467,377.6522294)
\curveto(524.82326467,376.79285454)(524.57022652,375.98122273)(524.06415021,375.21733396)
\curveto(523.56762252,374.46299381)(522.8467025,373.87575432)(521.90139015,373.45561551)
\curveto(520.95607781,373.0450253)(519.88663354,372.83973019)(518.69305735,372.83973019)
\curveto(516.71649518,372.83973019)(515.20781487,373.2503204)(514.16701643,374.07150082)
\curveto(513.1357666,374.89268124)(512.47691254,376.11012896)(512.19045426,377.72384397)
\closepath
}
}
{
\newrgbcolor{curcolor}{0 0 0}
\pscustom[linestyle=none,fillstyle=solid,fillcolor=curcolor]
{
\newpath
\moveto(537.81414884,378.7550938)
\lineto(540.34930467,378.42566677)
\curveto(540.072395,376.67827122)(539.36102359,375.30804575)(538.21519044,374.31499036)
\curveto(537.07890591,373.33148358)(535.68003461,372.83973019)(534.01857655,372.83973019)
\curveto(531.93697968,372.83973019)(530.2611987,373.51768147)(528.99123363,374.87358402)
\curveto(527.73081718,376.23903518)(527.10060895,378.19172583)(527.10060895,380.73165597)
\curveto(527.10060895,382.37401681)(527.37274432,383.81108254)(527.91701506,385.04285317)
\curveto(528.46128581,386.2746238)(529.28724053,387.19606462)(530.39487924,387.80717563)
\curveto(531.51206655,388.42783525)(532.72473996,388.73816506)(534.03289947,388.73816506)
\curveto(535.68480892,388.73816506)(537.03593717,388.31802624)(538.08628421,387.47774861)
\curveto(539.13663126,386.64701958)(539.80980824,385.46299199)(540.10581513,383.92566586)
\lineto(537.59930513,383.53894717)
\curveto(537.36058989,384.56064839)(536.93567677,385.32931146)(536.32456576,385.84493638)
\curveto(535.72300336,386.36056129)(534.99253473,386.61837375)(534.13315987,386.61837375)
\curveto(532.83454897,386.61837375)(531.77942762,386.15049188)(530.96779581,385.21472815)
\curveto(530.156164,384.28851302)(529.75034809,382.81802715)(529.75034809,380.80327054)
\curveto(529.75034809,378.7598681)(530.14184108,377.27505932)(530.92482706,376.34884419)
\curveto(531.70781305,375.42262907)(532.72951427,374.95952151)(533.98993073,374.95952151)
\curveto(535.00208334,374.95952151)(535.84713528,375.26985132)(536.52508656,375.89051094)
\curveto(537.20303783,376.51117056)(537.63272526,377.46603151)(537.81414884,378.7550938)
\closepath
}
}
{
\newrgbcolor{curcolor}{0 0 0}
\pscustom[linestyle=none,fillstyle=solid,fillcolor=curcolor]
{
\newpath
\moveto(542.52638855,373.18348013)
\lineto(542.52638855,388.39441512)
\lineto(544.84670067,388.39441512)
\lineto(544.84670067,386.08842592)
\curveto(545.43871446,387.1674188)(545.9829852,387.87879021)(546.4795129,388.22254015)
\curveto(546.9855892,388.56629009)(547.53940855,388.73816506)(548.14097096,388.73816506)
\curveto(549.00989442,388.73816506)(549.8931408,388.46125539)(550.7907101,387.90743603)
\lineto(549.90268941,385.51550935)
\curveto(549.27248118,385.88790512)(548.64227296,386.074103)(548.01206473,386.074103)
\curveto(547.44869676,386.074103)(546.94262046,385.90222803)(546.49383581,385.55847809)
\curveto(546.04505116,385.22427676)(545.72517274,384.75639489)(545.53420055,384.15483249)
\curveto(545.24774227,383.23816597)(545.10451312,382.23556197)(545.10451312,381.14702048)
\lineto(545.10451312,373.18348013)
\closepath
}
}
{
\newrgbcolor{curcolor}{0 0 0}
\pscustom[linestyle=none,fillstyle=solid,fillcolor=curcolor]
{
\newpath
\moveto(562.73601928,378.08191682)
\lineto(565.40008134,377.7524898)
\curveto(564.97994252,376.19606644)(564.20173084,374.98816734)(563.06544631,374.12879248)
\curveto(561.92916177,373.26941762)(560.47777312,372.83973019)(558.71128036,372.83973019)
\curveto(556.48645434,372.83973019)(554.71996157,373.52245577)(553.41180207,374.88790694)
\curveto(552.11319117,376.26290671)(551.46388572,378.18695153)(551.46388572,380.6600414)
\curveto(551.46388572,383.21906875)(552.12273978,385.20517954)(553.4404479,386.61837375)
\curveto(554.75815601,388.03156796)(556.46735712,388.73816506)(558.56805122,388.73816506)
\curveto(560.60190505,388.73816506)(562.26336311,388.04589087)(563.55242539,386.66134249)
\curveto(564.84148768,385.27679411)(565.48601882,383.32887776)(565.48601882,380.81759346)
\curveto(565.48601882,380.6648157)(565.48124452,380.43564907)(565.47169591,380.13009357)
\lineto(554.12794778,380.13009357)
\curveto(554.22343388,378.4590869)(554.69609005,377.17957322)(555.5459163,376.29155254)
\curveto(556.39574255,375.40353185)(557.45563821,374.95952151)(558.72560327,374.95952151)
\curveto(559.67091562,374.95952151)(560.47777312,375.20778535)(561.14617579,375.70431305)
\curveto(561.81457846,376.20084075)(562.34452629,376.99337534)(562.73601928,378.08191682)
\closepath
\moveto(554.27117693,382.24988489)
\lineto(562.76466511,382.24988489)
\curveto(562.65008179,383.52939856)(562.32542907,384.48903382)(561.79070693,385.12879066)
\curveto(560.96952651,386.12184605)(559.90485655,386.61837375)(558.59669704,386.61837375)
\curveto(557.41266946,386.61837375)(556.41483977,386.22210645)(555.60320796,385.42957186)
\curveto(554.80112475,384.63703727)(554.35711441,383.57714161)(554.27117693,382.24988489)
\closepath
}
}
{
\newrgbcolor{curcolor}{0 0 0}
\pscustom[linestyle=none,fillstyle=solid,fillcolor=curcolor]
{
\newpath
\moveto(574.26596206,375.48946934)
\lineto(574.63835783,373.21212596)
\curveto(573.9126635,373.05934821)(573.26335806,372.98295933)(572.69044148,372.98295933)
\curveto(571.75467775,372.98295933)(571.02898343,373.13096278)(570.51335851,373.42696968)
\curveto(569.9977336,373.72297657)(569.63488643,374.10969526)(569.42481702,374.58712573)
\curveto(569.21474761,375.07410482)(569.10971291,376.09103174)(569.10971291,377.63790648)
\lineto(569.10971291,386.38920712)
\lineto(567.21908822,386.38920712)
\lineto(567.21908822,388.39441512)
\lineto(569.10971291,388.39441512)
\lineto(569.10971291,392.16134158)
\lineto(571.67351457,393.70821633)
\lineto(571.67351457,388.39441512)
\lineto(574.26596206,388.39441512)
\lineto(574.26596206,386.38920712)
\lineto(571.67351457,386.38920712)
\lineto(571.67351457,377.49467734)
\curveto(571.67351457,376.7594344)(571.71648331,376.28677823)(571.8024208,376.07670882)
\curveto(571.89790689,375.86663941)(572.04591034,375.69953875)(572.24643114,375.57540682)
\curveto(572.45650055,375.4512749)(572.75250745,375.38920894)(573.13445183,375.38920894)
\curveto(573.42091011,375.38920894)(573.79808019,375.42262907)(574.26596206,375.48946934)
\closepath
}
}
{
\newrgbcolor{curcolor}{0 0 0}
\pscustom[linestyle=none,fillstyle=solid,fillcolor=curcolor]
{
\newpath
\moveto(576.80112292,391.21602924)
\lineto(576.80112292,394.1808725)
\lineto(579.37924749,394.1808725)
\lineto(579.37924749,391.21602924)
\closepath
\moveto(576.80112292,373.18348013)
\lineto(576.80112292,388.39441512)
\lineto(579.37924749,388.39441512)
\lineto(579.37924749,373.18348013)
\closepath
}
}
{
\newrgbcolor{curcolor}{0 0 0}
\pscustom[linestyle=none,fillstyle=solid,fillcolor=curcolor]
{
\newpath
\moveto(581.94304863,373.18348013)
\lineto(581.94304863,375.27462562)
\lineto(591.62533869,386.38920712)
\curveto(590.5272486,386.33191546)(589.55806473,386.30326963)(588.71778709,386.30326963)
\lineto(582.5159652,386.30326963)
\lineto(582.5159652,388.39441512)
\lineto(594.94825481,388.39441512)
\lineto(594.94825481,386.68998832)
\lineto(586.71257909,377.03634408)
\lineto(585.1227356,375.27462562)
\curveto(586.27811736,375.36056311)(587.36188454,375.40353185)(588.37403715,375.40353185)
\lineto(595.40658807,375.40353185)
\lineto(595.40658807,373.18348013)
\closepath
}
}
{
\newrgbcolor{curcolor}{0 0 0}
\pscustom[linestyle=none,fillstyle=solid,fillcolor=curcolor]
{
\newpath
\moveto(608.38314572,378.08191682)
\lineto(611.04720778,377.7524898)
\curveto(610.62706896,376.19606644)(609.84885729,374.98816734)(608.71257275,374.12879248)
\curveto(607.57628822,373.26941762)(606.12489957,372.83973019)(604.3584068,372.83973019)
\curveto(602.13358078,372.83973019)(600.36708802,373.52245577)(599.05892851,374.88790694)
\curveto(597.76031762,376.26290671)(597.11101217,378.18695153)(597.11101217,380.6600414)
\curveto(597.11101217,383.21906875)(597.76986623,385.20517954)(599.08757434,386.61837375)
\curveto(600.40528246,388.03156796)(602.11448356,388.73816506)(604.21517766,388.73816506)
\curveto(606.24903149,388.73816506)(607.91048955,388.04589087)(609.19955184,386.66134249)
\curveto(610.48861413,385.27679411)(611.13314527,383.32887776)(611.13314527,380.81759346)
\curveto(611.13314527,380.6648157)(611.12837096,380.43564907)(611.11882235,380.13009357)
\lineto(599.77507423,380.13009357)
\curveto(599.87056032,378.4590869)(600.3432165,377.17957322)(601.19304274,376.29155254)
\curveto(602.04286899,375.40353185)(603.10276465,374.95952151)(604.37272972,374.95952151)
\curveto(605.31804206,374.95952151)(606.12489957,375.20778535)(606.79330224,375.70431305)
\curveto(607.4617049,376.20084075)(607.99165273,376.99337534)(608.38314572,378.08191682)
\closepath
\moveto(599.91830337,382.24988489)
\lineto(608.41179155,382.24988489)
\curveto(608.29720824,383.52939856)(607.97255551,384.48903382)(607.43783338,385.12879066)
\curveto(606.61665296,386.12184605)(605.551983,386.61837375)(604.24382349,386.61837375)
\curveto(603.05979591,386.61837375)(602.06196621,386.22210645)(601.2503344,385.42957186)
\curveto(600.4482512,384.63703727)(600.00424086,383.57714161)(599.91830337,382.24988489)
\closepath
}
}
{
\newrgbcolor{curcolor}{0 0 0}
\pscustom[linestyle=none,fillstyle=solid,fillcolor=curcolor]
{
\newpath
\moveto(624.15267834,373.18348013)
\lineto(624.15267834,375.10275065)
\curveto(623.18826878,373.59407034)(621.77030027,372.83973019)(619.8987728,372.83973019)
\curveto(618.68609939,372.83973019)(617.56891207,373.17393152)(616.54721085,373.84233419)
\curveto(615.53505824,374.51073686)(614.74729795,375.44172629)(614.18392999,376.63530248)
\curveto(613.63011064,377.83842728)(613.35320096,379.21820136)(613.35320096,380.77462471)
\curveto(613.35320096,382.29285363)(613.60623911,383.6678534)(614.11231542,384.89962403)
\curveto(614.61839173,386.14094327)(615.37750618,387.09102992)(616.38965879,387.74988398)
\curveto(617.4018114,388.40873803)(618.53332163,388.73816506)(619.78418948,388.73816506)
\curveto(620.700856,388.73816506)(621.51726211,388.54241857)(622.23340783,388.15092558)
\curveto(622.94955354,387.7689812)(623.53201872,387.2676792)(623.98080337,386.64701958)
\lineto(623.98080337,394.1808725)
\lineto(626.54460503,394.1808725)
\lineto(626.54460503,373.18348013)
\closepath
\moveto(616.00294011,380.77462471)
\curveto(616.00294011,378.82670837)(616.41353032,377.37054541)(617.23471074,376.40613585)
\curveto(618.05589116,375.44172629)(619.02507502,374.95952151)(620.14226234,374.95952151)
\curveto(621.26899827,374.95952151)(622.22385922,375.41785476)(623.0068452,376.33452128)
\curveto(623.79937979,377.2607364)(624.19564709,378.66915631)(624.19564709,380.559781)
\curveto(624.19564709,382.64137788)(623.79460549,384.1691554)(622.99252229,385.14311357)
\curveto(622.19043909,386.11707175)(621.202158,386.60405083)(620.02767903,386.60405083)
\curveto(618.88184588,386.60405083)(617.92221062,386.13616897)(617.14877325,385.20040523)
\curveto(616.38488449,384.2646415)(616.00294011,382.78938132)(616.00294011,380.77462471)
\closepath
}
}
{
\newrgbcolor{curcolor}{0 0 0}
\pscustom[linestyle=none,fillstyle=solid,fillcolor=curcolor]
{
}
}
{
\newrgbcolor{curcolor}{0 0 0}
\pscustom[linestyle=none,fillstyle=solid,fillcolor=curcolor]
{
\newpath
\moveto(638.74772982,367.35405401)
\lineto(638.74772982,388.39441512)
\lineto(641.09668776,388.39441512)
\lineto(641.09668776,386.41785295)
\curveto(641.65050711,387.19129032)(642.27594104,387.7689812)(642.97298953,388.15092558)
\curveto(643.67003803,388.54241857)(644.51508997,388.73816506)(645.50814537,388.73816506)
\curveto(646.80675626,388.73816506)(647.95258941,388.40396373)(648.9456448,387.73556106)
\curveto(649.93870019,387.0671584)(650.68826604,386.12184605)(651.19434234,384.89962403)
\curveto(651.70041865,383.68695062)(651.9534568,382.35491959)(651.9534568,380.90353094)
\curveto(651.9534568,379.34710759)(651.67177282,377.94346199)(651.10840486,376.69259414)
\curveto(650.5545855,375.4512749)(649.74295369,374.49641394)(648.67350943,373.82801128)
\curveto(647.61361377,373.16915722)(646.49642645,372.83973019)(645.32194748,372.83973019)
\curveto(644.46257262,372.83973019)(643.68913525,373.02115377)(643.00163536,373.38400093)
\curveto(642.32368409,373.7468481)(641.76509043,374.20518135)(641.32585439,374.75900071)
\lineto(641.32585439,367.35405401)
\closepath
\moveto(641.08236485,380.70301014)
\curveto(641.08236485,378.74554519)(641.47863214,377.29893084)(642.27116673,376.36316711)
\curveto(643.06370132,375.42740337)(644.02333658,374.95952151)(645.15007251,374.95952151)
\curveto(646.29590565,374.95952151)(647.27463813,375.44172629)(648.08626994,376.40613585)
\curveto(648.90745036,377.38009402)(649.31804057,378.88400003)(649.31804057,380.91785386)
\curveto(649.31804057,382.85622159)(648.91699897,384.30761024)(648.11491577,385.2720198)
\curveto(647.32238118,386.23642937)(646.37229453,386.71863415)(645.26465582,386.71863415)
\curveto(644.16656573,386.71863415)(643.19260755,386.20300923)(642.3427813,385.1717594)
\curveto(641.50250367,384.15005818)(641.08236485,382.6604751)(641.08236485,380.70301014)
\closepath
}
}
{
\newrgbcolor{curcolor}{0 0 0}
\pscustom[linestyle=none,fillstyle=solid,fillcolor=curcolor]
{
\newpath
\moveto(655.03287293,373.18348013)
\lineto(655.03287293,388.39441512)
\lineto(657.35318505,388.39441512)
\lineto(657.35318505,386.08842592)
\curveto(657.94519884,387.1674188)(658.48946959,387.87879021)(658.98599728,388.22254015)
\curveto(659.49207359,388.56629009)(660.04589294,388.73816506)(660.64745534,388.73816506)
\curveto(661.51637881,388.73816506)(662.39962519,388.46125539)(663.29719449,387.90743603)
\lineto(662.4091738,385.51550935)
\curveto(661.77896557,385.88790512)(661.14875734,386.074103)(660.51854911,386.074103)
\curveto(659.95518115,386.074103)(659.44910484,385.90222803)(659.0003202,385.55847809)
\curveto(658.55153555,385.22427676)(658.23165713,384.75639489)(658.04068494,384.15483249)
\curveto(657.75422665,383.23816597)(657.61099751,382.23556197)(657.61099751,381.14702048)
\lineto(657.61099751,373.18348013)
\closepath
}
}
{
\newrgbcolor{curcolor}{0 0 0}
\pscustom[linestyle=none,fillstyle=solid,fillcolor=curcolor]
{
\newpath
\moveto(663.87011331,380.78894763)
\curveto(663.87011331,383.60578744)(664.65309929,385.69215862)(666.21907126,387.04806118)
\curveto(667.52723076,388.1747971)(669.12184856,388.73816506)(671.00292463,388.73816506)
\curveto(673.09407012,388.73816506)(674.80327123,388.05066518)(676.13052795,386.6756654)
\curveto(677.45778468,385.31021424)(678.12141304,383.41958955)(678.12141304,381.00379134)
\curveto(678.12141304,379.04632639)(677.82540615,377.50422595)(677.23339235,376.37749002)
\curveto(676.65092717,375.26030271)(675.79632662,374.39137924)(674.66959069,373.77071962)
\curveto(673.55240338,373.15006)(672.33018136,372.83973019)(671.00292463,372.83973019)
\curveto(668.87358471,372.83973019)(667.15006069,373.52245577)(665.83235257,374.88790694)
\curveto(664.52419306,376.2533581)(663.87011331,378.22037166)(663.87011331,380.78894763)
\closepath
\moveto(666.51985246,380.78894763)
\curveto(666.51985246,378.84103128)(666.94476558,377.38009402)(667.79459183,376.40613585)
\curveto(668.64441808,375.44172629)(669.71386235,374.95952151)(671.00292463,374.95952151)
\curveto(672.28243831,374.95952151)(673.34710827,375.44650059)(674.19693452,376.42045877)
\curveto(675.04676077,377.39441694)(675.4716739,378.87922572)(675.4716739,380.87488511)
\curveto(675.4716739,382.75596119)(675.04198647,384.17870401)(674.18261161,385.14311357)
\curveto(673.33278536,386.11707175)(672.2728897,386.60405083)(671.00292463,386.60405083)
\curveto(669.71386235,386.60405083)(668.64441808,386.12184605)(667.79459183,385.15743649)
\curveto(666.94476558,384.19302693)(666.51985246,382.73686397)(666.51985246,380.78894763)
\closepath
}
}
{
\newrgbcolor{curcolor}{0 0 0}
\pscustom[linestyle=none,fillstyle=solid,fillcolor=curcolor]
{
\newpath
\moveto(683.52115571,373.18348013)
\lineto(681.12922902,373.18348013)
\lineto(681.12922902,394.1808725)
\lineto(683.70735359,394.1808725)
\lineto(683.70735359,386.68998832)
\curveto(684.79589508,388.05543948)(686.18521777,388.73816506)(687.87532166,388.73816506)
\curveto(688.81108539,388.73816506)(689.69433177,388.54719287)(690.5250608,388.16524849)
\curveto(691.36533844,387.79285272)(692.05283833,387.26290489)(692.58756046,386.575405)
\curveto(693.1318312,385.89745373)(693.55674433,385.07627331)(693.86229983,384.11186374)
\curveto(694.16785534,383.14745418)(694.32063309,382.11620435)(694.32063309,381.01811426)
\curveto(694.32063309,378.41134385)(693.67610195,376.39658724)(692.38703966,374.97384442)
\curveto(691.09797737,373.5511016)(689.55110263,372.83973019)(687.74641543,372.83973019)
\curveto(685.95127684,372.83973019)(684.54285693,373.58929604)(683.52115571,375.08842774)
\closepath
\moveto(683.49250988,380.90353094)
\curveto(683.49250988,379.07974652)(683.74077373,377.7620384)(684.23730142,376.95040659)
\curveto(685.04893323,375.62314987)(686.14702333,374.95952151)(687.53157171,374.95952151)
\curveto(688.65830764,374.95952151)(689.63226581,375.44650059)(690.45344623,376.42045877)
\curveto(691.27462665,377.40396555)(691.68521686,378.86490281)(691.68521686,380.80327054)
\curveto(691.68521686,382.78938132)(691.28894956,384.25509289)(690.49641497,385.20040523)
\curveto(689.71342899,386.14571758)(688.76334234,386.61837375)(687.64615503,386.61837375)
\curveto(686.5194191,386.61837375)(685.54546093,386.12662036)(684.72428051,385.14311357)
\curveto(683.90310009,384.1691554)(683.49250988,382.75596119)(683.49250988,380.90353094)
\closepath
}
}
{
\newrgbcolor{curcolor}{0 0 0}
\pscustom[linestyle=none,fillstyle=solid,fillcolor=curcolor]
{
\newpath
\moveto(697.40006364,373.18348013)
\lineto(697.40006364,394.1808725)
\lineto(699.97818822,394.1808725)
\lineto(699.97818822,373.18348013)
\closepath
}
}
{
\newrgbcolor{curcolor}{0 0 0}
\pscustom[linestyle=none,fillstyle=solid,fillcolor=curcolor]
{
\newpath
\moveto(714.38703948,378.08191682)
\lineto(717.05110154,377.7524898)
\curveto(716.63096272,376.19606644)(715.85275104,374.98816734)(714.71646651,374.12879248)
\curveto(713.58018197,373.26941762)(712.12879332,372.83973019)(710.36230056,372.83973019)
\curveto(708.13747454,372.83973019)(706.37098177,373.52245577)(705.06282227,374.88790694)
\curveto(703.76421137,376.26290671)(703.11490592,378.18695153)(703.11490592,380.6600414)
\curveto(703.11490592,383.21906875)(703.77375998,385.20517954)(705.0914681,386.61837375)
\curveto(706.40917621,388.03156796)(708.11837732,388.73816506)(710.21907142,388.73816506)
\curveto(712.25292525,388.73816506)(713.91438331,388.04589087)(715.20344559,386.66134249)
\curveto(716.49250788,385.27679411)(717.13703902,383.32887776)(717.13703902,380.81759346)
\curveto(717.13703902,380.6648157)(717.13226472,380.43564907)(717.12271611,380.13009357)
\lineto(705.77896798,380.13009357)
\curveto(705.87445408,378.4590869)(706.34711025,377.17957322)(707.1969365,376.29155254)
\curveto(708.04676275,375.40353185)(709.10665841,374.95952151)(710.37662347,374.95952151)
\curveto(711.32193582,374.95952151)(712.12879332,375.20778535)(712.79719599,375.70431305)
\curveto(713.46559866,376.20084075)(713.99554649,376.99337534)(714.38703948,378.08191682)
\closepath
\moveto(705.92219713,382.24988489)
\lineto(714.41568531,382.24988489)
\curveto(714.30110199,383.52939856)(713.97644927,384.48903382)(713.44172713,385.12879066)
\curveto(712.62054671,386.12184605)(711.55587675,386.61837375)(710.24771724,386.61837375)
\curveto(709.06368966,386.61837375)(708.06585997,386.22210645)(707.25422816,385.42957186)
\curveto(706.45214496,384.63703727)(706.00813461,383.57714161)(705.92219713,382.24988489)
\closepath
}
}
{
\newrgbcolor{curcolor}{0 0 0}
\pscustom[linestyle=none,fillstyle=solid,fillcolor=curcolor]
{
\newpath
\moveto(720.28808415,373.18348013)
\lineto(720.28808415,388.39441512)
\lineto(722.59407335,388.39441512)
\lineto(722.59407335,386.26030089)
\curveto(723.07150383,387.00509243)(723.70648636,387.60188053)(724.49902095,388.05066518)
\curveto(725.29155554,388.50899843)(726.19389914,388.73816506)(727.20605175,388.73816506)
\curveto(728.33278768,388.73816506)(729.2542285,388.50422413)(729.97037421,388.03634226)
\curveto(730.69606854,387.5684604)(731.20691915,386.91438064)(731.50292604,386.074103)
\curveto(732.70605084,387.85014438)(734.27202281,388.73816506)(736.20084193,388.73816506)
\curveto(737.70952224,388.73816506)(738.8696783,388.31802624)(739.68131011,387.47774861)
\curveto(740.49294192,386.64701958)(740.89875782,385.36273159)(740.89875782,383.62488466)
\lineto(740.89875782,373.18348013)
\lineto(738.33495616,373.18348013)
\lineto(738.33495616,382.7655098)
\curveto(738.33495616,383.79675963)(738.24901868,384.53677687)(738.07714371,384.98556152)
\curveto(737.91481735,385.44389477)(737.61403614,385.81151624)(737.17480011,386.08842592)
\curveto(736.73556407,386.36533559)(736.21993915,386.50379043)(735.62792536,386.50379043)
\curveto(734.55848109,386.50379043)(733.67046041,386.14571758)(732.9638633,385.42957186)
\curveto(732.2572662,384.72297475)(731.90396764,383.58669022)(731.90396764,382.02071826)
\lineto(731.90396764,373.18348013)
\lineto(729.32584307,373.18348013)
\lineto(729.32584307,383.066291)
\curveto(729.32584307,384.21212414)(729.11577366,385.071499)(728.69563484,385.64441557)
\curveto(728.27549602,386.21733215)(727.58799613,386.50379043)(726.63313518,386.50379043)
\curveto(725.90744086,386.50379043)(725.23426388,386.31281824)(724.61360426,385.93087386)
\curveto(724.00249325,385.54892948)(723.55848291,384.99033582)(723.28157323,384.25509289)
\curveto(723.00466356,383.51984995)(722.86620872,382.4599543)(722.86620872,381.07540591)
\lineto(722.86620872,373.18348013)
\closepath
}
}
{
\newrgbcolor{curcolor}{0 0 0}
\pscustom[linewidth=1.51181105,linecolor=curcolor]
{
\newpath
\moveto(355.4266885,25.86433274)
\lineto(308.63681764,25.66704141)
\curveto(308.63681764,25.66704141)(303.84192,25.15408393)(299.44489323,29.62511385)
\curveto(295.04775307,34.09614377)(295.53905386,38.89100361)(295.53905386,38.89100361)
\lineto(295.42680189,78.22144771)
}
}
{
\newrgbcolor{curcolor}{0 0 0}
\pscustom[linestyle=none,fillstyle=solid,fillcolor=curcolor]
{
\newpath
\moveto(291.79044092,70.30252265)
\lineto(295.40656212,80.22360088)
\lineto(299.07925413,70.32332491)
\curveto(296.92307928,71.90049574)(293.97926653,71.88297073)(291.79044092,70.30252265)
\closepath
}
}
{
\newrgbcolor{curcolor}{0 0 0}
\pscustom[linewidth=0.56692914,linecolor=curcolor]
{
\newpath
\moveto(291.79044092,70.30252265)
\lineto(295.40656212,80.22360088)
\lineto(299.07925413,70.32332491)
\curveto(296.92307928,71.90049574)(293.97926653,71.88297073)(291.79044092,70.30252265)
\closepath
}
}
{
\newrgbcolor{curcolor}{0 0 0}
\pscustom[linewidth=1.51181105,linecolor=curcolor]
{
\newpath
\moveto(455.42686488,270.82586975)
\lineto(455.28286488,347.55088393)
\curveto(455.28286488,347.55088393)(454.76986961,352.34581936)(459.24093732,356.74288393)
\curveto(463.71200504,361.14002408)(468.50690268,360.6487233)(468.50690268,360.6487233)
\lineto(477.71277354,360.75455007)
}
}
{
\newrgbcolor{curcolor}{0 0 0}
\pscustom[linestyle=none,fillstyle=solid,fillcolor=curcolor]
{
\newpath
\moveto(469.76272271,364.32234868)
\lineto(479.71467708,360.79208947)
\lineto(469.84650607,357.03398734)
\curveto(471.40498673,359.20370986)(471.3620252,362.14726127)(469.76272271,364.32234868)
\closepath
}
}
{
\newrgbcolor{curcolor}{0 0 0}
\pscustom[linewidth=0.56692914,linecolor=curcolor]
{
\newpath
\moveto(469.76272271,364.32234868)
\lineto(479.71467708,360.79208947)
\lineto(469.84650607,357.03398734)
\curveto(471.40498673,359.20370986)(471.3620252,362.14726127)(469.76272271,364.32234868)
\closepath
}
}
{
\newrgbcolor{curcolor}{0 0 0}
\pscustom[linewidth=1.51181105,linecolor=curcolor]
{
\newpath
\moveto(775.42687748,273.57842408)
\lineto(775.37434205,347.66060361)
\curveto(775.37434205,347.66060361)(775.88733732,352.45553904)(771.41626961,356.85260361)
\curveto(766.94516409,361.24974377)(761.7931011,360.75844298)(761.7931011,360.75844298)
\lineto(750.42685228,361.07856897)
}
}
{
\newrgbcolor{curcolor}{0 0 0}
\pscustom[linestyle=none,fillstyle=solid,fillcolor=curcolor]
{
\newpath
\moveto(779.08021549,281.48953125)
\lineto(775.44282275,271.5762321)
\lineto(771.79137443,281.4843629)
\curveto(773.94416142,279.9025709)(776.88800499,279.91378164)(779.08021549,281.48953125)
\closepath
}
}
{
\newrgbcolor{curcolor}{0 0 0}
\pscustom[linewidth=0.56692914,linecolor=curcolor]
{
\newpath
\moveto(779.08021549,281.48953125)
\lineto(775.44282275,271.5762321)
\lineto(771.79137443,281.4843629)
\curveto(773.94416142,279.9025709)(776.88800499,279.91378164)(779.08021549,281.48953125)
\closepath
}
}
{
\newrgbcolor{curcolor}{0 0 0}
\pscustom[linewidth=1.51181105,linecolor=curcolor]
{
\newpath
\moveto(220.49850497,130.96905738)
\lineto(370.3557584,130.96905738)
\curveto(384.24530053,130.96905738)(395.42713409,119.78722382)(395.42713409,105.89768169)
\curveto(395.42713409,92.00813956)(384.24530053,80.826306)(370.3557584,80.826306)
\lineto(220.49850497,80.826306)
\curveto(206.60896284,80.826306)(195.42712928,92.00813956)(195.42712928,105.89768169)
\curveto(195.42712928,119.78722382)(206.60896284,130.96905738)(220.49850497,130.96905738)
\closepath
}
}
{
\newrgbcolor{curcolor}{0 0 0}
\pscustom[linestyle=none,fillstyle=solid,fillcolor=curcolor]
{
\newpath
\moveto(209.77936287,97.9965976)
\lineto(209.77936287,113.20753259)
\lineto(212.09967499,113.20753259)
\lineto(212.09967499,110.90154339)
\curveto(212.69168878,111.98053626)(213.23595952,112.69190767)(213.73248722,113.03565762)
\curveto(214.23856352,113.37940756)(214.79238287,113.55128253)(215.39394527,113.55128253)
\curveto(216.26286874,113.55128253)(217.14611512,113.27437286)(218.04368442,112.7205535)
\lineto(217.15566373,110.32862681)
\curveto(216.5254555,110.70102259)(215.89524727,110.88722047)(215.26503905,110.88722047)
\curveto(214.70167108,110.88722047)(214.19559478,110.7153455)(213.74681013,110.37159556)
\curveto(213.29802548,110.03739422)(212.97814706,109.56951236)(212.78717487,108.96794996)
\curveto(212.50071659,108.05128344)(212.35748744,107.04867944)(212.35748744,105.96013795)
\lineto(212.35748744,97.9965976)
\closepath
}
}
{
\newrgbcolor{curcolor}{0 0 0}
\pscustom[linestyle=none,fillstyle=solid,fillcolor=curcolor]
{
\newpath
\moveto(229.9889945,102.89503429)
\lineto(232.65305656,102.56560726)
\curveto(232.23291774,101.00918391)(231.45470606,99.8012848)(230.31842153,98.94190995)
\curveto(229.18213699,98.08253509)(227.73074834,97.65284766)(225.96425558,97.65284766)
\curveto(223.73942956,97.65284766)(221.97293679,98.33557324)(220.66477729,99.7010244)
\curveto(219.36616639,101.07602418)(218.71686094,103.000069)(218.71686094,105.47315887)
\curveto(218.71686094,108.03218622)(219.375715,110.018297)(220.69342312,111.43149122)
\curveto(222.01113123,112.84468543)(223.72033234,113.55128253)(225.82102644,113.55128253)
\curveto(227.85488027,113.55128253)(229.51633833,112.85900834)(230.80540061,111.47445996)
\curveto(232.0944629,110.08991158)(232.73899404,108.14199523)(232.73899404,105.63071092)
\curveto(232.73899404,105.47793317)(232.73421974,105.24876654)(232.72467113,104.94321104)
\lineto(221.380923,104.94321104)
\curveto(221.4764091,103.27220437)(221.94906527,101.99269069)(222.79889152,101.10467001)
\curveto(223.64871777,100.21664932)(224.70861343,99.77263898)(225.97857849,99.77263898)
\curveto(226.92389084,99.77263898)(227.73074834,100.02090282)(228.39915101,100.51743052)
\curveto(229.06755368,101.01395821)(229.59750151,101.80649281)(229.9889945,102.89503429)
\closepath
\moveto(221.52415215,107.06300235)
\lineto(230.01764033,107.06300235)
\curveto(229.90305701,108.34251603)(229.57840429,109.30215129)(229.04368215,109.94190813)
\curveto(228.22250173,110.93496352)(227.15783177,111.43149122)(225.84967226,111.43149122)
\curveto(224.66564468,111.43149122)(223.66781499,111.03522392)(222.85618318,110.24268933)
\curveto(222.05409998,109.45015474)(221.61008963,108.39025908)(221.52415215,107.06300235)
\closepath
}
}
{
\newrgbcolor{curcolor}{0 0 0}
\pscustom[linestyle=none,fillstyle=solid,fillcolor=curcolor]
{
\newpath
\moveto(235.41737849,96.73618114)
\lineto(237.92388849,96.36378537)
\curveto(238.0289232,95.590348)(238.32015579,95.02698004)(238.79758626,94.67368148)
\curveto(239.4373431,94.19625101)(240.31104087,93.95753577)(241.41867958,93.95753577)
\curveto(242.61225577,93.95753577)(243.53369659,94.19625101)(244.18300204,94.67368148)
\curveto(244.83230749,95.15111196)(245.27154353,95.81951463)(245.50071016,96.67888949)
\curveto(245.63439069,97.20406301)(245.69645665,98.30692741)(245.68690804,99.98748269)
\curveto(244.56017212,98.66022596)(243.15652652,97.9965976)(241.47597124,97.9965976)
\curveto(239.38482575,97.9965976)(237.76633643,98.75093776)(236.62050329,100.25961806)
\curveto(235.47467015,101.76829837)(234.90175357,103.57775987)(234.90175357,105.68800258)
\curveto(234.90175357,107.13939123)(235.16434034,108.47619657)(235.68951386,109.69841859)
\curveto(236.21468738,110.93018922)(236.97380184,111.88027586)(237.96685723,112.54867853)
\curveto(238.96946124,113.2170812)(240.14394021,113.55128253)(241.49029415,113.55128253)
\curveto(243.28543274,113.55128253)(244.76546722,112.82558821)(245.93039759,111.37419956)
\lineto(245.93039759,113.20753259)
\lineto(248.30800136,113.20753259)
\lineto(248.30800136,100.05909726)
\curveto(248.30800136,97.6910421)(248.06451182,96.01526112)(247.57753273,95.03175434)
\curveto(247.10010225,94.03869895)(246.33621349,93.25571297)(245.28586644,92.6827964)
\curveto(244.245068,92.10987982)(242.96078002,91.82342154)(241.43300249,91.82342154)
\curveto(239.61876668,91.82342154)(238.15305512,92.23401175)(237.0358678,93.05519217)
\curveto(235.91868049,93.86682398)(235.37918405,95.0938203)(235.41737849,96.73618114)
\closepath
\moveto(237.55149272,105.87420047)
\curveto(237.55149272,103.87854107)(237.94776002,102.42237812)(238.74029461,101.50571161)
\curveto(239.5328292,100.58904509)(240.52588459,100.13071183)(241.71946078,100.13071183)
\curveto(242.90348836,100.13071183)(243.89654375,100.58427079)(244.69862696,101.49138869)
\curveto(245.50071016,102.40805521)(245.90175176,103.84034664)(245.90175176,105.78826298)
\curveto(245.90175176,107.65024184)(245.48638724,109.05388744)(244.65565821,109.99919979)
\curveto(243.83447779,110.94451213)(242.8414224,111.4171683)(241.67649204,111.4171683)
\curveto(240.53065889,111.4171683)(239.55670072,110.94928643)(238.75461752,110.0135227)
\curveto(237.95253432,109.08730758)(237.55149272,107.7075335)(237.55149272,105.87420047)
\closepath
}
}
{
\newrgbcolor{curcolor}{0 0 0}
\pscustom[linestyle=none,fillstyle=solid,fillcolor=curcolor]
{
\newpath
\moveto(262.17258277,97.9965976)
\lineto(262.17258277,100.23097223)
\curveto(260.98855519,98.51222252)(259.37961449,97.65284766)(257.34576066,97.65284766)
\curveto(256.44819136,97.65284766)(255.60791372,97.82472263)(254.82492774,98.16847257)
\curveto(254.05149037,98.51222252)(253.47379949,98.94190995)(253.09185511,99.45753486)
\curveto(252.71945934,99.98270838)(252.45687257,100.62246522)(252.30409482,101.37680538)
\curveto(252.19906012,101.88288168)(252.14654276,102.68496488)(252.14654276,103.78305498)
\lineto(252.14654276,113.20753259)
\lineto(254.72466734,113.20753259)
\lineto(254.72466734,104.77133607)
\curveto(254.72466734,103.42498212)(254.77718469,102.51786422)(254.8822194,102.04998235)
\curveto(255.04454576,101.37203107)(255.3882957,100.83730894)(255.91346922,100.44581595)
\curveto(256.43864275,100.06387157)(257.0879482,99.87289938)(257.86138557,99.87289938)
\curveto(258.63482294,99.87289938)(259.36051727,100.06864587)(260.03846854,100.46013886)
\curveto(260.71641982,100.86118046)(261.1938503,101.4006769)(261.47075997,102.07862818)
\curveto(261.75721826,102.76612806)(261.9004474,103.75918346)(261.9004474,105.05779435)
\lineto(261.9004474,113.20753259)
\lineto(264.47857198,113.20753259)
\lineto(264.47857198,97.9965976)
\closepath
}
}
{
\newrgbcolor{curcolor}{0 0 0}
\pscustom[linestyle=none,fillstyle=solid,fillcolor=curcolor]
{
\newpath
\moveto(268.46034252,97.9965976)
\lineto(268.46034252,118.99398997)
\lineto(271.0384671,118.99398997)
\lineto(271.0384671,97.9965976)
\closepath
}
}
{
\newrgbcolor{curcolor}{0 0 0}
\pscustom[linestyle=none,fillstyle=solid,fillcolor=curcolor]
{
\newpath
\moveto(284.96033927,99.87289938)
\curveto(284.00547832,99.06126756)(283.0840375,98.48835099)(282.19601681,98.15414966)
\curveto(281.31754474,97.81994833)(280.37223239,97.65284766)(279.36007978,97.65284766)
\curveto(277.68907311,97.65284766)(276.40478513,98.05866356)(275.50721583,98.87029537)
\curveto(274.60964654,99.69147579)(274.16086189,100.73704854)(274.16086189,102.00701361)
\curveto(274.16086189,102.75180515)(274.32796256,103.42975643)(274.66216389,104.04086744)
\curveto(275.00591383,104.66152706)(275.44992418,105.15805475)(275.99419492,105.53045052)
\curveto(276.54801427,105.9028463)(277.16867389,106.18453028)(277.85617378,106.37550247)
\curveto(278.36225009,106.509183)(279.12613885,106.63808923)(280.14784007,106.76222115)
\curveto(282.22943695,107.010485)(283.76198878,107.3064919)(284.74549556,107.65024184)
\curveto(284.75504417,108.00354039)(284.75981847,108.22793272)(284.75981847,108.32341881)
\curveto(284.75981847,109.37376586)(284.51632893,110.1137831)(284.02934984,110.54347053)
\curveto(283.37049579,111.12593571)(282.39176331,111.4171683)(281.09315241,111.4171683)
\curveto(279.880479,111.4171683)(278.98290971,111.20232459)(278.40044452,110.77263716)
\curveto(277.82752795,110.35249834)(277.40261483,109.60293249)(277.12570515,108.52393961)
\lineto(274.60487223,108.86768956)
\curveto(274.83403886,109.94668243)(275.21120894,110.8156059)(275.73638246,111.47445996)
\curveto(276.26155599,112.14286263)(277.02067045,112.65371324)(278.01372584,113.00701179)
\curveto(279.00678123,113.36985895)(280.15738868,113.55128253)(281.46554818,113.55128253)
\curveto(282.76415908,113.55128253)(283.81928043,113.39850478)(284.63091224,113.09294927)
\curveto(285.44254405,112.78739377)(286.03933215,112.40067508)(286.42127653,111.93279322)
\curveto(286.80322091,111.47445996)(287.07058198,110.89199478)(287.22335973,110.18539767)
\curveto(287.30929722,109.74616163)(287.35226596,108.95362704)(287.35226596,107.8077939)
\lineto(287.35226596,104.37029447)
\curveto(287.35226596,101.97359347)(287.40478331,100.45536456)(287.50981802,99.81560772)
\curveto(287.62440133,99.18539949)(287.84401935,98.57906278)(288.16867208,97.9965976)
\lineto(285.47596419,97.9965976)
\curveto(285.20860312,98.53131974)(285.03672815,99.15675366)(284.96033927,99.87289938)
\closepath
\moveto(284.74549556,105.63071092)
\curveto(283.80973182,105.24876654)(282.40608622,104.92411382)(280.53455875,104.65675275)
\curveto(279.4746631,104.503975)(278.72509725,104.33210003)(278.28586121,104.14112784)
\curveto(277.84662517,103.95015565)(277.50764953,103.66847166)(277.26893429,103.29607589)
\curveto(277.03021906,102.93322873)(276.91086144,102.52741283)(276.91086144,102.07862818)
\curveto(276.91086144,101.39112829)(277.16867389,100.81821172)(277.68429881,100.35987846)
\curveto(278.20947233,99.9015452)(278.9733611,99.67237858)(279.9759651,99.67237858)
\curveto(280.96902049,99.67237858)(281.85226687,99.88722229)(282.62570424,100.31690972)
\curveto(283.39914161,100.75614576)(283.96728388,101.35293385)(284.33013104,102.10727401)
\curveto(284.60704072,102.68973919)(284.74549556,103.54911405)(284.74549556,104.68539858)
\closepath
}
}
{
\newrgbcolor{curcolor}{0 0 0}
\pscustom[linestyle=none,fillstyle=solid,fillcolor=curcolor]
{
\newpath
\moveto(291.3197172,97.9965976)
\lineto(291.3197172,113.20753259)
\lineto(293.64002932,113.20753259)
\lineto(293.64002932,110.90154339)
\curveto(294.23204311,111.98053626)(294.77631385,112.69190767)(295.27284155,113.03565762)
\curveto(295.77891785,113.37940756)(296.3327372,113.55128253)(296.93429961,113.55128253)
\curveto(297.80322307,113.55128253)(298.68646945,113.27437286)(299.58403875,112.7205535)
\lineto(298.69601806,110.32862681)
\curveto(298.06580983,110.70102259)(297.43560161,110.88722047)(296.80539338,110.88722047)
\curveto(296.24202541,110.88722047)(295.73594911,110.7153455)(295.28716446,110.37159556)
\curveto(294.83837981,110.03739422)(294.51850139,109.56951236)(294.3275292,108.96794996)
\curveto(294.04107092,108.05128344)(293.89784177,107.04867944)(293.89784177,105.96013795)
\lineto(293.89784177,97.9965976)
\closepath
}
}
{
\newrgbcolor{curcolor}{0 0 0}
\pscustom[linestyle=none,fillstyle=solid,fillcolor=curcolor]
{
\newpath
\moveto(301.13090854,116.02914671)
\lineto(301.13090854,118.99398997)
\lineto(303.70903311,118.99398997)
\lineto(303.70903311,116.02914671)
\closepath
\moveto(301.13090854,97.9965976)
\lineto(301.13090854,113.20753259)
\lineto(303.70903311,113.20753259)
\lineto(303.70903311,97.9965976)
\closepath
}
}
{
\newrgbcolor{curcolor}{0 0 0}
\pscustom[linestyle=none,fillstyle=solid,fillcolor=curcolor]
{
\newpath
\moveto(306.27283425,97.9965976)
\lineto(306.27283425,100.08774309)
\lineto(315.95512432,111.20232459)
\curveto(314.85703422,111.14503293)(313.88785035,111.1163871)(313.04757271,111.1163871)
\lineto(306.84575082,111.1163871)
\lineto(306.84575082,113.20753259)
\lineto(319.27804043,113.20753259)
\lineto(319.27804043,111.50310579)
\lineto(311.04236471,101.84946155)
\lineto(309.45252122,100.08774309)
\curveto(310.60790298,100.17368058)(311.69167016,100.21664932)(312.70382277,100.21664932)
\lineto(319.73637369,100.21664932)
\lineto(319.73637369,97.9965976)
\closepath
}
}
{
\newrgbcolor{curcolor}{0 0 0}
\pscustom[linestyle=none,fillstyle=solid,fillcolor=curcolor]
{
\newpath
\moveto(332.22595947,99.87289938)
\curveto(331.27109851,99.06126756)(330.34965769,98.48835099)(329.46163701,98.15414966)
\curveto(328.58316493,97.81994833)(327.63785259,97.65284766)(326.62569998,97.65284766)
\curveto(324.95469331,97.65284766)(323.67040533,98.05866356)(322.77283603,98.87029537)
\curveto(321.87526673,99.69147579)(321.42648208,100.73704854)(321.42648208,102.00701361)
\curveto(321.42648208,102.75180515)(321.59358275,103.42975643)(321.92778409,104.04086744)
\curveto(322.27153403,104.66152706)(322.71554437,105.15805475)(323.25981512,105.53045052)
\curveto(323.81363447,105.9028463)(324.43429409,106.18453028)(325.12179397,106.37550247)
\curveto(325.62787028,106.509183)(326.39175904,106.63808923)(327.41346026,106.76222115)
\curveto(329.49505714,107.010485)(331.02760897,107.3064919)(332.01111575,107.65024184)
\curveto(332.02066436,108.00354039)(332.02543867,108.22793272)(332.02543867,108.32341881)
\curveto(332.02543867,109.37376586)(331.78194912,110.1137831)(331.29497004,110.54347053)
\curveto(330.63611598,111.12593571)(329.6573835,111.4171683)(328.35877261,111.4171683)
\curveto(327.1460992,111.4171683)(326.2485299,111.20232459)(325.66606472,110.77263716)
\curveto(325.09314815,110.35249834)(324.66823502,109.60293249)(324.39132534,108.52393961)
\lineto(321.87049243,108.86768956)
\curveto(322.09965906,109.94668243)(322.47682913,110.8156059)(323.00200266,111.47445996)
\curveto(323.52717618,112.14286263)(324.28629064,112.65371324)(325.27934603,113.00701179)
\curveto(326.27240142,113.36985895)(327.42300887,113.55128253)(328.73116838,113.55128253)
\curveto(330.02977927,113.55128253)(331.08490063,113.39850478)(331.89653244,113.09294927)
\curveto(332.70816425,112.78739377)(333.30495234,112.40067508)(333.68689673,111.93279322)
\curveto(334.06884111,111.47445996)(334.33620217,110.89199478)(334.48897993,110.18539767)
\curveto(334.57491741,109.74616163)(334.61788616,108.95362704)(334.61788616,107.8077939)
\lineto(334.61788616,104.37029447)
\curveto(334.61788616,101.97359347)(334.67040351,100.45536456)(334.77543821,99.81560772)
\curveto(334.89002153,99.18539949)(335.10963955,98.57906278)(335.43429227,97.9965976)
\lineto(332.74158438,97.9965976)
\curveto(332.47422332,98.53131974)(332.30234834,99.15675366)(332.22595947,99.87289938)
\closepath
\moveto(332.01111575,105.63071092)
\curveto(331.07535202,105.24876654)(329.67170642,104.92411382)(327.80017895,104.65675275)
\curveto(326.74028329,104.503975)(325.99071744,104.33210003)(325.5514814,104.14112784)
\curveto(325.11224536,103.95015565)(324.77326973,103.66847166)(324.53455449,103.29607589)
\curveto(324.29583925,102.93322873)(324.17648163,102.52741283)(324.17648163,102.07862818)
\curveto(324.17648163,101.39112829)(324.43429409,100.81821172)(324.949919,100.35987846)
\curveto(325.47509253,99.9015452)(326.23898129,99.67237858)(327.24158529,99.67237858)
\curveto(328.23464068,99.67237858)(329.11788706,99.88722229)(329.89132444,100.31690972)
\curveto(330.66476181,100.75614576)(331.23290408,101.35293385)(331.59575124,102.10727401)
\curveto(331.87266091,102.68973919)(332.01111575,103.54911405)(332.01111575,104.68539858)
\closepath
}
}
{
\newrgbcolor{curcolor}{0 0 0}
\pscustom[linestyle=none,fillstyle=solid,fillcolor=curcolor]
{
\newpath
\moveto(344.24288854,100.3025868)
\lineto(344.61528431,98.02524343)
\curveto(343.88958999,97.87246568)(343.24028454,97.7960768)(342.66736797,97.7960768)
\curveto(341.73160423,97.7960768)(341.00590991,97.94408025)(340.490285,98.24008714)
\curveto(339.97466008,98.53609404)(339.61181292,98.92281273)(339.40174351,99.4002432)
\curveto(339.1916741,99.88722229)(339.08663939,100.9041492)(339.08663939,102.45102395)
\lineto(339.08663939,111.20232459)
\lineto(337.19601471,111.20232459)
\lineto(337.19601471,113.20753259)
\lineto(339.08663939,113.20753259)
\lineto(339.08663939,116.97445905)
\lineto(341.65044105,118.52133379)
\lineto(341.65044105,113.20753259)
\lineto(344.24288854,113.20753259)
\lineto(344.24288854,111.20232459)
\lineto(341.65044105,111.20232459)
\lineto(341.65044105,102.30779481)
\curveto(341.65044105,101.57255187)(341.6934098,101.0998957)(341.77934728,100.88982629)
\curveto(341.87483338,100.67975688)(342.02283683,100.51265621)(342.22335763,100.38852429)
\curveto(342.43342704,100.26439237)(342.72943393,100.2023264)(343.11137831,100.2023264)
\curveto(343.3978366,100.2023264)(343.77500667,100.23574654)(344.24288854,100.3025868)
\closepath
}
}
{
\newrgbcolor{curcolor}{0 0 0}
\pscustom[linestyle=none,fillstyle=solid,fillcolor=curcolor]
{
\newpath
\moveto(346.7780422,116.02914671)
\lineto(346.7780422,118.99398997)
\lineto(349.35616677,118.99398997)
\lineto(349.35616677,116.02914671)
\closepath
\moveto(346.7780422,97.9965976)
\lineto(346.7780422,113.20753259)
\lineto(349.35616677,113.20753259)
\lineto(349.35616677,97.9965976)
\closepath
}
}
{
\newrgbcolor{curcolor}{0 0 0}
\pscustom[linestyle=none,fillstyle=solid,fillcolor=curcolor]
{
\newpath
\moveto(352.3210095,105.6020651)
\curveto(352.3210095,108.41890491)(353.10399549,110.50527609)(354.66996745,111.86117864)
\curveto(355.97812696,112.98791457)(357.57274475,113.55128253)(359.45382083,113.55128253)
\curveto(361.54496631,113.55128253)(363.25416742,112.86378265)(364.58142415,111.48878287)
\curveto(365.90868087,110.12333171)(366.57230923,108.23270702)(366.57230923,105.81690881)
\curveto(366.57230923,103.85944386)(366.27630234,102.31734342)(365.68428855,101.19060749)
\curveto(365.10182336,100.07342018)(364.24722281,99.20449671)(363.12048689,98.58383709)
\curveto(362.00329957,97.96317747)(360.78107755,97.65284766)(359.45382083,97.65284766)
\curveto(357.3244809,97.65284766)(355.60095688,98.33557324)(354.28324876,99.7010244)
\curveto(352.97508926,101.06647557)(352.3210095,103.03348913)(352.3210095,105.6020651)
\closepath
\moveto(354.97074865,105.6020651)
\curveto(354.97074865,103.65414875)(355.39566177,102.19321149)(356.24548802,101.21925332)
\curveto(357.09531427,100.25484376)(358.16475854,99.77263898)(359.45382083,99.77263898)
\curveto(360.7333345,99.77263898)(361.79800447,100.25961806)(362.64783071,101.23357623)
\curveto(363.49765696,102.20753441)(363.92257009,103.69234319)(363.92257009,105.68800258)
\curveto(363.92257009,107.56907866)(363.49288266,108.99182148)(362.6335078,109.95623104)
\curveto(361.78368155,110.93018922)(360.72378589,111.4171683)(359.45382083,111.4171683)
\curveto(358.16475854,111.4171683)(357.09531427,110.93496352)(356.24548802,109.97055396)
\curveto(355.39566177,109.00614439)(354.97074865,107.54998144)(354.97074865,105.6020651)
\closepath
}
}
{
\newrgbcolor{curcolor}{0 0 0}
\pscustom[linestyle=none,fillstyle=solid,fillcolor=curcolor]
{
\newpath
\moveto(369.59444813,97.9965976)
\lineto(369.59444813,113.20753259)
\lineto(371.91476024,113.20753259)
\lineto(371.91476024,111.04477253)
\curveto(373.03194756,112.7157792)(374.64566257,113.55128253)(376.75590528,113.55128253)
\curveto(377.67257179,113.55128253)(378.51284943,113.38418187)(379.27673819,113.04998053)
\curveto(380.05017557,112.72532781)(380.62786644,112.29564038)(381.00981082,111.76091824)
\curveto(381.39175521,111.22619611)(381.65911627,110.59121358)(381.81189403,109.85597064)
\curveto(381.90738012,109.37854017)(381.95512317,108.54303683)(381.95512317,107.34946064)
\lineto(381.95512317,97.9965976)
\lineto(379.37699859,97.9965976)
\lineto(379.37699859,107.24920024)
\curveto(379.37699859,108.29954729)(379.27673819,109.08253327)(379.07621739,109.59815819)
\curveto(378.87569659,110.12333171)(378.51762374,110.53869622)(378.00199882,110.84425173)
\curveto(377.49592252,111.15935584)(376.89913442,111.3169079)(376.21163453,111.3169079)
\curveto(375.11354444,111.3169079)(374.16345779,110.96838365)(373.36137459,110.27133516)
\curveto(372.56884,109.57428666)(372.1725727,108.25180424)(372.1725727,106.3038879)
\lineto(372.1725727,97.9965976)
\closepath
}
}
{
\newrgbcolor{curcolor}{0 0 0}
\pscustom[linewidth=1.51181105,linecolor=curcolor]
{
\newpath
\moveto(540.49913154,130.96899971)
\lineto(690.35638136,130.96899971)
\curveto(704.24592449,130.96899971)(715.42775885,119.78716535)(715.42775885,105.89762222)
\curveto(715.42775885,92.00807909)(704.24592449,80.82624473)(690.35638136,80.82624473)
\lineto(540.49913154,80.82624473)
\curveto(526.60958841,80.82624473)(515.42775405,92.00807909)(515.42775405,105.89762222)
\curveto(515.42775405,119.78716535)(526.60958841,130.96899971)(540.49913154,130.96899971)
\closepath
}
}
{
\newrgbcolor{curcolor}{0 0 0}
\pscustom[linestyle=none,fillstyle=solid,fillcolor=curcolor]
{
\newpath
\moveto(566.16564106,98.30078235)
\lineto(566.16564106,113.51171734)
\lineto(568.48595317,113.51171734)
\lineto(568.48595317,111.20572814)
\curveto(569.07796696,112.28472101)(569.62223771,112.99609242)(570.1187654,113.33984237)
\curveto(570.62484171,113.68359231)(571.17866106,113.85546728)(571.78022346,113.85546728)
\curveto(572.64914693,113.85546728)(573.53239331,113.5785576)(574.42996261,113.02473825)
\lineto(573.54194192,110.63281156)
\curveto(572.91173369,111.00520734)(572.28152546,111.19140522)(571.65131723,111.19140522)
\curveto(571.08794927,111.19140522)(570.58187297,111.01953025)(570.13308832,110.67578031)
\curveto(569.68430367,110.34157897)(569.36442525,109.87369711)(569.17345306,109.27213471)
\curveto(568.88699477,108.35546819)(568.74376563,107.35286419)(568.74376563,106.2643227)
\lineto(568.74376563,98.30078235)
\closepath
}
}
{
\newrgbcolor{curcolor}{0 0 0}
\pscustom[linestyle=none,fillstyle=solid,fillcolor=curcolor]
{
\newpath
\moveto(586.37527269,103.19921904)
\lineto(589.03933475,102.86979201)
\curveto(588.61919593,101.31336866)(587.84098425,100.10546955)(586.70469971,99.24609469)
\curveto(585.56841518,98.38671984)(584.11702653,97.95703241)(582.35053377,97.95703241)
\curveto(580.12570775,97.95703241)(578.35921498,98.63975799)(577.05105548,100.00520915)
\curveto(575.75244458,101.38020893)(575.10313913,103.30425375)(575.10313913,105.77734362)
\curveto(575.10313913,108.33637097)(575.76199319,110.32248175)(577.0797013,111.73567596)
\curveto(578.39740942,113.14887018)(580.10661053,113.85546728)(582.20730462,113.85546728)
\curveto(584.24115845,113.85546728)(585.90261651,113.16319309)(587.1916788,111.77864471)
\curveto(588.48074109,110.39409633)(589.12527223,108.44617998)(589.12527223,105.93489567)
\curveto(589.12527223,105.78211792)(589.12049793,105.55295129)(589.11094932,105.24739579)
\lineto(577.76720119,105.24739579)
\curveto(577.86268729,103.57638912)(578.33534346,102.29687544)(579.18516971,101.40885475)
\curveto(580.03499596,100.52083407)(581.09489161,100.07682372)(582.36485668,100.07682372)
\curveto(583.31016903,100.07682372)(584.11702653,100.32508757)(584.7854292,100.82161527)
\curveto(585.45383187,101.31814296)(585.98377969,102.11067755)(586.37527269,103.19921904)
\closepath
\moveto(577.91043033,107.3671871)
\lineto(586.40391851,107.3671871)
\curveto(586.2893352,108.64670078)(585.96468248,109.60633604)(585.42996034,110.24609288)
\curveto(584.60877992,111.23914827)(583.54410996,111.73567596)(582.23595045,111.73567596)
\curveto(581.05192287,111.73567596)(580.05409317,111.33940867)(579.24246136,110.54687408)
\curveto(578.44037816,109.75433949)(577.99636782,108.69444383)(577.91043033,107.3671871)
\closepath
}
}
{
\newrgbcolor{curcolor}{0 0 0}
\pscustom[linestyle=none,fillstyle=solid,fillcolor=curcolor]
{
\newpath
\moveto(591.24506302,102.84114618)
\lineto(593.79454176,103.24218778)
\curveto(593.93777091,102.22048656)(594.3340382,101.43750058)(594.98334365,100.89322984)
\curveto(595.64219771,100.3489591)(596.55886422,100.07682372)(597.7333432,100.07682372)
\curveto(598.91737078,100.07682372)(599.79584286,100.31553896)(600.36875943,100.79296944)
\curveto(600.941676,101.27994853)(601.22813429,101.84809079)(601.22813429,102.49739624)
\curveto(601.22813429,103.07986142)(600.97509613,103.53819468)(600.46901983,103.87239601)
\curveto(600.11572128,104.10156264)(599.2372492,104.39279523)(597.8336036,104.74609379)
\curveto(595.94297891,105.22352426)(594.6300451,105.63411447)(593.89480216,105.97786442)
\curveto(593.16910784,106.33116297)(592.61528849,106.81336775)(592.23334411,107.42447876)
\curveto(591.86094833,108.04513838)(591.67475045,108.72786396)(591.67475045,109.47265551)
\curveto(591.67475045,110.15060678)(591.8275282,110.77604071)(592.13308371,111.34895728)
\curveto(592.44818782,111.93142246)(592.87310094,112.41362724)(593.40782308,112.79557162)
\curveto(593.80886468,113.09157852)(594.35313542,113.33984237)(595.04063531,113.54036317)
\curveto(595.7376838,113.75043258)(596.48247535,113.85546728)(597.27500994,113.85546728)
\curveto(598.46858613,113.85546728)(599.51415888,113.68359231)(600.41172817,113.33984237)
\curveto(601.31884608,112.99609242)(601.98724874,112.52821056)(602.41693617,111.93619676)
\curveto(602.8466236,111.35373158)(603.1426305,110.5707456)(603.30495686,109.58723882)
\lineto(600.78412394,109.24348888)
\curveto(600.66954063,110.02647486)(600.33533929,110.63758587)(599.78151994,111.07682191)
\curveto(599.2372492,111.51605795)(598.46381183,111.73567596)(597.46120783,111.73567596)
\curveto(596.27718024,111.73567596)(595.4321283,111.53992947)(594.92605199,111.14843648)
\curveto(594.41997569,110.75694349)(594.16693754,110.29861023)(594.16693754,109.77343671)
\curveto(594.16693754,109.43923537)(594.27197224,109.13845417)(594.48204165,108.8710931)
\curveto(594.69211106,108.59418343)(595.02153809,108.3650168)(595.47032274,108.18359322)
\curveto(595.72813519,108.08810712)(596.48724965,107.8684891)(597.74766611,107.52473916)
\curveto(599.57145053,107.03776007)(600.8414156,106.63671847)(601.55756132,106.32161436)
\curveto(602.28325564,106.01605885)(602.85139791,105.56727421)(603.26198812,104.97526041)
\curveto(603.67257833,104.38324662)(603.87787343,103.64800369)(603.87787343,102.76953161)
\curveto(603.87787343,101.91015675)(603.62483528,101.09852494)(603.11875897,100.33463618)
\curveto(602.62223128,99.58029603)(601.90131126,98.99305654)(600.95599891,98.57291772)
\curveto(600.01068657,98.16232751)(598.9412423,97.95703241)(597.74766611,97.95703241)
\curveto(595.77110394,97.95703241)(594.26242363,98.36762262)(593.22162519,99.18880304)
\curveto(592.19037536,100.00998346)(591.5315213,101.22743117)(591.24506302,102.84114618)
\closepath
}
}
{
\newrgbcolor{curcolor}{0 0 0}
\pscustom[linestyle=none,fillstyle=solid,fillcolor=curcolor]
{
\newpath
\moveto(606.95730091,116.33333146)
\lineto(606.95730091,119.29817472)
\lineto(609.53542548,119.29817472)
\lineto(609.53542548,116.33333146)
\closepath
\moveto(606.95730091,98.30078235)
\lineto(606.95730091,113.51171734)
\lineto(609.53542548,113.51171734)
\lineto(609.53542548,98.30078235)
\closepath
}
}
{
\newrgbcolor{curcolor}{0 0 0}
\pscustom[linestyle=none,fillstyle=solid,fillcolor=curcolor]
{
\newpath
\moveto(623.32839143,98.30078235)
\lineto(623.32839143,100.22005287)
\curveto(622.36398186,98.71137256)(620.94601335,97.95703241)(619.07448588,97.95703241)
\curveto(617.86181247,97.95703241)(616.74462515,98.29123374)(615.72292393,98.95963641)
\curveto(614.71077132,99.62803908)(613.92301104,100.55902851)(613.35964307,101.7526047)
\curveto(612.80582372,102.9557295)(612.52891405,104.33550358)(612.52891405,105.89192693)
\curveto(612.52891405,107.41015585)(612.7819522,108.78515562)(613.2880285,110.01692625)
\curveto(613.79410481,111.25824549)(614.55321927,112.20833214)(615.56537188,112.86718619)
\curveto(616.57752449,113.52604025)(617.70903472,113.85546728)(618.95990257,113.85546728)
\curveto(619.87656908,113.85546728)(620.6929752,113.65972079)(621.40912091,113.26822779)
\curveto(622.12526663,112.88628341)(622.70773181,112.38498141)(623.15651646,111.76432179)
\lineto(623.15651646,119.29817472)
\lineto(625.72031812,119.29817472)
\lineto(625.72031812,98.30078235)
\closepath
\moveto(615.17865319,105.89192693)
\curveto(615.17865319,103.94401059)(615.5892434,102.48784763)(616.41042382,101.52343807)
\curveto(617.23160424,100.55902851)(618.20078811,100.07682372)(619.31797542,100.07682372)
\curveto(620.44471135,100.07682372)(621.3995723,100.53515698)(622.18255828,101.4518235)
\curveto(622.97509287,102.37803862)(623.37136017,103.78645853)(623.37136017,105.67708322)
\curveto(623.37136017,107.75868009)(622.97031857,109.28645762)(622.16823537,110.26041579)
\curveto(621.36615217,111.23437396)(620.37787108,111.72135305)(619.20339211,111.72135305)
\curveto(618.05755897,111.72135305)(617.09792371,111.25347118)(616.32448633,110.31770745)
\curveto(615.56059757,109.38194371)(615.17865319,107.90668354)(615.17865319,105.89192693)
\closepath
}
}
{
\newrgbcolor{curcolor}{0 0 0}
\pscustom[linestyle=none,fillstyle=solid,fillcolor=curcolor]
{
\newpath
\moveto(639.74245159,98.30078235)
\lineto(639.74245159,100.53515698)
\curveto(638.55842401,98.81640727)(636.9494833,97.95703241)(634.91562947,97.95703241)
\curveto(634.01806017,97.95703241)(633.17778253,98.12890738)(632.39479655,98.47265732)
\curveto(631.62135918,98.81640727)(631.0436683,99.24609469)(630.66172392,99.76171961)
\curveto(630.28932815,100.28689313)(630.02674139,100.92664997)(629.87396364,101.68099013)
\curveto(629.76892893,102.18706643)(629.71641158,102.98914963)(629.71641158,104.08723973)
\lineto(629.71641158,113.51171734)
\lineto(632.29453615,113.51171734)
\lineto(632.29453615,105.07552082)
\curveto(632.29453615,103.72916687)(632.3470535,102.82204897)(632.45208821,102.3541671)
\curveto(632.61441457,101.67621582)(632.95816451,101.14149369)(633.48333804,100.7500007)
\curveto(634.00851156,100.36805631)(634.65781701,100.17708412)(635.43125438,100.17708412)
\curveto(636.20469176,100.17708412)(636.93038608,100.37283062)(637.60833736,100.76432361)
\curveto(638.28628863,101.16536521)(638.76371911,101.70486165)(639.04062879,102.38281293)
\curveto(639.32708707,103.07031281)(639.47031622,104.0633682)(639.47031622,105.3619791)
\lineto(639.47031622,113.51171734)
\lineto(642.04844079,113.51171734)
\lineto(642.04844079,98.30078235)
\closepath
}
}
{
\newrgbcolor{curcolor}{0 0 0}
\pscustom[linestyle=none,fillstyle=solid,fillcolor=curcolor]
{
\newpath
\moveto(656.01328621,100.17708412)
\curveto(655.05842526,99.36545231)(654.13698444,98.79253574)(653.24896375,98.45833441)
\curveto(652.37049167,98.12413307)(651.42517933,97.95703241)(650.41302672,97.95703241)
\curveto(648.74202005,97.95703241)(647.45773207,98.36284831)(646.56016277,99.17448012)
\curveto(645.66259348,99.99566054)(645.21380883,101.04123329)(645.21380883,102.31119836)
\curveto(645.21380883,103.0559899)(645.38090949,103.73394118)(645.71511083,104.34505219)
\curveto(646.05886077,104.96571181)(646.50287111,105.4622395)(647.04714186,105.83463527)
\curveto(647.60096121,106.20703104)(648.22162083,106.48871503)(648.90912072,106.67968722)
\curveto(649.41519702,106.81336775)(650.17908579,106.94227398)(651.20078701,107.0664059)
\curveto(653.28238388,107.31466975)(654.81493571,107.61067665)(655.7984425,107.95442659)
\curveto(655.80799111,108.30772514)(655.81276541,108.53211747)(655.81276541,108.62760356)
\curveto(655.81276541,109.67795061)(655.56927587,110.41796785)(655.08229678,110.84765528)
\curveto(654.42344272,111.43012046)(653.44471025,111.72135305)(652.14609935,111.72135305)
\curveto(650.93342594,111.72135305)(650.03585664,111.50650934)(649.45339146,111.07682191)
\curveto(648.88047489,110.65668309)(648.45556176,109.90711724)(648.17865209,108.82812436)
\lineto(645.65781917,109.1718743)
\curveto(645.8869858,110.25086718)(646.26415588,111.11979065)(646.7893294,111.77864471)
\curveto(647.31450293,112.44704737)(648.07361738,112.95789799)(649.06667277,113.31119654)
\curveto(650.05972817,113.6740437)(651.21033561,113.85546728)(652.51849512,113.85546728)
\curveto(653.81710602,113.85546728)(654.87222737,113.70268953)(655.68385918,113.39713402)
\curveto(656.49549099,113.09157852)(657.09227909,112.70485983)(657.47422347,112.23697797)
\curveto(657.85616785,111.77864471)(658.12352892,111.19617953)(658.27630667,110.48958242)
\curveto(658.36224416,110.05034638)(658.4052129,109.25781179)(658.4052129,108.11197865)
\lineto(658.4052129,104.67447921)
\curveto(658.4052129,102.27777822)(658.45773025,100.75954931)(658.56276496,100.11979247)
\curveto(658.67734827,99.48958424)(658.89696629,98.88324753)(659.22161901,98.30078235)
\lineto(656.52891113,98.30078235)
\curveto(656.26155006,98.83550448)(656.08967509,99.46093841)(656.01328621,100.17708412)
\closepath
\moveto(655.7984425,105.93489567)
\curveto(654.86267876,105.55295129)(653.45903316,105.22829857)(651.58750569,104.9609375)
\curveto(650.52761003,104.80815975)(649.77804418,104.63628478)(649.33880815,104.44531259)
\curveto(648.89957211,104.2543404)(648.56059647,103.97265641)(648.32188123,103.60026064)
\curveto(648.08316599,103.23741348)(647.96380837,102.83159757)(647.96380837,102.38281293)
\curveto(647.96380837,101.69531304)(648.22162083,101.12239647)(648.73724575,100.66406321)
\curveto(649.26241927,100.20572995)(650.02630803,99.97656332)(651.02891203,99.97656332)
\curveto(652.02196743,99.97656332)(652.90521381,100.19140704)(653.67865118,100.62109447)
\curveto(654.45208855,101.06033051)(655.02023082,101.6571186)(655.38307798,102.41145876)
\curveto(655.65998766,102.99392394)(655.7984425,103.85329879)(655.7984425,104.98958333)
\closepath
}
}
{
\newrgbcolor{curcolor}{0 0 0}
\pscustom[linestyle=none,fillstyle=solid,fillcolor=curcolor]
{
\newpath
\moveto(662.3440111,98.30078235)
\lineto(662.3440111,119.29817472)
\lineto(664.92213567,119.29817472)
\lineto(664.92213567,98.30078235)
\closepath
}
}
{
\newrgbcolor{curcolor}{0 0 0}
\pscustom[linestyle=none,fillstyle=solid,fillcolor=curcolor]
{
\newpath
\moveto(398.28536902,14.97027232)
\lineto(398.28536902,28.1759993)
\lineto(396.00802565,28.1759993)
\lineto(396.00802565,30.18120731)
\lineto(398.28536902,30.18120731)
\lineto(398.28536902,31.79969662)
\curveto(398.28536902,32.82139784)(398.37608081,33.5805123)(398.55750439,34.07703999)
\curveto(398.80576824,34.74544266)(399.24022997,35.2849391)(399.86088959,35.69552931)
\curveto(400.49109782,36.11566813)(401.3695699,36.32573754)(402.49630583,36.32573754)
\curveto(403.22200015,36.32573754)(404.02408335,36.23980005)(404.90255543,36.06792508)
\lineto(404.51583674,33.81922754)
\curveto(403.98111461,33.91471363)(403.4750383,33.96245668)(402.99760783,33.96245668)
\curveto(402.21462184,33.96245668)(401.66080249,33.79535601)(401.33614977,33.46115468)
\curveto(401.01149704,33.12695335)(400.84917068,32.50151942)(400.84917068,31.58485291)
\lineto(400.84917068,30.18120731)
\lineto(403.81401394,30.18120731)
\lineto(403.81401394,28.1759993)
\lineto(400.84917068,28.1759993)
\lineto(400.84917068,14.97027232)
\closepath
}
}
{
\newrgbcolor{curcolor}{0 0 0}
\pscustom[linestyle=none,fillstyle=solid,fillcolor=curcolor]
{
\newpath
\moveto(416.23198026,19.86870901)
\lineto(418.89604232,19.53928198)
\curveto(418.4759035,17.98285863)(417.69769183,16.77495952)(416.56140729,15.91558466)
\curveto(415.42512276,15.0562098)(413.97373411,14.62652237)(412.20724135,14.62652237)
\curveto(409.98241532,14.62652237)(408.21592256,15.30924796)(406.90776305,16.67469912)
\curveto(405.60915216,18.04969889)(404.95984671,19.97374371)(404.95984671,22.44683358)
\curveto(404.95984671,25.00586094)(405.61870077,26.99197172)(406.93640888,28.40516593)
\curveto(408.254117,29.81836014)(409.9633181,30.52495725)(412.0640122,30.52495725)
\curveto(414.09786603,30.52495725)(415.75932409,29.83268306)(417.04838638,28.44813467)
\curveto(418.33744867,27.06358629)(418.98197981,25.11566995)(418.98197981,22.60438564)
\curveto(418.98197981,22.45160789)(418.9772055,22.22244126)(418.9676569,21.91688575)
\lineto(407.62390877,21.91688575)
\curveto(407.71939486,20.24587909)(408.19205104,18.96636541)(409.04187728,18.07834472)
\curveto(409.89170353,17.19032403)(410.95159919,16.74631369)(412.22156426,16.74631369)
\curveto(413.1668766,16.74631369)(413.97373411,16.99457754)(414.64213678,17.49110523)
\curveto(415.31053944,17.98763293)(415.84048727,18.78016752)(416.23198026,19.86870901)
\closepath
\moveto(407.76713791,24.03667707)
\lineto(416.26062609,24.03667707)
\curveto(416.14604278,25.31619075)(415.82139005,26.27582601)(415.28666792,26.91558284)
\curveto(414.4654875,27.90863824)(413.40081754,28.40516593)(412.09265803,28.40516593)
\curveto(410.90863045,28.40516593)(409.91080075,28.00889864)(409.09916894,27.21636404)
\curveto(408.29708574,26.42382945)(407.8530754,25.3639338)(407.76713791,24.03667707)
\closepath
}
}
{
\newrgbcolor{curcolor}{0 0 0}
\pscustom[linestyle=none,fillstyle=solid,fillcolor=curcolor]
{
\newpath
\moveto(432.54578002,19.86870901)
\lineto(435.20984208,19.53928198)
\curveto(434.78970326,17.98285863)(434.01149159,16.77495952)(432.87520705,15.91558466)
\curveto(431.73892252,15.0562098)(430.28753387,14.62652237)(428.52104111,14.62652237)
\curveto(426.29621508,14.62652237)(424.52972232,15.30924796)(423.22156281,16.67469912)
\curveto(421.92295192,18.04969889)(421.27364647,19.97374371)(421.27364647,22.44683358)
\curveto(421.27364647,25.00586094)(421.93250053,26.99197172)(423.25020864,28.40516593)
\curveto(424.56791676,29.81836014)(426.27711787,30.52495725)(428.37781196,30.52495725)
\curveto(430.41166579,30.52495725)(432.07312385,29.83268306)(433.36218614,28.44813467)
\curveto(434.65124843,27.06358629)(435.29577957,25.11566995)(435.29577957,22.60438564)
\curveto(435.29577957,22.45160789)(435.29100527,22.22244126)(435.28145666,21.91688575)
\lineto(423.93770853,21.91688575)
\curveto(424.03319463,20.24587909)(424.5058508,18.96636541)(425.35567705,18.07834472)
\curveto(426.20550329,17.19032403)(427.26539895,16.74631369)(428.53536402,16.74631369)
\curveto(429.48067636,16.74631369)(430.28753387,16.99457754)(430.95593654,17.49110523)
\curveto(431.6243392,17.98763293)(432.15428703,18.78016752)(432.54578002,19.86870901)
\closepath
\moveto(424.08093767,24.03667707)
\lineto(432.57442585,24.03667707)
\curveto(432.45984254,25.31619075)(432.13518981,26.27582601)(431.60046768,26.91558284)
\curveto(430.77928726,27.90863824)(429.7146173,28.40516593)(428.40645779,28.40516593)
\curveto(427.22243021,28.40516593)(426.22460051,28.00889864)(425.4129687,27.21636404)
\curveto(424.6108855,26.42382945)(424.16687516,25.3639338)(424.08093767,24.03667707)
\closepath
}
}
{
\newrgbcolor{curcolor}{0 0 0}
\pscustom[linestyle=none,fillstyle=solid,fillcolor=curcolor]
{
\newpath
\moveto(448.31530724,14.97027232)
\lineto(448.31530724,16.88954283)
\curveto(447.35089768,15.38086253)(445.93292916,14.62652237)(444.06140169,14.62652237)
\curveto(442.84872828,14.62652237)(441.73154097,14.96072371)(440.70983975,15.62912638)
\curveto(439.69768714,16.29752904)(438.90992685,17.22851847)(438.34655889,18.42209466)
\curveto(437.79273953,19.62521947)(437.51582986,21.00499354)(437.51582986,22.5614169)
\curveto(437.51582986,24.07964581)(437.76886801,25.45464559)(438.27494431,26.68641622)
\curveto(438.78102062,27.92773546)(439.54013508,28.8778221)(440.55228769,29.53667616)
\curveto(441.5644403,30.19553022)(442.69595053,30.52495725)(443.94681838,30.52495725)
\curveto(444.86348489,30.52495725)(445.67989101,30.32921075)(446.39603672,29.93771776)
\curveto(447.11218244,29.55577338)(447.69464762,29.05447138)(448.14343227,28.43381176)
\lineto(448.14343227,35.96766468)
\lineto(450.70723393,35.96766468)
\lineto(450.70723393,14.97027232)
\closepath
\moveto(440.165569,22.5614169)
\curveto(440.165569,20.61350055)(440.57615921,19.1573376)(441.39733963,18.19292804)
\curveto(442.21852005,17.22851847)(443.18770392,16.74631369)(444.30489123,16.74631369)
\curveto(445.43162716,16.74631369)(446.38648811,17.20464695)(447.1694741,18.12131346)
\curveto(447.96200869,19.04752859)(448.35827598,20.45594849)(448.35827598,22.34657318)
\curveto(448.35827598,24.42817006)(447.95723438,25.95594759)(447.15515118,26.92990576)
\curveto(446.35306798,27.90386393)(445.36478689,28.39084302)(444.19030792,28.39084302)
\curveto(443.04447478,28.39084302)(442.08483952,27.92296115)(441.31140215,26.98719742)
\curveto(440.54751338,26.05143368)(440.165569,24.57617351)(440.165569,22.5614169)
\closepath
}
}
{
\newrgbcolor{curcolor}{0 0 0}
\pscustom[linestyle=none,fillstyle=solid,fillcolor=curcolor]
{
\newpath
\moveto(457.13822642,14.97027232)
\lineto(454.74629974,14.97027232)
\lineto(454.74629974,35.96766468)
\lineto(457.32442431,35.96766468)
\lineto(457.32442431,28.4767805)
\curveto(458.4129658,29.84223167)(459.80228848,30.52495725)(461.49239237,30.52495725)
\curveto(462.42815611,30.52495725)(463.31140249,30.33398506)(464.14213152,29.95204068)
\curveto(464.98240916,29.5796449)(465.66990904,29.04969708)(466.20463118,28.36219719)
\curveto(466.74890192,27.68424591)(467.17381504,26.86306549)(467.47937055,25.89865593)
\curveto(467.78492605,24.93424637)(467.93770381,23.90299654)(467.93770381,22.80490644)
\curveto(467.93770381,20.19813604)(467.29317266,18.18337943)(466.00411038,16.76063661)
\curveto(464.71504809,15.33789378)(463.16817335,14.62652237)(461.36348614,14.62652237)
\curveto(459.56834755,14.62652237)(458.15992764,15.37608822)(457.13822642,16.87521992)
\closepath
\moveto(457.1095806,22.69032313)
\curveto(457.1095806,20.8665387)(457.35784444,19.54883059)(457.85437214,18.73719878)
\curveto(458.66600395,17.40994205)(459.76409405,16.74631369)(461.14864243,16.74631369)
\curveto(462.27537835,16.74631369)(463.24933653,17.23329278)(464.07051695,18.20725095)
\curveto(464.89169737,19.19075773)(465.30228758,20.65169499)(465.30228758,22.59006273)
\curveto(465.30228758,24.57617351)(464.90602028,26.04188507)(464.11348569,26.98719742)
\curveto(463.33049971,27.93250976)(462.38041306,28.40516593)(461.26322574,28.40516593)
\curveto(460.13648982,28.40516593)(459.16253165,27.91341254)(458.34135123,26.92990576)
\curveto(457.52017081,25.95594759)(457.1095806,24.54275338)(457.1095806,22.69032313)
\closepath
}
}
{
\newrgbcolor{curcolor}{0 0 0}
\pscustom[linestyle=none,fillstyle=solid,fillcolor=curcolor]
{
\newpath
\moveto(481.00019842,16.84657409)
\curveto(480.04533746,16.03494228)(479.12389664,15.46202571)(478.23587596,15.12782438)
\curveto(477.35740388,14.79362304)(476.41209154,14.62652237)(475.39993893,14.62652237)
\curveto(473.72893226,14.62652237)(472.44464428,15.03233828)(471.54707498,15.84397009)
\curveto(470.64950568,16.66515051)(470.20072103,17.71072325)(470.20072103,18.98068832)
\curveto(470.20072103,19.72547987)(470.3678217,20.40343114)(470.70202304,21.01454215)
\curveto(471.04577298,21.63520177)(471.48978332,22.13172947)(472.03405407,22.50412524)
\curveto(472.58787342,22.87652101)(473.20853304,23.15820499)(473.89603292,23.34917718)
\curveto(474.40210923,23.48285772)(475.16599799,23.61176395)(476.18769921,23.73589587)
\curveto(478.26929609,23.98415972)(479.80184792,24.28016661)(480.7853547,24.62391656)
\curveto(480.79490331,24.97721511)(480.79967762,25.20160743)(480.79967762,25.29709353)
\curveto(480.79967762,26.34744058)(480.55618807,27.08745782)(480.06920899,27.51714525)
\curveto(479.41035493,28.09961043)(478.43162245,28.39084302)(477.13301156,28.39084302)
\curveto(475.92033815,28.39084302)(475.02276885,28.1759993)(474.44030367,27.74631187)
\curveto(473.8673871,27.32617305)(473.44247397,26.57660721)(473.16556429,25.49761433)
\lineto(470.64473138,25.84136427)
\curveto(470.87389801,26.92035715)(471.25106808,27.78928062)(471.77624161,28.44813467)
\curveto(472.30141513,29.11653734)(473.06052959,29.62738795)(474.05358498,29.9806865)
\curveto(475.04664037,30.34353367)(476.19724782,30.52495725)(477.50540733,30.52495725)
\curveto(478.80401822,30.52495725)(479.85913958,30.3721795)(480.67077139,30.06662399)
\curveto(481.4824032,29.76106849)(482.07919129,29.3743498)(482.46113568,28.90646793)
\curveto(482.84308006,28.44813467)(483.11044112,27.86566949)(483.26321888,27.15907239)
\curveto(483.34915636,26.71983635)(483.39212511,25.92730176)(483.39212511,24.78146861)
\lineto(483.39212511,21.34396918)
\curveto(483.39212511,18.94726819)(483.44464246,17.42903927)(483.54967716,16.78928243)
\curveto(483.66426048,16.1590742)(483.8838785,15.5527375)(484.20853122,14.97027232)
\lineto(481.51582333,14.97027232)
\curveto(481.24846227,15.50499445)(481.07658729,16.13042838)(481.00019842,16.84657409)
\closepath
\moveto(480.7853547,22.60438564)
\curveto(479.84959097,22.22244126)(478.44594537,21.89778853)(476.5744179,21.63042747)
\curveto(475.51452224,21.47764972)(474.76495639,21.30577474)(474.32572035,21.11480255)
\curveto(473.88648431,20.92383036)(473.54750868,20.64214638)(473.30879344,20.26975061)
\curveto(473.0700782,19.90690345)(472.95072058,19.50108754)(472.95072058,19.05230289)
\curveto(472.95072058,18.36480301)(473.20853304,17.79188644)(473.72415795,17.33355318)
\curveto(474.24933148,16.87521992)(475.01322024,16.64605329)(476.01582424,16.64605329)
\curveto(477.00887963,16.64605329)(477.89212601,16.86089701)(478.66556339,17.29058443)
\curveto(479.43900076,17.72982047)(480.00714303,18.32660857)(480.36999019,19.08094872)
\curveto(480.64689986,19.6634139)(480.7853547,20.52278876)(480.7853547,21.6590733)
\closepath
}
}
{
\newrgbcolor{curcolor}{0 0 0}
\pscustom[linestyle=none,fillstyle=solid,fillcolor=curcolor]
{
\newpath
\moveto(497.31400178,20.54188598)
\lineto(499.84915761,20.21245895)
\curveto(499.57224794,18.46506341)(498.86087653,17.09483794)(497.71504338,16.10178255)
\curveto(496.57875885,15.11827577)(495.17988755,14.62652237)(493.51842949,14.62652237)
\curveto(491.43683261,14.62652237)(489.76105164,15.30447365)(488.49108657,16.66037621)
\curveto(487.23067011,18.02582737)(486.60046189,19.97851802)(486.60046189,22.51844815)
\curveto(486.60046189,24.16080899)(486.87259726,25.59787473)(487.416868,26.82964536)
\curveto(487.96113874,28.06141599)(488.78709347,28.98285681)(489.89473217,29.59396782)
\curveto(491.01191949,30.21462744)(492.2245929,30.52495725)(493.53275241,30.52495725)
\curveto(495.18466186,30.52495725)(496.53579011,30.10481843)(497.58613715,29.26454079)
\curveto(498.6364842,28.43381176)(499.30966117,27.24978418)(499.60566807,25.71245804)
\lineto(497.09915807,25.32573936)
\curveto(496.86044283,26.34744058)(496.43552971,27.11610364)(495.82441869,27.63172856)
\curveto(495.22285629,28.14735347)(494.49238766,28.40516593)(493.63301281,28.40516593)
\curveto(492.33440191,28.40516593)(491.27928056,27.93728406)(490.46764875,27.00152033)
\curveto(489.65601694,26.07530521)(489.25020103,24.60481934)(489.25020103,22.59006273)
\curveto(489.25020103,20.54666029)(489.64169402,19.0618515)(490.42468,18.13563638)
\curveto(491.20766599,17.20942125)(492.22936721,16.74631369)(493.48978366,16.74631369)
\curveto(494.50193627,16.74631369)(495.34698822,17.0566435)(496.0249395,17.67730312)
\curveto(496.70289077,18.29796274)(497.1325782,19.25282369)(497.31400178,20.54188598)
\closepath
}
}
{
\newrgbcolor{curcolor}{0 0 0}
\pscustom[linestyle=none,fillstyle=solid,fillcolor=curcolor]
{
\newpath
\moveto(502.06920663,14.97027232)
\lineto(502.06920663,35.96766468)
\lineto(504.6473312,35.96766468)
\lineto(504.6473312,23.99370833)
\lineto(510.74889269,30.18120731)
\lineto(514.08613172,30.18120731)
\lineto(508.27102852,24.53797907)
\lineto(514.67337121,14.97027232)
\lineto(511.49368424,14.97027232)
\lineto(506.46634132,22.74761478)
\lineto(504.6473312,21.00021924)
\lineto(504.6473312,14.97027232)
\closepath
}
}
{
\newrgbcolor{curcolor}{0 0 0}
\pscustom[linewidth=1.51181105,linecolor=curcolor]
{
\newpath
\moveto(380.49932558,50.89867952)
\lineto(530.3565718,50.89867952)
\curveto(544.24611593,50.89867952)(555.42795109,39.71684436)(555.42795109,25.82730023)
\curveto(555.42795109,11.9377561)(544.24611593,0.75592093)(530.3565718,0.75592093)
\lineto(380.49932558,0.75592093)
\curveto(366.60978145,0.75592093)(355.42794629,11.9377561)(355.42794629,25.82730023)
\curveto(355.42794629,39.71684436)(366.60978145,50.89867952)(380.49932558,50.89867952)
\closepath
}
}
{
\newrgbcolor{curcolor}{0 0 0}
\pscustom[linewidth=1.51181105,linecolor=curcolor]
{
\newpath
\moveto(615.42677669,81.07846818)
\lineto(615.41543811,38.94966188)
\curveto(615.41543811,38.94966188)(615.92839559,34.15472645)(611.45736567,29.75766188)
\curveto(606.98633575,25.36052172)(602.19147591,25.85182251)(602.19147591,25.85182251)
\lineto(558.14081764,25.79286188)
}
}
{
\newrgbcolor{curcolor}{0 0 0}
\pscustom[linestyle=none,fillstyle=solid,fillcolor=curcolor]
{
\newpath
\moveto(566.05422236,22.14450322)
\lineto(556.13863608,25.77565659)
\lineto(566.04446699,29.43333959)
\curveto(564.46403009,27.27955756)(564.47709346,24.33572163)(566.05422236,22.14450322)
\closepath
}
}
{
\newrgbcolor{curcolor}{0 0 0}
\pscustom[linewidth=0.56692914,linecolor=curcolor]
{
\newpath
\moveto(566.05422236,22.14450322)
\lineto(556.13863608,25.77565659)
\lineto(566.04446699,29.43333959)
\curveto(564.46403009,27.27955756)(564.47709346,24.33572163)(566.05422236,22.14450322)
\closepath
}
}
{
\newrgbcolor{curcolor}{0 0 0}
\pscustom[linewidth=1.51181105,linecolor=curcolor]
{
\newpath
\moveto(395.4267515,105.82594534)
\lineto(422.53197354,105.85240204)
\curveto(422.53197354,105.85240204)(427.32690898,105.33944456)(431.72397354,109.81047448)
\curveto(436.1211515,114.2815044)(435.62985071,119.07636424)(435.62985071,119.07636424)
\lineto(435.78405543,168.32742566)
}
}
{
\newrgbcolor{curcolor}{0 0 0}
\pscustom[linestyle=none,fillstyle=solid,fillcolor=curcolor]
{
\newpath
\moveto(432.10036473,160.4304061)
\lineto(435.77579894,170.32966411)
\lineto(439.38917189,160.40758435)
\curveto(437.24247505,161.99763166)(434.29861014,161.99772572)(432.10036473,160.4304061)
\closepath
}
}
{
\newrgbcolor{curcolor}{0 0 0}
\pscustom[linewidth=0.56692914,linecolor=curcolor]
{
\newpath
\moveto(432.10036473,160.4304061)
\lineto(435.77579894,170.32966411)
\lineto(439.38917189,160.40758435)
\curveto(437.24247505,161.99763166)(434.29861014,161.99772572)(432.10036473,160.4304061)
\closepath
}
}
{
\newrgbcolor{curcolor}{0 0 0}
\pscustom[linewidth=1.51181105,linecolor=curcolor]
{
\newpath
\moveto(155.38859717,171.04532172)
\lineto(155.44226646,119.13090282)
\curveto(155.44226646,119.13090282)(154.92930898,114.33596739)(159.4003389,109.93890282)
\curveto(163.87136882,105.54176267)(168.66622866,106.03306345)(168.66622866,106.03306345)
\lineto(192.88859339,106.04440204)
}
}
{
\newrgbcolor{curcolor}{0 0 0}
\pscustom[linestyle=none,fillstyle=solid,fillcolor=curcolor]
{
\newpath
\moveto(184.97836708,109.69964689)
\lineto(194.89078916,106.05986469)
\lineto(184.98177847,102.41080479)
\curveto(186.56408935,104.56321044)(186.55358821,107.50705662)(184.97836708,109.69964689)
\closepath
}
}
{
\newrgbcolor{curcolor}{0 0 0}
\pscustom[linewidth=0.56692914,linecolor=curcolor]
{
\newpath
\moveto(184.97836708,109.69964689)
\lineto(194.89078916,106.05986469)
\lineto(184.98177847,102.41080479)
\curveto(186.56408935,104.56321044)(186.55358821,107.50705662)(184.97836708,109.69964689)
\closepath
}
}
{
\newrgbcolor{curcolor}{0 0 0}
\pscustom[linewidth=1.51181105,linecolor=curcolor]
{
\newpath
\moveto(718.39180346,106.091117)
\lineto(743.53012157,105.74453432)
\curveto(743.53012157,105.74453432)(747.97707591,105.20942881)(752.05503496,109.873517)
\curveto(756.1331074,114.53760519)(755.67744756,119.53954534)(755.67744756,119.53954534)
\lineto(755.84677039,171.28997054)
}
}
{
\newrgbcolor{curcolor}{0 0 0}
\pscustom[linestyle=none,fillstyle=solid,fillcolor=curcolor]
{
\newpath
\moveto(726.24912513,102.3234929)
\lineto(716.3895907,106.10419476)
\lineto(726.3496076,109.61164314)
\curveto(724.73677756,107.48201006)(724.70531672,104.53831326)(726.24912513,102.3234929)
\closepath
}
}
{
\newrgbcolor{curcolor}{0 0 0}
\pscustom[linewidth=0.56692914,linecolor=curcolor]
{
\newpath
\moveto(726.24912513,102.3234929)
\lineto(716.3895907,106.10419476)
\lineto(726.3496076,109.61164314)
\curveto(724.73677756,107.48201006)(724.70531672,104.53831326)(726.24912513,102.3234929)
\closepath
}
}
{
\newrgbcolor{curcolor}{0 0 0}
\pscustom[linestyle=none,fillstyle=solid,fillcolor=curcolor]
{
\newpath
\moveto(521.38090583,353.06458723)
\lineto(540.97397669,353.06458723)
\lineto(540.97397669,354.22112267)
\lineto(521.38090583,354.22112267)
}
}
{
\newrgbcolor{curcolor}{0 0 0}
\pscustom[linewidth=0,linecolor=curcolor]
{
\newpath
\moveto(521.38090583,353.06458723)
\lineto(540.97397669,353.06458723)
\lineto(540.97397669,354.22112267)
\lineto(521.38090583,354.22112267)
}
}
{
\newrgbcolor{curcolor}{0 0 0}
\pscustom[linestyle=none,fillstyle=solid,fillcolor=curcolor]
{
\newpath
\moveto(539.26752,338.24505967)
\lineto(537.94090583,338.24505967)
\curveto(537.71980346,336.29482345)(537.2379137,331.82742188)(532.18090583,331.82742188)
\lineto(529.26689008,331.82742188)
\lineto(529.26689008,348.54616204)
\lineto(533.08232315,348.54616204)
\lineto(533.08232315,349.87277621)
\curveto(531.84641764,349.79340613)(528.76232315,349.79340613)(527.37901606,349.79340613)
\curveto(526.13744126,349.79340613)(523.28578772,349.79340613)(522.20862236,349.87277621)
\lineto(522.20862236,348.54616204)
\lineto(525.26437039,348.54616204)
\lineto(525.26437039,331.82742188)
\lineto(522.20862236,331.82742188)
\lineto(522.20862236,330.50080771)
\lineto(538.38877984,330.50080771)
\closepath
}
}
{
\newrgbcolor{curcolor}{0 0 0}
\pscustom[linewidth=0,linecolor=curcolor]
{
\newpath
\moveto(539.26752,338.24505967)
\lineto(537.94090583,338.24505967)
\curveto(537.71980346,336.29482345)(537.2379137,331.82742188)(532.18090583,331.82742188)
\lineto(529.26689008,331.82742188)
\lineto(529.26689008,348.54616204)
\lineto(533.08232315,348.54616204)
\lineto(533.08232315,349.87277621)
\curveto(531.84641764,349.79340613)(528.76232315,349.79340613)(527.37901606,349.79340613)
\curveto(526.13744126,349.79340613)(523.28578772,349.79340613)(522.20862236,349.87277621)
\lineto(522.20862236,348.54616204)
\lineto(525.26437039,348.54616204)
\lineto(525.26437039,331.82742188)
\lineto(522.20862236,331.82742188)
\lineto(522.20862236,330.50080771)
\lineto(538.38877984,330.50080771)
\closepath
}
}
{
\newrgbcolor{curcolor}{0 0 0}
\pscustom[linestyle=none,fillstyle=solid,fillcolor=curcolor]
{
\newpath
\moveto(549.97681134,323.72600456)
\lineto(549.97681134,323.73734314)
\lineto(549.97681134,323.75435101)
\lineto(549.97681134,323.7600203)
\lineto(549.97681134,323.77135889)
\lineto(549.97114205,323.77702818)
\lineto(549.97114205,323.78836676)
\lineto(549.96547276,323.79403605)
\lineto(549.96547276,323.80537463)
\lineto(549.95980346,323.81671322)
\lineto(549.95413417,323.8280518)
\lineto(549.94279559,323.83939038)
\lineto(549.9371263,323.85639826)
\lineto(549.92578772,323.86773684)
\lineto(549.92011843,323.88474471)
\lineto(549.90311055,323.90175259)
\lineto(549.89177197,323.91876046)
\lineto(549.87476409,323.93576834)
\lineto(549.86342551,323.9584455)
\lineto(549.85208693,323.96978408)
\lineto(549.84074835,323.98112267)
\lineto(549.83507906,323.99246125)
\lineto(549.82374047,324.00379983)
\lineto(549.81240189,324.01513841)
\lineto(549.80106331,324.03214629)
\lineto(549.78972472,324.04348487)
\lineto(549.77838614,324.05482345)
\lineto(549.76137827,324.07183133)
\lineto(549.75003969,324.08316991)
\lineto(549.73303181,324.10017778)
\lineto(549.72169323,324.11718566)
\lineto(549.70468535,324.13419353)
\lineto(549.68767748,324.15120141)
\lineto(549.6763389,324.16820928)
\lineto(549.65933102,324.18521715)
\lineto(549.64232315,324.20222503)
\lineto(549.61964598,324.2192329)
\lineto(549.60263811,324.24191007)
\lineto(549.58563024,324.25891794)
\lineto(549.56295307,324.28159511)
\lineto(549.54027591,324.29860298)
\lineto(549.52326803,324.32128015)
\lineto(549.50059087,324.34395731)
\curveto(545.97429165,327.90427227)(545.07287433,333.23907542)(545.07287433,337.55907542)
\curveto(545.07287433,342.47435101)(546.14437039,347.39529589)(549.61397669,350.91592582)
\curveto(549.97681134,351.26175259)(549.97681134,351.3184455)(549.97681134,351.40348487)
\curveto(549.97681134,351.60191007)(549.8690948,351.68128015)(549.70468535,351.68128015)
\curveto(549.42122079,351.68128015)(546.87570898,349.76505967)(545.21460661,346.18206755)
\curveto(543.76893732,343.0696266)(543.42877984,339.93450849)(543.42877984,337.55907542)
\curveto(543.42877984,335.35372109)(543.74059087,331.946477)(545.29397669,328.74899668)
\curveto(546.98342551,325.27372109)(549.42122079,323.44253999)(549.70468535,323.44253999)
\curveto(549.8690948,323.44253999)(549.97681134,323.52191007)(549.97681134,323.72600456)
\closepath
}
}
{
\newrgbcolor{curcolor}{0 0 0}
\pscustom[linewidth=0,linecolor=curcolor]
{
\newpath
\moveto(549.97681134,323.72600456)
\lineto(549.97681134,323.73734314)
\lineto(549.97681134,323.75435101)
\lineto(549.97681134,323.7600203)
\lineto(549.97681134,323.77135889)
\lineto(549.97114205,323.77702818)
\lineto(549.97114205,323.78836676)
\lineto(549.96547276,323.79403605)
\lineto(549.96547276,323.80537463)
\lineto(549.95980346,323.81671322)
\lineto(549.95413417,323.8280518)
\lineto(549.94279559,323.83939038)
\lineto(549.9371263,323.85639826)
\lineto(549.92578772,323.86773684)
\lineto(549.92011843,323.88474471)
\lineto(549.90311055,323.90175259)
\lineto(549.89177197,323.91876046)
\lineto(549.87476409,323.93576834)
\lineto(549.86342551,323.9584455)
\lineto(549.85208693,323.96978408)
\lineto(549.84074835,323.98112267)
\lineto(549.83507906,323.99246125)
\lineto(549.82374047,324.00379983)
\lineto(549.81240189,324.01513841)
\lineto(549.80106331,324.03214629)
\lineto(549.78972472,324.04348487)
\lineto(549.77838614,324.05482345)
\lineto(549.76137827,324.07183133)
\lineto(549.75003969,324.08316991)
\lineto(549.73303181,324.10017778)
\lineto(549.72169323,324.11718566)
\lineto(549.70468535,324.13419353)
\lineto(549.68767748,324.15120141)
\lineto(549.6763389,324.16820928)
\lineto(549.65933102,324.18521715)
\lineto(549.64232315,324.20222503)
\lineto(549.61964598,324.2192329)
\lineto(549.60263811,324.24191007)
\lineto(549.58563024,324.25891794)
\lineto(549.56295307,324.28159511)
\lineto(549.54027591,324.29860298)
\lineto(549.52326803,324.32128015)
\lineto(549.50059087,324.34395731)
\curveto(545.97429165,327.90427227)(545.07287433,333.23907542)(545.07287433,337.55907542)
\curveto(545.07287433,342.47435101)(546.14437039,347.39529589)(549.61397669,350.91592582)
\curveto(549.97681134,351.26175259)(549.97681134,351.3184455)(549.97681134,351.40348487)
\curveto(549.97681134,351.60191007)(549.8690948,351.68128015)(549.70468535,351.68128015)
\curveto(549.42122079,351.68128015)(546.87570898,349.76505967)(545.21460661,346.18206755)
\curveto(543.76893732,343.0696266)(543.42877984,339.93450849)(543.42877984,337.55907542)
\curveto(543.42877984,335.35372109)(543.74059087,331.946477)(545.29397669,328.74899668)
\curveto(546.98342551,325.27372109)(549.42122079,323.44253999)(549.70468535,323.44253999)
\curveto(549.8690948,323.44253999)(549.97681134,323.52191007)(549.97681134,323.72600456)
\closepath
}
}
{
\newrgbcolor{curcolor}{0 0 0}
\pscustom[linestyle=none,fillstyle=solid,fillcolor=curcolor]
{
\newpath
\moveto(551.42814992,346.26143763)
\lineto(569.52452787,346.26143763)
\lineto(569.52452787,347.41797306)
\lineto(551.42814992,347.41797306)
}
}
{
\newrgbcolor{curcolor}{0 0 0}
\pscustom[linewidth=0,linecolor=curcolor]
{
\newpath
\moveto(551.42814992,346.26143763)
\lineto(569.52452787,346.26143763)
\lineto(569.52452787,347.41797306)
\lineto(551.42814992,347.41797306)
}
}
{
\newrgbcolor{curcolor}{0 0 0}
\pscustom[linestyle=none,fillstyle=solid,fillcolor=curcolor]
{
\newpath
\moveto(563.95728378,332.5360833)
\lineto(563.95728378,330.33072897)
\lineto(568.95192945,330.50080771)
\lineto(568.95192945,331.82742188)
\curveto(567.20011843,331.82742188)(567.00736252,331.82742188)(567.00736252,332.9272644)
\lineto(567.00736252,343.21135889)
\lineto(561.83129953,342.97891794)
\lineto(561.83129953,341.65230377)
\curveto(563.58877984,341.65230377)(563.78153575,341.65230377)(563.78153575,340.55246125)
\lineto(563.78153575,335.16096519)
\curveto(563.78153575,332.84789432)(562.34720504,331.35120141)(560.28358299,331.35120141)
\curveto(558.11224441,331.35120141)(558.02153575,332.04852424)(558.02153575,333.58490219)
\lineto(558.02153575,343.21135889)
\lineto(552.85114205,342.97891794)
\lineto(552.85114205,341.65230377)
\curveto(554.60295307,341.65230377)(554.80137827,341.65230377)(554.80137827,340.55246125)
\lineto(554.80137827,333.970414)
\curveto(554.80137827,330.95435101)(557.09177197,330.33072897)(559.88106331,330.33072897)
\curveto(560.61807118,330.33072897)(562.65334677,330.33072897)(563.95728378,332.5360833)
\closepath
}
}
{
\newrgbcolor{curcolor}{0 0 0}
\pscustom[linewidth=0,linecolor=curcolor]
{
\newpath
\moveto(563.95728378,332.5360833)
\lineto(563.95728378,330.33072897)
\lineto(568.95192945,330.50080771)
\lineto(568.95192945,331.82742188)
\curveto(567.20011843,331.82742188)(567.00736252,331.82742188)(567.00736252,332.9272644)
\lineto(567.00736252,343.21135889)
\lineto(561.83129953,342.97891794)
\lineto(561.83129953,341.65230377)
\curveto(563.58877984,341.65230377)(563.78153575,341.65230377)(563.78153575,340.55246125)
\lineto(563.78153575,335.16096519)
\curveto(563.78153575,332.84789432)(562.34720504,331.35120141)(560.28358299,331.35120141)
\curveto(558.11224441,331.35120141)(558.02153575,332.04852424)(558.02153575,333.58490219)
\lineto(558.02153575,343.21135889)
\lineto(552.85114205,342.97891794)
\lineto(552.85114205,341.65230377)
\curveto(554.60295307,341.65230377)(554.80137827,341.65230377)(554.80137827,340.55246125)
\lineto(554.80137827,333.970414)
\curveto(554.80137827,330.95435101)(557.09177197,330.33072897)(559.88106331,330.33072897)
\curveto(560.61807118,330.33072897)(562.65334677,330.33072897)(563.95728378,332.5360833)
\closepath
}
}
{
\newrgbcolor{curcolor}{0 0 0}
\pscustom[linestyle=none,fillstyle=solid,fillcolor=curcolor]
{
\newpath
\moveto(577.83570898,337.55907542)
\lineto(577.83570898,337.7688392)
\lineto(577.83570898,337.98994156)
\lineto(577.83003969,338.21104393)
\lineto(577.8187011,338.44348487)
\lineto(577.79602394,338.91970534)
\lineto(577.76200819,339.41860298)
\lineto(577.71665386,339.94017778)
\lineto(577.65429165,340.47876046)
\lineto(577.58059087,341.02868172)
\lineto(577.4898822,341.60128015)
\lineto(577.37649638,342.17954786)
\lineto(577.24610268,342.76348487)
\lineto(577.09303181,343.35876046)
\lineto(576.92295307,343.95970534)
\lineto(576.72452787,344.56631952)
\lineto(576.50342551,345.17293369)
\lineto(576.3787011,345.47340613)
\lineto(576.25397669,345.77387857)
\lineto(576.1179137,346.07435101)
\lineto(575.97618142,346.37482345)
\curveto(574.2867326,349.84442975)(571.85460661,351.68128015)(571.57114205,351.68128015)
\curveto(571.4067326,351.68128015)(571.28767748,351.57356361)(571.28767748,351.40348487)
\curveto(571.28767748,351.3184455)(571.28767748,351.26175259)(571.82059087,350.75151637)
\curveto(574.59287433,347.96222503)(576.20862236,343.466477)(576.20862236,337.55907542)
\curveto(576.20862236,332.73450849)(575.15980346,327.76253999)(571.65618142,324.20222503)
\curveto(571.28767748,323.86773684)(571.28767748,323.80537463)(571.28767748,323.72600456)
\curveto(571.28767748,323.55592582)(571.4067326,323.44253999)(571.57114205,323.44253999)
\curveto(571.85460661,323.44253999)(574.38877984,325.36442975)(576.06689008,328.94742188)
\curveto(577.50122079,332.04852424)(577.83570898,335.18364235)(577.83570898,337.55907542)
\closepath
}
}
{
\newrgbcolor{curcolor}{0 0 0}
\pscustom[linewidth=0,linecolor=curcolor]
{
\newpath
\moveto(577.83570898,337.55907542)
\lineto(577.83570898,337.7688392)
\lineto(577.83570898,337.98994156)
\lineto(577.83003969,338.21104393)
\lineto(577.8187011,338.44348487)
\lineto(577.79602394,338.91970534)
\lineto(577.76200819,339.41860298)
\lineto(577.71665386,339.94017778)
\lineto(577.65429165,340.47876046)
\lineto(577.58059087,341.02868172)
\lineto(577.4898822,341.60128015)
\lineto(577.37649638,342.17954786)
\lineto(577.24610268,342.76348487)
\lineto(577.09303181,343.35876046)
\lineto(576.92295307,343.95970534)
\lineto(576.72452787,344.56631952)
\lineto(576.50342551,345.17293369)
\lineto(576.3787011,345.47340613)
\lineto(576.25397669,345.77387857)
\lineto(576.1179137,346.07435101)
\lineto(575.97618142,346.37482345)
\curveto(574.2867326,349.84442975)(571.85460661,351.68128015)(571.57114205,351.68128015)
\curveto(571.4067326,351.68128015)(571.28767748,351.57356361)(571.28767748,351.40348487)
\curveto(571.28767748,351.3184455)(571.28767748,351.26175259)(571.82059087,350.75151637)
\curveto(574.59287433,347.96222503)(576.20862236,343.466477)(576.20862236,337.55907542)
\curveto(576.20862236,332.73450849)(575.15980346,327.76253999)(571.65618142,324.20222503)
\curveto(571.28767748,323.86773684)(571.28767748,323.80537463)(571.28767748,323.72600456)
\curveto(571.28767748,323.55592582)(571.4067326,323.44253999)(571.57114205,323.44253999)
\curveto(571.85460661,323.44253999)(574.38877984,325.36442975)(576.06689008,328.94742188)
\curveto(577.50122079,332.04852424)(577.83570898,335.18364235)(577.83570898,337.55907542)
\closepath
}
}
{
\newrgbcolor{curcolor}{0 0 0}
\pscustom[linestyle=none,fillstyle=solid,fillcolor=curcolor]
{
\newpath
\moveto(607.9283074,339.74175259)
\lineto(608.00767748,339.74175259)
\lineto(608.08704756,339.74175259)
\lineto(608.17208693,339.74175259)
\lineto(608.21177197,339.74742188)
\lineto(608.25145701,339.74742188)
\lineto(608.29681134,339.75309117)
\lineto(608.33649638,339.75876046)
\lineto(608.37618142,339.76442975)
\lineto(608.41586646,339.77009904)
\lineto(608.4555515,339.77576834)
\lineto(608.48956724,339.78710692)
\lineto(608.52925228,339.7984455)
\lineto(608.56326803,339.80978408)
\lineto(608.59728378,339.82112267)
\lineto(608.63129953,339.83813054)
\lineto(608.65964598,339.85513841)
\lineto(608.69366173,339.87781558)
\lineto(608.72200819,339.89482345)
\lineto(608.74468535,339.92316991)
\lineto(608.77303181,339.94584708)
\lineto(608.79570898,339.97419353)
\lineto(608.80704756,339.99120141)
\lineto(608.81271685,340.00820928)
\lineto(608.82405543,340.01954786)
\lineto(608.83539402,340.03655574)
\lineto(608.84106331,340.05356361)
\lineto(608.8467326,340.07624078)
\lineto(608.85240189,340.09324865)
\lineto(608.86374047,340.11592582)
\lineto(608.86940976,340.13293369)
\lineto(608.86940976,340.15561086)
\lineto(608.87507906,340.17828802)
\lineto(608.88074835,340.20096519)
\lineto(608.88074835,340.22364235)
\lineto(608.88641764,340.24631952)
\lineto(608.88641764,340.26899668)
\lineto(608.88641764,340.29734314)
\curveto(608.88641764,340.86427227)(608.34783496,340.86427227)(607.95665386,340.86427227)
\lineto(591.03381921,340.86427227)
\curveto(590.63696882,340.86427227)(590.10405543,340.86427227)(590.10405543,340.29734314)
\curveto(590.10405543,339.74175259)(590.63696882,339.74175259)(591.06216567,339.74175259)
\closepath
}
}
{
\newrgbcolor{curcolor}{0 0 0}
\pscustom[linewidth=0,linecolor=curcolor]
{
\newpath
\moveto(607.9283074,339.74175259)
\lineto(608.00767748,339.74175259)
\lineto(608.08704756,339.74175259)
\lineto(608.17208693,339.74175259)
\lineto(608.21177197,339.74742188)
\lineto(608.25145701,339.74742188)
\lineto(608.29681134,339.75309117)
\lineto(608.33649638,339.75876046)
\lineto(608.37618142,339.76442975)
\lineto(608.41586646,339.77009904)
\lineto(608.4555515,339.77576834)
\lineto(608.48956724,339.78710692)
\lineto(608.52925228,339.7984455)
\lineto(608.56326803,339.80978408)
\lineto(608.59728378,339.82112267)
\lineto(608.63129953,339.83813054)
\lineto(608.65964598,339.85513841)
\lineto(608.69366173,339.87781558)
\lineto(608.72200819,339.89482345)
\lineto(608.74468535,339.92316991)
\lineto(608.77303181,339.94584708)
\lineto(608.79570898,339.97419353)
\lineto(608.80704756,339.99120141)
\lineto(608.81271685,340.00820928)
\lineto(608.82405543,340.01954786)
\lineto(608.83539402,340.03655574)
\lineto(608.84106331,340.05356361)
\lineto(608.8467326,340.07624078)
\lineto(608.85240189,340.09324865)
\lineto(608.86374047,340.11592582)
\lineto(608.86940976,340.13293369)
\lineto(608.86940976,340.15561086)
\lineto(608.87507906,340.17828802)
\lineto(608.88074835,340.20096519)
\lineto(608.88074835,340.22364235)
\lineto(608.88641764,340.24631952)
\lineto(608.88641764,340.26899668)
\lineto(608.88641764,340.29734314)
\curveto(608.88641764,340.86427227)(608.34783496,340.86427227)(607.95665386,340.86427227)
\lineto(591.03381921,340.86427227)
\curveto(590.63696882,340.86427227)(590.10405543,340.86427227)(590.10405543,340.29734314)
\curveto(590.10405543,339.74175259)(590.63696882,339.74175259)(591.06216567,339.74175259)
\closepath
}
}
{
\newrgbcolor{curcolor}{0 0 0}
\pscustom[linestyle=none,fillstyle=solid,fillcolor=curcolor]
{
\newpath
\moveto(607.95665386,334.25387857)
\lineto(607.99066961,334.25387857)
\lineto(608.03035465,334.25387857)
\lineto(608.10972472,334.25954786)
\lineto(608.18342551,334.25954786)
\lineto(608.22311055,334.25954786)
\lineto(608.26279559,334.26521715)
\lineto(608.30248063,334.27088645)
\lineto(608.34216567,334.27088645)
\lineto(608.38185071,334.27655574)
\lineto(608.42153575,334.28789432)
\lineto(608.46122079,334.29356361)
\lineto(608.49523654,334.30490219)
\lineto(608.52925228,334.31624078)
\lineto(608.56893732,334.32757936)
\lineto(608.60295307,334.33891794)
\lineto(608.63129953,334.35592582)
\lineto(608.66531528,334.37293369)
\lineto(608.69366173,334.39561086)
\lineto(608.72200819,334.41828802)
\lineto(608.75035465,334.44096519)
\lineto(608.77303181,334.46931164)
\lineto(608.78437039,334.48065023)
\lineto(608.79570898,334.4976581)
\lineto(608.80704756,334.50899668)
\lineto(608.81271685,334.52600456)
\lineto(608.82405543,334.54301243)
\lineto(608.83539402,334.5600203)
\lineto(608.84106331,334.57702818)
\lineto(608.8467326,334.59403605)
\lineto(608.85240189,334.61671322)
\lineto(608.86374047,334.63372109)
\lineto(608.86940976,334.65639826)
\lineto(608.86940976,334.67907542)
\lineto(608.87507906,334.70175259)
\lineto(608.88074835,334.72442975)
\lineto(608.88074835,334.74710692)
\lineto(608.88641764,334.76978408)
\lineto(608.88641764,334.79813054)
\lineto(608.88641764,334.82080771)
\curveto(608.88641764,335.38773684)(608.34783496,335.38773684)(607.9283074,335.38773684)
\lineto(591.06216567,335.38773684)
\curveto(590.63696882,335.38773684)(590.10405543,335.38773684)(590.10405543,334.82080771)
\curveto(590.10405543,334.25387857)(590.63696882,334.25387857)(591.03381921,334.25387857)
\closepath
}
}
{
\newrgbcolor{curcolor}{0 0 0}
\pscustom[linewidth=0,linecolor=curcolor]
{
\newpath
\moveto(607.95665386,334.25387857)
\lineto(607.99066961,334.25387857)
\lineto(608.03035465,334.25387857)
\lineto(608.10972472,334.25954786)
\lineto(608.18342551,334.25954786)
\lineto(608.22311055,334.25954786)
\lineto(608.26279559,334.26521715)
\lineto(608.30248063,334.27088645)
\lineto(608.34216567,334.27088645)
\lineto(608.38185071,334.27655574)
\lineto(608.42153575,334.28789432)
\lineto(608.46122079,334.29356361)
\lineto(608.49523654,334.30490219)
\lineto(608.52925228,334.31624078)
\lineto(608.56893732,334.32757936)
\lineto(608.60295307,334.33891794)
\lineto(608.63129953,334.35592582)
\lineto(608.66531528,334.37293369)
\lineto(608.69366173,334.39561086)
\lineto(608.72200819,334.41828802)
\lineto(608.75035465,334.44096519)
\lineto(608.77303181,334.46931164)
\lineto(608.78437039,334.48065023)
\lineto(608.79570898,334.4976581)
\lineto(608.80704756,334.50899668)
\lineto(608.81271685,334.52600456)
\lineto(608.82405543,334.54301243)
\lineto(608.83539402,334.5600203)
\lineto(608.84106331,334.57702818)
\lineto(608.8467326,334.59403605)
\lineto(608.85240189,334.61671322)
\lineto(608.86374047,334.63372109)
\lineto(608.86940976,334.65639826)
\lineto(608.86940976,334.67907542)
\lineto(608.87507906,334.70175259)
\lineto(608.88074835,334.72442975)
\lineto(608.88074835,334.74710692)
\lineto(608.88641764,334.76978408)
\lineto(608.88641764,334.79813054)
\lineto(608.88641764,334.82080771)
\curveto(608.88641764,335.38773684)(608.34783496,335.38773684)(607.9283074,335.38773684)
\lineto(591.06216567,335.38773684)
\curveto(590.63696882,335.38773684)(590.10405543,335.38773684)(590.10405543,334.82080771)
\curveto(590.10405543,334.25387857)(590.63696882,334.25387857)(591.03381921,334.25387857)
\closepath
}
}
{
\newrgbcolor{curcolor}{0 0 0}
\pscustom[linestyle=none,fillstyle=solid,fillcolor=curcolor]
{
\newpath
\moveto(633.2587011,339.50931164)
\lineto(633.2587011,339.77576834)
\lineto(633.2587011,340.06490219)
\lineto(633.25303181,340.37671322)
\lineto(633.24736252,340.70553211)
\lineto(633.23602394,341.0456896)
\lineto(633.21334677,341.40285495)
\lineto(633.18500031,341.77135889)
\lineto(633.15098457,342.14553211)
\lineto(633.10563024,342.53104393)
\lineto(633.04893732,342.92222503)
\lineto(632.97523654,343.31907542)
\lineto(632.89586646,343.71592582)
\lineto(632.79381921,344.11277621)
\lineto(632.68043339,344.5096266)
\lineto(632.55003969,344.90080771)
\lineto(632.39696882,345.28631952)
\lineto(632.22689008,345.66049274)
\lineto(632.03413417,346.02899668)
\lineto(631.8187011,346.38049274)
\lineto(631.58059087,346.72065023)
\lineto(631.31413417,347.04946912)
\lineto(631.02500031,347.35561086)
\lineto(630.87192945,347.50301243)
\lineto(630.70752,347.63907542)
\lineto(630.54311055,347.77513841)
\lineto(630.36736252,347.90553211)
\lineto(630.18027591,348.03025652)
\lineto(629.98752,348.14931164)
\lineto(629.7890948,348.25702818)
\lineto(629.58500031,348.36474471)
\lineto(629.36956724,348.46112267)
\lineto(629.14279559,348.55183133)
\lineto(628.91035465,348.6368707)
\lineto(628.67224441,348.71057148)
\lineto(628.42279559,348.77860298)
\lineto(628.16200819,348.83529589)
\lineto(627.8955515,348.88631952)
\lineto(627.62342551,348.92600456)
\lineto(627.33429165,348.9600203)
\lineto(627.0394885,348.98836676)
\lineto(626.73901606,348.99970534)
\lineto(626.42153575,349.00537463)
\lineto(626.42153575,347.98490219)
\curveto(627.47035465,347.98490219)(628.62689008,347.44631952)(629.15980346,346.20474471)
\curveto(629.64736252,345.16159511)(629.64736252,342.16820928)(629.64736252,339.84379983)
\curveto(629.64736252,338.26773684)(629.64736252,335.69954786)(629.44326803,334.23120141)
\curveto(629.05208693,331.60065023)(627.27192945,331.20946912)(626.42153575,331.20946912)
\lineto(626.42153575,331.20946912)
\curveto(625.40106331,331.20946912)(623.82500031,331.82742188)(623.45649638,333.89104393)
\curveto(623.20704756,335.35372109)(623.20704756,338.2960833)(623.20704756,339.84379983)
\curveto(623.20704756,341.88474471)(623.20704756,343.71592582)(623.42814992,345.18994156)
\curveto(623.76263811,347.70143763)(625.74689008,347.98490219)(626.42153575,347.98490219)
\lineto(626.42153575,349.00537463)
\curveto(619.61838614,349.00537463)(619.58437039,342.28159511)(619.58437039,339.50931164)
\curveto(619.58437039,336.70868172)(619.61838614,330.18899668)(626.42153575,330.18899668)
\curveto(633.1963389,330.18899668)(633.2587011,336.68600456)(633.2587011,339.50931164)
\closepath
}
}
{
\newrgbcolor{curcolor}{0 0 0}
\pscustom[linewidth=0,linecolor=curcolor]
{
\newpath
\moveto(633.2587011,339.50931164)
\lineto(633.2587011,339.77576834)
\lineto(633.2587011,340.06490219)
\lineto(633.25303181,340.37671322)
\lineto(633.24736252,340.70553211)
\lineto(633.23602394,341.0456896)
\lineto(633.21334677,341.40285495)
\lineto(633.18500031,341.77135889)
\lineto(633.15098457,342.14553211)
\lineto(633.10563024,342.53104393)
\lineto(633.04893732,342.92222503)
\lineto(632.97523654,343.31907542)
\lineto(632.89586646,343.71592582)
\lineto(632.79381921,344.11277621)
\lineto(632.68043339,344.5096266)
\lineto(632.55003969,344.90080771)
\lineto(632.39696882,345.28631952)
\lineto(632.22689008,345.66049274)
\lineto(632.03413417,346.02899668)
\lineto(631.8187011,346.38049274)
\lineto(631.58059087,346.72065023)
\lineto(631.31413417,347.04946912)
\lineto(631.02500031,347.35561086)
\lineto(630.87192945,347.50301243)
\lineto(630.70752,347.63907542)
\lineto(630.54311055,347.77513841)
\lineto(630.36736252,347.90553211)
\lineto(630.18027591,348.03025652)
\lineto(629.98752,348.14931164)
\lineto(629.7890948,348.25702818)
\lineto(629.58500031,348.36474471)
\lineto(629.36956724,348.46112267)
\lineto(629.14279559,348.55183133)
\lineto(628.91035465,348.6368707)
\lineto(628.67224441,348.71057148)
\lineto(628.42279559,348.77860298)
\lineto(628.16200819,348.83529589)
\lineto(627.8955515,348.88631952)
\lineto(627.62342551,348.92600456)
\lineto(627.33429165,348.9600203)
\lineto(627.0394885,348.98836676)
\lineto(626.73901606,348.99970534)
\lineto(626.42153575,349.00537463)
\lineto(626.42153575,347.98490219)
\curveto(627.47035465,347.98490219)(628.62689008,347.44631952)(629.15980346,346.20474471)
\curveto(629.64736252,345.16159511)(629.64736252,342.16820928)(629.64736252,339.84379983)
\curveto(629.64736252,338.26773684)(629.64736252,335.69954786)(629.44326803,334.23120141)
\curveto(629.05208693,331.60065023)(627.27192945,331.20946912)(626.42153575,331.20946912)
\lineto(626.42153575,331.20946912)
\curveto(625.40106331,331.20946912)(623.82500031,331.82742188)(623.45649638,333.89104393)
\curveto(623.20704756,335.35372109)(623.20704756,338.2960833)(623.20704756,339.84379983)
\curveto(623.20704756,341.88474471)(623.20704756,343.71592582)(623.42814992,345.18994156)
\curveto(623.76263811,347.70143763)(625.74689008,347.98490219)(626.42153575,347.98490219)
\lineto(626.42153575,349.00537463)
\curveto(619.61838614,349.00537463)(619.58437039,342.28159511)(619.58437039,339.50931164)
\curveto(619.58437039,336.70868172)(619.61838614,330.18899668)(626.42153575,330.18899668)
\curveto(633.1963389,330.18899668)(633.2587011,336.68600456)(633.2587011,339.50931164)
\closepath
}
}
{
\newrgbcolor{curcolor}{0 0 0}
\pscustom[linestyle=none,fillstyle=solid,fillcolor=curcolor]
{
\newpath
\moveto(640.30563024,330.52915416)
\lineto(640.30563024,330.6992329)
\lineto(640.29996094,330.86931164)
\lineto(640.28862236,331.03372109)
\lineto(640.27728378,331.19246125)
\lineto(640.25460661,331.34553211)
\lineto(640.23759874,331.49293369)
\lineto(640.20925228,331.64033526)
\lineto(640.18090583,331.77639826)
\lineto(640.14689008,331.91246125)
\lineto(640.11287433,332.04285495)
\lineto(640.07318929,332.16757936)
\lineto(640.02783496,332.28663448)
\lineto(639.98248063,332.4000203)
\lineto(639.93145701,332.50773684)
\lineto(639.87476409,332.61545337)
\lineto(639.81807118,332.71183133)
\lineto(639.76137827,332.80253999)
\lineto(639.69334677,332.89324865)
\lineto(639.63098457,332.97261873)
\lineto(639.55728378,333.04631952)
\lineto(639.48358299,333.1200203)
\lineto(639.4098822,333.18238251)
\lineto(639.33051213,333.23907542)
\lineto(639.25114205,333.29576834)
\lineto(639.16610268,333.34112267)
\lineto(639.08106331,333.38080771)
\lineto(638.99035465,333.41482345)
\lineto(638.89397669,333.44316991)
\lineto(638.80326803,333.46584708)
\lineto(638.70122079,333.48285495)
\lineto(638.60484283,333.48852424)
\lineto(638.50279559,333.49419353)
\curveto(637.57303181,333.49419353)(637.00610268,332.78553211)(637.00610268,331.99750062)
\curveto(637.00610268,331.23781558)(637.57303181,330.50080771)(638.50279559,330.50080771)
\curveto(638.83728378,330.50080771)(639.21145701,330.61419353)(639.49492157,330.86364235)
\curveto(639.57429165,330.92600456)(639.6083074,330.95435101)(639.63665386,330.95435101)
\curveto(639.65933102,330.95435101)(639.68767748,330.92600456)(639.68767748,330.52915416)
\curveto(639.68767748,328.43718566)(638.70689008,326.74206755)(637.76578772,325.81230377)
\curveto(637.45397669,325.50616204)(637.45397669,325.44379983)(637.45397669,325.36442975)
\curveto(637.45397669,325.16033526)(637.59570898,325.05261873)(637.73744126,325.05261873)
\curveto(638.04925228,325.05261873)(640.30563024,327.22395731)(640.30563024,330.52915416)
\closepath
}
}
{
\newrgbcolor{curcolor}{0 0 0}
\pscustom[linewidth=0,linecolor=curcolor]
{
\newpath
\moveto(640.30563024,330.52915416)
\lineto(640.30563024,330.6992329)
\lineto(640.29996094,330.86931164)
\lineto(640.28862236,331.03372109)
\lineto(640.27728378,331.19246125)
\lineto(640.25460661,331.34553211)
\lineto(640.23759874,331.49293369)
\lineto(640.20925228,331.64033526)
\lineto(640.18090583,331.77639826)
\lineto(640.14689008,331.91246125)
\lineto(640.11287433,332.04285495)
\lineto(640.07318929,332.16757936)
\lineto(640.02783496,332.28663448)
\lineto(639.98248063,332.4000203)
\lineto(639.93145701,332.50773684)
\lineto(639.87476409,332.61545337)
\lineto(639.81807118,332.71183133)
\lineto(639.76137827,332.80253999)
\lineto(639.69334677,332.89324865)
\lineto(639.63098457,332.97261873)
\lineto(639.55728378,333.04631952)
\lineto(639.48358299,333.1200203)
\lineto(639.4098822,333.18238251)
\lineto(639.33051213,333.23907542)
\lineto(639.25114205,333.29576834)
\lineto(639.16610268,333.34112267)
\lineto(639.08106331,333.38080771)
\lineto(638.99035465,333.41482345)
\lineto(638.89397669,333.44316991)
\lineto(638.80326803,333.46584708)
\lineto(638.70122079,333.48285495)
\lineto(638.60484283,333.48852424)
\lineto(638.50279559,333.49419353)
\curveto(637.57303181,333.49419353)(637.00610268,332.78553211)(637.00610268,331.99750062)
\curveto(637.00610268,331.23781558)(637.57303181,330.50080771)(638.50279559,330.50080771)
\curveto(638.83728378,330.50080771)(639.21145701,330.61419353)(639.49492157,330.86364235)
\curveto(639.57429165,330.92600456)(639.6083074,330.95435101)(639.63665386,330.95435101)
\curveto(639.65933102,330.95435101)(639.68767748,330.92600456)(639.68767748,330.52915416)
\curveto(639.68767748,328.43718566)(638.70689008,326.74206755)(637.76578772,325.81230377)
\curveto(637.45397669,325.50616204)(637.45397669,325.44379983)(637.45397669,325.36442975)
\curveto(637.45397669,325.16033526)(637.59570898,325.05261873)(637.73744126,325.05261873)
\curveto(638.04925228,325.05261873)(640.30563024,327.22395731)(640.30563024,330.52915416)
\closepath
}
}
{
\newrgbcolor{curcolor}{0 0 0}
\pscustom[linestyle=none,fillstyle=solid,fillcolor=curcolor]
{
\newpath
\moveto(647.23917354,353.63151637)
\lineto(709.48799244,353.63151637)
\lineto(709.48799244,354.7880518)
\lineto(647.23917354,354.7880518)
}
}
{
\newrgbcolor{curcolor}{0 0 0}
\pscustom[linewidth=0,linecolor=curcolor]
{
\newpath
\moveto(647.23917354,353.63151637)
\lineto(709.48799244,353.63151637)
\lineto(709.48799244,354.7880518)
\lineto(647.23917354,354.7880518)
}
}
{
\newrgbcolor{curcolor}{0 0 0}
\pscustom[linestyle=none,fillstyle=solid,fillcolor=curcolor]
{
\newpath
\moveto(658.90090583,331.71403605)
\lineto(658.90090583,330.33072897)
\lineto(664.06563024,330.50080771)
\lineto(664.06563024,331.82742188)
\curveto(662.3194885,331.82742188)(662.11539402,331.82742188)(662.11539402,332.9272644)
\lineto(662.11539402,350.10521715)
\lineto(657.12074835,349.87277621)
\lineto(657.12074835,348.54616204)
\curveto(658.87255937,348.54616204)(659.07098457,348.54616204)(659.07098457,347.44631952)
\lineto(659.07098457,341.93576834)
\curveto(657.65933102,343.04128015)(656.19098457,343.21135889)(655.2498822,343.21135889)
\lineto(655.53334677,342.19088645)
\curveto(657.1490948,342.19088645)(658.28295307,341.28946912)(658.90090583,340.47309117)
\lineto(658.90090583,340.47309117)
\lineto(658.90090583,333.40348487)
\curveto(658.5267326,332.89891794)(657.37586646,331.35120141)(655.22720504,331.35120141)
\curveto(651.78027591,331.35120141)(651.78027591,334.7584455)(651.78027591,336.74269747)
\curveto(651.78027591,338.10332739)(651.78027591,339.62269747)(652.51161449,340.72253999)
\curveto(653.33933102,341.90742188)(654.6659452,342.19088645)(655.53334677,342.19088645)
\lineto(655.2498822,343.21135889)
\curveto(651.13397669,343.21135889)(648.16326803,340.72253999)(648.16326803,336.74269747)
\curveto(648.16326803,333.01797306)(650.77114205,330.33072897)(654.94374047,330.33072897)
\curveto(656.66720504,330.33072897)(658.02216567,331.00537463)(658.90090583,331.71403605)
\closepath
}
}
{
\newrgbcolor{curcolor}{0 0 0}
\pscustom[linewidth=0,linecolor=curcolor]
{
\newpath
\moveto(658.90090583,331.71403605)
\lineto(658.90090583,330.33072897)
\lineto(664.06563024,330.50080771)
\lineto(664.06563024,331.82742188)
\curveto(662.3194885,331.82742188)(662.11539402,331.82742188)(662.11539402,332.9272644)
\lineto(662.11539402,350.10521715)
\lineto(657.12074835,349.87277621)
\lineto(657.12074835,348.54616204)
\curveto(658.87255937,348.54616204)(659.07098457,348.54616204)(659.07098457,347.44631952)
\lineto(659.07098457,341.93576834)
\curveto(657.65933102,343.04128015)(656.19098457,343.21135889)(655.2498822,343.21135889)
\lineto(655.53334677,342.19088645)
\curveto(657.1490948,342.19088645)(658.28295307,341.28946912)(658.90090583,340.47309117)
\lineto(658.90090583,340.47309117)
\lineto(658.90090583,333.40348487)
\curveto(658.5267326,332.89891794)(657.37586646,331.35120141)(655.22720504,331.35120141)
\curveto(651.78027591,331.35120141)(651.78027591,334.7584455)(651.78027591,336.74269747)
\curveto(651.78027591,338.10332739)(651.78027591,339.62269747)(652.51161449,340.72253999)
\curveto(653.33933102,341.90742188)(654.6659452,342.19088645)(655.53334677,342.19088645)
\lineto(655.2498822,343.21135889)
\curveto(651.13397669,343.21135889)(648.16326803,340.72253999)(648.16326803,336.74269747)
\curveto(648.16326803,333.01797306)(650.77114205,330.33072897)(654.94374047,330.33072897)
\curveto(656.66720504,330.33072897)(658.02216567,331.00537463)(658.90090583,331.71403605)
\closepath
}
}
{
\newrgbcolor{curcolor}{0 0 0}
\pscustom[linestyle=none,fillstyle=solid,fillcolor=curcolor]
{
\newpath
\moveto(675.68767748,332.67781558)
\lineto(675.68767748,332.50773684)
\lineto(675.69334677,332.33198881)
\lineto(675.69901606,332.24128015)
\lineto(675.70468535,332.15057148)
\lineto(675.71602394,332.05986282)
\lineto(675.73303181,331.96915416)
\lineto(675.75570898,331.8784455)
\lineto(675.77838614,331.78773684)
\lineto(675.8067326,331.69702818)
\lineto(675.84074835,331.60631952)
\lineto(675.88610268,331.51561086)
\lineto(675.93145701,331.43057148)
\lineto(675.98814992,331.34553211)
\lineto(676.05618142,331.26049274)
\lineto(676.1298822,331.18112267)
\lineto(676.20925228,331.10175259)
\lineto(676.29996094,331.0280518)
\lineto(676.35098457,330.99403605)
\lineto(676.40200819,330.9600203)
\lineto(676.4587011,330.92600456)
\lineto(676.51539402,330.89198881)
\lineto(676.57775622,330.85797306)
\lineto(676.64011843,330.8296266)
\lineto(676.70814992,330.80128015)
\lineto(676.77618142,330.77293369)
\lineto(676.8498822,330.74458723)
\lineto(676.92358299,330.71624078)
\lineto(677.00295307,330.69356361)
\lineto(677.08799244,330.67088645)
\lineto(677.17303181,330.64820928)
\lineto(677.25807118,330.62553211)
\lineto(677.35444913,330.60852424)
\lineto(677.45082709,330.59151637)
\lineto(677.54720504,330.57450849)
\lineto(677.64925228,330.55750062)
\lineto(677.75696882,330.54616204)
\lineto(677.87035465,330.53482345)
\lineto(677.98374047,330.52348487)
\lineto(678.10279559,330.51781558)
\lineto(678.22752,330.506477)
\lineto(678.35224441,330.506477)
\lineto(678.48263811,330.50080771)
\lineto(678.6187011,330.50080771)
\lineto(679.97366173,330.50080771)
\curveto(680.54059087,330.50080771)(680.88074835,330.50080771)(680.88074835,331.18112267)
\curveto(680.88074835,331.82742188)(680.5179137,331.82742188)(680.14374047,331.82742188)
\curveto(678.4259452,331.85576834)(678.4259452,332.22994156)(678.4259452,332.87624078)
\lineto(678.4259452,338.94238251)
\curveto(678.4259452,341.45954786)(676.41334677,343.29072897)(672.2067326,343.29072897)
\curveto(670.60232315,343.29072897)(667.15539402,343.18301243)(667.15539402,340.69419353)
\curveto(667.15539402,339.45828802)(668.14185071,338.91403605)(668.90153575,338.91403605)
\curveto(669.78027591,338.91403605)(670.68169323,339.50931164)(670.68169323,340.69419353)
\curveto(670.68169323,341.54458723)(670.14877984,342.026477)(670.06374047,342.07750062)
\curveto(670.85177197,342.24757936)(671.75318929,342.28159511)(672.09901606,342.28159511)
\curveto(674.21933102,342.28159511)(675.20011843,341.09671322)(675.20011843,338.94238251)
\lineto(675.20011843,337.98427227)
\lineto(675.20011843,337.11120141)
\lineto(675.20011843,334.42395731)
\curveto(675.20011843,331.71403605)(672.60358299,331.35120141)(671.89492157,331.35120141)
\curveto(670.46059087,331.35120141)(669.3267326,332.33198881)(669.3267326,333.58490219)
\curveto(669.3267326,336.62931164)(673.79413417,337.02616204)(675.20011843,337.11120141)
\lineto(675.20011843,337.11120141)
\lineto(675.20011843,337.98427227)
\curveto(673.19885858,337.8992329)(666.02153575,337.64978408)(666.02153575,333.54521715)
\curveto(666.02153575,330.83529589)(669.5194885,330.33072897)(671.46972472,330.33072897)
\curveto(673.73744126,330.33072897)(675.05838614,331.49293369)(675.68767748,332.67781558)
\closepath
}
}
{
\newrgbcolor{curcolor}{0 0 0}
\pscustom[linewidth=0,linecolor=curcolor]
{
\newpath
\moveto(675.68767748,332.67781558)
\lineto(675.68767748,332.50773684)
\lineto(675.69334677,332.33198881)
\lineto(675.69901606,332.24128015)
\lineto(675.70468535,332.15057148)
\lineto(675.71602394,332.05986282)
\lineto(675.73303181,331.96915416)
\lineto(675.75570898,331.8784455)
\lineto(675.77838614,331.78773684)
\lineto(675.8067326,331.69702818)
\lineto(675.84074835,331.60631952)
\lineto(675.88610268,331.51561086)
\lineto(675.93145701,331.43057148)
\lineto(675.98814992,331.34553211)
\lineto(676.05618142,331.26049274)
\lineto(676.1298822,331.18112267)
\lineto(676.20925228,331.10175259)
\lineto(676.29996094,331.0280518)
\lineto(676.35098457,330.99403605)
\lineto(676.40200819,330.9600203)
\lineto(676.4587011,330.92600456)
\lineto(676.51539402,330.89198881)
\lineto(676.57775622,330.85797306)
\lineto(676.64011843,330.8296266)
\lineto(676.70814992,330.80128015)
\lineto(676.77618142,330.77293369)
\lineto(676.8498822,330.74458723)
\lineto(676.92358299,330.71624078)
\lineto(677.00295307,330.69356361)
\lineto(677.08799244,330.67088645)
\lineto(677.17303181,330.64820928)
\lineto(677.25807118,330.62553211)
\lineto(677.35444913,330.60852424)
\lineto(677.45082709,330.59151637)
\lineto(677.54720504,330.57450849)
\lineto(677.64925228,330.55750062)
\lineto(677.75696882,330.54616204)
\lineto(677.87035465,330.53482345)
\lineto(677.98374047,330.52348487)
\lineto(678.10279559,330.51781558)
\lineto(678.22752,330.506477)
\lineto(678.35224441,330.506477)
\lineto(678.48263811,330.50080771)
\lineto(678.6187011,330.50080771)
\lineto(679.97366173,330.50080771)
\curveto(680.54059087,330.50080771)(680.88074835,330.50080771)(680.88074835,331.18112267)
\curveto(680.88074835,331.82742188)(680.5179137,331.82742188)(680.14374047,331.82742188)
\curveto(678.4259452,331.85576834)(678.4259452,332.22994156)(678.4259452,332.87624078)
\lineto(678.4259452,338.94238251)
\curveto(678.4259452,341.45954786)(676.41334677,343.29072897)(672.2067326,343.29072897)
\curveto(670.60232315,343.29072897)(667.15539402,343.18301243)(667.15539402,340.69419353)
\curveto(667.15539402,339.45828802)(668.14185071,338.91403605)(668.90153575,338.91403605)
\curveto(669.78027591,338.91403605)(670.68169323,339.50931164)(670.68169323,340.69419353)
\curveto(670.68169323,341.54458723)(670.14877984,342.026477)(670.06374047,342.07750062)
\curveto(670.85177197,342.24757936)(671.75318929,342.28159511)(672.09901606,342.28159511)
\curveto(674.21933102,342.28159511)(675.20011843,341.09671322)(675.20011843,338.94238251)
\lineto(675.20011843,337.98427227)
\lineto(675.20011843,337.11120141)
\lineto(675.20011843,334.42395731)
\curveto(675.20011843,331.71403605)(672.60358299,331.35120141)(671.89492157,331.35120141)
\curveto(670.46059087,331.35120141)(669.3267326,332.33198881)(669.3267326,333.58490219)
\curveto(669.3267326,336.62931164)(673.79413417,337.02616204)(675.20011843,337.11120141)
\lineto(675.20011843,337.11120141)
\lineto(675.20011843,337.98427227)
\curveto(673.19885858,337.8992329)(666.02153575,337.64978408)(666.02153575,333.54521715)
\curveto(666.02153575,330.83529589)(669.5194885,330.33072897)(671.46972472,330.33072897)
\curveto(673.73744126,330.33072897)(675.05838614,331.49293369)(675.68767748,332.67781558)
\closepath
}
}
{
\newrgbcolor{curcolor}{0 0 0}
\pscustom[linestyle=none,fillstyle=solid,fillcolor=curcolor]
{
\newpath
\moveto(683.80610268,341.71466597)
\lineto(683.80610268,333.970414)
\curveto(683.80610268,330.95435101)(686.27224441,330.33072897)(688.27350425,330.33072897)
\curveto(690.42783496,330.33072897)(691.69208693,331.946477)(691.69208693,333.99876046)
\lineto(691.69208693,335.49545337)
\lineto(690.36547276,335.49545337)
\lineto(690.36547276,334.06112267)
\curveto(690.36547276,332.1392329)(689.45838614,331.45891794)(688.67035465,331.45891794)
\curveto(687.03192945,331.45891794)(687.03192945,333.26175259)(687.03192945,333.91939038)
\lineto(687.03192945,341.71466597)
\lineto(691.15350425,341.71466597)
\lineto(691.15350425,343.04128015)
\lineto(687.03192945,343.04128015)
\lineto(687.03192945,348.4384455)
\lineto(685.70531528,348.4384455)
\curveto(685.67696882,345.61513841)(684.29366173,342.81450849)(681.4987011,342.73513841)
\lineto(681.4987011,341.71466597)
\closepath
}
}
{
\newrgbcolor{curcolor}{0 0 0}
\pscustom[linewidth=0,linecolor=curcolor]
{
\newpath
\moveto(683.80610268,341.71466597)
\lineto(683.80610268,333.970414)
\curveto(683.80610268,330.95435101)(686.27224441,330.33072897)(688.27350425,330.33072897)
\curveto(690.42783496,330.33072897)(691.69208693,331.946477)(691.69208693,333.99876046)
\lineto(691.69208693,335.49545337)
\lineto(690.36547276,335.49545337)
\lineto(690.36547276,334.06112267)
\curveto(690.36547276,332.1392329)(689.45838614,331.45891794)(688.67035465,331.45891794)
\curveto(687.03192945,331.45891794)(687.03192945,333.26175259)(687.03192945,333.91939038)
\lineto(687.03192945,341.71466597)
\lineto(691.15350425,341.71466597)
\lineto(691.15350425,343.04128015)
\lineto(687.03192945,343.04128015)
\lineto(687.03192945,348.4384455)
\lineto(685.70531528,348.4384455)
\curveto(685.67696882,345.61513841)(684.29366173,342.81450849)(681.4987011,342.73513841)
\lineto(681.4987011,341.71466597)
\closepath
}
}
{
\newrgbcolor{curcolor}{0 0 0}
\pscustom[linestyle=none,fillstyle=solid,fillcolor=curcolor]
{
\newpath
\moveto(704.12484283,332.67781558)
\lineto(704.12484283,332.50773684)
\lineto(704.13051213,332.33198881)
\lineto(704.13618142,332.24128015)
\lineto(704.14185071,332.15057148)
\lineto(704.15318929,332.05986282)
\lineto(704.17019717,331.96915416)
\lineto(704.18720504,331.8784455)
\lineto(704.2098822,331.78773684)
\lineto(704.24389795,331.69702818)
\lineto(704.2779137,331.60631952)
\lineto(704.31759874,331.51561086)
\lineto(704.36862236,331.43057148)
\lineto(704.42531528,331.34553211)
\lineto(704.48767748,331.26049274)
\lineto(704.56137827,331.18112267)
\lineto(704.64641764,331.10175259)
\lineto(704.7371263,331.0280518)
\lineto(704.78814992,330.99403605)
\lineto(704.83917354,330.9600203)
\lineto(704.89586646,330.92600456)
\lineto(704.95255937,330.89198881)
\lineto(705.00925228,330.85797306)
\lineto(705.07728378,330.8296266)
\lineto(705.13964598,330.80128015)
\lineto(705.21334677,330.77293369)
\lineto(705.28137827,330.74458723)
\lineto(705.36074835,330.71624078)
\lineto(705.44011843,330.69356361)
\lineto(705.5194885,330.67088645)
\lineto(705.60452787,330.64820928)
\lineto(705.69523654,330.62553211)
\lineto(705.7859452,330.60852424)
\lineto(705.88232315,330.59151637)
\lineto(705.98437039,330.57450849)
\lineto(706.08641764,330.55750062)
\lineto(706.19413417,330.54616204)
\lineto(706.30752,330.53482345)
\lineto(706.42090583,330.52348487)
\lineto(706.53996094,330.51781558)
\lineto(706.65901606,330.506477)
\lineto(706.78940976,330.506477)
\lineto(706.91980346,330.50080771)
\lineto(707.05586646,330.50080771)
\lineto(708.41082709,330.50080771)
\curveto(708.97775622,330.50080771)(709.31224441,330.50080771)(709.31224441,331.18112267)
\curveto(709.31224441,331.82742188)(708.94940976,331.82742188)(708.58090583,331.82742188)
\curveto(706.85744126,331.85576834)(706.85744126,332.22994156)(706.85744126,332.87624078)
\lineto(706.85744126,338.94238251)
\curveto(706.85744126,341.45954786)(704.85051213,343.29072897)(700.64389795,343.29072897)
\curveto(699.0394885,343.29072897)(695.59255937,343.18301243)(695.59255937,340.69419353)
\curveto(695.59255937,339.45828802)(696.57334677,338.91403605)(697.3387011,338.91403605)
\curveto(698.21177197,338.91403605)(699.11885858,339.50931164)(699.11885858,340.69419353)
\curveto(699.11885858,341.54458723)(698.5859452,342.026477)(698.49523654,342.07750062)
\curveto(699.28326803,342.24757936)(700.19035465,342.28159511)(700.53618142,342.28159511)
\curveto(702.65082709,342.28159511)(703.63728378,341.09671322)(703.63728378,338.94238251)
\lineto(703.63728378,337.98427227)
\lineto(703.63728378,337.11120141)
\lineto(703.63728378,334.42395731)
\curveto(703.63728378,331.71403605)(701.04074835,331.35120141)(700.33208693,331.35120141)
\curveto(698.89775622,331.35120141)(697.76389795,332.33198881)(697.76389795,333.58490219)
\curveto(697.76389795,336.62931164)(702.22563024,337.02616204)(703.63728378,337.11120141)
\lineto(703.63728378,337.11120141)
\lineto(703.63728378,337.98427227)
\curveto(701.63602394,337.8992329)(694.4587011,337.64978408)(694.4587011,333.54521715)
\curveto(694.4587011,330.83529589)(697.95665386,330.33072897)(699.90689008,330.33072897)
\curveto(702.17460661,330.33072897)(703.4955515,331.49293369)(704.12484283,332.67781558)
\closepath
}
}
{
\newrgbcolor{curcolor}{0 0 0}
\pscustom[linewidth=0,linecolor=curcolor]
{
\newpath
\moveto(704.12484283,332.67781558)
\lineto(704.12484283,332.50773684)
\lineto(704.13051213,332.33198881)
\lineto(704.13618142,332.24128015)
\lineto(704.14185071,332.15057148)
\lineto(704.15318929,332.05986282)
\lineto(704.17019717,331.96915416)
\lineto(704.18720504,331.8784455)
\lineto(704.2098822,331.78773684)
\lineto(704.24389795,331.69702818)
\lineto(704.2779137,331.60631952)
\lineto(704.31759874,331.51561086)
\lineto(704.36862236,331.43057148)
\lineto(704.42531528,331.34553211)
\lineto(704.48767748,331.26049274)
\lineto(704.56137827,331.18112267)
\lineto(704.64641764,331.10175259)
\lineto(704.7371263,331.0280518)
\lineto(704.78814992,330.99403605)
\lineto(704.83917354,330.9600203)
\lineto(704.89586646,330.92600456)
\lineto(704.95255937,330.89198881)
\lineto(705.00925228,330.85797306)
\lineto(705.07728378,330.8296266)
\lineto(705.13964598,330.80128015)
\lineto(705.21334677,330.77293369)
\lineto(705.28137827,330.74458723)
\lineto(705.36074835,330.71624078)
\lineto(705.44011843,330.69356361)
\lineto(705.5194885,330.67088645)
\lineto(705.60452787,330.64820928)
\lineto(705.69523654,330.62553211)
\lineto(705.7859452,330.60852424)
\lineto(705.88232315,330.59151637)
\lineto(705.98437039,330.57450849)
\lineto(706.08641764,330.55750062)
\lineto(706.19413417,330.54616204)
\lineto(706.30752,330.53482345)
\lineto(706.42090583,330.52348487)
\lineto(706.53996094,330.51781558)
\lineto(706.65901606,330.506477)
\lineto(706.78940976,330.506477)
\lineto(706.91980346,330.50080771)
\lineto(707.05586646,330.50080771)
\lineto(708.41082709,330.50080771)
\curveto(708.97775622,330.50080771)(709.31224441,330.50080771)(709.31224441,331.18112267)
\curveto(709.31224441,331.82742188)(708.94940976,331.82742188)(708.58090583,331.82742188)
\curveto(706.85744126,331.85576834)(706.85744126,332.22994156)(706.85744126,332.87624078)
\lineto(706.85744126,338.94238251)
\curveto(706.85744126,341.45954786)(704.85051213,343.29072897)(700.64389795,343.29072897)
\curveto(699.0394885,343.29072897)(695.59255937,343.18301243)(695.59255937,340.69419353)
\curveto(695.59255937,339.45828802)(696.57334677,338.91403605)(697.3387011,338.91403605)
\curveto(698.21177197,338.91403605)(699.11885858,339.50931164)(699.11885858,340.69419353)
\curveto(699.11885858,341.54458723)(698.5859452,342.026477)(698.49523654,342.07750062)
\curveto(699.28326803,342.24757936)(700.19035465,342.28159511)(700.53618142,342.28159511)
\curveto(702.65082709,342.28159511)(703.63728378,341.09671322)(703.63728378,338.94238251)
\lineto(703.63728378,337.98427227)
\lineto(703.63728378,337.11120141)
\lineto(703.63728378,334.42395731)
\curveto(703.63728378,331.71403605)(701.04074835,331.35120141)(700.33208693,331.35120141)
\curveto(698.89775622,331.35120141)(697.76389795,332.33198881)(697.76389795,333.58490219)
\curveto(697.76389795,336.62931164)(702.22563024,337.02616204)(703.63728378,337.11120141)
\lineto(703.63728378,337.11120141)
\lineto(703.63728378,337.98427227)
\curveto(701.63602394,337.8992329)(694.4587011,337.64978408)(694.4587011,333.54521715)
\curveto(694.4587011,330.83529589)(697.95665386,330.33072897)(699.90689008,330.33072897)
\curveto(702.17460661,330.33072897)(703.4955515,331.49293369)(704.12484283,332.67781558)
\closepath
}
}
{
\newrgbcolor{curcolor}{0 0 0}
\pscustom[linewidth=1.51181105,linecolor=curcolor]
{
\newpath
\moveto(512.5778948,105.73417841)
\lineto(487.57889008,105.745517)
\curveto(487.57889008,105.745517)(483.15654047,105.21041148)(479.10122079,109.87449967)
\curveto(475.04571213,114.53858786)(475.49887748,119.54052802)(475.49887748,119.54052802)
\lineto(475.32879874,171.29095322)
}
}
{
\newrgbcolor{curcolor}{0 0 0}
\pscustom[linestyle=none,fillstyle=solid,fillcolor=curcolor]
{
\newpath
\moveto(504.67104077,109.39671229)
\lineto(514.58010397,105.7477957)
\lineto(504.66773429,102.10787015)
\curveto(506.25202829,104.25881652)(506.2442404,107.20267113)(504.67104077,109.39671229)
\closepath
}
}
{
\newrgbcolor{curcolor}{0 0 0}
\pscustom[linewidth=0.56692914,linecolor=curcolor]
{
\newpath
\moveto(504.67104077,109.39671229)
\lineto(514.58010397,105.7477957)
\lineto(504.66773429,102.10787015)
\curveto(506.25202829,104.25881652)(506.2442404,107.20267113)(504.67104077,109.39671229)
\closepath
}
}
\end{pspicture}

        }
        %    \includegraphics[width=0.9\textwidth]{figures/figure_1.png}
    \end{center}
    \caption{Graphical presentation of the error-minimizing two step method}\label{fig:two_step_method}
\end{figure}
%
Notation agreements:
\begin{symbollist}
    \item[$u$] Non-regularized/non-smoothed function, to be determined numerically.
    \item[$\widetilde u$] Regularized/smoothed function, denoted by the wavy line.
    \item[$\overline u$] Piecewise linear function, denoted by the bar.
    \item[$\widehat u$] Numerical solution on the nodes.
\end{symbollist}

A tilde ($\widetilde u$) indicates the variables and differential operators of the easy
problem.
Their discretizations are indicated by a bar, $\overline {u}$.
Next, we define the smooth and infinitely differentiable function $\widehat u$ that is a very close approximation of numerical solution $\overline {u}$.
By means of an error analysis we determine the differential problem that $\widehat u$ is a solution of.
Note that the data pertaining to the computational model are also included in the procedure.
Independent variables describing, e.g., the geometry and initial and boundary conditions also need to be discretized and hence need to be sufficiently smooth, to ensure that all higher-order error terms are sufficiently small and can be neglected.
Sufficient smoothness is obtained automatically by using smoothing coefficients that are a function of the discretization errors.
See also \citet{Borsboom2001}.

