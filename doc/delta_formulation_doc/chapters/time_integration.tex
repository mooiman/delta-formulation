%------------------------------------------------------------------------------
\chapter{Time integration scheme}\label{sec:time_integration}
To derive a time integration we start from the PDE:
\begin{align}
    \pdiff{\vec{u}}{t} + \vec{f}(\vec{u}) = 0\label{eq:pde}
\end{align}
%for the two dimensional linear wave  equations this equation reads
%\begin{align}
%    \pdiff{h}{t} + \pdiff{q}{x}  + \pdiff{r}{y}= 0
%    \\
%    \pdiff{q}{t} + gh\pdiff{\zeta}{x} = 0
%    \\
%    \pdiff{r}{t} + gh \pdiff{\zeta}{y} = 0
%\end{align}
For conservation types it can be written as:
\begin{align}
    \pdiff{\vec{u}}{t} + \nabla \dotp \vec{f}(\vec{u}) = \mat{S}
\end{align}
and when the finite volume approach is applied, we get
\begin{align}
    \int_\Omega \pdiff{\vec{u}}{t}\, d\Omega + \int_\Omega \nabla \dotp \vec{f}(\vec{u})\, d\Omega&= \int_\Omega \mat{S}\, d\Omega,
\end{align}
and after Green's theorem
\begin{align}
    \int_\Omega \pdiff{\vec{u}}{t}\, d\Omega + \int_\Gamma \vec{f}(\vec{u}) \dotp \vec{n}\, d\Gamma&= \int_\Omega \mat{S}\, d\Omega,
\end{align}
%--------------------------------------------------------------------------------
\section{Fully implicit time integration by adding an iteration process}\label{sec:fully_implicit}
The system of Equations \eqref{eq:pde} can be written as, including the $\theta$ method:
\begin{align}
    &\Dt_{inv} \mat{M} \left( \vec{u}^{n+1} - \vec{u}^{n} \right)  +
    \vec{f} \left( \vec{u}^{n+\theta} \right) = 0 \label{eq:discretized}
\end{align}
with $\Dt_{\it inv} = 1/\Dt$, $\mat{M}$ a mass-matrix and $0 \leq \theta \leq 1$.
The mass-matrix used in this document for three adjacent grid nodes reads:
\begin{align}
    \mat{M} = \begin{pmatrix} \frac{1}{8} \quad \frac{6}{8} \quad \frac{1}{8} \end{pmatrix}
    \label{eq:definition_mass_matrix}
\end{align}
representing a piecewise linear approximation of the function between the grid nodes.

To reach a fully implicit time integration an iteration process $p$ is added \citep[eqs.\ 15/16]{Borsboom2019a}:
\begin{align}
    &\Dt_{inv} \mat{M} \left( \vec{u}^{n+1,p+1} - \vec{u}^{n} \right)  +
    \vec{f} \left( \vec{u}^{n+\theta, p+1} \right) = 0
    \label{eq:1D_iteration}
\end{align}
iterating from $p \rightarrow p+1$ until convergence.

The \textbf{first} term is split to get a so called "Delta" formulation, taking into account the previous iteration:
\begin{align}
    &\Dt_{inv} \mat{M} \left( \vec{u}^{n+1,p+1} - \vec{u}^{n+1,p} + \vec{u}^{n+1,p} - \vec{u}^{n} \right) + \dots = 0
    \\
    & \Dt_{inv} \mat{M} \Delta \vec{u}^{n+1,p+1} + \dots  =
    - \Dt_{inv} \mat{M} \left( \vec{u}^{n+1,p} - \vec{u}^{n} \right)
\end{align}
with $\Delta \vec{u}^{n+1,p+1} = \vec{u}^{n+1,p+1} - \vec{u}^{n+1,p}$.
The right hand side is fully explicit (i.e.\ known at the previous iteration level $p$).
And if the iteration process is converged, the term at the left hand side is zero (i.e. $\Delta \vec{u}^{n+1,p+1} = 0$) so the right handside represent the time derivative.

The \textbf{second} term of \autoref{eq:discretized}
\begin{align}
    &\ldots +
    \vec{f} \left( \vec{u}^{n+\theta, p+1} \right) = 0
    \label{eq:second_term}
\end{align}
%
will be linearized around the iteration step $p$ (Newton linearization) and yields
\begin{align}
    \vec{f}( \vec{u}^{n+\theta, p+1} ) & = \vec{f}( \vec{u}^{n+\theta, p} ) + \pdiff{\vec{f}(\vec{u}^{n+\theta,p})}{\vec{u}^{n+1,p}} (\vec{u}^{n+\theta,p+1} - \vec{u}^{n+\theta,p})
    \\
    & = \vec{f}( \vec{u}^{n+\theta, p} ) + \pdiff{\vec{f}(\vec{u}^{n+\theta,p})}{\vec{u}^{n+1,p}} \Delta\vec{u}^{n+\theta,p+1} \label{eq:NewtonLinearization}
\end{align}
with
\begin{align}
    \Delta \vec{u}^{n+\theta, p+1} & = \vec{u}^{n+\theta, p+1} - \vec{u}^{n+\theta, p}
    \\
    & = \theta \vec{u}^{n+1, p+1} + (1-\theta) \vec{u}^{n} - \theta \vec{u}^{n+1, p} - (1-\theta) \vec{u}^{n}
    \\
    & = \theta \vec{u}^{n+1, p+1} - \theta \vec{u}^{n+1, p}
    \\
    &= \theta \Delta \vec{u}^{n+1, p+1} \label{eq:delta_n_theta}
\end{align}
After subtitution of \autoref{eq:delta_n_theta} into \autoref{eq:NewtonLinearization} we get:
\begin{align}
    \vec{f}( \vec{u}^{n+\theta, p+1} ) & = \vec{f}( \vec{u}^{n+\theta, p} ) + \theta \pdiff{\vec{f}(\vec{u}^{n+\theta,p})}{\vec{u}^{n+1,p}} \Delta \vec{u}^{n+1, p+1}
\end{align}
The Jacobian
\begin{align}
    \mat{J}^{n+1,p} & = \pdiff{\vec{f}(\vec{u}^{n+\theta,p})}{\vec{u}^{n+1,p}}
     = \pdiff{\vec{f}(\theta \vec{u^{n+1,p}} + (1 - \theta) \vec{u}^{n})}{\vec{u}^{n+1,p}}
\end{align}
is the approximate linearization of $\vec{f}$ as a
function of $\theta \vec{u}^{n+1} + (1-\theta)\vec{u}^{n}$ with respect to $\vec{u}^{n+1, p}$.
The Jacobians needed for the shallow water equations are described in \autoref{sec:jacobians}.


The total time integration method read:
\begin{empheq}[box=\fbox]{align}
    &\left(\Dt_{inv} \mat{M} + \theta\mat{J}^{n+1,p}\right)  \Delta \mat{u}^{n+1,p+1} =
    \nonumber \\
& \qquad = - \left( \Dt_{inv} \mat{M} \left( \vec{u}^{n+1,p} - \vec{u}^{n} \right) + \vec{f} \left( \theta \vec{u}^{n+1,p} + (1-\theta) \vec{u}^{n} \right) \right) \label{eq:Ax=b_b}
\end{empheq}
%
with $\vec{u}^{n+1,p+1}  = \vec{u}^{n+1,p} + \Delta \vec{u}^{n+1,p+1}$ and right hand side is explicit w.r.t.\  the iterator $p$.
In case the Newton iteration process converges, i.e.:
\begin{align}
    \lim_{p\rightarrow \infty}\left( \Delta \vec{u}^{n+1,p+1}\right) = \lim_{p\rightarrow \infty}\left( \vec{u}^{n+1,p+1} - \vec{u}^{n+1,p} \right) = 0.
\end{align}
%
then the left hand side of \autoref{eq:Ax=b_b} is equal to zero and thus it solves the original system of equations:
\begin{align}
    0 = \Dt_{inv} \mat{M} \left( \vec{u}^{n+1,p} - \vec{u}^{n} \right) + \vec{f} \left( \theta \vec{u}^{n+1,p} + (1-\theta) \vec{u}^{n} \right).
\end{align}
%
Because in the previous part we consider only the first derivative (Jacobian) and assumed that the second derivative is nearly zero, which is not always through.
Therefor we will extend the iteration process, see  \autoref{sec:pseudo_time_steppping}.
See also: \citet{Borsboom1998,Pulliam2014}


%--------------------------------------------------------------------------------
\subsection{Pseudo time stepping}\label{sec:psuedo_time_step}
\label{sec:pseudo_time_steppping}
In  \autoref{sec:fully_implicit} we assumed that only the Jacobian is relevant and the second derivative is negligible.
But in some cases it is not the case, so we have to assure that the following inequality is true:
\begin{align}
    \abs{ \half \frac{\partial^2 \vec{f}(\theta \vec{u}^{n+1,p}+ (1 - \theta)\vec{u}^n)}{(\partial \vec{u}^{n+1,p})^2} \Delta \vec{u}^{n+1,p+1}}
    < O\left(
    \abs{ \pdiff{\vec{f}(\theta \vec{u}^{n+1,p}+ (1 - \theta)\vec{u}^n)}{\vec{u}^{n+1,p} } }
    \right)
\end{align}
Therefor the time integration is extended with a (so called) pseudo timestep method, which read:
\begin{empheq}[box=\fbox]{align}
    & \left(\mat{M}_{pseu}\vec{T}^{n+1, p}_{pseu} + \Dt_{inv} \mat{M} + \theta\mat{J}^{n+1,p}\right)  \Delta \mat{u}^{n+1,p+1} =
    \nonumber \\
& \qquad = - \left( \Dt_{inv} \mat{M} \left( \vec{u}^{n+1,p} - \vec{u}^{n} \right) + \vec{f} \left( \theta \vec{u}^{n+1,p} + (1-\theta) \vec{u}^{n} \right) \right)
\end{empheq}
where $\vec{T}^{n+1, p}_{pseu}$ is a vector containing the inverse of the pseudo timestep, which may vary for all grid nodes, and $\mat{M}_{pseu}$ a mass-matrix operating on the pseudo timestep vector.
See for a more detailed description and how to choose the term $\mat{M}_{pseu}\vec{T}^{n+1, p}_{pseu}$
\citet{Borsboom2019a, Buijs2024}.

%--------------------------------------------------------------------------------
\section{Jacobians}\label{sec:jacobians}
As seen in \autoref{sec:fully_implicit} Jacobians need to be computed.
These Jacobians does not contain only derivatives to the major variables but also to place derivatives, which need special attention (\autoref{sec:jacobians_with_operator}).
For example for the two dimensional convection flux ($\vec{q}{\vec{q}^T}$) and presure term $gh\nabla \zeta$, where $\vec{q} = (q, r)^T$ and $\zeta$ the water level.

As example we take the integral form of the two dimensional non-linear wave equation.
This equation reads:
%
\begin{align}
    \int_\Omega \pdiff{\vec{u}}{t}\, d\Omega  + \int_\Omega \nabla \dotp \vec{F}\, d\Omega = 0
    \Leftrightarrow
    \int_\Omega \pdiff{\vec{u}}{t}\, d\Omega  + \oint_\Gamma \vec{F} \dotp \vec{n} \, d\Gamma = 0
\end{align}
with vector $\vec{u} = (h, q, r)^T$ the Jacobian read.
With
\begin{symbollist}
    \item[$h$] The total water depth, [$\si{\metre}$]
    \item[$q$] The water flux in $x$-direction, [$\si{\square\metre\per\second}$]
    \item[$r$] The water flux in $y$-direction, [$\si{\square\metre\per\second}$]
\end{symbollist}

The Jacobian of the function $F(h, q, r)$ as used in the two dimensional shallow water equations reads:
\begin{align}
    \mat{J} =
    \begin{pmatrix}
        J_{11} & J_{12} & J_{13} \\
        J_{21} & J_{22} & J_{23} \\
        J_{31} & J_{32} & J_{33}
    \end{pmatrix}
    =
    \begin{pmatrix}
        \pdiff{F_1}{h} & \pdiff{F_1}{q} &  \pdiff{F_1}{r} \\
        \pdiff{F_2}{h} & \pdiff{F_2}{q} &  \pdiff{F_2}{r} \\
        \pdiff{F_3}{h} & \pdiff{F_3}{q} &  \pdiff{F_3}{r}
    \end{pmatrix}
\end{align}


%--------------------------------------------------------------------------------
\subsection{Non-linear term, product}\label{sec:jacobians_with_non_linear_product_term}
If the Jacobian contains non-linear terms, each variable of $\vec{u}$ is linearized before using it in the non-linear term.
As an example a product of two quantities is taken, say $q$ and $r$ (like the convection term in two dimensional shallow water equations) and  $\Delta q^{n+1, p+1} = \Delta q$ and  $\Delta r^{n+1, p+1} = \Delta r$:
\begin{align}
    \left.(qr)\right|^{n+\theta, p+1} & = \left( q^{n+\theta, p} + \theta  \Delta q \right) \left( r^{n+\theta, p} + \theta  \Delta r\right) \approx
    \\
    & \approx  q^{n+\theta, p} r^{n+\theta, p} + q^{n+\theta, p} \theta \Delta r + r^{n+\theta, p} \theta \Delta q
\end{align}
omitting the quadratic term $O((\Delta q)^2, \Delta q \Delta r, (\Delta r)^2)$.
%--------------------------------------------------------------------------------
\subparagraph*{Jacobian}
When the Jacobian notation is  used for the function $F(q,r) = qr$
\begin{align}
    \mat{J} =
    \begin{pmatrix}
        J_{11} & J_{12}
    \end{pmatrix}
    =
    \begin{pmatrix}
        \pdiff{F}{q} & \pdiff{F}{r}
    \end{pmatrix}
    =
    \begin{pmatrix}
        \pdiff{qr}{q} & \pdiff{qr}{r}
    \end{pmatrix}
    =
    \begin{pmatrix}
        r & q
    \end{pmatrix}
\end{align}
 it leads to following approximation
\begin{align}
    \left.(qr)\right|^{n+\theta, p+1} &  \approx  q^{n+\theta, p} r^{n+\theta, p}
    + J_{11}^{n+\theta, p} \theta \Delta q
    + J_{12}^{n+\theta, p} \theta \Delta r =
    \\
    & = q^{n+\theta, p} r^{n+\theta, p}
    + r^{n+\theta, p} \theta \Delta q + q^{n+\theta, p} \theta \Delta r
\end{align}

%--------------------------------------------------------------------------------
\subsection{Non-linear term, quotient}\label{sec:jacobians_with_non_linear_quotient_term}
If the Jacobian contains non-linear terms, each variable of $\vec{u}$ is linearized before using it in the non-linear term.
In this example a quotient of two quantities is taken, say $q$ and $h$ (representing the velocity in the two dimensional shallow water equations) and  $\Delta q^{n+1, p+1} = \Delta q$ and  $\Delta h^{n+1, p+1} = \Delta h$:
\begin{align}
    \left.\left(\frac{q}{h}\right)\right|^{n+\theta, p+1} & = \frac{ q^{n+\theta, p} + \theta  \Delta q }{ h^{n+\theta, p} + \theta  \Delta h} \approx
    \\
    & \approx \frac{ q^{n+\theta, p} + \theta  \Delta q }{ h^{n+\theta, p}} \left( 1 - \frac{\theta}{h^{n+\theta, p}} \Delta h  + O\left(  (\Delta h)^2 \right) \right)  \approx
    \\
    & \approx \left( \frac{ q^{n+\theta, p}}{ h^{n+\theta, p}} + \frac{1}{ h^{n+\theta, p}}  \theta\Delta q \right)
    \left( 1 - \frac{1}{h^{n+\theta, p}} \theta\Delta h  + O\left(  (\Delta h)^2 \right) \right)  \approx
    \\
    & \approx  \frac{ q^{n+\theta, p}}{ h^{n+\theta, p}}
    - \frac{ q^{n+\theta, p}}{ (h^{n+\theta, p})^2} \theta \Delta h
    + \frac{1}{ h^{n+\theta, p}}  \theta\Delta q
\end{align}
omitting the quadratic term $O((\Delta q)^2, \Delta q \Delta h, (\Delta h)^2)$.
%--------------------------------------------------------------------------------
\subparagraph*{Jacobian}
When the Jacobian notation is  used for the function $F(h,q) = q/h$
\begin{align}
    \mat{J} =
    \begin{pmatrix}
        J_{11} & J_{12}
    \end{pmatrix}
    =
    \begin{pmatrix}
        \pdiff{F}{h} & \pdiff{F}{q}
    \end{pmatrix}
    =
    \begin{pmatrix}
        \pdiff{q/h}{h} & \pdiff{q/h}{q}
    \end{pmatrix}
    =
    \begin{pmatrix}
        -\frac{q}{h^2} & \frac{1}{h}
    \end{pmatrix}
\end{align}
it leads to following approximation
\begin{align}
    \left.\left(\frac{q}{h}\right)\right|^{n+\theta, p+1} &  \approx  \frac{q^{n+\theta, p}}{h^{n+\theta, p}}
    + J_{11}^{n+\theta, p} \theta \Delta h
    + J_{12}^{n+\theta, p} \theta \Delta q =
    \\
    & = \frac{q^{n+\theta, p}}{h^{n+\theta, p}}
    - \frac{q^{n+\theta, p}}{(h^{n+\theta, p})^2} \theta \Delta h
    + \frac{1}{h^{n+\theta, p}} \theta \Delta q
\end{align}
%--------------------------------------------------------------------------------
\subsection{Non-linear term, product and quotient}
For this example we look  to the Jacobian as they appear for the convection term in two dimensions:
\begin{align}
    \left.\left(\frac{qr}{h}\right)\right|^{n+\theta, p+1} & = \frac{ \left(q^{n+\theta, p} + \theta  \Delta q\right)\left( r^{n+\theta, p} + \theta  \Delta r \right) }{ h^{n+\theta, p} + \theta  \Delta h}.
\end{align}

%--------------------------------------------------------------------------------
\subparagraph*{Jacobian}
When the Jacobian notation is  used for the function $F(h,q, r) = qr/h$
\begin{align}
    \mat{J} & =
    \begin{pmatrix}
        J_{11} & J_{12} & J_{13}
    \end{pmatrix}
    =
    \begin{pmatrix}
        \pdiff{F}{h} & \pdiff{F}{q} & \pdiff{F}{r}
    \end{pmatrix}
    =
    \\
    & = \begin{pmatrix}
        \pdiff{qr/h}{h} & \pdiff{qr/h}{q} & \pdiff{qr/h}{r}
    \end{pmatrix}
    =
    \begin{pmatrix}
        -\frac{qr}{h^2} & \frac{r}{h} & \frac{q}{h}
    \end{pmatrix}
\end{align}
it leads to following approximation
\begin{align}
    \left.\left(\frac{qr}{h}\right)\right|^{n+\theta, p+1} &  \approx  \frac{q^{n+\theta, p}r^{n+\theta, p}}{h^{n+\theta, p}}
    + J_{11}^{n+\theta, p} \theta\Delta h
    + J_{12}^{n+\theta, p} \theta\Delta q
    + J_{13}^{n+\theta, p} \theta\Delta r=
    \\
    & = \frac{q^{n+\theta, p}r^{n+\theta, p}}{h^{n+\theta, p}}
    - \frac{q^{n+\theta, p}r^{n+\theta, p}}{(h^{n+\theta, p})^2} \theta\Delta h
    + \frac{r^{n+\theta, p}}{h^{n+\theta, p}} \theta\Delta q
    + \frac{q^{n+\theta, p}}{h^{n+\theta, p}} \theta\Delta r
    \label{eq:2d_convection_q_equation}
\end{align}
in case $q$ is substitute for $r$ then $\lpdiff{}{r}=0$ and the approximation reads:
\begin{align}
    \left.\left(\frac{qq}{h}\right)\right|^{n+\theta, p+1} &  \approx
\frac{q^{n+\theta, p}q^{n+\theta, p}}{h^{n+\theta, p}}
- \frac{q^{n+\theta, p}q^{n+\theta, p}}{(h^{n+\theta, p})^2} \theta\Delta h
+ \frac{2 q^{n+\theta, p}}{h^{n+\theta, p}} \theta\Delta q.
\end{align}

%--------------------------------------------------------------------------------
\subsection{Terms with an operator}\label{sec:jacobians_with_operator}
The Jacobian of an operator is also applied to the argument of the operator.
As an example the pressure term of the shallow water equations is taken, where $\Delta h^{n+1, p+1} = \Delta h$ and  $\Delta \zeta^{n+1, p+1} = \Delta \zeta$:
\begin{align}
    \left. gh \pdiff{\zeta}{x}\right|^{n+\theta, p+1} & \approx g \left( h^{n+\theta, p} + \theta \Delta h \right) \pdiff{}{x} \left( \zeta^{n+\theta, p} + \theta\Delta \zeta  \right) \approx
%    & = g h^{n+\theta, p} \pdiff{\zeta^{n+\theta, p}}{x} + gh^{n+\theta, p} \pdiff{\theta\Delta \zeta}{x} + g \theta\Delta h \pdiff{\zeta^{n+\theta, p}}{x} + O(\Delta h \Delta \zeta) \approx
%    \\
%    & \approx g h^{n+\theta, p} \pdiff{\zeta^{n+\theta, p}}{x} + gh^{n+\theta, p} \pdiff{\theta\Delta \zeta}{x} + g \pdiff{\zeta^{n+\theta, p}}{x} \theta\Delta h =
    \\
    & \approx g h^{n+\theta, p} \pdiff{\zeta^{n+\theta, p}}{x}
    +  gh^{n+\theta, p} \pdiff{}{x} (\theta\Delta \zeta)
    +  g \pdiff{\zeta^{n+\theta, p}}{x} \theta\Delta h
\end{align}
omitting the quadratic  term $O(\Delta h\ \lpdiff{\Delta \zeta}{x})$.
The term  $\lpdiff{\Delta \zeta}{x}$ is not always small and therefor a psuedo time step method is introduced, see \autoref{sec:psuedo_time_step}.
In discrete form it reads on location $x_{i+\half}$:
\begin{align}
    \left. gh \pdiff{\zeta}{x}\right|^{n+\theta, p+1}_{i+\half} & \approx g h^{n+\theta, p}_{i+\half} \frac{\zeta^{n+\theta, p}_{i+1} - \zeta^{n+\theta, p}_{i}}{\Dx_{i}}
    + gh^{n+\theta, p}_{n+\half} \frac{\theta \Delta \zeta_{i+1} - \theta \Delta \zeta_{i}}{\Dx_{i}} +
    \\
    & \quad
    + g \frac{\zeta^{n+\theta, p}_{i+1} - \zeta^{n+\theta, p}_{i}}{\Dx_{i}} \theta \Delta h_{i+\half}
\end{align}
Remember that the gradient over a grid cell $\Dx_i$ is constant, due to the piecewise linear approximation between two nodes.
%--------------------------------------------------------------------------------
\subparagraph*{Jacobian}
When the Jacobian notation is  used for the function $F(q,h) = gh\, \lpdiff{\zeta}{x}$
\begin{align}
    \mat{J} =
    \begin{pmatrix}
        J_{11} & J_{12}
    \end{pmatrix}
    =
    \begin{pmatrix}
        \pdiff{F}{h} & \pdiff{F}{\zeta}
    \end{pmatrix}
    =
    \begin{pmatrix}
        \pdiff{(gh\, \lpdiff{\zeta}{x})}{h} & \pdiff{(gh\, \lpdiff{\zeta}{x})}{\zeta}
    \end{pmatrix}
    =
    \begin{pmatrix}
        g\, \pdiff{\zeta}{x} & gh\, \pdiff{}{x}
    \end{pmatrix}
\end{align}
it leads to following approximation
\begin{align}
    \left.gh \pdiff{\zeta}{x}\right|^{n+\theta, p+1}_{i+\half} & \approx g h^{n+\theta, p} \pdiff{\zeta^{n+\theta, p}}{x}
    + J_{11} \theta \Delta h
    + J_{12} \pdiff{}{x}(\theta \Delta \zeta) \approx
    \\
    & \approx g h^{n+\theta, p} \frac{\zeta^{n+\theta, p}_{i+1} - \zeta^{n+\theta, p}_{i}}{\Dx_{i}}
    + g \frac{\zeta^{n+\theta, p}_{i+1} - \zeta^{n+\theta, p}_{i}}{\Dx_{i}} \theta\Delta h_{i+\half} +
    \\
    & \quad + gh^{n+\theta, p}_{n+\half} \frac{\theta\Delta \zeta_{i+1} - \theta\Delta \zeta_{i}}{\Dx_{i}}
\end{align}
