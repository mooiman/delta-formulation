\documentclass[subfooter, backcover]{mooiman_memo}
\hypersetup
{
    pdfauthor   = {Mooiman},
}
\addbibresource{./common/literature_references.bib}

\usepackage{empheq}
\usepackage{cancel}

\usepackage{bm}
\renewcommand{\vec}[1]{\mbox{\boldmath ${#1}$} }

\sisetup{mode=math}
\captionsetup[subfigure]{justification=RaggedRight}

\allowdisplaybreaks[1]

\newcommand{\ii}{{\rm i}}
\newcommand{\Dx}{\Delta x}
\newcommand{\Dy}{\Delta y}
\newcommand{\Dh}{\Delta h}
\newcommand{\Dq}{\Delta q}
\newcommand{\Dr}{\Delta r}
\newcommand{\Dxi}{\Delta \xi}
\newcommand{\Deta}{\Delta \eta}
\newcommand{\Dt}{\Delta t}
\newcommand{\Dtinv}{\Delta t_{\it{inv}}}

\newcommand{\mat}[1]{\mbox{\boldmath ${\rm #1}$} }
\newcommand{\half}{\frac{1}{2}}
\newcommand{\quart}{\frac{1}{4}}
\newcommand{\abs}[1]{\left\lvert #1 \right\rvert}
\newcommand{\Fabs}[1]{F_{\it abs}(#1)\xspace}  % Continue function for the absolute value
\newcommand{\eps}{\varepsilon}
\newcommand{\norm}[1]{\left\lVert #1 \right\rVert}
\newcommand{\bunit}[1]{\text{\normalfont [}{\unit{#1}}\text{\normalfont ]}}
\newcommand{\bqty}[2]{\text{\num{#1}}\ \text{\normalfont [}{\unit{#2}}\text{\normalfont ]}}
\newcommand{\Pe}{{Pe}\xspace}
\newcommand{\bPsi}{\boldsymbol{\Psi}}

\newcommand{\utilde}{\widetilde{u}}
\newcommand{\ubar}{\overline{u}}
\newcommand{\uhat }{\widehat{u}}
\newcommand{\htilde}{\widetilde{h}}
\renewcommand{\hbar}{\overline{h}}  % overrule the Planck constant
\newcommand{\hhat }{\widehat{h}}
\newcommand{\qtilde}{\widetilde{q}}
\newcommand{\qbar}{\overline{q}}
\newcommand{\qhat }{\widehat{q}}

\newcommand{\maplesoft}{\textbf{Maplesoft}}
\newcommand{\deltaformulation}{$\Delta$-formulation\xspace}
\newcommand{\notyet}{\textbf{Not yet documented}\newline}

\newcommand{\red}[1]{\textcolor{red}{#1}}
\newcommand{\white}[1]{\textcolor{white}{#1}}

\begin{document}

\title{Manual, Two step FVE method}
\subtitle{A numerical modeling technique designed for error insight}
\contact{J.\ Mooiman}
\author{Jan Mooiman}
\partner{}

\documentid{ID-20251012}
\version{000}
\date{\today~\currenttime} % start of writing document
\status{\textbf{Work in progress}}

\memoSubject{Manual, Two step FVE method}
\memoTo{}

\mooimantitle
%\deltarestitle
%------------------------------------------------------------------------------
\chapter{Description of input file}
Nine observation points are hard coded in the software, these are located at:
\begin{enumerate}
    \item the centre of the grid.
    \item in the middle of the north boundary.
    \item in the middle of the east boundary.
    \item in the middle of the south boundary.
    \item in the middle of the west boundary.
    \item in the North-East corner of the grid.
    \item in the South-East corner of the grid.
    \item in the South-West corner of the grid.
    \item in the North-West corner of the grid.
\end{enumerate}
Extra observation points can be specified in the input file according the description in \autoref{sec:file_description}
%------------------------------------------------------------------------------
\section{File description} \label{sec:file_description}
Items in the color red are only to be used in the development stage of the software.
An example input file is given in \autoref{sec:example_input}.
%------------------------------------------------------------------------------
{{\small%\setlength{\tabcolsep}{1pt}
\begin{longtable}{p{35mm} p{20mm} p{\textwidth-55mm-36pt}}
    \caption{Standard input-file with default settings.\label{tab:appmdudefaults}} \\ \topline
    \rowcolor{mgreen1}
    \white{\textbf{Keyword}}   &  \white{\textbf{Default}}  &  \white{\textbf{Description}}   \\ \hline
    \endfirsthead
    \multicolumn{3}{@{}l}{\textsl{(continued from previous page)}} \\
    \topline
    \rowcolor{mgreen1}
    \white{\textbf{Keyword}}   &  \white{\textbf{Default}}  &  \white{\textbf{Description}}   \\
    \topline
    \endhead
    \multicolumn{3}{@{}l}{\textsl{(continued on next page)}} \\  % no horizontal lines, they should only be written by the table itself
    \endfoot
    \bottomline
    \endlastfoot
%
logging & "None" & "iterations" | "matrix" | "pattern" \\*
\midline
%
\textbf{\texttt{[Boundary]}} & & \\*
treg         & 150.0 & Regularization time boundary signal \\*
eps\_bc\_corr  & 0.01  & Boundary correction term to force the boundary value to the given boundary value \\*
bc\_type      & [\ ,\ ,\ ,\ ] &  ["free-slip" | "no-slip" | "borsboom" | "mooiman" ] \\*
bc\_vars      & [\ ,\ ,\ ,\ ] & ["-" | "zeta" | "q"] \\*
bc\_vals      & [\ ,\ ,\ ,\ ] & [Real value] \\*
bc\_absorbing & [\ ,\ ,\ ,\ ] & [Boolean value] \\*
\midline
%
\textbf{\texttt{[Domain]}} & & \\*
bed-level-file    &-& Name of bed-level-file between  double quotes \\*
mesh-file         &-& Name of mesh file point between double quotes \\*
\midline
%
\textbf{\texttt{[Initial]}}  && \\*
ini\_vars & None & ["zeta", "zeta\_gauss\_hump" | "zeta\_constant" | "zeta\_gauss\_hump\_x" |  "zeta\_gauss\_hump\_y"]  \\*
gauss\_amp & 0.0 & \\*
gauss\_mu & 0.0& \\*
gauss\_mu\_x & 0.0 & \\*
gauss\_mu\_y & 0.0 & \\*
gauss\_sigma & 350.0 & \\*
gauss\_sigma\_x & 350.0 & Optional if "gauss\_sigma" is not specified \\*
gauss\_sigma\_y & 350.0 & Optional if "gauss\_sigma" is not specified \\
\midline
%
\multicolumn{3}{@{}l}{\textbf{\texttt{[Numerics]}} } \\*
dt & 5.0  & Time step size [s], if dt == 0: then stationary simulation \\*
theta & 0.501  & Implicitness factor (0.5 <= theta <= 1.0) \\*
c\_psi & 4.0 & Smoothness factor used for regularization \\*
iter\_max & 25  & Maximum number of nonlinear iterations \\*
eps\_newton & 1e-12 & Accuracy of the Newton solver (non-linear)\\*
eps\_bicgstab & 1e-06 & Accuracy of the linear solver\\*
eps\_abs\_function & 0.01 & Smoothing factor for the absolute-function\\*
linear\_solver & "bicgstab" & Type of linear solver "bicgstab", {\color{red} "multigrid" } \\*
regularization\_init & false & true, false \\*
regularization\_iter & false & true, false \\*
regularization\_time & false & true, false \\*
\midline
%
\multicolumn{3}{@{}l}{\textbf{\texttt{[Physics]}} } \\*
\red{do\_linear\_waves} & \red{true} &  \red{true, false} \\*
\red{do\_continuity}    & \red{true} &  \red{true, false} \\*
\red{do\_q\_equation}   & \red{true} &  \red{true, false} \\*
\red{do\_r\_equation}   & \red{true} &  \red{true, false} \\*
do\_convection & false & true, false \\*
do\_bed\_shear\_stress & false & true, false \\*
do\_viscosity & false & true, false \\*
chezy\_coefficient & 25.0 & Bed friction coefficient\\*
viscosity & 0.01 & Diffusity coefficient\\*
\midline
%
%
\multicolumn{3}{@{}l}{\textbf{\texttt{[Output]}} } \\*
dt\_his & dt & Time interval to write the history file, \bunit{\second}\\*
dt\_map & $60.0$ & Time interval to write the history file, \bunit{\second} \\*
\midline
%
%
\multicolumn{3}{@{}l}{\textbf{\texttt{[Time]}} } \\*
start & 0.0 & Start time of the simulation, \bunit{\second} \\*
stop & 60.0 & Stop time of the simulation, \bunit{\second} \\*
\midline
%
\multicolumn{3}{@{}l}{\textbf{\texttt{[[ObservationPoint]]}} } \\*
{x}    &-& $x$-coordinate of the observation point\\*
{y}    &-& $y$-coordinate of the observation point\\*
{name} &-& Name of observation point between double quotes \\
\midline

\end{longtable}

%
%================================================================================
\newpage
\section{Example input file}\label{sec:example_input}
\begin{Verbatim}[fontsize=\scriptsize]
logging = "None"  # "iterations", "matrix", "pattern"

[Boundary]  # north, east, south, west
bc_type = ["borsboom", "borsboom", "borsboom", "borsboom"]  # Type "neumann", "dirichlet"
bc_vars = ["zeta", "zeta", "zeta", "zeta"]
bc_absorbing = [true, true, true, true]
bc_vals = [0.0, 0.0, 0.0, 0.0]
treg = 150.0  # Regularization time boundary signal
eps_bc_corr = 0.01

[Domain]
mesh_file = "geometry/62x62_6x6km_net.nc"
bed_level_file = "geometry/62x62_6x6km_flat.dep"

[Initial]
ini_vars = ["zeta", "zeta_gauss_hump"]
Gauss_amp = 1.0
Gauss_mu = 0.0
Gauss_mu_x = 0.0
Gauss_mu_y = 0.0
Gauss_sigma = 350.0
Gauss_sigma_x = 350.0
Gauss_sigma_y = 350.0

[Numerics]
dt = 5.0  # Time step size [s], if dt == 0: then stationary problem
theta = 0.501  # Implicitness factor (0.5 <= theta <= 1.0)
c_psi = 4.0
iter_max = 25  # Maximum number of nonlinear iterations
eps_newton = 1e-12
eps_bicgstab = 1e-06
eps_abs_function = 0.01
linear_solver = "bicgstab"  # "bicgstab", "multigrid"
regularization_init = false
regularization_iter = false
regularization_time = false

[Physics]
do_linear_waves = true
do_continuity = true
do_q_equation = true
do_r_equation = true
do_convection = false
do_bed_shear_stress = false
do_viscosity = false
chezy_coefficient = 25.0
viscosity = 100.0

[Output]
dt_his = 0.01
dt_map = 30.0

[Time]
tstart = 0.0
tstop = 1800.0
\end{Verbatim}

\end{document}