%-------------------------------------------------------------------------------
\chapter{2D Heat equation}
%-------------------------------------------------------------------------------
Consider the non-linear heat equation
\begin{subequations}
    \begin{align}
    \pdiff{u}{t}
    - \nabla \dotp \left( \bPsi \nabla u \right) = 0,
\end{align}
\end{subequations}
with
\begin{symbollist}
    \item[$u$] Temperature, \bunit{\celsius}.
    \item[$\bPsi$] 2-dimensional thermal conductivity coefficient, \bunit{\square\metre\per\second}.
\end{symbollist}
$\bPsi$ the thermal conductivity coefficient is defined as:
\begin{align}
    \bPsi =
    \begin{pmatrix}
        \Psi_{11} & \Psi_{12} \\
        \Psi_{21} & \Psi_{22}
    \end{pmatrix}
\end{align}
The 2D-heat equation in curvilinear ($\xi,\eta$)-coordinates reads:
\begin{align}
    &J \pdiff{u}{t} - \nabla \dotp \left( \bPsi \nabla u \right) = 0,
    \label{eq:heat_2d_curvi}
\end{align}
\subsection*{Finite Volume approach}
Integrating the equations over a finite volume $\Omega$ yields:
\begin{align}
    \int_\Omega \pdiff{u}{t}\, d\omega &
    - \int_\Omega \nabla \dotp \left( \bPsi \nabla u \right) \, d\omega = 0
    \label{eq:heat_2d_equation}
\end{align}
%
The 2D-heat equation in curvilinear coordinates reads:
\begin{align}
    &\int_{\Omega_{\xi\eta}} J \pdiff{u}{t}\, d\xi d\eta - \int_{\Omega_{\xi\eta}} \nabla \dotp \left( \bPsi \nabla u \right)\, d\xi d\eta = 0,
    \label{eq:heat_2d_curvi}
\end{align}


%-------------------------------------------------------------------------------
\section{Space discretization, structured}
\begin{figure}[H]
    \begin{center}
        \def\svgwidth{0.80\textwidth} % scaling text
        \resizebox{0.65\textwidth}{!}{
            \input{figures/cartesian_grid_interior_2d.pdf_tex}
        }
    \end{center}
    \caption[Definition of the grid to solve the 2D-shallow water equations in the interior area]{Coefficients for the mass-matrix and the control volume in 2-dimensions in the interior area, on a structured grid. The black dots indicate the location of the quadrature points, and black diamonds the flux points.}
    \label{fig:2d_structured_grid}
\end{figure}
For the space discretization of an arbitrary function $u$ on the quadrature point of a sub-control volume the following space interpolations are used:
\begin{align}
    \left. u\right|_{i+\quart, j+\quart} & \approx \frac{1}{16}\left( 9u_{i, j} + 3 u_{i+1,j} + u_{i+1, j+1} + 3  u_{i, j+1}\right)
    \\
    \left.u\right|_{i+\half, j+\quart} & \approx \frac{1}{8} \left( 3 u_{i,j} + 3 u_{i+1,j} + u_{i+1, j+1} + u_{i, j+1} \right)
    \\
    \left. u \right|_{i+\quart, j+\half} & \approx \frac{1}{8} \left( 3 u_{i, j} + u_{i+1, j} + u_{i+1, j+1}  + 3 u_{i, j+1} \right)
\end{align}
See for the locations \autoref{fig:2d_structured_grid}.
%-------------------------------------------------------------------------------
\subsection{Discretizations heat equation}
The discretization of heat \autoref{eq:heat_2d_equation} will be presented term by term.
%-------------------------------------------------------------------------------
\subsubsection{Time derivative}
The time derivative term of the heat equation reads:
\begin{align}
    \int_{\Omega_{\xi\eta}} J \pdiff{u}{t}\, d\xi d\eta
\end{align}
which will be approximated by the sum of the integral over the sub-control volumes,
with taking into account that the Jacobian $J$ is constant in time.
On a structured grid one control volume ($cv$) around a node consist of four sub-control volumes ($scv_i$, $i\in\{0,1,2,3\}$).
\begin{align}
    J_{cv}\int_{cv} \pdiff{u}{t}\, d\omega &=
    J_{scv_0}\int_{scv_0} \pdiff{u}{t}\, d\omega +
    J_{scv_1}\int_{scv_1} \pdiff{u}{t}\, d\omega +
    \nonumber \\*
    &\quad
    +J_{scv_2}\int_{scv_2} \pdiff{u}{t}\, d\omega +
    J_{scv_3}\int_{scv_3} \pdiff{u}{t}\, d\omega
\end{align}
Fore the discretization on a curvilinear grid we get:
\begin{align}
    J_{cv}\int_{cv} \pdiff{u}{t}\, d\omega \approx &
    J_{scv_0}\Dtinv \left( u^{n+1,p+1}_{i-\quart, j-\quart} -  u^{n+1,n}_{i-\quart, j-\quart} \right) +
    \nonumber \\*
    &J_{scv_1}\Dtinv \left( u^{n+1,p+1}_{i+\quart, j-\quart} -  u^{n+1,n}_{i+\quart, j-\quart} \right) +
    \nonumber \\*
    &J_{scv_2}\Dtinv \left( u^{n+1,p+1}_{i+\quart, j+\quart} -  u^{n+1,n}_{i+\quart, j+\quart} \right) +
    \nonumber \\*
    &J_{scv_3}\Dtinv \left( u^{n+1,p+1}_{i-\quart, j+\quart} -  u^{n+1,n}_{i-\quart, j+\quart} \right)
\end{align}
with $J_{scv_i}$ the area of the sub control volume $i$.
For cartesian grids we have $J_{scv_i} = \quart \Dx\Dy$.

Just looking to the quadrature point of $scv_2$ as part of the control volume for node $(i,j)$ the discretization reads:
\begin{align}
    &J_{scv_2}\Dtinv \left( u^{n+1}_{i+\quart, j+\quart} -  u^{n+1,n}_{i+\quart, j+\quart} \right) =
    \\*
    &= J_{scv_2}\Dtinv \left[ \frac{1}{16}\left( 9u^{n+1}_{i, j} + 3 u^{n+1}_{i+1,j}  + 3  u^{n+1}_{i, j+1} + u^{n+1}_{i+1, j+1}\right) \right. +
    \\*
    &\quad - \left. \frac{1}{16}\left( 9 u^{n}_{i, j} +  3 u^{n}_{i+1,j}  + 3  u^{n}_{i, j+1} + u^{n}_{i+1, j+1}\right)\right]
\end{align}
Written in \deltaformulation it yields (using $\Delta u^{n+1,p+1} = \Delta u$):
\begin{align}
    &J_{scv_2}\Dtinv \left( u^{n+1}_{i+\quart, j+\quart} -  u^{n}_{i+\quart, j+\quart} \right) =
    \\*
    &=J_{scv_2}\Dtinv \left[ \frac{1}{16}\left( 9 \Delta u_{i, j} + 3 \Delta u_{i+1,j+1}  + 3 \Delta u_{i, j+1} + \Delta u_{i+1, j+1}\right) \right] +
    \nonumber \\*
    & + J_{scv_2}\Dtinv \Bigg[ \frac{1}{16}\bigg(
    9 \left( u^{n+1,p}_{i, j} - u^{n}_{i, j}\right) +
    3 \left( u^{n+1,p}_{i+1,j} - u^{n}_{i+1,j}\right)  +
    \nonumber \\*
    & \qquad \qquad \qquad 3 \left(u^{n+1,p}_{i, j+1} - u^{n}_{i, j+1}\right) +
    \left(u^{n+1,p+1}_{i+1, j+1} - u^{n}_{i+1, j+1}\right) \Bigg) \Bigg]
\end{align}
with
\begin{align}
    J_{scv_2} &= x_\xi y_\eta - y_\xi x_\eta
\end{align}
with  $J_{scv_2}$ representing the area of $scv_2$, this area is computed
by
\begin{align}
    J = \half \sum_{i=0}^{3} (x_{i} y_{i+1}  - x_{i+1} y_{i}),
    \quad \text{with } x_4 = x_0 \text{ and } y_4 = y_0
\end{align}


%--------------------------------------------------------------------------------
\subsubsection{Thermal conductivity}
The thermal conductivity term in vector notation reads:
\begin{align}
    \int_\Omega & \nabla \dotp \left( \mat{\Psi} \nabla u \right) \, d\omega  =
    \oint_{\Omega}  \left(  \mat{\Psi} \nabla u \right) \dotp \vec{n}\, dl =
    \\
    = & \oint_{\Omega} \left(
      \left( \Psi_{11} \pdiff{{u}}{x} + \Psi_{12} \pdiff{{u}}{y}  \right) \nx
    + \left( \Psi_{21} \pdiff{{u}}{x} + \Psi_{22} \pdiff{{u}}{y}  \right) \ny
    \right) \norm{dl}
\end{align}
with $\vecn = (\nx, \ny)^T$ the outward normal vector.

For the heat-equation the thermal conductivity term:
\begin{align}
    \pdiff{}{x} \left( \Psi_{11} \pdiff{{u}}{x} + \Psi_{12} \pdiff{{u}}{y}  \right) +
    \pdiff{}{y} \left( \Psi_{21} \pdiff{{u}}{x} + \Psi_{22} \pdiff{{u}}{y}  \right)
\end{align}
can be transformed as\newline
\textit{step 1a}
\begin{align}
&\frac{1}{J} \Bigg[
y_n \pdiff{}{\xi} \left(  \Psi_{11} \pdiff{{u}}{x} + \Psi_{12} \pdiff{{u}}{y} \right) - y_\xi \pdiff{}{\eta} \left(  \Psi_{11} \pdiff{{u}}{x} + \Psi_{12} \pdiff{{u}}{y} \right)
+
\\*
& \qquad
-x_\eta \pdiff{}{\xi} \left(  \Psi_{21} \pdiff{{u}}{x} + \Psi_{22} \pdiff{{u}}{y}  \right) + x_\xi \pdiff{}{\eta} \left( \Psi_{21} \pdiff{{u}}{x} + \Psi_{22} \pdiff{{u}}{y}   \right)
\Bigg]
\end{align}
\textit{step 1b}\newline
Assume $x_{\xi\eta} = x_{\eta\xi}$ and $y_{\xi\eta} = y_{\eta\xi}$
\begin{align}
    &\frac{1}{J} \Bigg[
    \pdiff{}{\xi} \left( y_\eta \left( \Psi_{11} \pdiff{{u}}{x} + \Psi_{12} \pdiff{{u}}{y} \right) \right)
    - \pdiff{}{\eta} \left( y_\xi \left(  \Psi_{11} \pdiff{{u}}{x} + \Psi_{12} \pdiff{{u}}{y} \right) \right)
    +
    \\*
    & \qquad
    -\pdiff{}{\xi} \left( x_\eta  \left( \Psi_{21} \pdiff{{u}}{x} + \Psi_{22} \pdiff{{u}}{y} \right) \right)
    + \pdiff{}{\eta} \left( x_\xi \left( \Psi_{21} \pdiff{{u}}{x} + \Psi_{22} \pdiff{{u}}{y} \right) \right)
    \Bigg]
\end{align}
Rearrange this equation to a divergence type in $\xi$- and $\eta$-direction to prepare to use Green's theorem, yields:
\begin{align}
    &\frac{1}{J} \Bigg[
    \pdiff{}{\xi} \left( y_\eta \left( \Psi_{11} \pdiff{{u}}{x} + \Psi_{12} \pdiff{{u}}{y} \right)
    - x_\eta  \left( \Psi_{21} \pdiff{{u}}{x} + \Psi_{22} \pdiff{{u}}{y} \right) \right)
    +
    \\*
    & \qquad
    + \pdiff{}{\eta} \left( -y_\xi \left(  \Psi_{11} \pdiff{{u}}{x} + \Psi_{12} \pdiff{{u}}{y} \right)
    +   x_\xi \left( \Psi_{21} \pdiff{{u}}{x} + \Psi_{22} \pdiff{{u}}{y} \right) \right)
    \Bigg]
\end{align}
 Applying Green's theorem, and multiplying with $J$ (because the whole heat equation is multiplied by $J$)\newline
 \textit{step 2a}:
\begin{align}
    &
    \left( y_\eta \left( \Psi_{11} \pdiff{{u}}{x} + \Psi_{12} \pdiff{{u}}{y} \right)
    - x_\eta  \left( \Psi_{21} \pdiff{{u}}{x} + \Psi_{22} \pdiff{{u}}{y} \right) \right) \nxi
    +
    \\*
    & \qquad
    + \left( -y_\xi \left(  \Psi_{11} \pdiff{{u}}{x} + \Psi_{12} \pdiff{{u}}{y} \right)
    +   x_\xi \left( \Psi_{21} \pdiff{{u}}{x} + \Psi_{22} \pdiff{{u}}{y} \right) \right) \neta
\end{align}
with $\vecn = (\nxi, \neta)^T$ the outward normal vector.

\textit{step 2b}:
\begin{align}
    &
    \frac{1}{J} \Bigg[ y_\eta \left(
      \Psi_{11} \left( y_\eta \pdiff{u}{\xi} - y_\xi  \pdiff{u}{\eta}\right)
    + \Psi_{12} \left( - x_\eta \pdiff{u}{\xi} + x_\xi \pdiff{u}{\eta} \right) \right) +
    \\*
    & \quad - x_\eta \left(
      \Psi_{21} \left( y_\eta \pdiff{u}{\xi} - y_\xi  \pdiff{u}{\eta}\right)
    + \Psi_{22} \left( - x_\eta \pdiff{u}{\xi} + x_\xi \pdiff{u}{\eta} \right)
    \right)
    \Bigg] \nxi
    +
    \\
    &\frac{1}{J} \Bigg[
    -y_\xi \left(
       \Psi_{11} \left( y_\eta \pdiff{u}{\xi} - y_\xi  \pdiff{u}{\eta}\right)
     + \Psi_{12} \left( - x_\eta \pdiff{u}{\xi} + x_\xi \pdiff{u}{\eta} \right)  \right) +
    \\*
    &\quad +   x_\xi \left(
       \Psi_{21} \left( y_\eta \pdiff{u}{\xi} - y_\xi  \pdiff{u}{\eta}\right)
     + \Psi_{22} \left( - x_\eta \pdiff{u}{\xi} + x_\xi \pdiff{u}{\eta} \right)  \right)
     \Bigg] \neta
\end{align}

%--------------------------------------------------------------------------------
\subsection*{Orthotropic conductivity}
When the thermal conductivity coefficient is orthotropic  ($\Psi_{12} = \Psi_{21} = 0$) reduces to:
%($\Psi_{11} = \Psi_{22} = \Psi(\xi, \eta)$ and $\Psi_{12} = \Psi_{21} = 0$)
\begin{align}
    &
    \frac{1}{J} \Bigg[ y_\eta
    \Psi_{11} \left( y_\eta \pdiff{u}{\xi} - y_\xi  \pdiff{u}{\eta}\right)
     - x_\eta
    \Psi_{22} \left( - x_\eta \pdiff{u}{\xi} + x_\xi \pdiff{u}{\eta} \right)
    \Bigg] \nxi
    +
    \\
    & + \frac{1}{J} \Bigg[
    -y_\xi
    \Psi_{11} \left( y_\eta \pdiff{u}{\xi} - y_\xi  \pdiff{u}{\eta}\right)
    +   x_\xi
     \Psi_{22} \left( - x_\eta \pdiff{u}{\xi} + x_\xi \pdiff{u}{\eta} \right)
    \Bigg] \neta
\end{align}

In case it is also orthogonal (i.e.\ $y_\xi = x_\eta =0$) it reads:
\begin{align}
    &
    \frac{1}{J} \Bigg[ y_\eta
    \Psi_{11} \left( y_\eta \pdiff{u}{\xi} \right)
    \Bigg] \nxi
    + \frac{1}{J} \Bigg[
    x_\xi
    \Psi_{22} \left( x_\xi \pdiff{u}{\eta} \right)
    \Bigg] \neta
\end{align}

\notyet


Define the components of $\bPsi$ at different locations of the volume faces where $(\mu, \nu) \in \{ 1, 2\}$ as
%
\begin{align}
    \left. \overline\Psi_{\mu\nu} \right|_{i+\half,j+\quart} = \frac{1}{8} \left(\left. 3\Psi_{\mu\nu} \right|_{i,j}
    + \left. 3\Psi_{\mu\nu} \right|_{i+1,j} + \left. \Psi_{\mu\nu} \right|_{i+1,j+1}  + \left. \Psi_{\mu\nu} \right|_{i,j+1}\right)
\end{align}
%
and define the partial differentials as
\begin{align}
    \left. \pdiff{u}{x} \right|_{i+\half,j+\quart} =  \quart \left(3\frac{u_{i+1,j} - u_{i,j}}{\Dx_{i+\half,j}} + \frac{u_{i+1,j+1} - u_{i,j+1}}{\Dx_{i+\half,j+1}}\right)
    \\
    \left. \pdiff{u}{y} \right|_{i+\half,j+\quart} =  \half \left(\frac{u_{i,j+1} - u_{i,j}}{\Dy_{i,j+\half}} + \frac{u_{i+1,j+1} - u_{i+1,j}}{\Dy_{i+1,j+\half}}\right)
\end{align}

%================================================================================
\chapter{Numerical experiment}
\section{Initial Dirac-delta function}
Th analytic solution for a 2D heat equation is
\begin{align}
T(t)= \frac{1}{4D\pi t} \exp\left(\frac{-x^2 - y^2}{4Dt}\right)
\end{align}
where
\begin{symbollist}
    \item[$T$] Temperature, \bunit{\celsius}.
    \item[$D$] Heat conductivity, \bunit{\watt\per\metre\per\kelvin}.
    \item[$t$] Time, \bunit{\second}.
    \item[$x$] Coordinate in $x$-direction, \bunit{\metre}.
    \item[$y$] Coordinate in $y$-direction, \bunit{\metre}.
\end{symbollist}

%================================================================================
\appendix
\chapter{Curvilinear coordinate transformation}\label{app:CoordinateTransform}
\notyet
