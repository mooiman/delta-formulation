%-------------------------------------------------------------------------------
\chapter{2D Heat equation}
%-------------------------------------------------------------------------------
Consider the non-linear heat equation
\begin{subequations}
    \begin{align}
    \pdiff{u}{t}
    - \nabla \dotp \left( \bPsi \nabla u \right) = 0,
\end{align}
\end{subequations}
with
\begin{symbollist}
    \item[$u$] Temperature, \bunit{\celsius}.
    \item[$\bPsi$] 2-dimensional thermal diffusivity, \bunit{\square\metre\per\second}.
\end{symbollist}
$\bPsi$ the thermal diffusivity is defined as:
\begin{align}
    \bPsi =
    \begin{pmatrix}
        \Psi_{11} & \Psi_{12} \\
        \Psi_{21} & \Psi_{22}
    \end{pmatrix}
\end{align}
The 2D-heat equation in curvilinear coordinates read:
\begin{align}
    &J \pdiff{u}{t} - \nabla \dotp \left( \bPsi \nabla u \right) = 0,
    \label{eq:heat_2d_curvi}
\end{align}
\subsection*{Finite Volume approach}
Integrating the equations over a finite volume $\Omega$ yields:
\begin{align}
    \int_\Omega \pdiff{u}{t}\, d\omega &
    - \int_\Omega \nabla \dotp \left( \bPsi \nabla u \right) \, d\omega = 0
    \label{eq:heat_2d_equation}
\end{align}
%
The 2D-heat equation in curvilinear coordinates read:
\begin{align}
    &\int_{\Omega_{\xi\eta}} J \pdiff{u}{t}\, d\xi d\eta - \int_{\Omega_{\xi\eta}} \nabla \dotp \left( \bPsi \nabla u \right)\, d\xi d\eta = 0,
    \label{eq:heat_2d_curvi}
\end{align}


%-------------------------------------------------------------------------------
\section{Space discretization, structured}
\begin{figure}[H]
    \begin{center}
        \def\svgwidth{0.80\textwidth} % scaling text
        \resizebox{0.65\textwidth}{!}{
            \input{figures/cartesian_grid_interior_2d.pdf_tex}
        }
    \end{center}
    \caption[Definition of the grid to solve the 2D-shallow water equations in the interior area]{Coefficients for the mass-matrix and the control volume in 2-dimensions in the interior area, on a structured grid. The black dots indicate the location of the quadrature points, and black diamonds the flux points.}
    \label{fig:2d_structured_grid}
\end{figure}
For the space discretization of an arbitrary function $u$ on the quadrature point of a sub-control volume the following space interpolations are used:
\begin{align}
    \left. u\right|_{i+\quart, j+\quart} & \approx \frac{1}{16}\left( 9u_{i, j} + 3 u_{i+1,j} + u_{i+1, j+1} + 3  u_{i, j+1}\right)
    \\
    \left.u\right|_{i+\half, j+\quart} & \approx \frac{1}{8} \left( 3 u_{i,j} + 3 u_{i+1,j} + u_{i+1, j+1} + u_{i, j+1} \right)
    \\
    \left. u \right|_{i+\quart, j+\half} & \approx \frac{1}{8} \left( 3 u_{i, j} + u_{i+1, j} + u_{i+1, j+1}  + 3 u_{i, j+1} \right)
\end{align}
See for the locations \autoref{fig:2d_structured_grid}.
%-------------------------------------------------------------------------------
\subsection{Discretizations heat equation}
The discretization of heat \autoref{eq:heat_2d_equation} will be presented term by term.
%-------------------------------------------------------------------------------
\subsubsection{Time derivative}
The time derivative term of the heat equation reads:
\begin{align}
    \int_{\Omega_{\xi\eta}} J \pdiff{u}{t}\, d\xi d\eta
\end{align}
which will be approximated by the sum of the integral over the sub-control volumes,
with taking into account that the Jacobian $J$ is constant in time.
On a structured grid one control volume ($cv$) around a node consist of four sub-control volumes ($scv_i$, $i\in\{0,1,2,3\}$).
\begin{align}
    J_{cv}\int_{cv} \pdiff{u}{t}\, d\omega &=
    J_{scv_0}\int_{scv_0} \pdiff{u}{t}\, d\omega +
    J_{scv_1}\int_{scv_1} \pdiff{u}{t}\, d\omega +
    \nonumber \\*
    &\quad
    +J_{scv_2}\int_{scv_2} \pdiff{u}{t}\, d\omega +
    J_{scv_3}\int_{scv_3} \pdiff{u}{t}\, d\omega
\end{align}
Fore the discretization on a curvilinear grid we get:
\begin{align}
    J_{cv}\int_{cv} \pdiff{u}{t}\, d\omega \approx &
    J_{scv_0}\Dtinv \left( u^{n+1,p+1}_{i-\quart, j-\quart} -  u^{n+1,n}_{i-\quart, j-\quart} \right) +
    \nonumber \\*
    &J_{scv_1}\Dtinv \left( u^{n+1,p+1}_{i+\quart, j-\quart} -  u^{n+1,n}_{i+\quart, j-\quart} \right) +
    \nonumber \\*
    &J_{scv_2}\Dtinv \left( u^{n+1,p+1}_{i+\quart, j+\quart} -  u^{n+1,n}_{i+\quart, j+\quart} \right) +
    \nonumber \\*
    &J_{scv_3}\Dtinv \left( u^{n+1,p+1}_{i-\quart, j+\quart} -  u^{n+1,n}_{i-\quart, j+\quart} \right)
\end{align}
with $J_{scv_i}$ the area of the sub control volume $i$.
For cartesian grids we have $J_{scv_i} = \quart \Dx\Dy$.

Just looking to the quadrature point of $scv_2$ as part of the control volume for node $(i,j)$ the discretization reads:
\begin{align}
    &J_{scv_2}\Dtinv \left( u^{n+1}_{i+\quart, j+\quart} -  u^{n+1,n}_{i+\quart, j+\quart} \right) =
    \\*
    &= J_{scv_2}\Dtinv \left[ \frac{1}{16}\left( 9u^{n+1}_{i, j} + 3 u^{n+1}_{i+1,j}  + 3  u^{n+1}_{i, j+1} + u^{n+1}_{i+1, j+1}\right) \right. +
    \\*
    &\quad - \left. \frac{1}{16}\left( 9 u^{n}_{i, j} +  3 u^{n}_{i+1,j}  + 3  u^{n}_{i, j+1} + u^{n}_{i+1, j+1}\right)\right]
\end{align}
Written in \deltaformulation it yields (using $\Delta u^{n+1,p+1} = \Delta u$):
\begin{align}
    &J_{scv_2}\Dtinv \left( u^{n+1}_{i+\quart, j+\quart} -  u^{n}_{i+\quart, j+\quart} \right) =
    \\*
    &=J_{scv_2}\Dtinv \left[ \frac{1}{16}\left( 9 \Delta u_{i, j} + 3 \Delta u_{i+1,j+1}  + 3 \Delta u_{i, j+1} + \Delta u_{i+1, j+1}\right) \right] +
    \nonumber \\*
    & + J_{scv_2}\Dtinv \Bigg[ \frac{1}{16}\bigg(
    9 \left( u^{n+1,p}_{i, j} - u^{n}_{i, j}\right) +
    3 \left( u^{n+1,p}_{i+1,j} - u^{n}_{i+1,j}\right)  +
    \nonumber \\*
    & \qquad \qquad \qquad 3 \left(u^{n+1,p}_{i, j+1} - u^{n}_{i, j+1}\right) +
    \left(u^{n+1,p+1}_{i+1, j+1} - u^{n}_{i+1, j+1}\right) \Bigg) \Bigg]
\end{align}
with
\begin{align}
    J_{scv_2} &= x_\xi y_\eta - y_\xi x_\eta
\end{align}
with  $J_{scv_2}$ representing the area of $scv_2$, this area is computed
by
\begin{align}
    J = \half \sum_{i=0}^{3} (x_{i} y_{i+1}  - x_{i+1} y_{i}),
    \quad \text{with } x_4 = x_0 \text{ and } y_4 = y_0
\end{align}


%--------------------------------------------------------------------------------
\subsubsection{Viscosity}
The viscosity term in vector notation reads:
\begin{align}
    \int_\Omega & \nabla \dotp \left( \mat{\Psi} \nabla u \right) \, d\omega  =
    \oint_{\Omega}  \left(  \mat{\Psi} \nabla u \right) \dotp \vec{n}\, dl =
    \\
    = & \oint_{\Omega}  \left( \left( \Psi_{11} \pdiff{{u}}{x} + \Psi_{12} \pdiff{{u}}{y}  \right) n_x + \left( \Psi_{21} \pdiff{{u}}{x} + \Psi_{22} \pdiff{{u}}{y}  \right) n_y \right) \norm{dl}
\end{align}
with $\vecn = (\nx, \ny)^T$ the outward normal vector.

For the heat-equation the diffusity term can be transformed as:
\begin{align}
    \pdiff{}{x} \left( \Psi_{11} \pdiff{{u}}{x} + \Psi_{12} \pdiff{{u}}{y}  \right) +
    \pdiff{}{y} \left( \Psi_{21} \pdiff{{u}}{x} + \Psi_{22} \pdiff{{u}}{y}  \right)
\end{align}
\textit{step 1a}
\begin{align}
&\frac{1}{J} \Bigg[
y_n \pdiff{}{\xi} \left(  \Psi_{11} \pdiff{{u}}{x} + \Psi_{12} \pdiff{{u}}{y} \right) - y_\xi \pdiff{}{\eta} \left(  \Psi_{11} \pdiff{{u}}{x} + \Psi_{12} \pdiff{{u}}{y} \right)
\Bigg] +
\\*
+
&\frac{1}{J} \Bigg[
-x_\eta \pdiff{}{\xi} \left(  \Psi_{21} \pdiff{{u}}{x} + \Psi_{22} \pdiff{{u}}{y}  \right) + x_\xi \pdiff{}{\eta} \left( \Psi_{21} \pdiff{{u}}{x} + \Psi_{22} \pdiff{{u}}{y}   \right)
\Bigg]
\end{align}
\textit{step 1b}\newline
Assume $x_{\xi\eta} = x_{\eta\xi}$ and $y_{\xi\eta} = y_{\eta\xi}$
\begin{align}
    &\frac{1}{J} \Bigg[
    \pdiff{}{\xi} \left( y_\eta \left( \Psi_{11} \pdiff{{u}}{x} + \Psi_{12} \pdiff{{u}}{y} \right) \right)
    - \pdiff{}{\eta} \left( y_\xi \left(  \Psi_{11} \pdiff{{u}}{x} + \Psi_{12} \pdiff{{u}}{y} \right) \right)
    \Bigg] +
    \\*
    +
    &\frac{1}{J} \Bigg[
    -\pdiff{}{\xi} \left( x_\eta  \left( \Psi_{21} \pdiff{{u}}{x} + \Psi_{22} \pdiff{{u}}{y} \right) \right)
    + \pdiff{}{\eta} \left( x_\xi \left( \Psi_{21} \pdiff{{u}}{x} + \Psi_{22} \pdiff{{u}}{y} \right) \right)
    \Bigg]
\end{align}
\textit{step 2}
\begin{align}
    &\frac{1}{J} \Bigg[
      \pdiff{}{\xi} \left( y_\eta  \Psi_{11} \left( y_\eta \pdiff{u}{\xi}
    - y_\xi  \pdiff{u}{\eta}\right)
    + y_\eta \Psi_{12} \left( - x_\eta \pdiff{u}{\xi} + x_\xi \pdiff{u}{\eta} \right) \right) +
    \\*
    &\qquad
    - \pdiff{}{\eta} \left( y_\xi  \Psi_{11} \left(y_\eta \pdiff{u}{\xi}  - y_\xi \pdiff{u}{\eta} \right)
    + y_\xi \Psi_{12} \left( -x_\eta \pdiff{u}{\xi} + x_\xi \pdiff{u}{\eta} \right) \right)
    \Bigg] +
    \\
    +
    &\frac{1}{J} \Bigg[
    -\pdiff{}{\xi} \left( x_\eta  \Psi_{21} \left( y_\eta \pdiff{u}{\xi} - y_\xi \pdiff{u}{\eta} \right)
    + x_\eta \Psi_{22} \left( -x_\eta \pdiff{u}{\xi} + x_\xi \pdiff{u}{\eta} \right)  \right) +
    \\*
    &\qquad
    + \pdiff{}{\eta} \left( x_\xi \Psi_{21} \left( y_\eta \pdiff{u}{\xi}  - y_\xi \pdiff{u}{\eta} \right)
    + x_\xi \Psi_{22} \left( -x_\eta \pdiff{u}{\xi} + x_\xi \pdiff{u}{\eta} \right)   \right)
    \Bigg]
\end{align}
{{
\color{gray}
\begin{align}
    &\frac{1}{J} \Bigg[
    y_\eta y_{\xi\eta} \Psi_{11} \pdiff{u}{\xi}
  + y_\eta y_{\eta}    \pdiff{}{\xi} \left(  \Psi_{11} \pdiff{u}{\xi} \right)
  - y_\eta y_{\xi\xi}  \Psi_{11} \pdiff{u}{\eta}
  - y_\eta y_{\xi}     \pdiff{}{\xi}\left(\Psi_{11} \pdiff{u}{\eta}\right) +
  \\*
  &
  - y_\eta x_{\xi\eta} \Psi_{12}\pdiff{u}{\xi}
  - y_\eta x_{\eta}    \pdiff{}{\xi} \left( \Psi_{12} \pdiff{u}{\xi}\right)
  + y_\eta x_{\xi\xi}  \Psi_{12} \pdiff{u}{\eta}
  + y_\eta x_{\xi}     \pdiff{}{\xi} \left( \Psi_{12} \pdiff{u}{\eta} \right) +
    \\
  &
- y_\xi y_{\eta\eta} \Psi_{11}\pdiff{u}{\xi}
- y_\xi y_{\eta}     \pdiff{}{\eta} \left( \Psi_{11} \pdiff{u}{\xi}\right)
+ y_\xi y_{\eta\xi}  \Psi_{11} \pdiff{u}{\eta}
+ y_\xi y_{\xi}     \pdiff{}{\eta} \left( \Psi_{11} \pdiff{u}{\eta} \right) +
\\
  &
+ y_\xi x_{\eta\eta} \Psi_{12}\pdiff{u}{\xi}
+ y_\xi x_{\eta}     \pdiff{}{\eta} \left( \Psi_{12} \pdiff{u}{\xi}\right)
- y_\xi x_{\eta\xi}  \Psi_{12} \pdiff{u}{\eta}
- y_\xi x_{\xi}      \pdiff{}{\eta} \left( \Psi_{12} \pdiff{u}{\eta} \right)
\Bigg]
+
\\
%
%
%
&\frac{1}{J} \Bigg[
- x_\eta y_{\xi\eta} \Psi_{21} \pdiff{u}{\xi}
- x_\eta y_{\eta}    \pdiff{}{\xi} \left(  \Psi_{21} \pdiff{u}{\xi} \right)
+ x_\eta y_{\xi\xi}  \Psi_{21} \pdiff{u}{\eta}
+ x_\eta y_{\xi}     \pdiff{}{\xi}\left(\Psi_{21} \pdiff{u}{\eta}\right) +
\\*
&
+ x_\eta x_{\xi\eta} \Psi_{2}\pdiff{u}{\xi}
+ x_\eta x_{\eta}    \pdiff{}{\xi} \left( \Psi_{22} \pdiff{u}{\xi}\right)
- x_\eta x_{\xi\xi}  \Psi_{22} \pdiff{u}{\eta}
- x_\eta x_{\xi}     \pdiff{}{\eta} \left( \Psi_{22} \pdiff{u}{\eta} \right) +
\\
&
+ x_\xi y_{\eta\eta} \Psi_{21}\pdiff{u}{\xi}
+ x_\xi y_{\eta}     \pdiff{}{\eta} \left( \Psi_{21} \pdiff{u}{\xi}\right)
- x_\xi y_{\eta\xi}  \Psi_{21} \pdiff{u}{\eta}
- x_\xi y_{\xi}     \pdiff{}{\eta} \left( \Psi_{21} \pdiff{u}{\eta} \right) +
\\*
  &
- x_\xi x_{\eta\eta} \Psi_{22}\pdiff{u}{\xi}
- x_\xi x_{\eta}     \pdiff{}{\eta} \left( \Psi_{22} \pdiff{u}{\xi}\right)
+ x_\xi x_{\eta\xi}  \Psi_{22} \pdiff{u}{\eta}
+ x_\xi x_{\xi}      \pdiff{}{\eta} \left( \Psi_{22} \pdiff{u}{\eta} \right)
\Bigg]
\end{align}
}}
%--------------------------------------------------------------------------------
\subsection*{Isotropic and on a cartesian grid}
When the heat conductivity coefficient is isotropic ($\Psi_{11} = \Psi_{22} = \Psi(\xi, \eta)$ and $\Psi_{12} = \Psi_{21} = 0$), on a cartesian grid ($y_\xi = x_\eta = \textit{constant}$) and multiplied with $J$ this term reduces to
reduce to:
\begin{align}
    &\frac{1}{J} \Bigg[
      y_\eta y_{\eta}    \pdiff{}{\xi} \left( \Psi \pdiff{u}{\xi}  \right)
    - y_\eta y_{\xi}     \pdiff{}{\xi} \left( \Psi \pdiff{u}{\eta} \right) +
    \\*
    & \qquad
    - y_\xi y_{\eta}     \pdiff{}{\eta} \left( \Psi \pdiff{u}{\xi}\right)
    + y_\xi y_{\xi}     \pdiff{}{\eta} \left( \Psi \pdiff{u}{\eta} \right)
    \Bigg]
    +
    \\
    %
    %
    %
    + &\frac{1}{J} \Bigg[
    x_\eta x_{\eta}    \pdiff{}{\xi} \left( \Psi \pdiff{u}{\xi}\right)
    - x_\eta x_{\xi}     \pdiff{}{\eta} \left( \Psi \pdiff{u}{\eta} \right) +
    \\
    & \qquad
    - x_\xi x_{\eta}     \pdiff{}{\eta} \left( \Psi \pdiff{u}{\xi}\right)
    + x_\xi x_{\xi}      \pdiff{}{\eta} \left( \Psi \pdiff{u}{\eta} \right)
    \Bigg]
\end{align}
\notyet
Integration of the term for the eight quadrature points $qp$, for each face of the control volume $scv$  we get:
%
\begin{align}
&
\left( \Psi_{11} \pdiff{{u}}{x} + \Psi_{12} \pdiff{{u}}{y} \right) n_x \norm{dl} +
\left( \Psi_{21} \pdiff{{u}}{x} + \Psi_{22} \pdiff{{u}}{y} \right) n_y \norm{dl}
\end{align}
If the  viscosity is isotropic ($\Psi_{12} = \Psi_{21} = 0$) these equations reduce to:
\begin{align}
    &
    \left( \Psi_{11} \pdiff{{u}}{x} \right) n_x \norm{dl} +
    \left( \Psi_{22} \pdiff{{u}}{y} \right) n_y \norm{dl}
\end{align}

Define the components of $\bPsi$ at different locations with $(\mu, \nu) \in \{ 1, 2\}$ as
%
\begin{align}
    \left. \overline\Psi_{\mu\nu} \right|_{i+\half,j+\quart} = \frac{1}{8} \left(\left. 3\Psi_{\mu\nu} \right|_{i,j}
    + \left. 3\Psi_{\mu\nu} \right|_{i+1,j} + \left. \Psi_{\mu\nu} \right|_{i+1,j+1}  + \left. \Psi_{\mu\nu} \right|_{i,j+1}\right)
\end{align}
%
and define the partial differentials as
\begin{align}
    \left. \pdiff{u}{x} \right|_{i+\half,j+\quart} =  \quart \left(3\frac{u_{i+1,j} - u_{i,j}}{\Dx_{i+\half,j}} + \frac{u_{i+1,j+1} - u_{i,j+1}}{\Dx_{i+\half,j+1}}\right)
    \\
    \left. \pdiff{u}{y} \right|_{i+\half,j+\quart} =  \half \left(\frac{u_{i,j+1} - u_{i,j}}{\Dy_{i,j+\half}} + \frac{u_{i+1,j+1} - u_{i+1,j}}{\Dy_{i+1,j+\half}}\right)
\end{align}

In case the viscosity is isotropic it reads:
\begin{align}
    \bPsi =
    \begin{pmatrix}
        \Psi & 0 \\
        0 & \Psi
    \end{pmatrix}
\end{align}
(isotropic, $\Psi_{11} = \Psi_{22} = \Psi$ and $\Psi_{12} = \Psi_{21} = 0$).
and then the discretization is straight forward.

%================================================================================
\newpage
\appendix
\chapter{Curvilinear coordinate transformation}\label{app:CoordinateTransform}

\subsection{Viscosity term}
The $x$ and $y$-momentum equation diffusion terms are treated here as well due to their complexity.
They can be transformed as:
\begin{align}
    \pdiff{}{x}\left(2\nu h \pdiff{(q/h)}{x}\right) =
\end{align}
\begin{align}
    =\frac{1}{J} \left[
    y_{\eta}\pdiff{}{\xi}\left( \frac{2\nu h}{J} \left(
    y_{\eta}\pdiff{(q/h)}{\xi}
    - y_{\xi}\pdiff{(q/h)}{\eta}\right) \right)
    - y_{\xi}\pdiff{}{\eta}\left(\frac{2\nu h}{J} \left(
    y_{\eta}\pdiff{(q/h)}{\xi}
    -y_{\xi}\pdiff{(q/h)}{\eta} \right) \right) \right] =
\end{align}
\begin{align}
    \begin{aligned}
        &=\frac{1}{J} \Bigg[
        y_{\eta} \left( \pdiff{}{\xi}\left( \frac{2\nu h}{J}\right) \left(
        y_{\eta}\pdiff{(q/h)}{\xi}
        - y_{\xi}\pdiff{(q/h)}{\eta}\right)
        + \frac{2\nu h}{J} \pdiff{}{\xi} \left(
        y_{\eta}\pdiff{(q/h)}{\xi}
        - y_{\xi}\pdiff{(q/h)}{\eta}\right)\right) +
        \\
        &- y_{\xi}\left(\pdiff{}{\eta} \left(\frac{2\nu h}{J}\right) \left(
        y_{\eta}\pdiff{(q/h)}{\xi}
        -y_{\xi}\pdiff{(q/h)}{\eta} \right)
        +\frac{2\nu h}{J} \pdiff{}{\eta}\left(
        y_{\eta}\pdiff{(q/h)}{\xi}
        -y_{\xi}\pdiff{(q/h)}{\eta} \right)
        \right) \Bigg] =
    \end{aligned}
\end{align}
\begin{align}
    \begin{aligned}
        &=\frac{2}{J} \Bigg[
        y_{\eta} \left( \pdiff{}{\xi}\left( \frac{\nu h}{J}\right) \left(
        y_{\eta}\pdiff{(q/h)}{\xi}
        - y_{\xi}\pdiff{(q/h)}{\eta}\right)
        + \frac{\nu h}{J} \pdiff{}{\xi} \left(
        y_{\eta}\pdiff{(q/h)}{\xi}
        - y_{\xi}\pdiff{(q/h)}{\eta}\right) \right) +
        \\
        &- y_{\xi}\left(\pdiff{}{\eta} \left(\frac{\nu h}{J}\right) \left(
        y_{\eta}\pdiff{(q/h)}{\xi}
        -y_{\xi}\pdiff{(q/h)}{\eta} \right)
        +\frac{\nu h}{J} \pdiff{}{\eta}\left(
        y_{\eta}\pdiff{(q/h)}{\xi}
        -y_{\xi}\pdiff{(q/h)}{\eta} \right)
        \right) \Bigg]
    \end{aligned}
\end{align}
%
	\begin{align}
    \begin{aligned}
        - \pdiff{}{x}\left(2\nu h\pdiff{(q/h)}{x} \right)
        - \pdiff{}{y}\left(\nu h\pdiff{(r/h)}{x} + \nu h \pdiff{(q/h)}{y} \right) =
    \end{aligned}
\end{align}