\documentclass[subfooter, backcover]{mooiman_report}
\hypersetup
{
    pdfauthor   = {Mooiman},
}
\addbibresource{./common/literature_references.bib}

\usepackage{empheq}
\usepackage{cancel}

\usepackage{bm}
\renewcommand{\vec}[1]{\mbox{\boldmath ${#1}$} }

\sisetup{mode=math}
\captionsetup[subfigure]{justification=RaggedRight}

\allowdisplaybreaks[1]

\newcommand{\ii}{{\rm i}}
\newcommand{\Dx}{\Delta x}
\newcommand{\Dy}{\Delta y}
\newcommand{\Dh}{\Delta h}
\newcommand{\Dq}{\Delta q}
\newcommand{\Dr}{\Delta r}
\newcommand{\Dxi}{\Delta \xi}
\newcommand{\Deta}{\Delta \eta}
\newcommand{\Dt}{\Delta t}
\newcommand{\Dtinv}{\Delta t_{\it{inv}}}

\newcommand{\mat}[1]{\mbox{\boldmath ${\rm #1}$} }
\newcommand{\half}{\frac{1}{2}}
\newcommand{\quart}{\frac{1}{4}}
\newcommand{\abs}[1]{\left\lvert #1 \right\rvert}
\newcommand{\Fabs}[1]{F_{\it abs}(#1)\xspace}  % Continue function for the absolute value
\newcommand{\eps}{\varepsilon}
\newcommand{\norm}[1]{\left\lVert #1 \right\rVert}
\newcommand{\bunit}[1]{\text{\normalfont [}\textit{\unit{#1}}\text{\normalfont ]}}
\newcommand{\bqty}[2]{\text{\num{#1}}\ \text{\normalfont [}\textit{\unit{#2}}\text{\normalfont ]}}
\newcommand{\Pe}{\textit{Pe}\xspace}
\newcommand{\bPsi}{\boldsymbol{\Psi}}

\newcommand{\utilde}{\widetilde{u}}
\newcommand{\ubar}{\overline{u}}
\newcommand{\uhat }{\widehat{u}}
\newcommand{\htilde}{\widetilde{h}}
\renewcommand{\hbar}{\overline{h}}  % overrule the Planck constant
\newcommand{\hhat }{\widehat{h}}
\newcommand{\qtilde}{\widetilde{q}}
\newcommand{\qbar}{\overline{q}}
\newcommand{\qhat }{\widehat{q}}

\newcommand{\maplesoft}{\textbf{Maplesoft}}
\newcommand{\deltaformulation}{$\Delta$-formulation\xspace}
\newcommand{\notyet}{\textbf{Not yet documented}\newline}

\begin{document}

\title{Two step FVE method}
\subtitle{A numerical modeling technique designed for error insight}
\contact{J.\ Mooiman}
\author{Jan Mooiman}
\partner{}

\documentid{ID-20241205}
\version{000}
\date{\today~\currenttime} % start of writing document
\status{\textbf{Work in progress}}

\mooimantitle
%------------------------------------------------------------------------------
\newrefsegment
\nomenclature{$t$}{\si{\second}}{Time coordinate}
\nomenclature{$x$}{\si{\metre}}{$x$-coordinate}
\nomenclature{$y$}{\si{\metre}}{$y$-coordinate}
\nomenclature{$i$}{\si{-}}{node counter}
\nomenclature{$\Omega$}{\si{-}}{Finite volume}
\nomenclature{$u$}{\si{\metre\per\second}}{Velocity in $x$-direction}
\nomenclature{$v$}{\si{\metre\per\second}}{Velocity in $y$-direction}
\nomenclature{$h$}{\si{\metre}}{Total water depth}
\nomenclature{$q$}{\si{\square\metre\per\second}}{The water flux in $x$-direction, $q = hu$}
\nomenclature{$r$}{\si{\square\metre\per\second}}{The water flux in $y$-direction, $r = hv$}
\nomenclature{$\xi$}{\si{-}}{Relative coordinate}
\nomenclature{$\Dt$}{\si{\second}}{Time increment}
\nomenclature{$\Dx$}{\si{\metre}}{Space increment, $\Dx_{i+\half} = x_{i+1} - x_{i}$}
\nomenclature{$\Psi$}{\si{\square\metre\per\second}}{Artificial smoothing coefficient}
\nomenclature{$c_\Psi$}{${(.)}^{-1}$}{Artificial smoothing variable}
\nomenclature{$\vec{E}$}{\si{-}}{Error vector function, defined in computational space}
\nomenclature{$\zeta$}{\si{\metre}}{Water level w.r.t.\ reference plane, positive upward}
\nomenclature{$z_b$}{\si{\metre}}{Bed level w.r.t.\ reference plane, positive upward            }
\nomenclature{$g$}{\si{\metre\per\square\second}}{Gravitational constant}
\nomenclature{$\theta$}{\si{-}}{$\theta$-method. If $\theta=1$ then it is a fully implicit method and if $\theta=0$ then it is a fully explicit method.}
\nomenclature{$\nu$}{\si{\square\metre\per\second}}{Kinematic viscosity}
\nomenclature{$\eps$}{\si{\metre\per\square\second}}{Multiplier of the correction term for the essential boundary condition, $\zeta$-boundary}
\nomenclature{$\eps$}{\si{\per\second}}{Multiplier of the correction term for the essential boundary condition, $q$-boundary}
\nomenclature{$c_f$}{\si{-}}{Bed shear stress coefficient}
\nomenclature{$C$}{\si{\metre^{\half}\, \second^{-1}}}{Ch\'ezy coefficient}



%------------------------------------------------------------------------------
\part{Heat 2D}
%-------------------------------------------------------------------------------
\chapter{Heat 2D equation}
%-------------------------------------------------------------------------------
Consider the non-linear heat equation
\begin{subequations}
    \begin{align}
    \pdiff{u}{t}
    - \nabla \dotp \left( \bPsi \nabla u \right) = 0,
\end{align}
\end{subequations}
with
\begin{symbollist}
    \item[$u$] Temperature, \bunit{\celsius}.
    \item[$\bPsi$] 2-dimensional thermal diffusivity, \bunit{\square\metre\per\second}.
\end{symbollist}
$\bPsi$ the thermal diffusivity is defined as:
\begin{align}
    \bPsi =
    \begin{pmatrix}
        \Psi_{11} & \Psi_{12} \\
        \Psi_{21} & \Psi_{22}
    \end{pmatrix}
\end{align}

\subsection*{Finite Volume approach}
Integrating the equations over a finite volume $\Omega$ yields:
\begin{align}
    \int_\Omega \pdiff{h}{t}\, d\omega &
    - \int_\Omega \nabla \dotp \left( \bPsi \nabla u \right) \, d\omega = 0
    \label{eq:heat_2d_equation}
\end{align}
with



%-------------------------------------------------------------------------------
\section{Space discretization, structured}
\begin{figure}[H]
    \begin{center}
        \def\svgwidth{0.80\textwidth} % scaling text
        \resizebox{0.65\textwidth}{!}{
            \input{figures/cartesian_grid_interior_2d.pdf_tex}
        }
    \end{center}
    \caption[Definition of the grid to solve the 2D-shallow water equations in the interior area]{Coefficients for the mass-matrix and the control volume in 2-dimensions in the interior area, on a structured grid. The black dots indicate the location of the quadrature points, and black diamonds the flux points.}
    \label{fig:2d_structured_grid}
\end{figure}
For the space discretization of an arbitrary function $u$ on the quadrature point of a sub-control volume the following space interpolations are used:
\begin{align}
    \left. u\right|_{i+\quart, j+\quart} & \approx \frac{1}{16}\left( 9u_{i, j} + 3 u_{i+1,j} + u_{i+1, j+1} + 3  u_{i, j+1}\right)
    \\
    \left.u\right|_{i+\half, j+\quart} & \approx \frac{1}{8} \left( 3 u_{i,j} + 3 u_{i+1,j} + u_{i+1, j+1} + u_{i, j+1} \right)
    \\
    \left. u \right|_{i+\quart, j+\half} & \approx \frac{1}{8} \left( 3 u_{i, j} + u_{i+1, j} + u_{i+1, j+1}  + 3 u_{i, j+1} \right)
\end{align}
See for the locations \autoref{fig:2d_structured_grid}.
%-------------------------------------------------------------------------------
\subsection{Discretizations heat equation}
The discretization of heat \autoref{eq:heat_2d_equation} will be presented term by term.
%-------------------------------------------------------------------------------
\subsubsection{Time derivative}
The discretization of the time derivative term of the continuity equation reads:
\begin{align}
    \int_{\Omega} \pdiff{u}{t}\, d\omega
\end{align}
which will be approximated by the sum of the integral over the sub-control volumes.
On a structured grid one control volume ($cv$) around a node consist of four sub-control volumes ($scv_i$, $i\in\{0,1,2,3\}$).
\begin{align}
    \int_{cv} \pdiff{u}{t}\, d\omega =
    \int_{scv_0} \pdiff{u}{t}\, d\omega +
    \int_{scv_1} \pdiff{u}{t}\, d\omega +
    \int_{scv_2} \pdiff{u}{t}\, d\omega +
    \int_{scv_3} \pdiff{u}{t}\, d\omega
\end{align}
For a cartesian grid we get:
\begin{align}
    \int_{cv} \pdiff{h}{t}\, d\omega \approx &
    \quart\Dx\Dy\Dtinv \left( h^{n+1,p+1}_{i-\quart, j-\quart} -  h^{n+1,n}_{i-\quart, j-\quart} \right) +
    \nonumber \\*
    &\quart\Dx\Dy\Dtinv \left( h^{n+1,p+1}_{i+\quart, j-\quart} -  h^{n+1,n}_{i+\quart, j-\quart} \right) +
    \nonumber \\*
    &\quart\Dx\Dy\Dtinv \left( h^{n+1,p+1}_{i+\quart, j+\quart} -  h^{n+1,n}_{i+\quart, j+\quart} \right) +
    \nonumber \\*
    &\quart\Dx\Dy\Dtinv \left( h^{n+1,p+1}_{i-\quart, j+\quart} -  h^{n+1,n}_{i-\quart, j+\quart} \right)
\end{align}
Just looking to the quadrature point of $scv_0$ as part of the control volume for node $(i,j)$ the discretization reads:
\begin{align}
    &\quart\Dx\Dy\Dtinv \left( h^{n+1}_{i-\quart, j-\quart} -  h^{n+1,n}_{i-\quart, j-\quart} \right) =
    \\*
    &= \quart\Dx\Dy\Dtinv \left[ \frac{1}{16}\left( 9h^{n+1}_{i, j} + 3 h^{n+1}_{i-1,j}  + 3  h^{n+1}_{i, j-1} + h^{n+1}_{i-1, j-1}\right) \right. +
    \\*
    &\quad - \left. \frac{1}{16}\left( 9 h^{n}_{i, j} +  3 h^{n}_{i-1,j}  + 3  h^{n}_{i, j-1} + h^{n}_{i-1, j-1}\right)\right]
\end{align}
Written in \deltaformulation  it reads:
\begin{align}
    &\quart\Dx\Dy\Dtinv \left( h^{n+1}_{i-\quart, j-\quart} -  h^{n}_{i-\quart, j-\quart} \right) =
    \nonumber \\*
    &=\quart\Dx\Dy\Dtinv \left[ \frac{1}{16}\left( 9 \Delta h^{n+1,p+1}_{i, j} + 3 \Delta q^{n+1,p+1}_{i-1,j}  + 3 \Delta q^{n+1,p+1}_{i, j-1} + \Delta h^{n+1,p+1}_{i-1, j-1}\right) \right] +
    \nonumber \\*
    & + \quart\Dx\Dy\Dtinv \left[ \frac{1}{16}\left( 9 h^{n+1,p}_{i, j} + 3 h^{n+1,p}_{i-1,j}  + 3 h^{n+1,p}_{i, j-1} + h^{n+1,p+1}_{i-1, j-1}\right) \right. +
    \nonumber \\*
    &\quad - \left. \frac{1}{16}\left( 9h^{n}_{i, j} +  3 h^{n}_{i-1,j}  + 3  h^{n}_{i, j-1} + h^{n}_{i-1, j-1}\right)\right]
\end{align}

%--------------------------------------------------------------------------------
\subsubsection{Viscosity}
The viscosity term in vector notation reads:
\begin{align}
    &\int_\Omega \nabla \dotp \left( \mat{\Psi} \nabla u \right) \, d\omega  =
    \oint_{\Omega}  \left(  \mat{\Psi} \nabla u \right) \dotp \vec{n}\, dl \approx
    \\
    & \oint_{\Omega}  \left( \left( \Psi_{11} \pdiff{{u}}{x} + \Psi_{12} \pdiff{{u}}{y}  \right) n_x + \left( \Psi_{21} \pdiff{{u}}{x} + \Psi_{22} \pdiff{{u}}{y}  \right) n_y \right) \norm{dl}
\end{align}
with $\vec{n} = (n_x, n_y)^T$ the outward normal vector.

Integration of the term for the eight quadrature points $qp$, for each face of the control volume $scv$  we get:
%
\begin{align}
&
\left( \Psi_{11} \pdiff{{u}}{x} + \Psi_{12} \pdiff{{u}}{y} \right) n_x \norm{dl} +
\left( \Psi_{21} \pdiff{{u}}{x} + \Psi_{22} \pdiff{{u}}{y} \right) n_y \norm{dl}
\end{align}
If the  viscosity is isotropic ($\Psi_{12} = \Psi_{21} = 0$) these equations reduce to:
\begin{align}
    &
    \left( \Psi_{11} \pdiff{{u}}{x} \right) n_x \norm{dl} +
    \left( \Psi_{22} \pdiff{{u}}{y} \right) n_y \norm{dl}
\end{align}

Define the components of $\bPsi$ at different locations with $(\mu, \nu) \in \{ 1, 2\}$ as
%
\begin{align}
    \left. \overline\Psi_{\mu\nu} \right|_{i+\half,j+\quart} = \frac{1}{8} \left(\left. 3\Psi_{\mu\nu} \right|_{i,j}
    + \left. 3\Psi_{\mu\nu} \right|_{i+1,j} + \left. \Psi_{\mu\nu} \right|_{i+1,j+1}  + \left. \Psi_{\mu\nu} \right|_{i,j+1}\right)
\end{align}
%
and define the partial differentials as
\begin{align}
    \left. \pdiff{u}{x} \right|_{i+\half,j+\quart} =  \quart \left(3\frac{u_{i+1,j} - u_{i,j}}{\Dx_{i+\half,j}} + \frac{u_{i+1,j+1} - u_{i,j+1}}{\Dx_{i+\half,j+1}}\right)
    \\
    \left. \pdiff{u}{y} \right|_{i+\half,j+\quart} =  \half \left(\frac{u_{i,j+1} - u_{i,j}}{\Dy_{i,j+\half}} + \frac{u_{i+1,j+1} - u_{i+1,j}}{\Dy_{i+1,j+\half}}\right)
\end{align}

In case the viscosity is isotropic it reads:
\begin{align}
    \bPsi =
    \begin{pmatrix}
        \Psi & 0 \\
        0 & \Psi
    \end{pmatrix}
\end{align}
(isotropic, $\Psi_{11} = \Psi_{22} = \Psi$ and $\Psi_{12} = \Psi_{21} = 0$).
and then the discretization is straight forward.
%--------------------------------------------------------------------------------
\end{document}